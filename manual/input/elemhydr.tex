% MBDyn (C) is a multibody analysis code. 
% http://www.mbdyn.org
% 
% Copyright (C) 1996-2024
% 
% Pierangelo Masarati	<pierangelo.masarati@polimi.it>
% Paolo Mantegazza	<paolo.mantegazza@polimi.it>
% 
% Dipartimento di Ingegneria Aerospaziale - Politecnico di Milano
% via La Masa, 34 - 20156 Milano, Italy
% http://www.aero.polimi.it
% 
% Changing this copyright notice is forbidden.
% 
% This program is free software; you can redistribute it and/or modify
% it under the terms of the GNU General Public License as published by
% the Free Software Foundation (version 2 of the License).
% 
% 
% This program is distributed in the hope that it will be useful,
% but WITHOUT ANY WARRANTY; without even the implied warranty of
% MERCHANTABILITY or FITNESS FOR A PARTICULAR PURPOSE.  See the
% GNU General Public License for more details.
% 
% You should have received a copy of the GNU General Public License
% along with this program; if not, write to the Free Software
% Foundation, Inc., 59 Temple Place, Suite 330, Boston, MA  02111-1307  USA
% 

\section{Hydraulic Element}
\label{sec:EL:HYDR}
{\em 
    Note: under development; syntax subjected to changes \\
    Initially implemented by: Lamberto Puggelli \\
    Reviewed by: Pierangelo Masarati
}

%\begin{verbatim}
\begin{Verbatim}[commandchars=\\\{\}]
    \bnt{elem_type} ::= \kw{hydraulic}

    \bnt{normal_arglist} ::= \bnt{hydr_elem_type} , \bnt{hydr_elem_data}
\end{Verbatim}
%\end{verbatim}
The field \nt{hydraulic\_element\_data} usually contains information
about fluid properties, which are handled by means of a \hty{HydraulicFluid}.
This can be directly inserted, following the syntax described in
Section~\ref{sec:HydraulicFluid} preceded by the keyword \kw{fluid}, or a
previously defined fluid can be recalled by using the keyword 
\kw{reference} followed by the label of the desired fluid:
%\begin{verbatim}
\begin{Verbatim}[commandchars=\\\{\}]
    \bnt{hydraulic_fluid_properties} ::=
        \{ \bnt{hydraulic_fluid_specification}
            | \kw{reference} , \bnt{hydraulic_fluid_label} \}
\end{Verbatim}
%\end{verbatim}



\subsection{Accumulator}
\label{sec:EL:HYDR:ACCUMULATOR}
Not documented yet.



\subsection{Actuator}
The \kw{actuator} element models the hydraulic and interactional aspects
of a two-way linear hydraulic actuator.
The two hydraulic nodes represent the pressure in the two chambers
of the cylinder.
The two structural nodes represent the cylinder and the piston.
Their relative motion is assumed to consist essentially
in a displacement along a given direction in the reference frame
of structural node 1.
The user must take care of this by constraining the two nodes
with an appropriate set of \kw{joint} elements.
Other hydraulic aspects of this component, like leakages between
the two chambers and between the chambers and the outside,
and the related pressure losses, must be explicitly taken care of,
e.g.\ by means of \kw{minor loss} elements.
\label{sec:EL:HYDR:ACTUATOR}
%\begin{verbatim}
\begin{Verbatim}[commandchars=\\\{\}]
    \bnt{hydr_elem_type} ::= \kw{actuator}

    \bnt{hydr_elem_data} ::=
        \bnt{node_1} , \bnt{node_2} , 
        \bnt{struct_node_1} , (\hty{Vec3}) \bnt{offset_1} ,
        \bnt{struct_node_2} , (\hty{Vec3}) \bnt{offset_2} ,
        [ \kw{direction} , ((\ty{Unit})\hty{Vec3}) \bnt{direction} , ]
        \bnt{area_1} ,
        \bnt{area_2} ,
        \bnt{cylinder_length} ,
        \kw{fluid} , (\hty{HydraulicFluid}) \bnt{fluid_1} ,
        \{ \kw{same} | \kw{fluid} , (\hty{HydraulicFluid}) \bnt{fluid_2} \}
\end{Verbatim}
%\end{verbatim}
The vector \nt{direction} is internally normalized to unity.
By default, it is direction 3 in the reference frame of the \kw{structural}
node \nt{struct\_node\_1}.

\paragraph{Example.} \
See the ``actuator'' example at
\begin{quote}
\htmladdnormallink{\kw{http://www.aero.polimi.it/mbdyn/documentation/examples/actuator}}{http://www.aero.polimi.it/mbdyn/documentation/examples/actuator}
\end{quote}
and the related chapter of the tutorials
\begin{quote}
\htmladdnormallink{\kw{http://www.aero.polimi.it/mbdyn/documentation/tutorials/}}{http://www.aero.polimi.it/mbdyn/documentation/tutorials/}
\end{quote}



\subsection{Control Valve}
\label{sec:EL:HYDR:CONTROL_VALVE}
%\begin{verbatim}
\begin{Verbatim}[commandchars=\\\{\}]
    \bnt{hydr_elem_type} ::= \kw{control valve}

    \bnt{hydr_elem_data} ::=
        \bnt{node_1} , \bnt{node_2} ,
        \bnt{node_3} , \bnt{node_4} ,
        \bnt{area} ,
        [ \kw{loss} , \bnt{loss_factor} , ]
        (\hty{DriveCaller}) \bnt{state} ,
        \kw{fluid} , (\hty{HydraulicFluid}) \bnt{fluid}
\end{Verbatim}
%\end{verbatim}
This element represents a valve that connects
\nt{node\_1} to \nt{node\_2} and \nt{node\_3} to \nt{node\_4}
when \nt{state} is positive, and \nt{node\_1} to \nt{node\_3}
and \nt{node\_2} to \nt{node\_4} when \nt{state} is negative.
The flow area is proportional to \nt{area} times the norm of \nt{state},
which must be comprised between $-1$ and $1$.
If \nt{loss\_factor} is defined, it represents the fraction
of area that leaks when \nt{state} is exactly zero.



\subsection{Dynamic Control Valve}
\label{sec:EL:HYDR:DYNAMIC_CONTROL_VALVE}
%\begin{verbatim}
\begin{Verbatim}[commandchars=\\\{\}]
    \bnt{hydr_elem_type} ::= \kw{dynamic control valve}

    \bnt{hydr_elem_data} ::=
        \bnt{node_1} , \bnt{node_2} ,
        \bnt{node_3} , \bnt{node_4} ,
        (\hty{DriveCaller}) \bnt{force} ,
        \bnt{initial_displacement} ,
        \bnt{max_displacement} ,
        \bnt{duct_width} ,
        [ \kw{loss} , \bnt{loss_factor} , ]
        \bnt{valve_diameter} ,
        \bnt{valve_density} ,
        \bnt{displacement_penalty} ,
        \bnt{velocity_penalty} ,
        \bnt{acceleration_penalty} ,
        \kw{fluid} , (\hty{HydraulicFluid}) \bnt{fluid}
\end{Verbatim}
%\end{verbatim}
This element represents a valve that connects
\nt{node\_1} to \nt{node\_2} and \nt{node\_3} to \nt{node\_4}
when the displacement is positive and \nt{node\_1} to \nt{node\_3}
and \nt{node\_2} to \nt{node\_4} when the displacement is negative,
accounting for the dynamics of the valve body.
The control force \nt{force} is applied to the valve, whose 
geometric and structural properties are described by 
\nt{initial\_displacement}, \nt{max\_displacement},
\nt{duct\_width}, \nt{valve\_diameter} and \nt{valve\_density}.
Again the \nt{loss\_factor}, if defined, represents the fraction
of the area that leaks when the displacement is zero.
Finally, \nt{displacement\_penalty}, \nt{velocity\_penalty} and \nt{acceleration\_penalty}
are the penalty coefficients for displacement, velocity and acceleration
when the maximum stroke is reached.




\subsection{Dynamic Pipe}
\label{sec:EL:HYDR:DYNAMIC_PIPE}
Same syntax as the \kw{pipe} hydraulic element
(Section~\ref{sec:EL:HYDR:PIPE}),
it also considers fluid compressibility in terms of pressure time derivative,
and thus the corresponding dynamic effect.
%\begin{verbatim}
\begin{Verbatim}[commandchars=\\\{\}]
    \bnt{hydr_elem_type} ::= \kw{dynamic pipe}
\end{Verbatim}
%\end{verbatim}

\subsubsection{Output}
\begin{itemize}
\item[1:] label
\item[2:] pressure at node 1
\item[3:] pressure at node 2
\item[4:] pressure derivative at node 1
\item[5:] pressure derivative at node 2
\item[6:] flow at node 1
\item[7:] flow at node 2
\item[8:] flow rate at node 1
\item[9:] flow rate at node 2
\item[10:] density at node 1
\item[11:] reference density
\item[12:] density at node 2
\item[13:] derivative of density with respect to pressure at node 1
\item[14:] derivative of density with respect to pressure at node 2
\item[15:] Reynolds number
\item[16:] flow type flag (boolean; true when flow is turbulent, false otherwise)
\end{itemize}



\subsection{Flow Valve}
\label{sec:EL:HYDR:FLOW_VALVE}
%\begin{verbatim}
\begin{Verbatim}[commandchars=\\\{\}]
    \bnt{hydr_elem_type} ::= \kw{flow valve}

    \bnt{hydr_elem_data} ::=
        \bnt{node_1} , \bnt{node_2} , \bnt{node_3} ,
        \bnt{area} ,
        \bnt{mass} ,
        \bnt{max_area} ,
        \bnt{spring_stiffness} ,
        \bnt{spring_preload} ,
        \bnt{width} ,
        \bnt{displacement_penalty} ,
        \bnt{velocity_penalty} ,
        \bnt{acceleration_penalty} ,
        \kw{fluid} , (\hty{HydraulicFluid}) \bnt{fluid}
\end{Verbatim}
%\end{verbatim}



\subsection{Imposed Flow}
\label{sec:EL:HYDR:IMPOSED_FLOW}
No specific element has been implemented to impose the flow at one node.
The \kw{abstract} variant of the \kw{force} element
(Section~\ref{sec:EL:FORCE:ABSTRACT}) can be used instead.
The magnitude of the abstract force is the mass flow extracted
from the circuit at that node.
In fact, a negative value of the abstract force means that the flow enters
the node.

\paragraph{Example.} \
\begin{verbatim}
    set: integer HYDRAULIC_NODE_LABEL = 100;
    set: integer IMPOSED_FLOW_LABEL = 200;
    force: IMPOSED_PRESSURE_LABEL, abstract,
        HYDRAULIC_NODE_LABEL, hydraulic,
        const, -1e-3;
\end{verbatim}



\subsection{Imposed Pressure}
\label{sec:EL:HYDR:IMPOSED_PRESSURE}
No specific element has been implemented to impose the pressure at one node.
The \kw{clamp} variant of the \kw{genel} element
(Section~\ref{sec:EL:GENEL:CLAMP}) can be used instead.

\paragraph{Example.} \
\begin{verbatim}
    set: integer HYDRAULIC_NODE_LABEL = 100;
    set: integer IMPOSED_PRESSURE_LABEL = 200;
    genel: IMPOSED_PRESSURE_LABEL, clamp,
        HYDRAULIC_NODE_LABEL, hydraulic,
        const, 101325.0;
\end{verbatim}




\subsection{Minor Loss}
A pressure loss between two pressure nodes.
\label{sec:EL:HYDR:MINOR_LOSS}
%\begin{verbatim}
\begin{Verbatim}[commandchars=\\\{\}]
    \bnt{hydr_elem_type} ::= \kw{minor loss}

    \bnt{hydr_elem_data} ::=
        \bnt{node_1} , \bnt{node_2} ,
        \bnt{k12} , \bnt{k21} , \bnt{area} ,
        \kw{fluid} , (\hty{HydraulicFluid}) \bnt{fluid}
\end{Verbatim}
%\end{verbatim}
Coefficients \nt{k12} and \nt{k21} characterize the pressure loss
when the flow goes from \nt{node\_1} to \nt{node\_2} and vice versa.
Turbulent flow is assumed.



\subsection{Orifice}
\label{sec:EL:HYDR:ORIFICE}
%\begin{verbatim}
\begin{Verbatim}[commandchars=\\\{\}]
    \bnt{hydr_elem_type} ::= \kw{orifice}

    \bnt{hydr_elem_data} ::=
        \bnt{node_1} , \bnt{node_2} ,
        \bnt{pipe_diameter} ,
        \bnt{orifice_area} , # diaphragm area
        [ \kw{area} , \bnt{pipe_area} , ] # defaults to (\bnt{pipe_diameter}/2)^2*pi
        [ \kw{reynolds} , \bnt{critical_reynolds_number} , ] # defaults to 10
        \kw{fluid} , (\hty{HydraulicFluid}) \bnt{fluid}
\end{Verbatim}
%\end{verbatim}



\subsection{Pipe}
\label{sec:EL:HYDR:PIPE}
This element models a simple pipeline connecting two \kw{hydraulic} nodes.
In detail, it models the pressure loss due to fluid viscosity
according to a constitutive law that depends on the Reynolds number
computed on average fluid velocity and hydraulic diameter.
Transition between laminar and turbulent flow is also modeled.
%\begin{verbatim}
\begin{Verbatim}[commandchars=\\\{\}]
    \bnt{hydr_elem_type} ::= \kw{pipe}

    \bnt{hydr_elem_data} ::=
        \bnt{node_1} , \bnt{node_2} ,
        \bnt{hydraulic_diameter} ,
        [ \kw{area} , \bnt{area} , ]
        \bnt{length} ,
        [ \kw{turbulent} , ]
        [ \kw{initial value} , \bnt{flow} , ]
        \kw{fluid} , (\hty{HydraulicFluid}) \bnt{fluid}
\end{Verbatim}
%\end{verbatim}
When \kw{area} is not given, it defaults to the equivalent area
computed according to the hydraulic diameter.
The flag \kw{turbulent} forces the flow to be considered turbulent
since the first iteration, when the initial conditions would fall
in the transition region.
The \kw{initial value} parameter refers to the initial value
of the flow internal state.

\paragraph{Example.} \
\begin{verbatim}
    set: integer NODE_1 = 10;
    set: integer NODE_2 = 20;
    set: integer PIPE = 100;
    set: integer FLUID = 1000;
    # ...
    hydraulic: PIPE, pipe, NODE_1, NODE_2,
        5e-3, 100.e-3,
        fluid, reference, FLUID;
\end{verbatim}



\subsection{Pressure Flow Control Valve}
\label{sec:EL:HYDR:PRESSURE_FLOW_CONTROL_VALVE}
%\begin{verbatim}
\begin{Verbatim}[commandchars=\\\{\}]
    \bnt{hydr_elem_type} ::= \kw{pressure flow control valve}

    \bnt{hydr_elem_data} ::=
        \bnt{node_1} , \bnt{node_2} ,
        \bnt{node_3} , \bnt{node_4} ,
        \bnt{node_5} , \bnt{node_6} ,
        (\hty{DriveCaller}) \bnt{force} ,
        \bnt{initial_displacement} ,
        \bnt{max_displacement} ,
        \bnt{duct_width} ,
        [ \kw{loss} , \bnt{loss_factor} , ]
        \bnt{valve_diameter} ,
        \bnt{valve_density} ,
        \bnt{displacement_penalty} ,
        \bnt{velocity_penalty} ,
        \bnt{acceleration_penalty} ,
        \kw{fluid} , (\hty{HydraulicFluid}) \bnt{fluid}
\end{Verbatim}
%\end{verbatim}
Same as Dynamic Control Valve (\ref{sec:EL:HYDR:DYNAMIC_CONTROL_VALVE}),
only the pressures at \nt{node\_5} and \nt{node\_6} are applied
at the sides of the valve body and participate in the force balance.



\subsection{Pressure Valve}
\label{sec:EL:HYDR:PRESSURE_VALVE}
%\begin{verbatim}
\begin{Verbatim}[commandchars=\\\{\}]
    \bnt{hydr_elem_type} ::= \kw{pressure valve}

    \bnt{hydr_elem_data} ::=
        \bnt{node_1} , \bnt{node_2} ,
        \bnt{area} ,
        \bnt{mass} ,
        \bnt{max_area} ,
        \bnt{spring_stiffness} ,
        \bnt{spring_preload} ,
        \bnt{width} ,
        \bnt{displacement_penalty} ,
        \bnt{velocity_penalty} ,
        \bnt{acceleration_penalty} ,
        \kw{fluid} , (\hty{HydraulicFluid}) \bnt{fluid}
\end{Verbatim}
%\end{verbatim}



\subsection{Tank}
\label{sec:EL:HYDR:TANK}
Not documented yet.



\subsection{Three Way Minor Loss}
A pressure loss between three pressure nodes,
depending on the sign of the pressure drop.
\label{sec:EL:HYDR:THREE_WAY_MINOR_LOSS}
%\begin{verbatim}
\begin{Verbatim}[commandchars=\\\{\}]
    \bnt{hydr_elem_type} ::= \kw{three way minor loss}

    \bnt{hydr_elem_data} ::=
        \bnt{node_1} , \bnt{node_2} , \bnt{node_3} ,
        \bnt{k12} , \bnt{k31} , \bnt{area_12} , \bnt{area_31} ,
        \kw{fluid} , (\hty{HydraulicFluid}) \bnt{fluid}
\end{Verbatim}
%\end{verbatim}
Coefficients \nt{k12} and \nt{k31} and the respective values of area
characterize the pressure loss when the flow goes
from \nt{node\_1} to \nt{node\_2} and from \nt{node\_3} to \nt{node\_1},
respectively.
Turbulent flow is assumed.




