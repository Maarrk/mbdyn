% $Header$
% MBDyn (C) is a multibody analysis code.
% http://www.mbdyn.org
%
% Copyright (C) 1996-2023
%
% Pierangelo Masarati  <pierangelo.masarati@polimi.it>
%
% Dipartimento di Ingegneria Aerospaziale - Politecnico di Milano
% via La Masa, 34 - 20156 Milano, Italy
% http://www.aero.polimi.it
%
% Changing this copyright notice is forbidden.
%
% This program is free software; you can redistribute it and/or modify
% it under the terms of the GNU General Public License as published by
% the Free Software Foundation (version 2 of the License).
% 
%
% This program is distributed in the hope that it will be useful,
% but WITHOUT ANY WARRANTY; without even the implied warranty of
% MERCHANTABILITY or FITNESS FOR A PARTICULAR PURPOSE.  See the
% GNU General Public License for more details.
%
% You should have received a copy of the GNU General Public License
% along with this program; if not, write to the Free Software
% Foundation, Inc., 59 Temple Place, Suite 330, Boston, MA  02111-1307  USA

\chapter{Nodes}\label{sec:NODES}
The \kw{nodes} section is enclosed in the cards:
%\begin{verbatim}
\begin{Verbatim}[commandchars=\\\{\}]
    \kw{begin} : \kw{nodes} ;
        # ...
    \kw{end} : \kw{nodes} ;
\end{Verbatim}
%\end{verbatim}
Every node card has the format:
%\begin{verbatim}
\begin{Verbatim}[commandchars=\\\{\}]
    \bnt{card} ::= \bnt{node_type} : \bnt{node_label}
        \bnt{additional_args}
        [ , \kw{scale} , \{ \kw{default} | \bnt{scale} \} ]
        [ , \kw{output} , \{ \kw{yes} | \kw{no} | (\ty{bool})\bnt{output_flag} \} ]
        [ , \bnt{extra_arglist} ] ;
\end{Verbatim}
%\end{verbatim}
where \nt{node\_type} is one of the following:
\begin{itemize}
    \item \kw{abstract}
    \item \kw{electric}
    \item \kw{hydraulic}
    \item \kw{parameter}
    \item \kw{structural}
    \item \kw{thermal}
\end{itemize}
The data manager reads the node type and the label and checks for
duplication.
If the node is not defined yet, the appropriate read function is
called, which parses the rest of the card and constructs the node.

The optional \kw{scale} keyword is only supported by
\begin{itemize}
    \item \kw{abstract}
    \item \kw{electric}
    \item \kw{hydraulic}
    \item \kw{structural}
    \item \kw{thermal}
\end{itemize}
nodes.
See Section~\ref{sec:CONTROLDATA:DEFAULTSCALE} for further details.



\section{Abstract Node}
\label{sec:NODE:ABSTRACT}
%\begin{verbatim}
\begin{Verbatim}[commandchars=\\\{\}]
    \bnt{additional_args} ::= , \{ \kw{algebraic} | \kw{differential} \}
        [ , \kw{value} , \bnt{initial_value}
        [ , \kw{derivative} , \bnt{derivative_initial_value} ] ]
\end{Verbatim}
%\end{verbatim}
\emph{
	Note: abstract nodes are ancestors of all scalar node types.
	Many \kw{genel} and \kw{electric} elements can be connected
	to \kw{abstract} nodes as well, since they directly use
	the ancestor class. 
}

\paragraph{Output.}
The value of abstract nodes is output with file extension \texttt{.abs}; for
each time step the output of the required nodes is written.
The format of each row is
\begin{itemize}
    \item the label of the node
    \item the value of the node
    \item the value of the node derivative, when \kw{differential} (the default)
\end{itemize}

\paragraph{Private Data.}
\label{sec:NODE:ABSTRACT:PRIV}
The following data are available:
\begin{enumerate}
\item \kw{"x"} state
\item \kw{"xP"} state time derivative, when \kw{differential} (the default)
\end{enumerate}



\section{Electric Node}
\label{sec:NODE:ELECTRIC}
%\begin{verbatim}
\begin{Verbatim}[commandchars=\\\{\}]
    \bnt{additional_args} ::= [ , \kw{value} , \bnt{initial_value}
        [ , \kw{derivative} , \bnt{derivative_initial_value} ] ]
\end{Verbatim}
%\end{verbatim}
\emph{Note: the keywords \kw{value} and \kw{derivative}
have been introduced recently; \kw{value} is not mandatory,
resulting in a warning, while \kw{derivative} is required.
The same applies to the \kw{abstract node} 
and to the \kw{hydraulic node}; the latter is an algebraic
node, so only \kw{value} is allowed.
}

\paragraph{Private Data.}
\label{sec:NODE:ELECTRIC:PRIV}
The following data are available:
\begin{enumerate}
\item \kw{"x"} voltage
\item \kw{"xP"} voltage time derivative
\end{enumerate}





\section{Hydraulic Node}
\label{sec:NODE:HYDRAULIC}
The hydraulic node represents the pressure
at a given location of a hydraulic circuit.
It is derived from the generic scalar algebraic node; as a consequence,
it can be connected to all elements that operate on generic scalar nodes. 

The syntax is
%\begin{verbatim}
\begin{Verbatim}[commandchars=\\\{\}]
    \bnt{additional_args} ::= [ , \kw{value} , \bnt{initial_value} ]
\end{Verbatim}
%\end{verbatim}

\paragraph{Private Data.}
\label{sec:NODE:HYDRAULIC:PRIV}
The following data are available:
\begin{enumerate}
\item \kw{"x"} pressure
\end{enumerate}



\section{Parameter Node}
\label{sec:NODE:PARAMETER}

\emph{NOTE: \kw{parameter} nodes are essentially obsolete;
in most cases their purpose can be better fulfilled
by \kw{element} and \kw{node} drive callers.}

%\begin{verbatim}
\begin{Verbatim}[commandchars=\\\{\}]
    \bnt{additional_args} ::= \{ [ , \kw{value} , \bnt{initial_value} ]
        | , \kw{element}
        | , \kw{sample and hold} , (\hty{NodeDof}) \bnt{signal} , \bnt{sample_time}
        | , \kw{beam strain gage} , \bnt{y} , \bnt{z} \}
\end{Verbatim}
%\end{verbatim}
The parameter node is derived from the class scalar algebraic node, but it
is used in a rather peculiar way: it doesn't own any degree of freedom,
so it does not participate in the solution; it is rather used as a sort of
placeholder for those elements that require to be connected to a scalar node
that is not otherwise significant to the analysis.
Thanks to the availability of the \kw{parameter} node, these elements
do not need be reformulated with a grounded node, while the parameter
node value can be changed during the solution by several means,
listed in the following.

\paragraph{Element.}
\label{sec:NODE:PARAMETER:ELEMENT}
When the argument list starts with the keyword \kw{element}, the parameter
node expects to be bound to an element, and to access bulk element data 
(see the \hyperref{\kwnd{bind} statement}{\kw{bind} statement, Section~}{}{sec:EL:MISC:BIND}).

\paragraph{Sample and Hold.}
\label{sec:NODE:PARAMETER:SAH}
When the argument list starts with the keyword \kw{sample and hold},
followed by a \hty{NodeDof} specification and a sample time,
the parameter node contains the value of the input signal, namely
the value of the node, for the duration of the \nt{sample\_time}.
This may be useful to preserve the value of some signal across
time steps.

\paragraph{Beam Strain Gage.}
\label{sec:NODE:PARAMETER:BEAM_STRAIN_GAGE}
When the argument list starts with the keyword \kw{beam strain gage},
followed by the coordinates of a point on the section of a beam,
the \kw{parameter} node expects to be bound to a \kw{beam} element,
and to access the measure of the axial strain at point \nt{x}, \nt{y}
in the section plane as a combination of section strains and curvatures:
\begin{equation}
	\varepsilon = \nu_x
		+ \nt{z} \cdot \kappa_y 
		- \nt{y} \cdot \kappa_z ,
\end{equation}
where
\begin{itemize}
\item $\nu_x$ is the axial strain of the beam;
\item $\kappa_y$ is the bending curvature of the beam about the $y$ axis;
\item $\kappa_z$ is the bending curvature of the beam about the $z$ axis.
\end{itemize}
The span-wise location of the point where the strain is evaluated
is set in the 
\hyperref{\kwnd{bind} statement}{\kw{bind} statement (see Section~}{)}{sec:EL:MISC:BIND}.

\emph{Note: measuring strains by means of derivatives of interpolated
positions and orientations may lead to inaccurate results; force summation
should be used instead.}





\section{Structural Node}
\label{sec:NODE:STRUCTURAL}
Structural nodes can have 6 degrees of freedom (position and orientation),
and thus describe the kinematics of rigid-body motion in space,
or 3 degrees of freedom (position) and thus describe the kinematics
of point mass motion in space.

The former has been originally implemented in MBDyn;
the latter has been added in MBDyn 1.5.0,
mainly to support the implementation of membrane elements.

The 6 dof structural node can be \kw{static},
\kw{dynamic}, \kw{modal} or \kw{dummy}.

The 3 dof structural node can be \kw{static} or \kw{dynamic}.

Elements that only require displacement can be connected to either type
of nodes; when connected to a 6 degree of freedom node, they only make use
of position/velocity and only contribute to force equilibrium equations.

Elements that require both displacement and orientation can only be connected
to 6 degree of freedom nodes.



\subsection{Static Node}
\label{sec:NODE:STRUCTURAL:STATIC}
The \kw{static} keyword means no inertia is related to that node, 
so it must be appropriately constrained or attached to elastic elements.
Static nodes are useful when there is no need to apply inertia
to them, thus saving 6 degrees of freedom.

\subsection{Dynamic Node}
\label{sec:NODE:STRUCTURAL:DYNAMIC}
The \kw{dynamic} keyword means inertia can be attached to the node, 
so it provides linear and angular momenta degrees of freedom, 
and automatically generates the so-called \kw{automatic structural}
elements.

\subsection{Modal Node}
\label{sec:NODE:STRUCTURAL:MODAL}
The \kw{modal} node is basically a regular \kw{dynamic} node
that must be used to describe the rigid reference motion
of a \kw{modal} joint.
See Section~\ref{sec:EL:STRUCT:JOINT:MODAL} for further details.

\subsection{Syntax}
\label{sec:NODE:STRUCTURAL:SYNTAX}
The syntax of the 6 degrees of freedom rigid-body motion structural node is
%\begin{verbatim}
\begin{Verbatim}[commandchars=\\\{\}]
    \bnt{additional_args} ::= , \{ \kw{static} | \kw{dynamic} | \kw{modal} \} ,
        [ \kw{position} , ]         (\hty{Vec3})              \bnt{absolute_position} ,
        [ \kw{orientation} , ]      (\hty{OrientationMatrix}) \bnt{absolute_orientation_matrix} ,
            [ \kw{orientation description} , \bnt{orientation_type} , ]
        [ \kw{velocity} , ]         (\hty{Vec3})              \bnt{absolute_velocity} ,
        [ \kw{angular velocity} , ] (\hty{Vec3})              \bnt{absolute_angular_velocity}
        [ , \kw{assembly}
          , (\ty{real}) \bnt{position_initial_stiffness}
          , (\ty{real}) \bnt{velocity_initial_stiffness}
          , \{ \kw{yes} | \kw{no} | (\ty{bool}) \bnt{omega_rotates?} \} ]
\end{Verbatim}
%\end{verbatim}
The \nt{orientation\_type} is defined
in Section~\ref{sec:CONTROLDATA:DEFAULTORIENTATION}.

The syntax of the 3 degrees of freedom point mass motion structural node is
%\begin{verbatim}
\begin{Verbatim}[commandchars=\\\{\}]
    \bnt{additional_args} ::= , \{ \kw{static displacement} | \kw{dynamic displacement} \} ,
        [ \kw{position} , ] (\hty{Vec3}) \bnt{absolute_position} ,
        [ \kw{velocity} , ] (\hty{Vec3}) \bnt{absolute_velocity} ,
        [ , \kw{assembly}
          , (\ty{real}) \bnt{position_initial_stiffness}
          , (\ty{real}) \bnt{velocity_initial_stiffness} ]
\end{Verbatim}
%\end{verbatim}

When a node's initial configuration is coincident with that of a single reference, instead of the \kw{position}, \kw{orientation}, \kw{velocity}, and \kw{angular velocity} one can respectively use
%\begin{verbatim}
\begin{Verbatim}[commandchars=\\\{\}]
    \bnt{additional_args} ::= , \{ \kw{static} | \kw{dynamic} | \kw{modal} \} ,
        \kw{at reference} , \bnt{reference_label}
            [ , \kw{orientation description} , \bnt{orientation_type} ]
        [ , \kw{assembly}
          , (\ty{real}) \bnt{position_initial_stiffness}
          , (\ty{real}) \bnt{velocity_initial_stiffness}
          , \{ \kw{yes} | \kw{no} | (\ty{bool}) \bnt{omega_rotates?} \} ]
\end{Verbatim}
%\end{verbatim}
or
%\begin{verbatim}
\begin{Verbatim}[commandchars=\\\{\}]
    \bnt{additional_args} ::= , \{ \kw{static displacement} | \kw{dynamic displacement} \} ,
        \kw{at reference} , \bnt{reference_label}
        [ , \kw{assembly}
          , (\ty{real}) \bnt{position_initial_stiffness}
          , (\ty{real}) \bnt{velocity_initial_stiffness} ]
\end{Verbatim}
%\end{verbatim}
The \nt{orientation\_type} is defined
in Section~\ref{sec:CONTROLDATA:DEFAULTORIENTATION}.


The stiffness parameters and the \nt{omega\_rotates?} flag
override the default values optionally set by the \kw{initial stiffness}
and \kw{omega rotates} keywords in the \kw{control data} block. 
They are optional, but they must be supplied all together if at least
one is to be input.

The \nt{omega\_rotates?} parameter determines whether 
the initial angular velocity should follow or not the node 
as it is rotated by the initial assembly procedure.
It can take values \kw{yes} or \kw{no};
a numerical value of 0 (no) or 1 (yes) is supported for backward
compatibility, but its use is deprecated.

The \kw{dynamic} and the \kw{modal} node types allow
the optional output keyword
\kw{accelerations} after the standard node output parameters:
%\begin{verbatim}
\begin{Verbatim}[commandchars=\\\{\}]
    \bnt{extra_arglist} ::= \kw{accelerations} , \{ \kw{yes} | \kw{no} | (\ty{bool}) \bnt{value} \}
\end{Verbatim}
%\end{verbatim}
to enable/disable the output of the linear and angular accelerations
of the node.  Since they are computed as a postprocessing, they are not required
by the regular analysis, so they should be enabled only when strictly required.

Also note that accelerations may be inaccurate, since they are reconstructed
from the momentum and the momenta moment derivatives through the inertia
properties associated to dynamic nodes.

The keyword \kw{accelerations} can be used for \kw{dummy} nodes;
however, the dummy node will compute and output the linear and angular accelerations
only if the node it is connected to can provide them.

Accelerations output can be controlled by means of the
\hyperref{\kw{default output} statement}{\kw{default output} statement (see Section~}{)}{sec:CONTROLDATA:DEFAULTOUTPUT}.

If the \kw{static} model type is used in the control data block,
all dynamic structural nodes are actually treated as static.
This is a shortcut to ease running static analyses without the need
to modify each node of a dynamic model.

\subsection{Dummy Node}
\label{sec:NODE:STRUCTURAL:DUMMY}
The \kw{dummy} structural node has been added to ease the visualization
of the kinematics of arbitrary points of the system
during the simulation. 
It does not provide any degrees of freedom, and it must be attached
to another node.

Two dummy structural node types are available.
%\begin{verbatim}
\begin{Verbatim}[commandchars=\\\{\}]
    \bnt{additional_args} ::= , \kw{dummy} , \bnt{base_node_label} , \bnt{type} , \bnt{dummy_node_data}
\end{Verbatim}
%\end{verbatim}

Dummy nodes take the label \nt{base\_node\_label} of the node they are attached to, 
followed by \nt{type}, the type of dummy node, possibly followed by specific data,
\nt{dummy\_node\_data}.
The following types are available:

\begin{itemize}
    \item \kw{offset}:
%\begin{verbatim}
\begin{Verbatim}[commandchars=\\\{\}]
    \bnt{type} ::= \kw{offset}

    \bnt{dummy_node_data} ::=
        (\hty{Vec3})              \bnt{relative_offset} ,
        (\hty{OrientationMatrix}) \bnt{relative_orientation_matrix}
            [ , \kw{orientation description} , \bnt{orientation_type} ]
\end{Verbatim}
%\end{verbatim}
    It outputs the configuration of a point offset from the base node.
            
    \item \kw{relative frame}:
%\begin{verbatim}
\begin{Verbatim}[commandchars=\\\{\}]
    \bnt{type} ::= \kw{relative frame}

    \bnt{dummy_node_data} ::= \bnt{reference_node_label}
        [ , \kw{position} , (\hty{Vec3}) \bnt{reference_offset} ]
        [ , \kw{orientation} , (\hty{OrientationMatrix}) \bnt{reference_orientation_matrix} ]
            [ \kw{orientation description} , \bnt{orientation_type} , ]
        [ , \kw{pivot node} , \bnt{pivot_node_label}
            [ , \kw{position} , (\hty{Vec3}) \bnt{pivot_offset} ]
            [ , \kw{orientation} , (\hty{OrientationMatrix}) \bnt{pivot_orientation_matrix} ] ]
\end{Verbatim}
%\end{verbatim}
    It outputs the configuration of the base node in the frame defined
    by the node of label \nt{reference\_node\_label}, optionally offset 
    by \nt{reference\_offset} and with optional relative orientation 
    \nt{reference\_orientation\_matrix}.

    If a \nt{pivot\_node\_label} is given, the relative frame motion
    is transformed as if it were expressed in the reference frame
    of the pivot node, optionally offset by \nt{pivot\_offset}
    and with optional relative orientation \nt{pivot\_orientation\_matrix}.
\end{itemize}
The \nt{orientation\_type} is defined
in Section~\ref{sec:CONTROLDATA:DEFAULTORIENTATION}.

\paragraph{Example.} \
\begin{verbatim}
    set: real Omega = 1.;
    structural: 1, static, null, eye, null, 0.,0.,Omega;
    structural: 1000, dummy, 1, offset, 1.,0.,0., eye;
    structural: 1001, dummy, 1, relative frame, 1000;
    structural: 2000, dynamic,
        0.,0.,1.,
        1, 1.,0.,0., 2, 0.,0.,1.,
        null,
        null,
        accelerations, yes;
\end{verbatim}

\paragraph{Output.}
\label{sec:NODE:STRUCTURAL:OUTPUT}
Structural nodes generate two kinds of output files:
\begin{itemize}
\item the \texttt{.mov} file;
\item the \texttt{.ine} file.
\end{itemize}
The output of displacement-only 3 degree of freedom nodes is identical
to that of 6 degree of freedom nodes, with the meaningless fields
filled with zeros.

\textbf{The \texttt{.mov} Output File.} \\
The first refers to the kinematics of the node; its extension is \texttt{.mov},
and for each time step it contains one row for each node whose output is
required.
The rows contain: \vspace{2mm} \\
\begin{tabular}{lp{140mm}}
	\hline
	1      & the label of the node \\
	2--4   & the three components of the position of the node \\
	5--7   & the three Euler angles that define the orientation of the node \\
	8--10  & the three components of the velocity of the node \\
	11--13 & the three components of the angular velocity of the node \\
	\hline
	14--16 & the three components of the linear acceleration
		of the \kw{dynamic} and \kw{modal} nodes (optional) \\
	17--19 & the three components of the angular acceleration
		of the \kw{dynamic} and \kw{modal} nodes (optional) \\
	\hline
\end{tabular}\vspace{2mm}\\
All the quantities are expressed in the global frame, except for
the \kw{relative frame} type of \kw{dummy} node, whose quantities are,
by definition, in the relative frame.

Other two variants of this output are available.
The output of the orientation can be modified by requesting
the three Euler angles, in 3 formats, the three components of the Euler vector,
or the nine components of the orientation matrix.

When the Euler angles are requested, columns 5 to 7 contain
them, in degrees, according to the requested output format:
by default, the `123' sequence is used (which results in the so-called Tait-Bryan or Cardano angles);
the `313' and `321' sequences are supported as well.

When the Euler vector is requested, columns 5 to 7 contain
its components.  Recall that the Euler vector's direction represents the rotation axis,
whereas its norm represents the magnitude, in radian.

Beware that it is not possible to discriminate between
the flavours of Euler angles and the Euler vector without knowing
what output type was requested.

When the orientation matrix is requested, columns 5 to 13
contains the elements of the matrix, written row-wise,
that is: $r_{11}$, $r_{12}$, $r_{13}$, $r_{21}$, \ldots, $r_{33}$.
Note that in this case the total column count changes and, if accelerations
are not requested, it corresponds to that of a structural node
with optional accelerations, so it might be hard to discriminate in this case
as well.

This ambiguity is resolved when results are output in NetCDF format.
See Section~\ref{sec:NetCDF:Node:Structural Node} for details.

\emph{Note: actually, the angles denoted as ``Euler angles'' 
are the three angles that describe a rotation made of a sequence
of three steps: first, a rotation about global axis 1,
followed by a rotation about axis 2 of the frame resulting from
the previous rotation, concluded by a rotation about axis 3
of the frame resulting from the two previous rotations.
To consistently transform this set of parameters into some other
representation, see the tools
\texttt{eu2rot(1)}, \texttt{rot2eu(1)}, \texttt{rot2eup(1)}, \texttt{rot2phi(1)}.
The functions that compute the relationship between an orientation
matrix and the set of three angles and vice versa are
\texttt{MatR2EulerAngles()} and \texttt{EulerAngles2MatR()}, in \texttt{matvec3.h}.
}

\textbf{The \texttt{.ine} Output File.} \\
The second output file refers only to dynamic nodes, and contains their
inertia; its extension is \texttt{.ine}.
For each time step, it contains information about the inertia of all the
nodes whose output is required.
Notice that more than one inertia body can be attached to one node; the
information in this file refers to the sum of all the inertia related to
the node.

The rows contain: \vspace{2mm} \\
\begin{tabular}{lp{140mm}}
        \hline
	1	& the label of the node \\
	2--4	& item the three components of the momentum
		in the absolute reference frame \\
	5--7	& item the three components of the momenta moment
		in the absolute reference frame,
		with respect to the coordinates of the node, 
		thus to a moving frame \\
    	8--10	& the three components of the derivative of the momentum \\
    	11--13	& the three components of the derivative of the momentum moment \\
	\hline
\end{tabular}\vspace{2mm}\\

\paragraph{Private Data.}
\label{sec:NODE:STRUCTURAL:PRIV}
The following data are available:
\begin{enumerate}
\item[] all structural nodes:
\item \kw{"X[1]"} position in global direction 1
\item \kw{"X[2]"} position in global direction 2
\item \kw{"X[3]"} position in global direction 3
\item \kw{"x[1]"} position in direction 1, in the reference frame of the node
\item \kw{"x[2]"} position in direction 2, in the reference frame of the node
\item \kw{"x[3]"} position in direction 3, in the reference frame of the node
\item \kw{"Phi[1]"} orientation vector in global direction 1
\item \kw{"Phi[2]"} orientation vector in global direction 2
\item \kw{"Phi[3]"} orientation vector in global direction 3
\item \kw{"XP[1]"} velocity in global direction 1
\item \kw{"XP[2]"} velocity in global direction 2
\item \kw{"XP[3]"} velocity in global direction 3
\item \kw{"xP[1]"} velocity in direction 1, in the reference frame of the node
\item \kw{"xP[2]"} velocity in direction 2, in the reference frame of the node
\item \kw{"xP[3]"} velocity in direction 3, in the reference frame of the node
\item \kw{"Omega[1]"} angular velocity in global direction 1
\item \kw{"Omega[2]"} angular velocity in global direction 2
\item \kw{"Omega[3]"} angular velocity in global direction 3
\item \kw{"omega[1]"} angular velocity in direction 1, in the reference frame of the node
\item \kw{"omega[2]"} angular velocity in direction 2, in the reference frame of the node
\item \kw{"omega[3]"} angular velocity in direction 3, in the reference frame of the node
\item \kw{"E[1]"} Cardan angle 1 (about global direction 1)
\item \kw{"E[2]"} Cardan angle 2 (about local direction 2)
\item \kw{"E[3]"} Cardan angle 3 (about local direction 3)
\item \kw{"E313[1]"} Cardan angle 1 (about global direction 3)
\item \kw{"E313[2]"} Cardan angle 2 (about local direction 1)
\item \kw{"E313[3]"} Cardan angle 3 (about local direction 3)
\item \kw{"E321[1]"} Cardan angle 1 (about global direction 3)
\item \kw{"E321[2]"} Cardan angle 2 (about local direction 2)
\item \kw{"E321[3]"} Cardan angle 3 (about local direction 1)
\item \kw{"PE[0]"} Euler parameter 0
\item \kw{"PE[1]"} Euler parameter 1
\item \kw{"PE[2]"} Euler parameter 2
\item \kw{"PE[3]"} Euler parameter 3
\item[] dynamic nodes only:
\item \kw{"XPP[1]"} acceleration in global direction 1
\item \kw{"XPP[2]"} acceleration in global direction 2
\item \kw{"XPP[3]"} acceleration in global direction 3
\item \kw{"xPP[1]"} acceleration in direction 1, in the reference frame of the node
\item \kw{"xPP[2]"} acceleration in direction 2, in the reference frame of the node
\item \kw{"xPP[3]"} acceleration in direction 3, in the reference frame of the node
\item \kw{"OmegaP[1]"} angular acceleration in global direction 1
\item \kw{"OmegaP[2]"} angular acceleration in global direction 2
\item \kw{"OmegaP[3]"} angular acceleration in global direction 3
\item \kw{"omegaP[1]"} angular acceleration in direction 1, in the reference frame of the node
\item \kw{"omegaP[2]"} angular acceleration in direction 2, in the reference frame of the node
\item \kw{"omegaP[3]"} angular acceleration in direction 3, in the reference frame of the node
\item \kw{"phi[1]"} orientation vector in direction 1, in the reference frame of the node
\item \kw{"phi[2]"} orientation vector in direction 2, in the reference frame of the node
\item \kw{"phi[3]"} orientation vector in direction 3, in the reference frame of the node
\end{enumerate}

Note: Euler parameters actually do not take into account 
the whole orientation of a node, since they are post-processed
from the orientation matrix.
As a consequence, they only parametrize the minimum norm orientation
that yields the current orientation matrix of the node.
The same applies to the orientation vector $\varphi$.

Note: if accelerations are requested using the \kw{string} form,
their computation is enabled even if it was not explicitly enabled
when the node was instantiated.
However, if the \kw{index} form is used, their computation must have
already been explicitly enabled.

Note: dummy nodes based on dynamic nodes inherit the capability to provide
access to linear and angular accelerations.


\section{Thermal Node}
\label{sec:NODE:THERMAL}
%\begin{verbatim}
\begin{Verbatim}[commandchars=\\\{\}]
    \bnt{additional_args} ::= [ , \kw{value} , \bnt{initial_value}
        [ , \kw{derivative} , \bnt{derivative_initial_value} ] ]
\end{Verbatim}
%\end{verbatim}
\emph{Note: the keywords \kw{value} and \kw{derivative}
have been introduced recently; \kw{value} is not mandatory,
resulting in a warning, while \kw{derivative} is required.
The same applies to the \kw{abstract node} 
and to the \kw{hydraulic node}; the latter is an algebraic
node, so only \kw{value} is allowed.
}

\paragraph{Private Data.}
\label{sec:NODE:THERMAL:PRIV}
The following data are available:
\begin{enumerate}
\item \kw{"x"} temperature
\item \kw{"xP"} temperature time derivative
\end{enumerate}







\section{Miscellaneous}

\subsection{Output}
\label{sec:NODE:MISC:OUTPUT}
There is an extra card, that is used to modify the output behavior of nodes:  
%\begin{verbatim}
\begin{Verbatim}[commandchars=\\\{\}]
    \bnt{card} ::= \kw{output} : \bnt{node_type} , \bnt{node_list} ;

    \bnt{node_list} ::= \{ \bnt{node_label}  [ , ... ] 
        | \kw{range} , \bnt{node_start_label} , \bnt{node_end_label} \}
\end{Verbatim}
%\end{verbatim}
\nt{node\_type} is a valid node type that can be read in the \kw{nodes}
block.
In case the keyword \kw{range} is used, all nodes of that type
with label between \nt{node\_start\_label} and \nt{node\_end\_label}
are set.

\noindent
{\em
   Note: if a node should never (\kw{no}) or always (\kw{yes}) be output,
   its output flag should be set directly on the node card. 
   The global behavior of all the nodes of a type can be set from the 
   \kw{control data} block by adding the node type to the item list in the 
   \kw{default output} card. 
   Then, the specific output flag of sets of nodes can be altered by means 
   of the \kw{output} card in the \kw{nodes} block. 
   This allows high flexibility in the selection of the desired output. 
   The same remarks apply to the output of the elements.
}

If \nt{node\_type} is \kw{structural}, the optional keyword \kw{accelerations}
can be used right after the \nt{node\_type} to enable the output
of the accelerations.

