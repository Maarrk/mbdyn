\subsection{Rigid body displacement joint}
\label{sec:EL:STRUCT:JOINT:RIGID_BODY_DISP_JOINT}
\emph{Author: Reinhard Resch}

This element is a load distributing coupling constraint similar to NASTRAN's \texttt{RBE3} element.
However, it may be used also in the context of large deformations and large rotations.
It is intended mainly to provide an interface between solid elements, which are using \htmlref{\kw{structural displacement node}}{sec:NODE:STRUCTURAL}s with only three degrees of freedom,
and the large family of elements (e.g. joints, forces, ...) which can be used only for \htmlref{\kw{structural node}}{sec:NODE:STRUCTURAL}s with six degrees of freedom.

\begin{Verbatim}[commandchars=\\\{\}]
    \bnt{joint_type} ::= \kw{rigid body displacement joint}

    \bnt{joint_arglist} ::=
       (\ty{StructNode}) \bnt{node_label_master} ,
       (\ty{integer}) \bnt{slave_node_number} ,
       \bnt{slave_node_data}
       [ , \kw{alpha}, (\ty{real}) \bnt{alpha} ] ;

    \bnt{slave_node_data} :: =
       \bnt{one_slave_node_1}
       [ , ... ]
       [ , \bnt{one_slave_node_N} ];

    \bnt{one_slave_node_i} :: =
       (\ty{StructDispNode}) \bnt{node_label_slave_i}
       [ , \kw{position} , (\hty{Vec3}) \bnt{position_slave_node_i} ]    
       [ , \kw{weight} , (\ty{real}) \bnt{weight_slave_node_i} ]

\end{Verbatim}

\paragraph{Kinematics}
The position and orientation of the \nt{node\_label\_master} is constrained in a way, that it's following the weighted average
of the rigid body motion of a set of \nt{slave\_node\_number} slave nodes. But, no artificial stiffness will be added to the slave nodes.
Since the slave nodes do not have any rotational degrees of freedom, a minimum number of three non coincident slave nodes is required.
If the optional parameter \nt{position\_slave\_node} is present, it is used to compute the relative offset between \nt{node\_label\_slave}
and \nt{node\_label\_master}. Otherwise, the initial position of \nt{node\_label\_slave} is used to compute the offset.
\paragraph{Corner nodes of quadratic elements}
There is also an optional weighting factor \nt{weight\_slave\_node} per node, which may be used for second order solid elements in order to achieve a smooth stress distribution.
Negative values for \nt{weight\_slave\_node} are allowed (e.g. for corner nodes of quadratic hexahedrons), but the sum of all weighting factors must be always greater than zero.
See also section~\ref{sec:EL:SOLID:preprocessing} for an example on how to create a \kw{rigid body displacement joint} automatically from a Finite Element mesh.
\paragraph{Scaling}
The optional parameter \nt{alpha} may be used to scale the constraint equations in order to improve the condition number of the Jacobian matrix.
However, it has no effect on the result and the default value should be fine for most applications.

\subsubsection{Output}
The output occurs in the \texttt{.jnt} file, which contains the reaction force and reaction couple related to the \nt{node\_label\_master}.

\subsubsection{Private Data}
The following data is available:
\begin{enumerate}
\item \kw{"Fx"} reaction force at \nt{node\_label\_master} along \nt{node\_label\_master} axis 1
\item \kw{"Fy"} reaction force at \nt{node\_label\_master} along \nt{node\_label\_master} axis 2
\item \kw{"Fz"} reaction force at \nt{node\_label\_master} along \nt{node\_label\_master} axis 3

\item \kw{"Mx"} reaction moment at \nt{node\_label\_master} along \nt{node\_label\_master} axis 1
\item \kw{"My"} reaction moment at \nt{node\_label\_master} along \nt{node\_label\_master} axis 2
\item \kw{"Mz"} reaction moment at \nt{node\_label\_master} along \nt{node\_label\_master} axis 3

\item \kw{"fx"} reaction force at \nt{node\_label\_master} along global axis 1
\item \kw{"fy"} reaction force at \nt{node\_label\_master} along global axis 2
\item \kw{"fz"} reaction force at \nt{node\_label\_master} along global axis 3

\item \kw{"mx"} reaction moment at \nt{node\_label\_master} along global axis 1
\item \kw{"my"} reaction moment at \nt{node\_label\_master} along global axis 2
\item \kw{"mz"} reaction moment at \nt{node\_label\_master} along global axis 3
\end{enumerate}

\paragraph{Example.}
This example mimics an instance of NASTRAN's \texttt{RBE3} element.
\begin{Verbatim}
  begin: nodes;
    structural: 609, static, reference, ref_id_solid, null,
                             reference, ref_id_solid, eye,
                             reference, ref_id_solid, null,
                             reference, ref_id_solid, null;
    structural: 1, dynamic displacement,
                             reference, ref_id_solid, 5.1e+02,  1.0e+01, -1.0e+01,
                             reference, ref_id_solid, null;
    structural: 2, dynamic displacement,
                             reference, ref_id_solid, 5.1e+02,  -1.0e+01, -1.0e+01,
                             reference, ref_id_solid, null;
    ...
    structural: 8, dynamic displacement,
                             reference, ref_id_solid, 9.9e+02,  -1.0e+01, 1.0e+01,
                             reference, ref_id_solid, null;
  end: nodes;
  begin: elements;
     joint: 100, rigid body displacement joint,
            609, ## master node label
              8, ## eight slave nodes
              5, weight, 1.3333333333333348e+02, ## midside nodes
              6, weight, 1.3333333333333351e+02,
              7, weight, 1.3333333333333351e+02,
              9, weight, 1.3333333333333351e+02,
             13, weight, -3.3333333333333414e+01, ## corner nodes
             14, weight, -3.3333333333333428e+01,
             15, weight, -3.3333333333333421e+01,
             16, weight, -3.3333333333333421e+01;
   end: elements;
\end{Verbatim}
