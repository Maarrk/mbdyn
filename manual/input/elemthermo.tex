% MBDyn (C) is a multibody analysis code.
% http://www.mbdyn.org
%
% Copyright (C) 1996-2023
%
% Pierangelo Masarati  <pierangelo.masarati@polimi.it>
%
% Dipartimento di Ingegneria Aerospaziale - Politecnico di Milano
% via La Masa, 34 - 20156 Milano, Italy
% http://www.aero.polimi.it
%
% Changing this copyright notice is forbidden.
%
% This program is free software; you can redistribute it and/or modify
% it under the terms of the GNU General Public License as published by
% the Free Software Foundation (version 2 of the License).
%
%
% This program is distributed in the hope that it will be useful,
% but WITHOUT ANY WARRANTY; without even the implied warranty of
% MERCHANTABILITY or FITNESS FOR A PARTICULAR PURPOSE.  See the
% GNU General Public License for more details.
%
% You should have received a copy of the GNU General Public License
% along with this program; if not, write to the Free Software
% Foundation, Inc., 59 Temple Place, Suite 330, Boston, MA  02111-1307  USA

\section{Thermal Elements}
\label{sec:EL:THERMO}

\subsection{Capacitance}
\label{sec:EL:THERMO:CAPACITANCE}

This element implements a simple linear thermal capacitance associated with a thermal node.
It computes the heat flow into the capacitance node as
\begin{align*}
	q &= -\nt{dCapacitance} \cdot \dot{T}
\end{align*}

The input format is:
%\begin{verbatim}
\begin{Verbatim}[commandchars=\\\{\}]
    \bnt{elem_type} ::= \kw{capacitance}

    \bnt{normal_arglist} ::= 
        \bnt{node_label} , 
        (\ty{real}) \bnt{dCapacitance}
\end{Verbatim}
%\end{verbatim}
where \nt{node\_label} is the thermal node the thermal capacitance is connected to, and \nt{dCapacitance} is the (positive) value of the thermal capacitance.

\paragraph{Example.} \
\begin{verbatim}
    thermal: 99, capacitance,
        11,    # node label
        0.11;  # capacitance
\end{verbatim}



\subsection{Resistance}
\label{sec:EL:THERMO:RESISTANCE}

This element implements a simple linear thermal resistance between two thermal nodes.
It computes the heat flow between the two nodes as
\begin{align*}
	q_{21} &= \frac{T_2 - T_1}{\nt{dResistance}}
\end{align*}

The input format is:
%\begin{verbatim}
\begin{Verbatim}[commandchars=\\\{\}]
    \bnt{elem_type} ::= \kw{resistance}

    \bnt{normal_arglist} ::= 
        \bnt{node_1_label} , 
        \bnt{node_2_label} , 
        (\ty{real}) \bnt{dResistance}
\end{Verbatim}
%\end{verbatim}
where \nt{node\_1\_label} and \nt{node\_2\_label} are the thermal nodes the thermal resistance is connected to, and \nt{dResistance} is the (positive) value of the thermal resistance.

\paragraph{Example.} \
\begin{verbatim}
    thermal: 98, resistance,
        11,    # node 1 label
        12,    # node 2 label
        0.99;  # resistance
\end{verbatim}



\subsection{Source}
\label{sec:EL:THERMO:SOURCE}

This element implements a simple thermal source associated with a thermal node.
It computes the heat flow into the node as
\begin{align*}
	q &= \nt{Source}
\end{align*}

The input format is:
%\begin{verbatim}
\begin{Verbatim}[commandchars=\\\{\}]
    \bnt{elem_type} ::= \kw{capacitance}

    \bnt{normal_arglist} ::= 
        \bnt{node_label} , 
        (\hty{DriveCaller}) \bnt{Source}
\end{Verbatim}
%\end{verbatim}
where \nt{node\_label} is the thermal node the thermal source is connected to, and \nt{Source} is a \ty{DriveCaller}.

\paragraph{Example.} \
\begin{verbatim}
    thermal: 97, source,
        12,    # node label
        # source; grows as 1-cos from 0 s to 1 s, raising from 0. to 0.9 
        cosine, 0., pi, 0.45, half, 0.;
\end{verbatim}





