% MBDyn (C) is a multibody analysis code. 
% http://www.mbdyn.org
% 
% Copyright (C) 1996-2023
% 
% Pierangelo Masarati	<pierangelo.masarati@polimi.it>
% Paolo Mantegazza	<paolo.mantegazza@polimi.it>
% 
% Dipartimento di Ingegneria Aerospaziale - Politecnico di Milano
% via La Masa, 34 - 20156 Milano, Italy
% http://www.aero.polimi.it
% 
% Changing this copyright notice is forbidden.
% 
% This program is free software; you can redistribute it and/or modify
% it under the terms of the GNU General Public License as published by
% the Free Software Foundation (version 2 of the License).
% 
% 
% This program is distributed in the hope that it will be useful,
% but WITHOUT ANY WARRANTY; without even the implied warranty of
% MERCHANTABILITY or FITNESS FOR A PARTICULAR PURPOSE.  See the
% GNU General Public License for more details.
% 
% You should have received a copy of the GNU General Public License
% along with this program; if not, write to the Free Software
% Foundation, Inc., 59 Temple Place, Suite 330, Boston, MA  02111-1307  USA
% 
% Author: Eduardo Okabe


\subsection{Module-fab-electric}
\label{sec:MODULE:FAB-ELECTRIC}
\emph{Author: Eduardo Okabe}

\noindent
This module implements several electric components.

\subsubsection{Resistor}
\label{sec:MODULE:FAB-ELECTRIC:RESISTOR}
\begin{Verbatim}[commandchars=\\\{\}]
    \bnt{name} ::= \kw{resistor}

    \bnt{module_data} ::=
        \bnt{electric_node_1_label} ,
        \bnt{electric_node_2_label} ,
        (\hty{real}) \bnt{resistance}
\end{Verbatim}

\paragraph{Output.}
\label{sec:MODULE:FAB-ELECTRIC:RESISTOR:OUTPUT}
The format is:
\begin{itemize}
\item the label of the element;
\item the current in the element (\kw{I} in NetCDF format);
\item the voltage on the first node (\kw{V1} in NetCDF format);
\item the voltage on the second node (\kw{V2} in NetCDF format).
\end{itemize}

\subsubsection{Capacitor}
\begin{Verbatim}[commandchars=\\\{\}]
    \bnt{name} ::= \kw{capacitor}

    \bnt{module_data} ::=
        \bnt{electric_node_1_label} ,
        \bnt{electric_node_2_label} ,
        (\hty{real}) \bnt{capacitance}
\end{Verbatim}

\paragraph{Output.}
\label{sec:MODULE:FAB-ELECTRIC:CAPACITOR:OUTPUT}
The format is:
\begin{itemize}
\item the label of the element;
\item the current in the element (\kw{I} in NetCDF format);
\item the voltage on the first node (\kw{V1} in NetCDF format);
\item the voltage on the second node (\kw{V2} in NetCDF format).
\end{itemize}

\subsubsection{Inductor}
\begin{Verbatim}[commandchars=\\\{\}]
    \bnt{name} ::= \kw{inductor}

    \bnt{module_data} ::=
        \bnt{electric_node_1_label} ,
        \bnt{electric_node_2_label} ,
        (\hty{real}) \bnt{inductance}
\end{Verbatim}

\paragraph{Output.}
\label{sec:MODULE:FAB-ELECTRIC:INDUCTOR:OUTPUT}
The format is:
\begin{itemize}
\item the label of the element;
\item the current in the element (\kw{I} in NetCDF format);
\item the voltage on the first node (\kw{V1} in NetCDF format);
\item the voltage on the second node (\kw{V2} in NetCDF format).
\end{itemize}

\subsubsection{Diode}
\begin{Verbatim}[commandchars=\\\{\}]
    \bnt{name} ::= \kw{diode}
\end{Verbatim}

\subsubsection{Switch}
\begin{Verbatim}[commandchars=\\\{\}]
    \bnt{name} ::= \kw{switch}
\end{Verbatim}

\subsubsection{Electrical source}
\begin{Verbatim}[commandchars=\\\{\}]
    \bnt{name} ::= \kw{electrical source}
\end{Verbatim}

\subsubsection{Ideal transformer}
\begin{Verbatim}[commandchars=\\\{\}]
    \bnt{name} ::= \kw{ideal transformer}
\end{Verbatim}

\subsubsection{Operational amplifier}
\begin{Verbatim}[commandchars=\\\{\}]
    \bnt{name} ::= \kw{operational amplifier}
\end{Verbatim}

\subsubsection{BJT}
\begin{Verbatim}[commandchars=\\\{\}]
    \bnt{name} ::= \kw{bjt}
\end{Verbatim}

\subsubsection{Proximity sensor}
\begin{Verbatim}[commandchars=\\\{\}]
    \bnt{name} ::= \kw{proximity sensor}
\end{Verbatim}



