% $Header$
% MBDyn (C) is a multibody analysis code.
% http://www.mbdyn.org
%
% Copyright (C) 1996-2023
%
% Pierangelo Masarati  <pierangelo.masarati@polimi.it>
%
% Dipartimento di Ingegneria Aerospaziale - Politecnico di Milano
% via La Masa, 34 - 20156 Milano, Italy
% http://www.aero.polimi.it
%
% Changing this copyright notice is forbidden.
%
% This program is free software; you can redistribute it and/or modify
% it under the terms of the GNU General Public License as published by
% the Free Software Foundation (version 2 of the License).
% 
%
% This program is distributed in the hope that it will be useful,
% but WITHOUT ANY WARRANTY; without even the implied warranty of
% MERCHANTABILITY or FITNESS FOR A PARTICULAR PURPOSE.  See the
% GNU General Public License for more details.
%
% You should have received a copy of the GNU General Public License
% along with this program; if not, write to the Free Software
% Foundation, Inc., 59 Temple Place, Suite 330, Boston, MA  02111-1307  USA

\subsection{Kinematics}
Position of an arbitrary point $P$
\begin{equation}
	\T{x}_P = \T{x}_0 + \T{f}_P + \T{u}_P
\end{equation}
where $\T{x}_0$ is the position of the point that describes the global
motion of the body,
$\T{f}_P$ is the relative position of the point when the body 
is undeformed,
and $\T{u}_P$ is the relative displacement of the point when the body
is deformed.

It can be rewritten as
\begin{equation}
	\T{x}_P = \T{x}_0 + \T{R}_0\plbr{\tilde{\T{f}}_P + \tilde{\T{u}}_P}
\end{equation}
where $\T{R}_0$ is the global orientation matrix of the body,
and the \emph{tilde} $\plbr{\tilde{\cdot}}$ indicates entities
expressed in the reference frame attached to the body.

The deformation of the body is expressed by a linear combination
of $M$ displacement (and rotation, for those models that consider
them, like beam trusses) shapes
\begin{equation}
	\tilde{\T{u}}_P = \sum_{j=1,M} \T{U}_{Pj} q_j = \T{U}_P \T{q}
\end{equation}
where $\T{U}_{Pj}$ is the vector containing the components
of the $j$-th displacement shape related to point $P$,
and $q_j$ is the $j$-th mode multiplier.

The orientation of the generic point $P$ is
\begin{equation}
	\T{R}_P = \T{R}_0 \tilde{\T{R}}_P
\end{equation}
and, assuming a representation of the relative orientation by a linear
combination of rotation shapes
\begin{equation}
	\tilde{\T{\phi}} = \sum_{j=1,M} \T{V}_{Pj} q_j = \T{V}_P \T{q}
\end{equation}
it results in a linearized orientation
\begin{equation}
	\T{R}_P \cong \T{R}_0 \plbr{\T{I} + \plbr{\T{V}_P \T{q}}\times{}}
\end{equation}
which is no longer orthogonal, because of matrix
\begin{equation}
	\tilde{\T{R}}_P = \T{I} + \plbr{\T{V}_P \T{q}}\times{}
\end{equation}
which represents a linearized rotation.

The first and second derivatives of position and orientation yield:
\begin{subequations}
\begin{align}
	\dot{\T{x}}_P &= \dot{\T{x}}_0
		+ \T{\omega}_0 \times \T{R}_0 \plbr{\tilde{\T{f}}_P + \T{U}_P \T{q}}
		+ \T{R}_0 \T{U}_P \dot{\T{q}} \\
	\T{\omega}_P &= \T{\omega}_0
		+ \T{R}_0 \T{V}_P \dot{\T{q}} \\
	\ddot{\T{x}}_P &= \ddot{\T{x}}_0
		+ \dot{\T{\omega}}_0 \times \T{R}_0 \plbr{\tilde{\T{f}}_P + \T{U}_P \T{q}}
		+ \T{\omega}_0 \times \T{\omega}_0 \times \T{R}_0 \plbr{\tilde{\T{f}}_P + \T{U}_P \T{q}} \nonumber \\
		& \mbox{} + 2 \T{\omega}_0 \times \T{R}_0 \T{U}_P \dot{\T{q}}
		+ \T{R}_0 \T{U}_P \ddot{\T{q}} \\
	\dot{\T{\omega}}_P &= \dot{\T{\omega}}_0
		+ \T{\omega}_0 \times \T{R}_0 \T{V}_P \dot{\T{q}}
		+ \T{R}_0 \T{V}_P \ddot{\T{q}}
\end{align}
\end{subequations}

The virtual perturbation of the position and orientation
of the generic point $P$ are:
\begin{subequations}
\begin{align}
	\delta{\T{x}}_P &= \delta{\T{x}}_0
		+ \delta\T{\phi}_0 \times \T{R}_0 \plbr{\tilde{\T{f}}_P + \T{U}_P \T{q}}
		+ \T{R}_0 \T{U}_P \delta{\T{q}} \\
	\delta\T{\phi}_P &= \delta\T{\phi}_0
		+ \T{R}_0 \T{V}_P \delta{\T{q}}
\end{align}
\end{subequations}

Without significant losses in generality, from now on it is assumed
that the structure of the problem is given in form of lumped inertia
parameters in specific points, corresponding to FEM nodes,
and that the position of each node corresponds to the center of mass
of each lumped mass.
A model made of $N$ FEM nodes is considered.
The nodal mass of the $i$-th FEM node is
\begin{equation}
	\T{M}_i = \sqbr{\matr{cc}{
		m_i \T{I} & \T{0} \\
		\T{0} & \T{J}_i
	}}
\end{equation}
There is no strict requirement for matrix $\T{J}_i$ to be diagonal.

The inertia forces and moments acting on each FEM node are:
\begin{subequations}
\begin{align}
	\T{F}_i &= - m_i \ddot{\T{x}}_i \\
	\T{C}_i &= - \T{R}_i \T{J}_i \T{R}_i^T \dot{\T{\omega}}_i
\end{align}
\end{subequations}
and the virtual work done by the inertia forces is
\begin{equation}
	\delta{L} = \sum_{i=1,N} \plbr{
		\delta{\T{x}}_i^T \T{F}_i
		+ \delta\T{\phi}_i^T \T{C}_i
	}
\end{equation}
which results in
\begin{equation}
	\delta{L} = \cubr{\cvvect{
		\delta\T{x}_0 \\
		\delta\T{\phi}_0 \\
		\delta\T{q}
	}}^T \plbr{
	\sqbr{\matr{ccc}{
		\T{M}_{xx} & \T{M}_{x\phi} & \T{M}_{xq} \\
			& \T{M}_{\phi\phi} & \T{M}_{\phi q} \\
		\llk{sym.} & & \T{M}_{qq}
	}} \cubr{\cvvect{
		\ddot{\T{x}}_0 \\
		\dot{\T{\omega}}_0 \\
		\ddot{\T{q}}
	}} + \cubr{\cvvect{
		\T{F}_x \\
		\T{F}_{\phi} \\
		\T{F}_q
	}}
	}
\end{equation}
with
\begin{subequations}
\begin{align}
	\T{M}_{xx}	&= \T{I} \sum_{i=1,N} m_i \\
	\T{M}_{x\phi}	&= \T{R}_0 \sum_{i=1,N} m_i \plbr{\tilde{\T{f}}_i + \T{U}_i \T{q}}\times^T \T{R}_0^T \\
	\T{M}_{xq}	&= \T{R}_0 \sum_{i=1,N} m_i \T{U}_i \\
	\T{M}_{\phi\phi}&= \T{R}_0 \sum_{i=1,N} \plbr{
		m_i \plbr{\tilde{\T{f}}_i + \T{U}_i \T{q}}\times
		\plbr{\tilde{\T{f}}_i + \T{U}_i \T{q}}\times^T
		+ \tilde{\T{R}}_i \T{J}_i \tilde{\T{R}}_i^T
	} \T{R}_0^T \\
	\T{M}_{\phi q}	&= \T{R}_0 \sum_{i=1,N} \plbr{
		m_i \plbr{\tilde{\T{f}}_i + \T{U}_i \T{q}}\times \T{U}_i
		+ \tilde{\T{R}}_i \T{J}_i \tilde{\T{R}}_i^T \T{V}_i
	} \\
	\T{M}_{qq}	&= \sum_{i=1,N} \plbr{
		m_i \T{U}_i^T \T{U}_i
		+ \T{V}_i^T \tilde{\T{R}}_i \T{J}_i \tilde{\T{R}}_i^T \T{V}_i
	} \\
	\T{F}_x		&= \sum_{i=1,N} m_i \plbr{
		\T{\omega}_0 \times \T{\omega}_0 \times \T{R}_0 \plbr{\tilde{\T{f}_i} + \T{U}_i \T{q}}
		+ 2 \T{\omega}_0 \times \T{R}_0 \T{U}_i \dot{\T{q}} 
	} \\
	\T{F}_{\phi}	&= \sum_{i=1,N} \T{R}_0 \lplbr{
		m_i \plbr{\tilde{\T{f}_i} + \T{U}_i \T{q}}\times \plbr{
			\T{\omega}_0 \times \T{\omega}_0 \times \T{R}_0 \plbr{\tilde{\T{f}_i} + \T{U}_i \T{q}}
			+ 2 \T{\omega}_0 \times \T{R}_0 \T{U}_i \dot{\T{q}}
		}
	} \nonumber \\
			& \rplbr{
		\mbox{} + \tilde{\T{R}}_i \T{J}_i \tilde{\T{R}}_i^T \T{R}_0^T \T{\omega}_0 \times \T{R}_0 \T{V}_i \dot{\T{q}}
	} \\
	\T{F}_q		&= \sum_{i=1,N} \lplbr{
		m_i \T{U}_i^T \T{R}_0^T \lplbr{
			\T{\omega}_0 \times \T{\omega}_0 \times \T{R}_0 \plbr{\tilde{\T{f}_i} + \T{U}_i \T{q}}
			+ 2 \T{\omega}_0 \times \T{R}_0 \T{U}_i \dot{\T{q}}
		}
	} \nonumber \\
			& \rplbr{
		\mbox{} + \T{V}_i^T \tilde{\T{R}}_i \T{J}_i \tilde{\T{R}}_i^T \T{R}_0^T \T{\omega}_0 \times \T{R}_0 \T{V}_i \dot{\T{q}}
	}
\end{align}
\end{subequations}
The $\T{M}_{jk}$ terms can be rewritten to highlight contributions of order
0, 1, and higher:
\begin{subequations}
\begin{align}
	\T{M}_{xx}	&= \T{I} \plbr{\sum_{i=1,N} m_i} \\
	\T{M}_{x\phi}	&= \T{R}_0 \plbr{\plbr{
		\sum_{i=1,N} m_i \tilde{\T{f}}_i
	} + \plbr{\plbr{
		\sum_{i=1,N} m_i \T{U}_i
	} \T{q}}} \times^T \T{R}_0^T \\
	\T{M}_{xq}	&= \T{R}_0 \plbr{\sum_{i=1,N} m_i \T{U}_i} \\
	\T{M}_{\phi\phi}&= \T{R}_0 \lplbr{
		\sum_{i=1,N} \plbr{
			m_i \tilde{\T{f}}_i \times \tilde{\T{f}}_i\times^T
			+ \T{J}_i
		}
	} \nonumber \\
			& \mbox{} + \sum_{i=1,N} \plbr{
			m_i \tilde{\T{f}}_i \times \plbr{\T{U}_i \T{q}}\times^T
			+ m_i \plbr{\T{U}_i \T{q}}\times \tilde{\T{f}}_i \times^T
			+ \T{J}_i \plbr{\T{V}_i \T{q}}\times^T
			+ \plbr{\T{V}_i \T{q}}\times \T{J}_i
	} \nonumber \\
			& \rplbr{ \mbox + \sum_{i=1,N} \plbr{
			m_i \plbr{\T{U}_i \T{q}}\times \plbr{\T{U}_i \T{q}}\times^T
			+ \plbr{\T{V}_i \T{q}}\times \T{J}_i \plbr{\T{V}_i \T{q}}\times^T
	}} \T{R}_0^T \\
	\T{M}_{\phi q}	&= \T{R}_0 \lplbr{
		\sum_{i=1,N} \plbr{
			m_i \tilde{\T{f}}_i \times \T{U}_i
			+ \T{J}_i \T{V}_i
		}
	} \nonumber \\
			& \mbox{} + \sum_{i=1,N} \plbr{
			m_i \plbr{\T{U}_i \T{q}} \times \T{U}_i
			+ \T{J}_i \plbr{\T{V}_i \T{q}}\times^T \T{V}_i
			+ \plbr{\T{V}_i \T{q}}\times \T{J}_i \T{V}_i
	} \nonumber \\
			& \rplbr{ \mbox{} + \sum_{i=1,N} \plbr{\T{V}_i \T{q}}\times \T{J}_i \plbr{\T{V}_i \T{q}}\times^T \T{V}_i
	} \\
	\T{M}_{qq}	&= \sum_{i=1,N} \plbr{
		m_i \T{U}_i^T \T{U}_i
		+ \T{V}_i^T \T{J}_i \T{V}_i
	} \nonumber \\
			& \mbox{} + \sum_{i=1,N} \plbr{
			\T{V}_i^T \T{J}_i \plbr{\T{V}_i \T{q}}\times^T \T{V}_i
			+ \T{V}_i^T \plbr{\T{V}_i \T{q}}\times \T{J}_i \T{V}_i
	} \nonumber \\
			& \mbox{} + \sum_{i=1,N}
			\T{V}_i^T \plbr{\T{V}_i \T{q}}\times \T{J}_i \plbr{\T{V}_i \T{q}}\times^T \T{V}_i
\end{align}
\end{subequations}

\subsection{Physics: Orthogonality}
Some noteworthy entities appear in the above equations, which may partially
simplify under special circumstances.

The overall mass of the body
\begin{equation}
	m = \sum_{i=1,N} m_i
\end{equation}

The static (first order) inertia moment
\begin{equation}
	\T{S}_{x\phi} = \sum_{i=1,N} m_i \tilde{\T{f}}_i
\end{equation}
vanishes if point $\T{x}_0$ is the center of mass of the undeformed body.

Similarly, the static (first order) inertia moment computed with 
the modal displacement shapes
\begin{equation}
	\label{eq:MODAL-Sxq}
	\T{S}_{xq} = \sum_{i=1,N} m_i \T{U}_i
\end{equation}
vanishes if the mode shapes have been inertially decoupled
from the rigid body displacements.
In fact, the decoupling of the rigid and the deformable modes
is expressed by
\begin{align}
	\sum_{i=1,N} \T{x}_r^T m_i \T{U}_i &= \nonumber \\
	\T{x}_r^T \sum_{i=1,N} m_i \T{U}_i &= 0
\end{align}
where $\T{x}_r^T$ describes three independent rigid translations,
which, for the arbitrariety of $\T{x}_r$,
implies the above Equation~(\ref{eq:MODAL-Sxq}).

In the same case, also the zero-order terms of the coupling
between the rigid body rotations and the modal variables,
\begin{equation}
	\label{eq:MODAL-Sphiq}
	\T{S}_{\phi q} = \sum_{i=1,N} \plbr{
		m_i \tilde{\T{f}}_i \times \T{U}_i
		+ \T{J}_i \T{V}_i
	}
\end{equation}
also vanishes.
In fact, the decoupling of the rigid and the deformable modes
is expressed by
\begin{align}
	\sum_{i=1,N} \plbr{
		m_i \T{\phi}_r^T \tilde{\T{f}}_i \times \T{U}_i
		+ \T{\phi}_r^T \T{J}_i \T{V}_i
	} &= \nonumber \\
	\T{\phi}_r^T \sum_{i=1,N} \plbr{
		m_i \tilde{\T{f}}_i \times \T{U}_i
		+ \T{J}_i \T{V}_i
	} &= 0
\end{align}
where $\T{\phi}_r^T$ describes three independent rigid rotations,
and $\T{\phi}_r^T \tilde{\T{f}}_i \times$ describes the corresponding
displacements, which, for the arbitrariety of $\T{\phi}_r$,
implies the above Equation~(\ref{eq:MODAL-Sphiq}).


The second order inertia moment is
\begin{equation}
	\T{J} = \sum_{i=1,N} \plbr{
		m_i \tilde{\T{f}}_i \times \tilde{\T{f}}_i\times^T
		+ \T{J}_i
	}
\end{equation}
It results in a diagonal matrix if the orientation of the body
is aligned with the principal inertia axes.

The modal mass matrix is
\begin{equation}
	\T{m} = \sum_{i=1,N} \plbr{
		m_i \T{U}_i^T \T{U}_i
		+ \T{V}_i^T \T{J}_i \T{V}_i
	}
\end{equation}
It is diagonal if only the normal modes are considered.

\subsection{Simplifications}
The problem, as stated up to now, already contains some simplifications.
First of all, those related to the lumped inertia model of a continuum;
moreover, those related to the mode superposition to describe the
straining of the body which, in the case of the FEM node rotation,
yields a non-orthogonal linearized rotation matrix.

Further simplifications are usually accepted in common modeling practice,
where some of the higher order terms are simply discarded.

When only the 0-th order coefficients are used in matrices $M_{uv}$,
the dynamics of the body are written referred to the undeformed shape.
This approximation can be quite drastic, but in some cases it may be
reasonable, if the reference straining, represented by $\T{q}$,
remains very small throughout the simulation.
This approximation is also required when the only available data
are the global inertia properties (e.g.\ $m$, the postion 
of the center of mass and the inertia matrix $\T{J}$),
and the modal mass matrix $\T{m}$.

More refined approximations include higher order terms: for example, the
first and second order contributions illustrated before.
This corresponds to using finer and finer descriptions of the inertia
properties of the system, corresponding to computing the inertia properties 
in the deformed condition with first and second order accuracy, respectively.

\subsection{Invariants}
The dynamics of the deformed body can be written without any detailed
knowledge of the mass distribution, provided some aggregate information
can be gathered in so-called \emph{invariants}.
They are:

\begin{enumerate}
\item[1.] Total mass
(scalar)
\begin{equation}
	\mathcal{I}_1 = \sum_{i=1,N} m_i
\end{equation}
where $m_i$ is the mass of the $i$-th FEM node\footnote{Although the input
format, because of NASTRAN legacy, allows each global direction to have
a separate mass value, invariants assume that the same value is given,
and only use the one associated to component 1.}.

\item[2.] Static moment
(matrix $3\times{1}$)
\begin{equation}
	\mathcal{I}_2 = \sum_{i=1,N} m_i \tilde{\T{f}}_i
\end{equation}

\item[3.] Static coupling between rigid body and FEM node displacements
(matrix $3\times{M}$)
\begin{equation}
	\mathcal{I}_3 = \sum_{i=1,N} m_i \T{U}_i
\end{equation}
where the portion related to the $k$-th mode is computed by summation
of the contribute of each FEM node, obtained by multiplying the FEM node
mass $m_i$ by the three components of the modal displacement $\T{U}_{ik}$
of the $k$-th mode.

\item[4.] Static coupling between rigid body rotations and FEM node 
displacements
(matrix $3\times{M}$)
\begin{equation}
	\mathcal{I}_4 = \sum_{i=1,N} \plbr{
		m_i \tilde{\T{f}}_i \times \T{U}_i
		+ \T{J}_i \T{V}_i
	}
\end{equation}
where the portion related to the $k$-th mode is computed by summation
of the contribute of each FEM node, obtained by multiplying the FEM node
mass $m_i$ by the cross product of the FEM node position $\tilde{\T{f}}_i$ 
and the three components of the modal displacement $\T{U}_{ik}$ of the
$k$-th mode.

\item[5.] Static coupling between FEM node displacements
(matrix $3\times{M}\times{M}$)
\begin{equation}
	\mathcal{I}_{5j} = \sum_{i=1,N} m_i \T{U}_{ij} \times \T{U}_i
\end{equation}
where the portion related to the $j$-th mode is computed by summation
of the contribute of each FEM node, obtained by multiplying the FEM node
mass $m_i$ by the cross product of the three components of the FEM node 
$j$-th modal displacement $\T{U}_{ij}$ and the three components of the 
$k$-th modal displacement $\T{U}_{ik}$.

\item[6.] Modal mass matrix
(matrix $M\times{M}$)
\begin{equation}
	\mathcal{I}_6 = \sum_{i=1,N} \plbr{
		m_i \T{U}_i^T \T{U}_i
		+ \T{V}_i^T \T{J}_i \T{V}_i
	}
\end{equation}

\item[7.] Inertia matrix
(matrix $3\times{3}$)
\begin{equation}
	\mathcal{I}_7 = \sum_{i=1,N} \plbr{
		m_i \tilde{\T{f}}_i \times \tilde{\T{f}}_i \times^T + \T{J}_i
	}
\end{equation}

\item[8.]
(matrix $3\times{M}\times{3}$)
\begin{equation}
	\mathcal{I}_{8j} =
		\sum_{i=1,N} m_i \tilde{\T{f}}_{i} \times \T{U}_{ij} \times{}^T
\end{equation}

\item[9.]
(matrix $3\times{M}\times{M}\times{3}$)
\begin{equation}
	\mathcal{I}_{9jk} = \sum_{i=1,N} m_i \T{U}_{ij} \times \T{U}_{ik} \times{}
\end{equation}

\item[10.]
(matrix $3\times{M}\times{3}$)
\begin{equation}
	\mathcal{I}_{10j} = \sum_{i=1,N} \T{V}_{ij}\times \T{J}_i
\end{equation}

\item[11.]
(matrix $3\times{M}$)
\begin{equation}
	\mathcal{I}_{11} = \sum_{i=1,N} \T{J}_i \T{V}_i
\end{equation}

\end{enumerate}

Using the invariants, the contributions to the inertia matrix of the body
become
\begin{subequations}
\begin{align}
	\T{M}_{xx}	&= \T{I} \, \mathcal{I}_1 \\
	\T{M}_{x\phi}	&= \T{R}_0 \plbr{
		\mathcal{I}_2 + \mathcal{I}_3 \T{q}}\times{}  \T{R}_0^T \\
	\T{M}_{xq}	&= \T{R}_0 \mathcal{I}_3 \\
	\T{M}_{\phi\phi}&= \T{R}_0 \plbr{
		\mathcal{I}_7
		+ \plbr{\mathcal{I}_{8j} + \mathcal{I}_{8j}^T} q_j
		+ \mathcal{I}_{9jk} q_j q_k
	} \T{R}_0^T \\
	\T{M}_{\phi q}	&= \T{R}_0 \lplbr{
		\mathcal{I}_4
		+ \mathcal{I}_{5j} q_j 
		+ \sum_{i=1,N} \plbr{
			\T{J}_i \plbr{\T{V}_i \T{q}}\times^T \T{V}_i
			+ \plbr{\T{V}_i \T{q}}\times \T{J}_i \T{V}_i
		}
	} \nonumber \\
			& \rplbr{ \mbox{} + \sum_{i=1,N} \plbr{\T{V}_i \T{q}}\times \T{J}_i \plbr{\T{V}_i \T{q}}\times^T \T{V}_i
	} \\
	\T{M}_{qq}	&= \mathcal{I}_6 \nonumber \\
			& \mbox{} + \sum_{i=1,N} \plbr{
			\T{V}_i^T \T{J}_i \plbr{\T{V}_i \T{q}}\times^T \T{V}_i
			+ \T{V}_i^T \plbr{\T{V}_i \T{q}}\times \T{J}_i \T{V}_i
	} \nonumber \\
			& \mbox{} + \sum_{i=1,N}
			\T{V}_i^T \plbr{\T{V}_i \T{q}}\times \T{J}_i \plbr{\T{V}_i \T{q}}\times^T \T{V}_i
\end{align}
\end{subequations}
where summation over repeated indices is assumed.
The remaining summation terms could be also cast into some invariant form;
however, in common practice (e.g.\ in ADAMS) they are simply neglected,
under the assumption that the finer the discretization, the smaller 
the FEM node inertia, so that linear and quadratic terms 
in the nodal rotation become reasonably small, yielding
\begin{subequations}
\begin{align}
	\T{M}_{xx}	&= \T{I} \, \mathcal{I}_1 \\
	\T{M}_{x\phi}	&= \T{R}_0 \plbr{
		\mathcal{I}_2 + \mathcal{I}_3 \T{q}}\times{}  \T{R}_0^T \\
	\T{M}_{xq}	&= \T{R}_0 \mathcal{I}_3 \\
	\T{M}_{\phi\phi}&= \T{R}_0 \plbr{
		\mathcal{I}_7
		+ \plbr{\mathcal{I}_{8j} + \mathcal{I}_{8j}^T} q_j
		+ \mathcal{I}_{9jk} q_j q_k
	} \T{R}_0^T \\
	\T{M}_{\phi q}	&= \T{R}_0 \plbr{
		\mathcal{I}_4
		+ \mathcal{I}_{5j} q_j 
	} \\
	\T{M}_{qq}	&= \mathcal{I}_6 
\end{align}
\end{subequations}
In some cases, the only remaining quadratic term in $\mathcal{I}_{9jk}$
is neglected as well.



\subsection{Interfacing}
The basic interface between the FEM and the multibody world occurs 
by clamping regular multibody nodes to selected nodes on the FEM mesh.
Whenever more sophisticated interfacing is required, for example 
connecting a multibody node to a combination of FEM nodes, 
an FEM node equivalent to the desired aggregate of nodes 
should either be prepared at the FEM side, for example by means of RBEs,
or at the FEM database side, for example by averaging existing mode shapes
according to the desired pattern, into an equivalent FEM node\footnote{%
%
For example, to constrain the displacement of a FEM node $P$ 
that represents the weighing of the displacement of a set of FEM nodes
according to a constant weighing matrix $\T{W}_P\in\mathbb{R}^{3n\times{3}}$,
simply use $\T{U}_P=\T{W}_P^T\T{U}$.
}.

The clamping is imposed by means of a coincidence and a parallelism
constraint between the locations and the orientations of the two points:
the multibody node $N$ and the FEM node $P$, according to the expressions
\begin{align}
	\T{x}_N + \T{f}_N &= \T{x}_P \\
	\mathrm{ax}\plbr{\mathrm{exp}^{-1}\plbr{\T{R}_N^T \T{R}_P}} &= \T{0}
\end{align}
which becomes
\begin{align}
	\T{x}_N + \T{R}_N \tilde{\T{f}}_N
	- \T{x}_0 - \T{R}_0 \plbr{\tilde{\T{f}}_P + \T{U}_P \T{q}} &= \T{0}
		\label{eq:modal:pos-constr} \\
	\mathrm{ax}\plbr{\mathrm{exp}^{-1}\plbr{\T{R}_N^T \T{R}_0 \plbr{\T{I} + \plbr{\T{V}_P \T{q}}\times{}}}} &= \T{0}
		\label{eq:modal:rot-constr}
\end{align}
The reaction forces exchanged are $\T{\lambda}$ in the global frame, 
while the reaction moments are $\T{R}_N \T{\alpha}$ in the reference frame 
of node $N$:
\begin{align}
	\T{F}_N &= \T{\lambda} \\
	\T{M}_N &= \T{f}_N \times \T{\lambda} + \T{R}_N \T{\alpha} \\
	\T{F}_P &= - \T{\lambda} \\
	\T{M}_P &= - \T{R}_N \T{\alpha}
\end{align}
The force and the moment apply on the rigid body displacement and rotation,
and on the modal equations as well, according to
\begin{equation}
	\cvvect{
		\delta\T{x}_P^T \T{F}_P \\
		\mbox{} + \delta\T{\phi}_P^T \T{M}_P
	}
	= \cubr{\cvvect{
		\delta\T{x}_0 \\
		\delta\T{\phi}_0 \\
		\delta\T{q}
	}}^T \cubr{\cvvect{
		- \T{\lambda} \\
		- \plbr{\T{R}_0 \plbr{\tilde{\T{f}}_P + \T{U}_P \T{q}}} \times \T{\lambda} - \T{R}_N \T{\alpha} \\
		- \T{U}_P^T \T{R}_0^T \T{\lambda} - \T{V}_P^T \T{R}_0^T \T{R}_N \T{\alpha}
	}}
	\label{eq:modal:modal-forces}
\end{equation}

The linearization of the constraint yields
\begin{equation}
	\sqbr{\matr{ccccc}{
		-\T{I} & \T{f}_N \times{} & \T{I} & -\plbr{\T{R}_0\plbr{\tilde{\T{f}}_P + \T{U}_P \T{q}}}\times{} & \T{R}_0 \T{U}_P \\
		\T{0} & -\T{\Gamma}\plbr{\T{\theta}}^{-1} \T{R}_N^T & \T{0} & \T{\Gamma}\plbr{\T{\theta}}^{-1} \T{R}_N^T & \T{\Gamma}\plbr{\T{\theta}}^{-1} \T{R}_N^T \T{R}_0 \T{V}_P
	}}\cubr{\cvvect{
		\delta\T{x}_N \\
		\delta\T{g}_N \\
		\delta\T{x}_0 \\
		\delta\T{g}_0 \\
		\delta\T{q}
	}} = \cubr{\cvvect{
			\eqref{eq:modal:pos-constr} \\
			\eqref{eq:modal:rot-constr}
	}}
\end{equation}
Note that $\T{\Gamma}\plbr{\T{\theta}}^{-1}\cong\T{I}$
since $\T{\theta}\rightarrow\T{0}$ when the constraint is satisfied.
The linearization of forces and moments yields
\begin{align}
	&\sqbr{\matr{cc}{
		\T{0} & \T{0} \\
		\T{0} & -\T{\lambda}\times\T{f}_N\times{} + \plbr{\T{R}_N \T{\alpha}}\times{} \\
		\T{0} & \T{0} \\
		\T{0} & -\plbr{\T{R}_N \T{\alpha}}\times{} \\
		\T{0} & -\T{V}_P^T \T{R}_0^T \plbr{\T{R}_N \T{\alpha}}\times{}
	}}\cubr{\cvvect{
		\delta\T{x}_N \\
		\delta\T{g}_N
	}} \\
	+& \sqbr{\matr{ccc}{
		\T{0} & \T{0} & \T{0} \\
		\T{0} & \T{0} & \T{0} \\
		\T{0} & \T{0} & \T{0} \\
		\T{0} & \T{\lambda}\times\plbr{\T{R}_0\plbr{\tilde{\T{f}}_P + \T{U}_P \T{q}}}\times{} & -\T{\lambda}\times\T{R}_0 \T{U}_P \\
		\T{0} & \T{U}_P^T \T{R}_0^T \T{\lambda}\times{} + \T{V}_P^T \T{R}_0^T \plbr{\T{R}_N \T{\alpha}} \times{} & \T{0}
	}}\cubr{\cvvect{
		\delta\T{x}_0 \\
		\delta\T{g}_0 \\
		\delta\T{q}
	}} \nonumber \\
	+& \sqbr{\matr{cc}{
		-\T{I} & \T{0} \\
		-\T{f}_N\times{} & -\T{R}_N \\
		\T{I} & \T{0} \\
		\plbr{\T{R}_0\plbr{\tilde{\T{f}}_P + \T{U}_P \T{q}}}\times{} & \T{R}_N \\
		\T{U}_P^T \T{R}_0^T & \T{V}_P^T \T{R}_0^T \T{R}_N
	}}\cubr{\cvvect{
		\delta\T{\lambda} \\
		\delta\T{\alpha}
	}} = \cubr{\cvvect{
			\T{\lambda} \\
			\T{f}_N \times \T{\lambda} + \T{R}_N \T{\alpha} \\
			\\
			\eqref{eq:modal:modal-forces} \\
			\mbox{}
	}} \nonumber
\end{align}


\subsection{Virtual strain energy}
\emph{Author: Reinhard Resch}
The modal element is based on the assumption of small elastic deformations but arbitrary superimposed rigid body motions. Since elastic deformations are small, the virtual strain energy $\delta W$ may be approximated as a linear function of the modal displacement $\T{q}$.
\begin{equation}
  \label{eq:modal:linear-elastic} \\
  \delta W \approx \delta \T{q}^T \, \T{K}_{qq} \, \T{q}
\end{equation}
\begin{enumerate}
  \item[$\T{K}_{qq}$] reduced order linear stiffness matrix
\end{enumerate}
\subsubsection{Quasi static corrections for stress stiffening}
However, there are applications of structures subject to small elastic deformations and linear elastic material, where those linear approximation of the virtual strain energy according equation~\ref{eq:modal:linear-elastic} is not applicable due to significant pre-stress effects. See also \cite{WALLRAPP1998}. Typical examples are:
\begin{enumerate}
\item rotor blades of helicopters, wind~turbines and gas~turbines
\item propellers of airplanes
\item rotors of turbo chargers for internal combustion engines
\item connecting~rods of high speed reciprocating engines
\item slender flexible arms of robots
\end{enumerate}
According to the following assumptions, which are described in \cite{WALLRAPP1998}, it is possible to apply a quasi static correction to the virtual strain energy.
\begin{enumerate}
\item The flexible body should have low stiffness in some directions but high stiffness in other directions.
\item There are only low loads applied in directions of low stiffness. So, the deformations will remain small despite of low stiffness in those directions.
\item There are also high loads applied in directions of high stiffness. Because of such a high stiffness in those directions, the deformations will remain small. However, those high loads will cause high stresses which are affecting the stiffness in other directions. 
\end{enumerate}
Based on those assumptions, the corrected expression for the virtual strain energy according \cite{WALLRAPP1998} is:
\begin{equation}
  \delta W \approx \delta \T{q}^T \, \left( \T{K}_{qq} + \T{K}_{qq_{geo}} \right) \, \T{q} \label{eq:modal:strain-energy}
\end{equation}
In equation~\ref{eq:modal:strain-energy} the matrix $\T{K}_{qq_{geo}}$ is called ``geometrical stiffness matrix'' or ``stress stiffening matrix''. According to \cite{WALLRAPP1998}, $\T{K}_{qq_{geo}}$ can be written as a linear combination of a set of constant matrices.
\begin{align}
  \T{K}_{qq_{geo}} = &\sum_{i=1,3} \left(\tilde{\ddot{x}}_{0_i} - \tilde{g}_{i}\right) \, \T{K}_{{0t}_i} 
  + \sum_{i=1,3} \tilde{\dot{\omega}}_{0_i} \, \T{K}_{{0r}_i} 
  + \sum_{m=1,6} \tilde{\omega}_{q_m} \, \T{K}_{{0\omega}_m} \\ \nonumber
  +& \sum_{i=1,3} \sum_{j=1,M} \tilde{F}_{P_{ij}} \, \T{K}_{{0F}_{ij}}
  + \sum_{i=1,3} \sum_{j=1,M} \tilde{M}_{P_{ij}} \, \T{K}_{{0M}_{ij}}
\end{align}
So, $\T{K}_{qq_{geo}}$ is a function of the rigid body motion of the modal node, and also a function of the Lagrange multipliers applied at the interface nodes.
\begin{align}
  \tilde{\ddot{\T{x}}}_0 = & \T{R}_0^T \, \ddot{\T{x}}_0 \\
  \tilde{\dot{\T{\omega}}}_0 = & \T{R}_0^T \, \dot{\T{\omega}}_0 \\
  \tilde{\T{\omega}}_q = & \begin{pmatrix}
    \tilde{\omega}_{0_1}^2 &
    \tilde{\omega}_{0_2}^2 &
    \tilde{\omega}_{0_3}^2 &
    \tilde{\omega}_{0_1} \, \tilde{\omega}_{0_2} &
    \tilde{\omega}_{0_2} \, \tilde{\omega}_{0_3} &
    \tilde{\omega}_{0_1} \, \tilde{\omega}_{0_3}
  \end{pmatrix}^T \\
  \tilde{\T{\omega}}_{0} = & \T{R}_0^T \, \T{\omega}_0 \\
  \tilde{\T{F}}_{P_j} = & -\T{R}_0^T \, \T{\lambda}_j \\
  \tilde{\T{M}}_{P_j} = & -\T{R}_0^T \, \T{R}_N \, \T{\alpha}_j
\end{align}
Those matrices are also called ``pre-stress stiffness matrices'' and can be a determined a priori based on the stress field for various static unit loads.
\begin{enumerate}
\item[$\T{K}_{{0t}_i}$] A unit linear acceleration is applied in direction $x_i$
\item[$\T{K}_{{0r}_i}$] A unit angular acceleration is applied along the $x_i$ axis
\item[$\T{K}_{{0\omega}_m}$] Can be computed from six angular velocity loads
\item[$\T{K}_{{0F}_{ij}}$] A unit force is applied in $x_i$ direction at interface node~$j$
\item[$\T{K}_{{0M}_{ij}}$] A unit moment is applied along the $x_i$ axis at interface node~$j$
\end{enumerate}
In theory, any structural Finite~Element solver like ANSYS, ABAQUS or NASTRAN could be used to generate those matrices. For a comprehensive description of the underlying procedure, the reader is referred to \cite{BATHE2016}.
Probably, the most convenient solution to get those matrices and all the modal element data is, to use a dedicated toolkit like \href{https://github.com/octave-user/mboct-fem-pkg}{mboct-fem-pkg} with builtin support for MBDyn's input files.
