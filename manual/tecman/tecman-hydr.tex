\section{Hydraulic Fluids}
\section{Hydraulic Nodes}

\section{Hydraulic Elements}

\subsection{Accumulator}
The accumulator defines two internal states $x$ and $v$ that represent 
the position and the velocity of the cap that, in a conventional
gas device, separates the fluid and the gas.
However, both the gravity effect and a linear spring effect
can be considered as well, and any combination of reaction forces
can be modeled by setting the appropriate parameters:
$g$ for a gravity device, $p_{g0}$ for a gas device, 
and $k$ for a linear spring device.
\begin{eqnarray*}
	0 & = & q \\
	m \dot{v} + k x & = & 
		- m g
		+ A p \plbr{p - p_g}
		- f_0 
		- \frac{1}{2} \rho A c_e \plbr{\frac{A}{A_p}}^2 \shbr{v} v \\
	& & \mbox{} - \step\plbr{x_{\llk{min}} - x}
		\plbr{c_1 \plbr{x - x_{\llk{min}}} + c_2 v + c_3 \dot{v}} \\
	& & \mbox{} - \step\plbr{x - x_{\llk{max}}}
		\plbr{c_1 \plbr{x - x_{\llk{max}}} + c_2 v + c_3 \dot{v}} \\
	\dot{x} & = & v
\end{eqnarray*}
where $p_g=p_{g0}\plbr{\frac{l}{l - x}}^{\gamma}$ and $q=\rho A v$.

\subsection{Actuator}
The hydraulic actuator element couples the hydraulic library 
with the structural library.
It connects the displacement of two structural nodes to the flow
through two hydraulic nodes, and the pressure at two hydraulic nodes
to the forces applied at two structural nodes.
In the spirit of the multibody analysis philosophy, this element
provides the essential connection between structural 
and hydraulic nodes; the constraints between the structural nodes, 
and other flow elements, e.g.\ leakages between the chambers, 
must be added by the user.

\subsection{Definitions}
In the following, $\plbr{\cdot}_{s1}$ and $\plbr{\cdot}_{s2}$
refer to structural nodes 1 and 2, 
and $\plbr{\cdot}_{h1}$ and $\plbr{\cdot}_{h2}$
refer to hydraulic nodes 1 and 2.
The structural node labeled as 1 is assumed as the cylinder,
and its orientation determines the axis of the actuator.
The relative orientation of the actuator is defined by the unit vector
$\tilde{\T{v}}$, and the absolute orientation is $\T{R}_{s1} \tilde{\T{v}}$.
It is assumed that appropriate kinematic constraints allow only 
a relative displacement of the structural nodes along 
the axis $\tilde{\T{v}}$, and the only relative rotation, 
if any, is about the axis itself.
This can be obtained by combining an inline joint with 
a revolute rotation or a prismatic joint.

\subsection{Equations}
\begin{eqnarray*}
	0 & = & -\T{F} \\
	0 & = & -\T{f}_{s1}\times\T{F} \\
	0 & = & \T{F} \\
	0 & = & \T{f}_{s2}\times\T{F} \\
	0 & = & q_{h1} \\
	0 & = & q_{h2} \\
	p_{h1} & = & P_{h1} \\
	p_{h2} & = & P_{h2}
\end{eqnarray*}
The first four equations apply the force resulting from the hydraulic 
pressure to the structural nodes; the fifth and the sixth apply 
the flow resulting from the actuator kinematics to the flow balance
equations of the hydraulic nodes.
The last two equations are required to associate two scalar
differential unknowns to the hydraulic node pressures, because 
the flow definitions require the derivative of the pressure, 
while the hydraulic nodes are defined as scalar algebraic.

The force is defined as
\begin{equation}
	\T{F} = \plbr{A_{h1} p_{h1} - A_{h2} p_{h2}}
		\plbr{\T{R}_{s1} \tilde{\T{v}}}
\end{equation}
The distance between the structural nodes, along the actuator axis, is
\begin{equation}
	l = \plbr{\T{R}_{s1} \tilde{\T{v}}}^T \plbr{
		\T{x}_{s2} + \T{f}_{s2} - \T{x}_{s1} - \T{f}_{s1}
	}
\end{equation}
The relative velocity of the structural nodes, along the actuator axis, is
\begin{equation}
	\dot{l} = \plbr{\T{R}_{s1} \tilde{\T{v}}}^T \plbr{
		\dot{\T{x}}_{s2} - \dot{\T{x}}_{s1} 
		+ \T{\omega}_{s1}\times\plbr{\T{x}_{s2} - \T{x}_{s1}}
		+ \plbr{\T{\omega}_{s2} - \T{\omega}_{s1}}\times\T{f}_{s2}
	}
\end{equation}
The flow at the two hydraulic nodes is
\begin{eqnarray*}
	q_{h1} & = & A_{h1} \plbr{
		l \frac{\partial\rho_{h1}}{\partial{p}}\dot{P}_{h1}
		+ \rho_{h1} \dot{d}
	} \\
	q_{h2} & = & A_{h2} \plbr{
		\plbr{L - l} \frac{\partial\rho_{h2}}{\partial{p}}\dot{P}_{h2}
		- \rho_{h2} \dot{d}
	}
\end{eqnarray*}
where $\rho$ is the fluid density (different fluids in the chambers 
are allowed), and $L$ is the total length of the actuator.
In case stroke limitations must be enforced, the kinematic 
constraints must account for them.


\subsection{Dynamic Pipe}
Finite Volume dynamic pipe.

\subsection{Definitions}
The dynamic pipe is formulated according to the finite volume approach.
The pipe is discretized by means of the pressures and the flow 
at the two ends, which are interpolated linearly.
The mass and the momentum balance equations are written by cutting
the pipe in two halves and adding the contribution of each resulting
subvolume to the respective nodal equations.


Consider the mass conservation and the momentum balance equations for a
one-dimensional flow:
\begin{eqnarray}
    \frac{D}{Dt}\plbr{dm} & = & 0 , \label{eq:MASS} \\
    \frac{D}{Dt}\plbr{dQ} & = & df . \label{eq:MOMENTUM}
\end{eqnarray}
When a rigid pipe is considered, the total derivative $D/Dt$ of the test
mass $dm=\rho{Adx}$ of Equation~(\ref{eq:MASS}) yields 
\begin{equation} 
    \frac{D}{Dt}\plbr{dm} \ = \ 
    \frac{\partial}{\partial{t}}\plbr{dm}
    + v \frac{\partial}{\partial{x}}\plbr{dm} ,
\end{equation}
which results in
\begin{equation}
    q_{/x} + A\rho_{/t} \ = \ 0 ,
\end{equation}
where $q=\rho{Av}$ is the mass flow.
Consider now the momentum equation~(\ref{eq:MOMENTUM}); the total derivative
of the momentum $dQ=vdm$ yields
\begin{equation}
    \frac{D}{Dt}\plbr{dQ} \ = \ \plbr{q_{/t} + \plbr{qv}_{/x}}dx ,
\end{equation}
while the pressure gradient and the viscous contributions can be isolated
from the force per unit length on the right hand side:
\begin{equation}
    df \ = \ - Adp + f_v dx + df^* ,
\end{equation}
so, by neglecting the deformability of the pipe and the extra forces $df^*$
acting on the fluid, the momentum balance equation yields
\begin{equation}
    q_{/t} + \plbr{qv + Ap}_{/x} \ = \ f_v ,
\end{equation}
which can be reduced to the pressure and flow unknowns simply by recalling
the definition of the flow:
\begin{equation}
    q_{/t} + \plbr{\frac{q^2}{\rho{A}} + Ap}_{/x} \ = \ f_v .
\end{equation}
A flexible pipe has been considered as well; the formulation is not reported
for simplicity, because such a level of detail is required only for very
specialized problems, and a first approximation can be obtained by altering
the bulk modulus of the fluid.

The pipe is discretized by considering a finite volume approach, based on
the use of constant stepwise ({\em Heavyside}) test functions with arbitrary
trial functions.
In the present case, linear trial functions have been considered both for
the flow and for the pressure:
\begin{eqnarray*}
    q\plbr{x} & = & \sqbr{\matr{cc}{
        \cfrac{1 - \xi}{2} &
        \cfrac{1 + \xi}{2}
    }}\cubr{\cvvect{
        q_1 \\
        q_2
    }} , \\
    p\plbr{x} & = & \sqbr{\matr{cc}{
        \cfrac{1 - \xi}{2} &
        \cfrac{1 + \xi}{2}
    }}\cubr{\cvvect{
        p_1 \\
        p_2
    }} ,
\end{eqnarray*}
with $\xi=\xi\plbr{x}\in\sqbr{-1,1}$ and $d\xi/dx=2/\plbr{b-a}$.
The discrete form of the pipe equations results in
\begin{eqnarray*}
    \lefteqn{
        q\plbr{b} - q\plbr{a} \ = \
        - \intg{a}{b}{\frac{\partial\rho}{\partial{p}}p_{/t}}{dx} ,
    } \hspace{20mm} \\
    \lefteqn{
        \frac{b-a}{2}\plbr{
            q\plbr{b}_{/t} + q\plbr{a}_{/t}
        } + \plbr{
            \frac{q\plbr{b}^2}{\rho\plbr{b} A} + A p\plbr{b}
        }
    } \hspace{40mm} \\
    \mbox{} - \plbr{
        \frac{q\plbr{a}^2}{\rho\plbr{a} A} + A p\plbr{a}
    } & = & \intg{a}{b}{f_v}{dx} ;
\end{eqnarray*}
by dividing the pipe in two portions, and by considering the domains
$\sqbr{-1,0}$ and $\sqbr{0,1}$ for $\xi$ in each portion, the discrete
equations of the finite volume pipe become
\begin{eqnarray*}
    \lefteqn{
    - \frac{1}{2}\plbr{
        q_1 + q_2
    } - \frac{\partial\rho\plbr{-1/2}}{\partial{p}}\frac{L}{8}\plbr{
        3 \dot{p}_1 + \dot{p}_2
    } \ = \ \phi_1 , 
    } \hspace{60mm} \\
    \lefteqn{
    \frac{1}{2}\plbr{
        q_1 + q_2
    } - \frac{\partial\rho\plbr{1/2}}{\partial{p}}\frac{L}{8}\plbr{
        \dot{p}_1 + 3 \dot{p}_2
    } \ = \ \phi_2 , 
    } \hspace{54.5mm} \\
    \lefteqn{
    \frac{L}{8}\plbr{
        3 \dot{q}_1 + \dot{q}_2
    } + % \frac{q\plbr{0}^2}{\rho\plbr{0}A}
    \frac{\plbr{q_1+q_2}^2}{4\rho\plbr{0}A}
    - \frac{q_1^2}{\rho\plbr{-1}A}
    } \hspace{60mm} \\
    \lefteqn{
    \mbox{} + \frac{A}{2}\plbr{
        p_2 - p_1
    } \ = \ \frac{L}{2}\intg{-1}{0}{f_v}{d\xi} , 
    } \hspace{40mm} \\
    \lefteqn{
    \frac{L}{8}\plbr{
        \dot{q}_1 + 3 \dot{q}_2
    } + \frac{q_2^2}{\rho\plbr{1}A}
    - % \frac{q\plbr{0}^2}{\rho\plbr{0}A}
    \frac{\plbr{q_1+q_2}^2}{4\rho\plbr{0}A}
    } \hspace{60mm} \\
    \lefteqn{
    \mbox{} + \frac{A}{2}\plbr{
        p_2 - p_1
    } \ = \ \frac{L}{2}\intg{0}{1}{f_v}{d\xi} ,
    } \hspace{40mm}
\end{eqnarray*}
where $\phi_1$ and $\phi_2$ are the contributions of the two portions of
pipe to the respective nodal flow balance equations.
The integral of the time derivative of the density is numerically computed.
The integral of the viscous forces per unit length is numerically performed
as well, accounting for the flow regime in the pipe as function of the
{\em Reynolds}\ number.
In fact, for the forces per unit length, the dependency on the flow is
considered linear for $0<Re<2000$, and quadratic for $Re>4000$, while a
polynomial fitting of the transition behavior, accounting also for the rate
of the {\em Reynolds}\ number, is modeled for $2000<Re<4000$.

\subsubsection{Equations}
\begin{eqnarray*}
	0 & = & \frac{1}{2}\plbr{q_1 + q_2}
		+ A L \frac{\partial\rho}{\partial{p}}_1 
		\plbr{\frac{3}{8} \dot{P}_1 + \frac{1}{8} \dot{P}_2} \\
	0 & = & -\frac{1}{2}\plbr{q_1 + q_2}
		+ A L \frac{\partial\rho}{\partial{p}}_2 
		\plbr{\frac{1}{8} \dot{P}_1 + \frac{3}{8} \dot{P}_2} \\
	0 & = & - L \plbr{\frac{3}{8} \dot{q}_1 + \frac{1}{8} \dot{q}_2}
		- \plbr{\frac{1}{2}\plbr{q_1 + q_2}}^2 \frac{1}{\rho_m A}
		+ q_1^2 \frac{1}{\rho_1 A}
		- \frac{A}{2}\plbr{p_2 - p_1}
		- f_1 \\
	0 & = & - L \plbr{\frac{1}{8} \dot{q}_1 + \frac{3}{8} \dot{q}_2}
		- q_2^2 \frac{1}{\rho_2 A}
		+ \plbr{\frac{1}{2}\plbr{q_1 + q_2}}^2 \frac{1}{\rho_m A}
		- \frac{A}{2}\plbr{p_2 - p_1}
		- f_2 \\
	p_1 & = & P_1 \\
	p_2 & = & P_2
\end{eqnarray*}
