% MBDyn (C) is a multibody analysis code.
% http://www.mbdyn.org
%
% Copyright (C) 1996-2023
%
% Pierangelo Masarati  <pierangelo.masarati@polimi.it>
%
% Dipartimento di Ingegneria Aerospaziale - Politecnico di Milano
% via La Masa, 34 - 20156 Milano, Italy
% http://www.aero.polimi.it
%
% Changing this copyright notice is forbidden.
%
% This program is free software; you can redistribute it and/or modify
% it under the terms of the GNU General Public License as published by
% the Free Software Foundation (version 2 of the License).
%
%
% This program is distributed in the hope that it will be useful,
% but WITHOUT ANY WARRANTY; without even the implied warranty of
% MERCHANTABILITY or FITNESS FOR A PARTICULAR PURPOSE.  See the
% GNU General Public License for more details.
%
% You should have received a copy of the GNU General Public License
% along with this program; if not, write to the Free Software
% Foundation, Inc., 59 Temple Place, Suite 330, Boston, MA  02111-1307  USA
\emph{Author: Reinhard Resch} \\
A rigid body displacement joint is a load distributing rigid body coupling constraint similar to NASTRAN's \texttt{RBE3} element.
\paragraph{Files.} \
It is implemented in files

\begin{tabular}{l}
\texttt{mbdyn/struct/rbdispjad.h} \\
\texttt{mbdyn/struct/rbdispjad.cc}
\end{tabular}

\paragraph{Kinematics}
The purpose of a rigid body displacement joint is, to apply a position and orientation constraint to a master node in such a way,
that the master node is following the weighted average rigid body motion of a set of displacement only slave nodes.
In contrast to NASTRAN's \texttt{RBE3} element, large deformations and large rotations are supported.
For each slave node the difference $\T{u}_j$, between the actual position of the slave node $\T{x}_j$ and the corresponding nominal offset
from the master node, can be written as:
\begin{eqnarray}
\T{u}_j & = & \T{x}_m + \T{R}_m \, \T{o}_j- \T{x}_j \label{eq:rbdispjoint:u}
\end{eqnarray}

\begin{description}
\item[$\T{x}_m$] Position of the master node
\item[$\T{R}_m$] Orientation matrix of the master node
\item[$\T{o}_j$] Initial distance between the slave node and the master node
\item[$\T{x}_j$] Position of the slave node
\end{description}
In order to derive the expressions for the kinematic constraints, the virtual perturbation of $\T{u}_j$ is needed.
\begin{eqnarray}
\delta\T{u}_j & = & \delta\T{x}_m + \delta\T{R}_m \, \T{o}_j - \delta\T{x}_j \\
\delta\T{R}_m & \approx & \delta\T{\varphi}_m \times \T{R}_m \\
\delta\T{u}_j & = & \delta\T{x}_m + \delta\T{\varphi}_m \times \left( \T{R}_m \, \T{o}_j \right) - \underbrace{\delta{\T{x}}_j}_{=\T{0}} \label{eq:rbdispjoint:deltau}
\end{eqnarray}
\begin{description}
\item[$\delta\T{x}_m$] Virtual perturbation of the position of the master node
\item[$\delta\T{R}_m$] Virtual perturbation of the orientation matrix of the master node
\item[$\delta\T{\varphi}_m$] Virtual perturbation of the rotation vector of the master node
\end{description}
Since the kinematic constraints should be applied only to the master node, $\delta\T{x}_j=\T{0}$ is assumed.
\paragraph{Fictitious strain energy}
In the following equation, the virtual perturbation of the fictitious strain energy $U^{\star}$ is defined.
\begin{eqnarray}
\delta U^{\star} & = & \sum_{j=1}^{N} w_j \, \delta\T{u}_j^T \, \T{u}_j \label{eq:rbdispjoint:deltaU}
\end{eqnarray}
\begin{description}
\item[$\delta U^{\star}$] Virtual perturbation of the fictitious strain energy
\item[$w_j$] Weighting factor for slave node~j
\item[$N$] Number of slave nodes
\end{description}
It is not required, that every weighting factor $w_j$ is greater than zero. However the following condition must hold:
\begin{eqnarray}
\sum_{j=1}^N w_j &>& 0
\end{eqnarray}
\paragraph{Physical interpretation}
One could give the following physical interpretation of $U^{\star}$: Between each slave node and the master node, a fictitious linear isotropic spring
with stiffness $w_j$ is attached at an offset $\T{o}_j$ with respect to the master node.
Then $U^{\star}$ is the overall strain energy of all fictitious springs.
In order to fulfill the constraint equations, the fictitious strain energy $U^{\star}$ must be minimized.
For that reason, we may write:
\begin{eqnarray}
\delta U^{\star} &  \equiv & \T{0} \label{eq:rbdispjoint:deltaU_0}
\end{eqnarray}
Within this context it is not obvious why $w_j < 0$ should be allowed.
However, this condition is required only for second order hexahedral elements which require negative weights at the corner nodes.
Otherwise it would not be possible to achieve a smooth stress distribution for such elements.
\paragraph{Constraint equations}
Now equation~\ref{eq:rbdispjoint:u} and equation~\ref{eq:rbdispjoint:deltau} are substituted into equation~\ref{eq:rbdispjoint:deltaU}:
\begin{eqnarray}
\delta U^{\star} & = & \sum_{j=1}^{N} w_j \, \left(\delta\T{x}_m^T + \delta\T{\varphi}_m^T \times \T{R}_m \, \T{o}_j\right) \left( \T{x}_m + \T{R}_m \, \T{o}_j - \T{x}_j \right) \\
& = & \sum_{j=1}^{N} w_j \, \left[\delta\T{x}_m^T\,\left(\T{x}_m + \T{R}_m \, \T{o}_j - \T{x}_j \right) + \delta\T{\varphi}_m^T \, \left(\T{R}_m \, \T{o}_j \right)\times \left(\T{x}_m + \T{R_m} \, \T{o}_j - \T{x}_j \right)\right] \label{eq:rbdispjoint:deltaUrt}
\end{eqnarray}
Since equation~\ref{eq:rbdispjoint:deltaUrt} and equation~\ref{eq:rbdispjoint:deltaU_0} must be valid for any $\delta\T{x}_m$ and any $\delta\T{\varphi}_m$ we can directly derive the constraint equations $\T{\Phi}_t$ and $\T{\Phi}_r$:
\begin{eqnarray}
\T{\Phi}_t & = & \sum_{j=1}^{N} w_j \, \left(\T{x}_m + \T{R}_m \, \T{o}_j - \T{x}_j\right) = \T{0} \\
\T{\Phi}_r & = & \sum_{j=1}^{N} w_j \, \left(\T{R}_m \, \T{o}_j \right)\times\left(\T{x}_m + \T{R}_m \, \T{o}_j - \T{x}_j \right) = \T{0}
\end{eqnarray}
\begin{description}
\item[$\T{\Phi}_t$] Constraint equations for the position of the master node
\item[$\T{\Phi}_r$] Constraint equations for the orientation of the master node
\end{description}
Because $\left(\T{R}_m \, \T{o}_j\right) \times \left(\T{R}_m \, \T{o}_j\right) = \T{0}$, the expression for $\T{\Phi}_r$ can be simplified.
\begin{eqnarray}
\T{\Phi}_r & = & \sum_{j=1}^{N} w_j \, \left(\T{R}_m \, \T{o}_j \right)\times\left(\T{x}_m - \T{x}_j \right) = \T{0}
\end{eqnarray}
\paragraph{Virtual perturbation of constraint equations}
In order to include Lagrange multipliers in the equations of motion, it is required to derive the virtual perturbation of the constraint equations $\delta\T{\Phi}_t$ and $\delta\T{\Phi}_r$.
\begin{eqnarray}
\delta\T{\Phi}_t & = & \sum_{j=1}^{N} w_j \, \left[\delta\T{x}_m + \delta\T{\varphi}_m \times \left(\T{R}_m \, \T{o}_j\right) - \delta\T{x}_j \right] \\
                 & = & \sum_{j=1}^{N} w_j \, \left[\delta\T{x}_m - \left(\T{R}_m \, \T{o}_j\right) \times \delta\T{\varphi}_m - \delta\T{x}_j \right] \\
\delta\T{\Phi}_t^T & = & \sum_{j=1}^{N} w_j \, \left[\delta\T{x}_m^T + \delta\T{\varphi}_m^T \, \left(\T{R}_m \, \T{o}_j\right)\times - \delta\T{x}_j^T \right] \label{eq:rbdispjoint:deltaPhit} \\
\end{eqnarray}
Now equation~\ref{eq:rbdispjoint:deltaPhit} can be rearranged and written in matrix form:
\begin{eqnarray}
\delta\T{\Phi}_t^T & = & \begin{pmatrix}
                   \delta\T{x}_m^T &
                   \delta\T{\varphi}_m^T &
                   \delta\T{x}_1^T &
                   \hdots{} &
                   \delta\T{x}_N^T
                   \end{pmatrix} \,
\begin{pmatrix}
        \sum_{j=1}^N w_j \, \T{I} \\
        \sum_{j=1}^N w_j \, \left(\T{R}_m \, \T{o}_j\right)\times \\
        -w_1 \, \T{I} \\
        \vdots{} \\
        -w_N \, \T{I}
\end{pmatrix}
\end{eqnarray}
The same procedure is also applied to $\T{\Phi}_r$.
\begin{eqnarray}
\delta\T{\Phi}_r & = & \sum_{j=1}^{N} w_j\,\left[-\left(\T{x}_m - \T{x}_j\right) \times \delta\T{R}_m \, \T{o}_j + \left(\T{R}_m\,\T{o}_j\right)\times\left(\delta\T{x}_m - \delta\T{x}_j\right)\right] \\
& = & \sum_{j=1}^{N} w_j \, \left[-\left(\T{x}_m - \T{x}_j\right)\times\delta\T{\varphi}_m\times\left(\T{R}_m \, \T{o}_j\right) + \left(\T{R}_m\,\T{o}_j\right) \times \left(\delta\T{x}_m - \delta\T{x}_j\right)\right] \\
& = & \sum_{j=1}^{N} w_j \, \left[\left(\T{x}_m - \T{x}_j\right)\times\left(\T{R}_m \, \T{o}_j\right)\times\delta\T{\varphi}_m + \left(\T{R}_m\,\T{o}_j\right) \times \left(\delta\T{x}_m - \delta\T{x}_j\right)\right] \\
\delta\T{\Phi}_r^T & = & \sum_{j=1}^{N} w_j \, \left[\delta\T{\varphi}_m^T \left(\T{R}_m \, \T{o}_j\right)\times\left(\T{x}_m - \T{x}_j\right)\times - \left(\delta\T{x}_m^T - \delta\T{x}_j^T\right) \left(\T{R}_m\,\T{o}_j\right) \times \right] \label{eq:rbdispjoint:deltaPhir}
\end{eqnarray}
Also equation~\ref{eq:rbdispjoint:deltaPhir} can be rearranged and written in matrix form:
\begin{eqnarray}
\delta\T{\Phi}_r^T & = &
\begin{pmatrix}
\delta\T{x}_m^T &
\delta\T{\varphi}_m^T &
\delta\T{x}_1^T &
\hdots{} &
\delta\T{x}_N^T
\end{pmatrix} \,
\begin{pmatrix}
-\sum_{j=1}^N w_j\,\left(\T{R}_m\,\T{o}_j\right)\times \\
\sum_{j=1}^N w_j\,\left(\T{R}_m\,\T{o}_j\right)\times\left(\T{x}_m - \T{x}_j\right)\times \\
w_1\,\left(\T{R}_m\,\T{o}_1\right)\times \\
\vdots{} \\
w_N\,\left(\T{R}_m\,\T{o}_N\right)\times
\end{pmatrix}
\end{eqnarray}
\paragraph{Adding contributions due to Lagrange multipliers}
In order to enforce the constraint equations $\T{\Phi}_t$ and $\T{\Phi}_r$, Lagrange multipliers $\T{\lambda}_t$ and $\T{\lambda}_r$ are required.
\begin{eqnarray}
\delta W_{\lambda} & = & \delta\T{\Phi}_t^T \, \T{\lambda}_t + \delta\T{\Phi}_r^T \, \T{\lambda}_r \\
& = &
\begin{pmatrix}
\delta\T{x}_m \\
\delta\T{\varphi}_m \\
\delta\T{x}_1 \\
\vdots{} \\
\delta\T{x}_N
\end{pmatrix}^T \,
\begin{pmatrix}
\sum_{j=1}^N w_j\,\left[\T{\lambda}_t - \,\left(\T{R}_m\,\T{o}_j\right)\times\T{\lambda}_r\right] \\
\sum_{j=1}^N w_j\left[\left(\T{R}_m\,\T{o}_j\right)\times\T{\lambda}_t+\left(\T{R}_m\,\T{o}_j\right)\times\left(\T{x}_m - \T{x}_j\right)\times\T{\lambda}_r\right] \\
w_1\left[\left(\T{R}_m\,\T{o}_1\right)\times\T{\lambda}_r - \T{\lambda}_t\right] \\
\vdots{} \\
w_N\left[\left(\T{R}_m\,\T{o}_N\right)\times\T{\lambda}_r - \T{\lambda}_t\right]
\end{pmatrix} \\
& = &
\begin{pmatrix}
\delta\T{x}_m \\
\delta\T{\varphi}_m \\
\delta\T{x}_1 \\
\vdots{} \\
\delta\T{x}_N
\end{pmatrix}^T \,
\begin{pmatrix}
\T{F}_m \\
\T{M}_m \\
\T{F}_1 \\
\vdots{} \\
\T{F}_N
\end{pmatrix}
\end{eqnarray}
\begin{description}
\item[$\T{\lambda}_t$] Lagrange multipliers for position constraints $\T{\Phi}_t$
\item[$\T{\lambda}_r$] Lagrange multipliers for orientation constraints $\T{\Phi}_r$
\item[$\T{F}_m$] Reaction force at the master node
\item[$\T{M}_m$] Reaction couple at the master node
\item[$\T{F}_j$] Reaction force at slave node~j
\end{description}
\paragraph{Contributions of the element}
Finally the contributions of the rigid body displacement joint can be summarized as:
\begin{eqnarray}
\begin{pmatrix}
\texttt{dCoef}^{-1} \, \T{\Phi}_t \\
\texttt{dCoef}^{-1} \, \T{\Phi}_r \\
\T{F}_m \\
\T{M}_m \\
\T{F}_1 \\
\vdots{} \\
\T{F}_N
\end{pmatrix}
& = &
\begin{pmatrix}
\texttt{dCoef}^{-1} \, \sum_{j=1}^{N} w_j \, \left(\T{x}_m + \T{R}_m \, \T{o}_j - \T{x}_j\right) \\
\texttt{dCoef}^{-1} \, \sum_{j=1}^{N} w_j \, \left(\T{R}_m \, \T{o}_j \right)\times\left(\T{x}_m - \T{x}_j \right) \\
\sum_{j=1}^N w_j\,\left[\T{\lambda}_t - \,\left(\T{R}_m\,\T{o}_j\right)\times\T{\lambda}_r\right] \\
\sum_{j=1}^N w_j\left(\T{R}_m\,\T{o}_j\right)\times\left[\T{\lambda}_t+\left(\T{x}_m - \T{x}_j\right)\times\T{\lambda}_r\right] \\
w_1\left[\left(\T{R}_m\,\T{o}_1\right)\times\T{\lambda}_r - \T{\lambda}_t\right] \\
\vdots{} \\
w_N\left[\left(\T{R}_m\,\T{o}_N\right)\times\T{\lambda}_r - \T{\lambda}_t\right]
\end{pmatrix} \label{eq:rbdispjoint:elemcontrib}
\end{eqnarray}
In equation~\ref{eq:rbdispjoint:elemcontrib} the constraint equations $\T{\Phi}_t$ and $\T{\Phi}_r$
are divided by $\texttt{dCoef}$ in order to improve the condition number of the Jacobian matrix.
