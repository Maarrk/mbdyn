% MBDyn (C) is a multibody analysis code.
% http://www.mbdyn.org
%
% Copyright (C) 1996-2023
%
% Pierangelo Masarati  <pierangelo.masarati@polimi.it>
%
% Dipartimento di Ingegneria Aerospaziale - Politecnico di Milano
% via La Masa, 34 - 20156 Milano, Italy
% http://www.aero.polimi.it
%
% Changing this copyright notice is forbidden.
%
% This program is free software; you can redistribute it and/or modify
% it under the terms of the GNU General Public License as published by
% the Free Software Foundation (version 2 of the License).
%
%
% This program is distributed in the hope that it will be useful,
% but WITHOUT ANY WARRANTY; without even the implied warranty of
% MERCHANTABILITY or FITNESS FOR A PARTICULAR PURPOSE.  See the
% GNU General Public License for more details.
%
% You should have received a copy of the GNU General Public License
% along with this program; if not, write to the Free Software
% Foundation, Inc., 59 Temple Place, Suite 330, Boston, MA  02111-1307  USA
\emph{Author: Reinhard Resch}
\subsection{Kinematics}
\label{sec:solid:interpol}
MBDyn's solid elements are based on the classical displacement based isoparametric Finite Element formulation \cite{BATHE2016}.
The three dimensional global Cartesian coordinates $\T{x}$ and the displacements $\T{u}=\T{x} - {^0}\T{x}$, of a particle inside an element,
are interpolated between coordinates $\hat{\T{x}}$ and displacements $\hat{\T{u}}$ at element nodes by means of shape functions $\T{h}=\mathit{f}\left(\T{r}\right)$ \cite{BATHE2016}, \cite{KUEBLER2005}.
\begin{eqnarray}
  \T{u}_{i} &=& \sum_{k=1}^{N_n}\hat{\T{u}}_{ik} \,\T{h}_{k} \label{eq:solid:u:interpol} \\
  \T{x}_{i} &=& \sum_{k=1}^{N_n}\hat{\T{x}}_{ik} \,\T{h}_{k} \\
  \T{u} &=& \begin{pmatrix} \T{u}_1 & \T{u}_2 & \T{u}_3 \end{pmatrix}^T \\
  \T{x} &=& \begin{pmatrix} \T{x}_1 & \T{x}_2 & \T{x}_3 \end{pmatrix}^T \\
  \T{r} &=& \begin{pmatrix} \T{r}_1 & \T{r}_2 & \T{r}_3 \end{pmatrix}^T \\
  \hat{\T{u}} &=& \begin{pmatrix}
    \hat{\T{u}}_{1,1} & \hat{\T{u}}_{1,2} & \ldots{} & \hat{\T{u}}_{1,N_n} \\
    \hat{\T{u}}_{2,1} & \hat{\T{u}}_{2,2} & \ldots{} & \hat{\T{u}}_{2,N_n} \\
    \hat{\T{u}}_{3,1} & \hat{\T{u}}_{2,2} & \ldots{} & \hat{\T{u}}_{3,N_n}
  \end{pmatrix} \\
  \T{h} & = & \begin{pmatrix}
    \T{h}_1 & \T{h}_2 & \ldots{} & \T{h}_{N_n}
  \end{pmatrix}^T
\end{eqnarray}

\begin{description}
\item[$\T{x}=\mathit{f}\left(\T{r}\right)$] global Cartesian coordinates of a particle inside the deformed body
\item[$^0\T{x}=\mathit{f}\left(\T{r}\right)$] global Cartesian coordinates of a particle inside the undeformed body
\item[$\T{u}=\T{x}-{^0}\T{x}$] deformation of a particle within the body
\item[$\T{r}$] curvilinear natural coordinates of a particle within the body
\item[$\T{h}=\mathit{f}\left(\T{r}\right)$] shape functions of a particular Finite Element type
\item[$N_n$] number of nodes per element
\end{description}

\paragraph{Shape functions}
All solid elements in MBDyn are based on templates and are using standard shape functions based on \cite{BATHE2016}, \cite{DHONDT2004} and \cite{CODEASTERR30301}.
In order to implement a new solid~element type in MBDyn, it is sufficient to provide a new C++ class which defines it's shape functions $\T{h}$,
derivatives of the shape functions versus the natural coordinates $\T{h}_d=\frac{\partial \T{h}}{\partial \T{r}}$ and also the integration points $\left.\T{r}\right\vert_g$ and weights $\left.\alpha\right\vert_g$.
The following types of elements are currently implemented:
\begin{itemize}
\item Hexahedrons with linear and quadratic shape functions \cite{BATHE2016}, \cite{DHONDT2004}
\item Pentahedrons with quadratic shape functions \cite{CODEASTERR30301}
\item Tetrahedrons with quadratic shape functions \cite{CODEASTERR30301}
\end{itemize}
See also MBDyn's input-manual for further information.

\paragraph{Deformation gradient}
Because MBDyn's solid elements are based on a Total~Lagrangian~Formulation, the deformation gradient $\T{F}$ is evaluated with respect to the undeformed state ${^0}\T{x}$.
As a consequence, the Jacobian matrix ${^0}\T{J}$ and the gradient operator $\T{h}_{0d}$ are constant.
\begin{IEEEeqnarray}{rCl}
  {^0}\T{J}_{ij} & = & \frac{\partial{^0}\T{x}_j}{\partial \T{r}_i} = \sum_{k=1}^{N_n} {^0}\hat{\T{x}}_{jk} \, \underbrace{\frac{\partial \T{h}_{k}}{\partial \T{r}_i}}_{\T{h}_{d_{ki}}} = \left({^0}\T{x}\,\T{h}_d\right)^T\\
  \left\{{^0}\T{J}^{-1}\right\}_{ij} & = & \frac{\partial\T{r}_j}{\partial {^0}\T{x}_i} \\
  %\T{h}_{d_{ij}} & = & \frac{\partial \T{h}_i}{\partial \T{r}_j} \\
  \T{F}_{ij} & = & \frac{\partial \T{x}_{i}}{\partial {^0}\T{x}_{j}} = \delta_{ij} + \frac{\partial \T{u}_i}{\partial ^0\T{x}_{j}} = \delta_{ij} + \sum_{k=1}^{N_n}\hat{\T{u}}_{ik} \, \underbrace{\frac{\partial \T{h}_k}{\partial ^0\T{x}_{j}}}_{\T{h}_{{0d}_{kj}}} \label{eq:solid:F} \\
  \dot{\T{F}}_{ij} & = & \sum_{k=1}^{N_n}\dot{\hat{\T{u}}}_{ik} \, \T{h}_{{0d}_{kj}} \label{eq:solid:Fdot} \\
  \T{h}_{0d_{ij}} & = & \frac{\partial \T{h}_i}{\partial {^0}\T{x}_{j}} = \sum_{k=1}^3 \frac{\partial \T{h}_i}{\partial \T{r}_{k}} \, \frac{\partial \T{r}_k}{\partial {^0}\T{x}_j} = \left\{\T{h}_{d}\,{^0}\T{J}^{-T}\right\}_{ij} \\
  %\T{J} & = & {^0}\T{x} \, \T{h}_d \\
  %\T{h}_{d_{ij}} & = & \frac{\partial \T{h}_i}{\partial \T{r}_j} \\
  \delta_{ij} & = & \left\{
  \begin{array}{lll}
  1 & \text{if} & i = j \cr
  0 & \text{if} & i \neq j
  \end{array}
  \right.
\end{IEEEeqnarray}
\begin{description}
\item[$\T{F}$] deformation gradient
\item[${^0}\T{J}$] Jacobian matrix $\det{\left({^0}\T{J}\right)}>0$
\item[$\T{h}_{0d}=\frac{\partial \T{h}}{\partial {^0}\T{x}}$] derivative of shape functions versus global Cartesian coordinates
\item[$\T{h}_d=\frac{\partial \T{h}}{\partial \T{r}}$] derivative of shape functions versus natural coordinates
\item[$\delta_{ij}$] Kronecker~delta
\end{description}

\paragraph{If elements become excessively distorted}
It is required that the deformation gradient $\T{F}$ and the Jacobian matrix ${^0}\T{J}$ are always invertible \cite{BATHE2016}, \cite{KUEBLER2005}.
For that reason $\det{\left(\T{F}\right)}>0$ will be checked by the solver at every iteration, and $\det{\left({^0}\T{J}\right)}>0$ will be checked
when the mesh is loaded. An exception will be thrown by the solver if any elements become excessively distorted and those conditions do not hold.

\paragraph{Strain tensor and stress tensor}
In order to derive the expressions for virtual strain energy, the Green-Lagrange strain tensor $\T{G}$ and the 2nd~Piola-Kirchhoff stress tensor $\T{S}$ are used.
It is important to note, that the 2nd~Piola-Kirchhoff stress tensor is work conjugate with the Green-Lagrange strain tensor \cite{WALLRAPP1998}, \cite{BATHE2016}, \cite{KUEBLER2005}.
Since MBDyn's solid element is pure displacement based, the stress tensor $\T{S}$ is a function of the strain tensor $\T{G}$ and optionally it's time derivative $\dot{\T{G}}$ and the strain history.
In case of viscoelastic constitutive laws, the strain rates $\dot{\T{G}}$ are scaled according to equation~\ref{eq:solid:Gdot} in order to make the effect of viscous damping independent on the strain \cite{KUEBLER2005}.
\begin{eqnarray}
\T{C}_{ij} & = & \sum_{k=1}^3\T{F}_{ki}\,\T{F}_{kj} \\
\T{G}_{ij} & = & \frac{1}{2} \, \left(\T{C}_{ij} - \delta_{ij} \right) = \frac{1}{2} \, \left(\sum_{k=1}^3\T{F}_{ki}\,\T{F}_{kj} - \delta_{ij}\right) \label{eq:solid:G} \\
\dot{\T{G}}_{ij} & = & \frac{1}{2} \, \left[\sum_{k=1}^3\left(\dot{\T{F}}_{ki}\,\T{F}_{kj} + \T{F}_{ki}\,\dot{\T{F}}_{kj} \right)\right] \\
\dot{\T{G}}^{\star} & = & \T{C}^{-1} \, \dot{\T{G}} \, \T{C}^{-1} \, \det{\left(\T{F}\right)} \label{eq:solid:Gdot} \\
\T{S} & = & \mathit{f}\left(\T{G}, \, \dot{\T{G}}^{\star}\right)
\end{eqnarray}

\begin{description}
\item[$\T{C}$] Right Cauchy-Green strain tensor
\item[$\T{G}$] Green-Lagrange strain tensor
\item[$\dot{\T{G}}^{\star}$] scaled Green-Lagrange strain rates
\item[$\T{S}$] 2nd Piola-Kirchhoff stress tensor
\end{description}

\paragraph{Constitutive laws}
The relationship between stress tensor $\T{S}$ and strain tensor $\T{G}$ is determined by a specific constitutive law.
Right now, the following types of constitutive laws are implemented:
\begin{itemize}
\item linear elastic generic
\item linear viscoelastic Kelvin-Voigt \cite{KUEBLER2005}
\item linear viscoelastic Maxwell \cite{bleyer2018numericaltours}
\item nonlinear hyperelastic Neo-Hookean \cite{KUEBLER2005}
\item nonlinear hyperelastic Mooney-Rivlin \cite{BATHE2016}
\item bilinear elasto-plastic with isotropic hardening \cite{BATHE2016}
\end{itemize}
See MBDyn's input-manual for further information about constitutive laws usable for solid elements.

\subsection{The principle of virtual displacements}
The equations of motion for displacement based Finite Element methods can be derived from the principle of virtual displacements \cite{WALLRAPP1998}, \cite{BATHE2016}, \cite{KUEBLER2005}.
\begin{eqnarray}
\underbrace{\bigcup_{e=1}^{N_e}\int_{^0V} {^0}\rho \, \left[\sum_{i=1}^3 \delta\T{u}_i\,\left(\ddot{\T{u}}_i - \T{b}_i\right) \right]\, d^0V}_{\delta ^mW} + \underbrace{\bigcup_{e=1}^{N_e}\int_{^0V} \left(\sum_{i=1}^3\sum_{j=1}^3 \delta\T{G}_{ij}\,\T{S}_{ij} \right) \, d^0V}_{\delta^iW} \nonumber \\
= \underbrace{\bigcup_{e=1}^{N_e}\int_{A} \sum_{i=1}^3 \delta\T{u}_i\,^A\T{f}_i \,dA}_{\delta^eW}
\end{eqnarray}

\begin{description}
\item[${^0}\rho$] Density of the undeformed body
\item[${^0}V$] Volume of the undeformed body
\item[$\T{b}$] Body loads due to gravity and rigid body kinematics
\item[${^A}\T{f}$] Surface loads due to pressure and surface traction's
\item[${^m}W$] Virtual work of body loads (e.g. inertia terms and gravity loads)
\item[${^i}W$] Internal virtual work (e.g. virtual strain energy)
\item[${^e}W$] External virtual work (e.g. due to surface loads ${^A}\T{f}$)
\item{$N_e$} Number of solid elements
\item[$\bigcup$] Summation over all elements
\end{description}

\subsection{Virtual strain energy}
In order to get the expression for the internal force vector at the element nodes $^i\hat{\T{k}}$, the following equivalence is used:
\begin{eqnarray}
\delta ^iW & = & \bigcup_{e=1}^{N_e} \int_{^0V} \left(\sum_{i=1}^3\sum_{j=1}^3 \delta \T{G}_{ij} \, \T{S}_{ij} \right) \, d^0V = \bigcup_{e=1}^{N_e}\sum_{k=1}^{3} \sum_{l=1}^{N_n} \delta \hat{\T{u}}_{kl} \, ^i\hat{\T{k}}_{kl} \label{eq:solid:deltaWi}
\end{eqnarray}

\paragraph{Virtual perturbation of the strain tensor}
As a first step, the virtual perturbation of the strain tensor $\delta\T{G}$ must be expressed in terms of the vector of virtual nodal displacements $\delta\hat{\T{u}}$.
For that purpose, the virtual perturbation of equation~\ref{eq:solid:G} must be derived.
\begin{eqnarray}
\delta \T{G}_{ij} & = & \frac{1}{2} \sum_{k=1}^3\left(\delta \T{F}_{ki} \, \T{F}_{kj} + \T{F}_{ki} \, \delta \T{F}_{kj}\right) \label{eq:solid:deltaG}
\end{eqnarray}

\paragraph{Virtual perturbation of the deformation gradient $\delta\T{F}$}
In the same way, the virtual perturbation of equation~\ref{eq:solid:F} is derived.
Since we are using a Total~Lagrangian~Formulation, $\T{h}_{0d}$ is a constant matrix.
\begin{eqnarray}
\delta \T{F}_{ij} & = & \sum_{k=1}^{N_n} \delta \hat{\T{u}}_{ik} \, \T{h}_{0d_{kj}} \label{eq:solid:deltaF}
\end{eqnarray}

\paragraph{Virtual perturbation of the strain tensor expressed by virtual displacements $\delta{\hat{\T{u}}}$}
Now equation~\ref{eq:solid:deltaF} is substituted into equation~\ref{eq:solid:deltaG}.
\begin{eqnarray}
\delta \T{G}_{ij} & = & \frac{1}{2} \, \sum_{k=1}^3 \sum_{l=1}^{N_n} \delta \hat{\T{u}}_{kl} \, \left( \T{h}_{0d_{li}} \, \T{F}_{kj} + \T{h}_{0d_{lj}} \, \T{F}_{ki} \right) \label{eq:solid:deltaG_F}
\end{eqnarray}

\paragraph{Virtual strain energy expressed by virtual displacements $\delta{\hat{\T{u}}}$}
Finally equation~\ref{eq:solid:deltaG_F} is substituted into equation~\ref{eq:solid:deltaWi}.
\begin{eqnarray}
\delta ^iW & = & \frac{1}{2} \bigcup_{e=1}^{N_e} \int_{^0V} \left[ \sum_{i=1}^3 \sum_{j=1}^3 \sum_{k=1}^3 \sum_{l=1}^{N_n} \delta \hat{\T{u}}_{kl}\left(\T{h}_{0d_{li}} \, \T{F}_{kj} + \T{h}_{0d_{lj}} \, \T{F}_{ki}\right) \, \T{S}_{ij} \right] \, d^0V \label{eq:solid:deltaWi_u}
\end{eqnarray}

\paragraph{Internal elastic reactions due to internal stress}
Because equation~\ref{eq:solid:deltaWi_u} and equation~\ref{eq:solid:deltaWi} must be valid for arbitrary virtual displacements $\delta\hat{\T{u}}$, we can get $^i\hat{\T{k}}$ just by comparing the coefficients of those two equations.
\begin{eqnarray}
^i\hat{\T{k}}_{kl} & = & \frac{1}{2} \, \int_{^0V} \left[ \sum_{i=1}^3 \sum_{j=1}^3 \left(\T{h}_{0d_{li}} \, \T{F}_{kj} + \T{h}_{0d_{lj}} \, \T{F}_{ki}\right) \, \T{S}_{ij} \right] \, d^0V
\end{eqnarray}

\paragraph{Numerical integration}
Finally standard numerical integration schemes (e.g. Gauss-Legendre) are applied, which sum up the weighted integrand at several integration points \cite{BATHE2016}, \cite{KUEBLER2005}.
Also in this case the values of $\det{\left({^0}\T{J}\right)}$ are constant for each integration point~$g$ because a Total Lagrangian Formulation is used.
\begin{eqnarray}
^i\hat{\T{k}}_{kl} & \approx & \frac{1}{2} \sum_{g=1}^{N_g} \left.\left[ \sum_{i=1}^3 \sum_{j=1}^3 \left(\T{h}_{0d_{li}} \, \T{F}_{kj} + \T{h}_{0d_{lj}} \, \T{F}_{ki}\right) \, \T{S}_{ij}  \, \alpha \, \det{\left({^0}\T{J}\right)}\right]\right\vert_{g}
\end{eqnarray}
\begin{description}
\item{$\alpha$} Weighting factor
\item{$N_g$} Number of integration points
\item{$\vert_g$} Expression is evaluated at integration point $g$
\end{description}

\subsection{Virtual work of inertia terms and body loads}
Also accelerations $\ddot{\hat{\T{u}}}$ and virtual displacements $\delta\hat{\T{u}}$ are interpolated by the same shape functions $\T{h}$.
\begin{eqnarray}
\delta^{m}W & = & \bigcup_{e=1}^{N_e} \int_{^0V} \sum_{i=1}^3 {^0}\rho \, \delta\T{u}_i \, \left(\ddot{\T{u}}_i - \T{b}_i\right) \, d^0V \label{eq:solid:delta_Wm} \\
\delta\T{u}_i & = & \sum_{k=1}^{N_n} \delta\hat{\T{u}}_{ik} \, \T{h}_k \label{eq:solid:delta_u} \\
\ddot{\T{u}}_i & = & \sum_{k=1}^{N_n} \ddot{\hat{\T{u}}}_{ik} \, \T{h}_k \label{eq:solid:delta_ddotu}
\end{eqnarray}

\paragraph{Body loads}
In MBDyn, body loads due to gravity $\T{g}$ and rigid body kinematics are considered. See also section~\ref{sec:nodes:structural nodes:relative frame dynamics}.
As a consequence, solid elements may be used also for typical rotor-dynamic analysis.
\begin{eqnarray}
\T{b} & = & \T{g} - \ddot{\overline{\T{x}}}_0 - \overline{\T{\omega}}_0 \times \left(\overline{\T{\omega}}_0 \times \T{x} \right) - \dot{\overline{\T{\omega}}}_0 \times \T{x} - 2 \, \overline{\T{\omega}}_0 \dot{\T{u}}
\end{eqnarray}

\begin{description}
\item[$\T{g}$] Gravity.
\item[$\ddot{\overline{\T{x}}}_0$] Acceleration of the global reference frame.
\item[$\overline{\T{\omega}}_0$]  Angular velocity of the global reference frame.
\item[$\dot{\overline{\T{\omega}}}_0$]  Angular acceleration of the global reference frame.
\end{description}

\paragraph{Mass matrix and body load vector}
When equation~\ref{eq:solid:delta_u} and equation~\ref{eq:solid:delta_ddotu} are substituted into equation~\ref{eq:solid:delta_Wm},
the virtual work $\delta^mW$ can be written in terms of the consistent mass matrix $\T{M}$ and body load vector $^b\hat{\T{f}}$.
Because of the isoparametric approach, the consistent mass matrix $\T{M}$ is constant and needs to be evaluated only once \cite{BATHE2016}, \cite{KUEBLER2005}.
In contradiction to that, the actual magnitudes of $\T{b}$ and $^b\hat{\T{f}}$ may be time dependent.
\begin{eqnarray}
\hat{\T{M}}_{kl} & = & \int_{^0V} {^0}\rho \, \T{h}_k \, \T{h}_l \, d^0V \approx \sum_{g=1}^{N_g} \left.\left[{^0}\rho \, \T{h}_k \, \T{h}_l \, \alpha \, \det{\left({^0}\T{J}\right)}\right]\right\vert_{g} \\
{^b}\hat{\T{f}}_{ik} & = & \int_{^0V} {^0}\rho \, \T{h}_k \, \T{b}_i \, d^0V \approx \sum_{g=1}^{N_g} \left.\left[{^0}\rho \, \T{h}_k \, \T{b}_i \, \alpha \, \det{\left({^0}\T{J}\right)}\right]\right\vert_{g} \\
\delta^{m}W & = & \bigcup_{e=1}^{N_e}\left(\sum_{i=1}^3\sum_{k=1}^{N_n}\sum_{l=1}^{N_n} \delta\hat{\T{u}}_{ik} \, \ddot{\hat{\T{u}}}_{il} \, \hat{\T{M}}_{kl}
 - \sum_{i=1}^3\sum_{k=1}^{N_n} \delta\hat{\T{u}}_{ik} \, {^b}\hat{\T{f}}_{ik}\right) = \bigcup_{e=1}^{N_e} \delta\hat{\T{u}}^T \, \left(\T{M} \, \ddot{\hat{\T{u}}} - {^b}\hat{\T{f}} \right)
\end{eqnarray}
\begin{description}
\item[$\hat{\T{M}}$] $N_n\times N_n$ element mass matrix in compact storage
\item[$\T{M}$] $3 N_n \times 3 N_n$ element mass matrix in redundant storage (e.g. $\T{M}_{iijk} = \hat{\T{M}}_{jk}$)
\item[${^b}\hat{\T{f}}$] Element body load vector
\end{description}

\paragraph{Lumped mass matrix}
In addition to a consistent mass matrix, also a lumped mass matrix can be used in MBDyn.
For that purpose, the diagonal of the consistent mass matrix is scaled, so that the overall mass is conserved.
\begin{eqnarray}
\hat{\T{M}}_{kk}^{\star} & = & \int_{^0V} {^0}\rho \, \T{h}_k^2 \, d^0V \approx \sum_{g=1}^{N_g} \left.\left[{^0}\rho \, \T{h}_k^2 \, \alpha \, \det{\left({^0}\T{J}\right)}\right]\right\vert_{g} \\
\hat{\T{M}}_{kk}&=&\hat{\T{M}}_{kk}^{\star} \, \frac{\,m}{\sum_{k=1}^{N_n}\hat{\T{M}}_{kk}^{\star}} \\
m&=&\int_{^0V}{^0}\rho\,d{^0V} \approx \sum_{g=1}^{N_g} \left.\left[{^0}\rho \, \alpha \, \det{\left(^0\T{J}\right)}\right]\right\vert_{g}
\end{eqnarray}
If a lumped mass matrix is used, the virtual work of the inertia terms becomes:
\begin{eqnarray}
\delta^{m}W & = & \bigcup_{e=1}^{N_e}\left(\sum_{i=1}^3\sum_{k=1}^{N_n}\delta\hat{\T{u}}_{ik} \, \ddot{\hat{\T{u}}}_{ik} \, \hat{\T{M}}_{kk}
 - \sum_{i=1}^3\sum_{k=1}^{N_n} \delta\hat{\T{u}}_{ik} \, {^b}\hat{\T{f}}_{ik}\right)
\end{eqnarray}

\subsection{Reformulation of inertia terms required for a first order DAE solver}
Since MBDyn does not use accelerations for the solution of equations of motion, a reformulation of the equations of motion is required for solid elements.
It is required also because rigid bodies might be present in the same model, and those rigid bodies might share common nodes with solid elements.
The principle of virtual displacements leads to the following expression, which is using accelerations $\ddot{\hat{\T{u}}}$.
So, it cannot be implemented directly.
\begin{eqnarray}
\bigcup_{e=1}^{N_e} \left(\T{M} \, \ddot{\hat{\T{u}}} + {^i}\hat{\T{k}} - {^b}\hat{\T{f}}\right) = \T{0} \label{eq:solid:global_eq1}
\end{eqnarray}

\paragraph{Introducing the momentum}
Since the mass matrix $\T{M}$ is constant,
equation~\ref{eq:solid:global_eq1} can be easily reformulated as a system of two first order ODE's
using the momentum $\hat{\T{\beta}}$ as a new unknown.
This approach is strictly consistent with MBDyn's conventions for rigid bodies.
\begin{eqnarray}
\hat{\T{\beta}}=\T{M}\,\dot{\hat{\T{u}}} \label{eq:solid:beta} \\
\dot{\hat{\T{\beta}}}=\T{M}\,\ddot{\hat{\T{u}}} \label{eq:solid:betadot}
\end{eqnarray}

\begin{description}
\item[$\hat{\T{\beta}}$] Momentum at element nodes
\end{description}

With equation~\ref{eq:solid:beta} and equation~\ref{eq:solid:betadot} the global system of equations is formulated as:
\begin{eqnarray}
\bigcup_{e=1}^{N_e} \left(\hat{\T{\beta}} - \T{M} \, \dot{\hat{\T{u}}}\right) = \T{0} \\
-\bigcup_{e=1}^{N_e} \left(\dot{\hat{\T{\beta}}} + {^i}\hat{\T{k}} - {^b}\hat{\T{f}} \right) = \T{0}
\end{eqnarray}

\paragraph{Limitations}
The only remaining limitation of this approach is, that it is not possible to compute accelerations
for a model using solid elements, unless a lumped mass matrix is used for all solid elements in the model.
See the MBDyn's input-manual for further information.

\subsection{Elements for surface loads}
In order to apply surface loads at solid elements, dedicated surface elements are required.
Basically the same isoparametric approach known from section~\ref{sec:solid:interpol} is applied also to surface elements.
Instead of the determinant of the Jacobian matrix, equation~\ref{eq:surface:dA} is used to transform infinitesimal surface areas
between natural curvilinear coordinates $\T{r}$ and global Cartesian coordinates $\T{x}$.
\begin{eqnarray}
\T{x}_i & = & \sum_{k=1}^{N_n} \T{h}_k \, \hat{\T{x}}_{ik} \\
\T{u}_i & = & \sum_{k=1}^{N_n} \T{h}_k \, \hat{\T{u}}_{ik} \\
dA & = & \left\Vert\frac{\partial \T{x}}{\partial \T{r}_1} \times \frac{\partial \T{x}}{\partial \T{r}_2} \right\Vert \,d\T{r}_1 \, d\T{r}_2 \label{eq:surface:dA} \\
\frac{\partial \T{x}_i}{\partial \T{r}_1} & = &\sum_{k=1}^{N_n} \frac{\partial \T{h}_{k}}{\partial \T{r}_1} \, \hat{\T{x}}_i \\
\frac{\partial \T{x}_i}{\partial \T{r}_2} & = &\sum_{k=1}^{N_n} \frac{\partial \T{h}_{k}}{\partial \T{r}_2} \, \hat{\T{x}}_i \\
\T{r}=\begin{pmatrix} \T{r}_1 & \T{r}_2 \end{pmatrix}^T
\end{eqnarray}
\paragraph{Shape functions}
Also surface elements are based on C++ templates in order to simplify the implementation of new element types.
The following types of elements are currently implemented:
\begin{itemize}
\item Quadrangular elements with linear and quadratic shape functions \cite{BATHE2016}, \cite{DHONDT2004}
\item Triangular elements with quadratic shape functions \cite{CODEASTERR30301}
\end{itemize}
See also MBDyn's input-manual for further information.

\paragraph{Virtual work of surface loads}
In order to derive the expressions for external virtual work, surface loads must be expressed in terms of
virtual displacements $\delta\hat{\T{u}}$ at the element nodes.
\begin{eqnarray}
\delta{^e}W & = & \bigcup_{e=1}^{N_e}\int_{A} \left(\sum_{i=1}^3 \delta\T{u}_i\,{^A}\T{f}_i \right)\,dA \\
          & = & \bigcup_{e=1}^{N_e}\int_{A} \left(\sum_{i=1}^3 \sum_{k=1}^{N_n} \delta\hat{\T{u}}_{ik} \, \T{h}_{k} \, {^A}\T{f}_i \right) \, dA
\end{eqnarray}
\paragraph{Pressure loads}
In case of pressure loads, ${^A}\T{f}$ will always act normal to the surface.
It should be emphasized, that the surface normal- and tangent vectors must be evaluated with respect to the deformed state.
\begin{eqnarray}
{^A}\T{f} & = & -\frac{1}{\left\Vert{^A}\T{n}\right\Vert} \, {^A}\T{n} \, p \\
p & = & \sum_{k=1}^{N_n} \T{h}_{k} \, \hat{\T{p}}_k \\
{^A}\T{n} & = & \frac{\partial \T{x}}{\partial \T{r}_1} \times \frac{\partial \T{x}}{\partial \T{r}_2}
%{^A}\T{n}_{1_i} & = & \frac{\partial \T{x}_i}{\partial \T{r}_1}  = \sum_{k=1}^{N_n} \frac{\partial \T{h}_{k}}{\partial \T{r}_1} \, \hat{\T{x}}_i \\
%{^A}\T{n}_{2_i} & = & \frac{\partial \T{x}_i}{\partial \T{r}_2}  = \sum_{k=1}^{N_n} \frac{\partial \T{h}_{k}}{\partial \T{r}_2} \, \hat{\T{x}}_i \\
%dA & = & \left\Vert{^A}\T{n}^{\star}\right\Vert \,d\T{r}_1 \, d\T{r}_2= \left\Vert{^A}\T{n}_1 \times {^A}\T{n}_2 \right\Vert \,d\T{r}_1 \, d\T{r}_2
\end{eqnarray}
\begin{description}
\item[$p$] pressure at a point $\T{x}$ at the surface
\item[$\hat{\T{p}}$] pressures at the element nodes
\item[${^A}\T{n}$] outward surface normal vector at point $\T{x}$
\end{description}
Finally, the virtual work of pressure loads can be evaluated by numerical integration across the surface area.
Due to the fact, that the surface may be moved or deformed during the simulation, pressure loads will not be constant,
even if the nodal pressures are constant values. In addition to that, pressure loads do contribute to the global stiffness matrix,
and they will change the natural frequencies of structures they are applied to.
\begin{eqnarray}
\delta{^e}W & = & -\bigcup_{e=1}^{N_e}\int_{A} \left(\sum_{i=1}^3 \sum_{k=1}^{N_n} \delta\hat{\T{u}}_{ik} \, \T{h}_{k} \, {^A}\T{n}_i \, \frac{p}{\left\Vert{^A}\T{n}\right\Vert}\right) \, dA \\
 & \approx & -\bigcup_{e=1}^{N_e}\sum_{g=1}^{N_g} \left.\left(\sum_{i=1}^3 \sum_{k=1}^{N_n} \delta\hat{\T{u}}_{ik} \, \T{h}_{k} \, p \, {^A}\T{n}_i \, \alpha \right)\right\vert_{g}
\end{eqnarray}
\paragraph{Surface traction's}
Surface traction's are similar to pressure loads, but they do not necessarily act normal to the surface.
Instead the initial orientation of traction loads will be specified by the user, by means of an orientation matrix $\T{R}_f$.
In that way, shear stresses may be imposed at the surface of a body.
\begin{eqnarray}
\T{e}_1^{\star} & = & \frac{\partial \T{x}}{\partial \T{r}_1} \\
\T{e}_2^{\star} & = & \frac{\partial \T{x}}{\partial \T{r}_2} \\
\T{e}_3^{\star} & = & \T{e}_1^{\star} \times \T{e}_2^{\star} \\
\T{e}_2^{\star\star} & = & \T{e}_3^{\star} \times \T{e}_1^{\star} \\
{^A}\T{R} & = & \begin{pmatrix}
\frac{1}{\left\Vert\T{e}_1^{\star}\right\Vert} \, \T{e}_1^{\star} & \frac{1}{\left\Vert\T{e}_2^{\star\star}\right\Vert} \, \T{e}_2^{\star\star} & \frac{1}{\left\Vert\T{e}_3^{\star}\right\Vert} \, \T{e}_3^{\star}
\end{pmatrix} \\
{^A}\T{f} & = & {^A}\T{R} \, {^{A_0}}\T{R}^T \, \T{R}_f \, {^A}\bar{\T{f}} \\
{^A}\bar{\T{f}}_i & = & \sum_{k=1}^{N_n} \T{h}_{k} \, {^A}\bar{\hat{\T{f}}}_{ik} \\
\delta{^e}W & = & \bigcup_{e=1}^{N_e}\int_{A} \left(\sum_{i=1}^3 \sum_{k=1}^{N_n} \delta\hat{\T{u}}_{ik} \, \T{h}_{k} \, {^A}\T{f}_i \right) \, dA \\
 & \approx & \bigcup_{e=1}^{N_e}\sum_{g=1}^{N_g} \left.\left(\sum_{i=1}^3 \sum_{k=1}^{N_n} \delta\hat{\T{u}}_{ik} \, \T{h}_{k} \, {^A}\T{f}_i \, \left\Vert \T{e}_3^{\star} \right\Vert \, \alpha \right)\right\vert_{g}
\end{eqnarray}
\begin{description}
\item[${^A}\T{R}$] relative orientation matrix at point $\T{x}$ at the deformed state
\item[${^{A_0}}\T{R}$] relative orientation matrix at point $\T{x}$ at the undeformed state
\item[$\T{R}_f$] absolute orientation matrix for the applied traction load with respect to the undeformed state
\item[${^A}\T{f}$] surface traction's at point $\T{x}$ with respect to the global reference frame
\item[${^A}\bar{\T{f}}$] surface traction's at point $\T{x}$ with respect to the local reference frame
\item[${^A}\bar{\hat{\T{f}}}$] surface traction's at element nodes
\end{description}

\subsection{Implementation notes}
Many authors of textbooks about nonlinear Finite Element Methods spent
a lot of effort to explain the consistent derivation of the tangent stiffness matrix,
since this is crucial for any Newton-like nonlinear solver \cite{WALLRAPP1998}, \cite{BATHE2016}.
In contradiction to that, the analytical derivation of a sparse tangent stiffness matrix is not required for MBDyn's solid elements,
because they are exploiting a technique called ``Automatic~Differentiation'' for Newton~based- as well as Newton~Krylov based solvers.
