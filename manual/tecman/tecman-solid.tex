% MBDyn (C) is a multibody analysis code.
% http://www.mbdyn.org
%
% Copyright (C) 1996-2023
%
% Pierangelo Masarati  <pierangelo.masarati@polimi.it>
%
% Dipartimento di Ingegneria Aerospaziale - Politecnico di Milano
% via La Masa, 34 - 20156 Milano, Italy
% http://www.aero.polimi.it
%
% Changing this copyright notice is forbidden.
%
% This program is free software; you can redistribute it and/or modify
% it under the terms of the GNU General Public License as published by
% the Free Software Foundation (version 2 of the License).
%
%
% This program is distributed in the hope that it will be useful,
% but WITHOUT ANY WARRANTY; without even the implied warranty of
% MERCHANTABILITY or FITNESS FOR A PARTICULAR PURPOSE.  See the
% GNU General Public License for more details.
%
% You should have received a copy of the GNU General Public License
% along with this program; if not, write to the Free Software
% Foundation, Inc., 59 Temple Place, Suite 330, Boston, MA  02111-1307  USA
\emph{Author: Reinhard Resch} \\
Solid elements are intended to model general three dimensional structures with complex shapes,
subject to large deformations, large strain and nonlinear constitutive laws.
Even fully incompressible constitutive laws are supported.
\paragraph{Files.} \
It is implemented in files\\
\begin{tabular}{l}
\texttt{mbdyn/struct/solid.h} \\
\texttt{mbdyn/struct/solid.cc} \\
\texttt{mbdyn/struct/solidshape.h} \\
\texttt{mbdyn/struct/solidshape.cc} \\
\texttt{mbdyn/struct/solidshapetest.cc} \\
\texttt{mbdyn/struct/solidinteg.h} \\
\texttt{mbdyn/struct/solidinteg.cc} \\
\texttt{mbdyn/struct/solidpress.h} \\
\texttt{mbdyn/struct/solidpress.cc} \\
\texttt{mbdyn/struct/solidcsl.h} \\
\texttt{mbdyn/struct/solidcsl.cc}
\end{tabular}
\subsection{Kinematics}
\label{sec:solid:interpol}
MBDyn's solid elements are based on the classical displacement based isoparametric Finite Element formulation \cite{BATHE2016}.
For enhancements, required for nearly or fully incompressible constitutive laws, see section~\ref{sec:solid:upc}.
\paragraph{Interpolation}
The three dimensional global Cartesian coordinates $\T{x}$ and the displacements $\T{u}=\T{x} - {^0}\T{x}$, of a particle inside an element,
are interpolated between coordinates $\hat{\T{x}}$ and displacements $\hat{\T{u}}$ at element nodes by means of shape functions $\T{h}=\mathit{f}\left(\T{r}\right)$ \cite{BATHE2016}, \cite{KUEBLER2005}.
\begin{eqnarray}
  \T{u}_{i} &=& \sum_{k=1}^{N_n}\hat{\T{u}}_{ik} \,\T{h}_{k} \label{eq:solid:u:interpol} \\
  \T{x}_{i} &=& \sum_{k=1}^{N_n}\hat{\T{x}}_{ik} \,\T{h}_{k} \\
  \T{u} &=& \begin{pmatrix} \T{u}_1 & \T{u}_2 & \T{u}_3 \end{pmatrix}^T \\
  \T{x} &=& \begin{pmatrix} \T{x}_1 & \T{x}_2 & \T{x}_3 \end{pmatrix}^T \\
  \T{r} &=& \begin{pmatrix} \T{r}_1 & \T{r}_2 & \T{r}_3 \end{pmatrix}^T \\
  \hat{\T{u}} &=& \begin{pmatrix}
    \hat{\T{u}}_{1,1} & \hat{\T{u}}_{1,2} & \ldots{} & \hat{\T{u}}_{1,N_n} \\
    \hat{\T{u}}_{2,1} & \hat{\T{u}}_{2,2} & \ldots{} & \hat{\T{u}}_{2,N_n} \\
    \hat{\T{u}}_{3,1} & \hat{\T{u}}_{2,2} & \ldots{} & \hat{\T{u}}_{3,N_n}
  \end{pmatrix} \\
  \T{h} & = & \begin{pmatrix}
    \T{h}_1 & \T{h}_2 & \ldots{} & \T{h}_{N_n}
  \end{pmatrix}^T
\end{eqnarray}

\begin{description}
\item[$\T{x}=\mathit{f}\left(\T{r}\right)$] global Cartesian coordinates of a particle inside the deformed body
\item[$^0\T{x}=\mathit{f}\left(\T{r}\right)$] global Cartesian coordinates of a particle inside the undeformed body
\item[$\T{u}=\T{x}-{^0}\T{x}$] deformation of a particle within the body
\item[$\T{r}$] curvilinear natural coordinates of a particle within the body
\item[$\T{h}=\mathit{f}\left(\T{r}\right)$] shape functions of a particular Finite Element type
\item[$N_n$] number of nodes per element
\end{description}

\paragraph{Shape functions}
All solid elements in MBDyn are based on templates and are using standard shape functions based on \cite{BATHE2016}, \cite{DHONDT2004} and \cite{CODEASTERR30301}.
In order to implement a new solid~element type in MBDyn, it is sufficient to provide a new C++ class which defines it's shape functions $\T{h}$,
derivatives of the shape functions versus the natural coordinates $\T{h}_d=\frac{\partial \T{h}}{\partial \T{r}}$ and also the integration points $\left.\T{r}\right\vert_g$ and weights $\left.\alpha\right\vert_g$.
The following types of elements are currently implemented:
\begin{itemize}
\item Hexahedrons with linear and quadratic shape functions \cite{BATHE2016}, \cite{DHONDT2004}
\item Pentahedrons with quadratic shape functions \cite{CODEASTERR30301}
\item Tetrahedrons with quadratic shape functions \cite{CODEASTERR30301}
\end{itemize}
See also MBDyn's \emph{input-manual} for further information.

\paragraph{Deformation gradient}
Because MBDyn's solid elements are based on a Total~Lagrangian~Formulation, the deformation gradient $\T{F}$ is evaluated with respect to the undeformed state ${^0}\T{x}$.
As a consequence, the Jacobian matrix ${^0}\T{J}$ and the gradient operator $\T{h}_{0d}$ are constant.
\begin{IEEEeqnarray}{rCl}
  {^0}\T{J}_{ij} & = & \frac{\partial{^0}\T{x}_j}{\partial \T{r}_i} = \sum_{k=1}^{N_n} {^0}\hat{\T{x}}_{jk} \, \underbrace{\frac{\partial \T{h}_{k}}{\partial \T{r}_i}}_{\T{h}_{d_{ki}}} = \left({^0}\T{x}\,\T{h}_d\right)^T\\
  \left\{{^0}\T{J}^{-1}\right\}_{ij} & = & \frac{\partial\T{r}_j}{\partial {^0}\T{x}_i} \\
  %\T{h}_{d_{ij}} & = & \frac{\partial \T{h}_i}{\partial \T{r}_j} \\
  \T{F}_{ij} & = & \frac{\partial \T{x}_{i}}{\partial {^0}\T{x}_{j}} = \delta_{ij} + \frac{\partial \T{u}_i}{\partial ^0\T{x}_{j}} = \delta_{ij} + \sum_{k=1}^{N_n}\hat{\T{u}}_{ik} \, \underbrace{\frac{\partial \T{h}_k}{\partial ^0\T{x}_{j}}}_{\T{h}_{{0d}_{kj}}} \label{eq:solid:F} \\
  \dot{\T{F}}_{ij} & = & \sum_{k=1}^{N_n}\dot{\hat{\T{u}}}_{ik} \, \T{h}_{{0d}_{kj}} \label{eq:solid:Fdot} \\
  \T{h}_{0d_{ij}} & = & \frac{\partial \T{h}_i}{\partial {^0}\T{x}_{j}} = \sum_{k=1}^3 \frac{\partial \T{h}_i}{\partial \T{r}_{k}} \, \frac{\partial \T{r}_k}{\partial {^0}\T{x}_j} = \left\{\T{h}_{d}\,{^0}\T{J}^{-T}\right\}_{ij} \\
  %\T{J} & = & {^0}\T{x} \, \T{h}_d \\
  %\T{h}_{d_{ij}} & = & \frac{\partial \T{h}_i}{\partial \T{r}_j} \\
  \delta_{ij} & = & \left\{
  \begin{array}{lll}
  1 & \text{if} & i = j \cr
  0 & \text{if} & i \neq j
  \end{array}
  \right.
\end{IEEEeqnarray}
\begin{description}
\item[$\T{F}$] deformation gradient
\item[${^0}\T{J}$] Jacobian matrix $\det{\left({^0}\T{J}\right)}>0$
\item[$\T{h}_{0d}=\frac{\partial \T{h}}{\partial {^0}\T{x}}$] derivative of shape functions versus global Cartesian coordinates
\item[$\T{h}_d=\frac{\partial \T{h}}{\partial \T{r}}$] derivative of shape functions versus natural coordinates
\item[$\delta_{ij}$] Kronecker~delta
\end{description}

\paragraph{If elements become excessively distorted}
It is required that the deformation gradient $\T{F}$ and the Jacobian matrix ${^0}\T{J}$ are always invertible \cite{BATHE2016}, \cite{KUEBLER2005}.
For that reason $\det{\left(\T{F}\right)}>0$ will be checked by the solver at every iteration, and $\det{\left({^0}\T{J}\right)}>0$ will be checked
once, when the mesh is loaded. An exception will be thrown by the solver, if any element becomes excessively distorted and those conditions are not valid any more.

\paragraph{Strain tensor and stress tensor}
In order to derive the expressions for virtual strain energy, the Green-Lagrange strain tensor $\T{G}$ and the 2nd~Piola-Kirchhoff stress tensor $\T{S}$ are used.
It is important to note, that the 2nd~Piola-Kirchhoff stress tensor is work conjugate with the Green-Lagrange strain tensor \cite{WALLRAPP1998}, \cite{BATHE2016}, \cite{KUEBLER2005}.
Since MBDyn's solid element is pure displacement based, the stress tensor $\T{S}$ is a function of the strain tensor $\T{G}$ and optionally it's time derivative $\dot{\T{G}}$ and the strain history.
In case of viscoelastic constitutive laws, the strain rates $\dot{\T{G}}$ are scaled according to equation~\ref{eq:solid:Gdot} in order to make the effect of viscous damping independent on the strain \cite{KUEBLER2005}.
\begin{eqnarray}
\T{C}_{ij} & = & \sum_{k=1}^3\T{F}_{ki}\,\T{F}_{kj} \\
\T{G}_{ij} & = & \frac{1}{2} \, \left(\T{C}_{ij} - \delta_{ij} \right) = \frac{1}{2} \, \left(\sum_{k=1}^3\T{F}_{ki}\,\T{F}_{kj} - \delta_{ij}\right) \label{eq:solid:G} \\
\dot{\T{G}}_{ij} & = & \frac{1}{2} \, \left[\sum_{k=1}^3\left(\dot{\T{F}}_{ki}\,\T{F}_{kj} + \T{F}_{ki}\,\dot{\T{F}}_{kj} \right)\right] \\
\dot{\T{G}}^{\star} & = & \T{C}^{-1} \, \dot{\T{G}} \, \T{C}^{-1} \, \det{\left(\T{F}\right)} \label{eq:solid:Gdot} \\
\T{S} & = & \mathit{f}\left(\T{G}, \, \dot{\T{G}}^{\star}\right)
\end{eqnarray}

\begin{description}
\item[$\T{C}$] Right Cauchy-Green strain tensor
\item[$\T{G}$] Green-Lagrange strain tensor
\item[$\dot{\T{G}}^{\star}$] scaled Green-Lagrange strain rates
\item[$\T{S}$] 2nd Piola-Kirchhoff stress tensor
\end{description}

\paragraph{Constitutive laws}
The relationship between stress tensor $\T{S}$ and strain tensor $\T{G}$ is determined by a specific constitutive law.
Right now, the following types of constitutive laws are implemented:
\begin{itemize}
\item linear elastic generic (isotropic or anisotropic)
\item linear Kelvin-Voigt viscoelastic generic (isotropic or anisotropic) \cite{KUEBLER2005}
\item Hookean linear elastic isotropic
\item Hookean linear Kelvin-Voigt viscoelastic isotropic
\item linear viscoelastic generalized Maxwell (isotropic or anisotropic) \cite{bleyer2018numericaltours}
\item nonlinear hyperelastic Neo-Hookean \cite{KUEBLER2005}
\item nonlinear Kelvin-Voigt viscoelastic Neo-Hookean \cite{KUEBLER2005}
\item nonlinear hyperelastic Mooney-Rivlin \cite{BATHE2016}
\item bilinear elasto-plastic with isotropic hardening \cite{BATHE2016}
\end{itemize}
See MBDyn's \emph{input-manual} for further information about constitutive laws usable for solid elements.

\subsection{The principle of virtual displacements}
The equations of motion for displacement based Finite Element methods can be derived from the principle of virtual displacements \cite{WALLRAPP1998}, \cite{BATHE2016}, \cite{KUEBLER2005}.
\begin{eqnarray}
\underbrace{\bigcup_{e=1}^{N_e}\int_{^0V} {^0}\rho \, \left[\sum_{i=1}^3 \delta\T{u}_i\,\left(\ddot{\T{u}}_i - \T{b}_i\right) \right]\, d^0V}_{\delta ^mW} + \underbrace{\bigcup_{e=1}^{N_e}\int_{^0V} \left(\sum_{i=1}^3\sum_{j=1}^3 \delta\T{G}_{ij}\,\T{S}_{ij} \right) \, d^0V}_{\delta^iW} \nonumber \\
= \underbrace{\bigcup_{e=1}^{N_e}\int_{A} \sum_{i=1}^3 \delta\T{u}_i\,^A\T{f}_i \,dA}_{\delta^eW} \label{eq:solid:pdVA}
\end{eqnarray}

\begin{description}
\item[${^0}\rho$] Density of the undeformed body
\item[${^0}V$] Volume of the undeformed body
\item[$\T{b}$] Body loads due to gravity and rigid body kinematics
\item[${^A}\T{f}$] Surface loads due to pressure and surface traction's
\item[${^m}W$] Virtual work of body loads (e.g. inertia terms and gravity loads)
\item[${^i}W$] Internal virtual work (e.g. virtual strain energy)
\item[${^e}W$] External virtual work (e.g. due to surface loads ${^A}\T{f}$)
\item{$N_e$} Number of solid elements
\item[$\bigcup$] Summation over all elements
\end{description}

\subsection{Virtual strain energy}
In order to get the expression for the internal force vector at the element nodes $^i\hat{\T{k}}$, the following equivalence is used:
\begin{eqnarray}
\delta ^iW & = & \bigcup_{e=1}^{N_e} \int_{^0V} \left(\sum_{i=1}^3\sum_{j=1}^3 \delta \T{G}_{ij} \, \T{S}_{ij} \right) \, d^0V = \bigcup_{e=1}^{N_e}\sum_{k=1}^{3} \sum_{l=1}^{N_n} \delta \hat{\T{u}}_{kl} \, ^i\hat{\T{k}}_{kl} \label{eq:solid:deltaWi}
\end{eqnarray}

\paragraph{Virtual perturbation of the strain tensor}
As a first step, the virtual perturbation of the strain tensor $\delta\T{G}$ must be expressed in terms of the vector of virtual nodal displacements $\delta\hat{\T{u}}$.
For that purpose, the virtual perturbation of equation~\ref{eq:solid:G} must be derived.
\begin{eqnarray}
\delta \T{G}_{ij} & = & \frac{1}{2} \sum_{k=1}^3\left(\delta \T{F}_{ki} \, \T{F}_{kj} + \T{F}_{ki} \, \delta \T{F}_{kj}\right) \label{eq:solid:deltaG}
\end{eqnarray}

\paragraph{Virtual perturbation of the deformation gradient $\delta\T{F}$}
In the same way, the virtual perturbation of equation~\ref{eq:solid:F} is derived.
Since we are using a Total~Lagrangian~Formulation, $\T{h}_{0d}$ is a constant matrix.
\begin{eqnarray}
\delta \T{F}_{ij} & = & \sum_{k=1}^{N_n} \delta \hat{\T{u}}_{ik} \, \T{h}_{0d_{kj}} \label{eq:solid:deltaF}
\end{eqnarray}

\paragraph{Virtual perturbation of the strain tensor expressed by virtual displacements $\delta{\hat{\T{u}}}$}
Now equation~\ref{eq:solid:deltaF} is substituted into equation~\ref{eq:solid:deltaG}.
\begin{eqnarray}
\delta \T{G}_{ij} & = & \frac{1}{2} \, \sum_{k=1}^3 \sum_{l=1}^{N_n} \delta \hat{\T{u}}_{kl} \, \left( \T{h}_{0d_{li}} \, \T{F}_{kj} + \T{h}_{0d_{lj}} \, \T{F}_{ki} \right) \label{eq:solid:deltaG_F}
\end{eqnarray}

\paragraph{Virtual strain energy expressed by virtual displacements $\delta{\hat{\T{u}}}$}
Finally equation~\ref{eq:solid:deltaG_F} is substituted into equation~\ref{eq:solid:deltaWi}.
\begin{eqnarray}
\delta ^iW & = & \frac{1}{2} \bigcup_{e=1}^{N_e} \int_{^0V} \left[ \sum_{i=1}^3 \sum_{j=1}^3 \sum_{k=1}^3 \sum_{l=1}^{N_n} \delta \hat{\T{u}}_{kl}\left(\T{h}_{0d_{li}} \, \T{F}_{kj} + \T{h}_{0d_{lj}} \, \T{F}_{ki}\right) \, \T{S}_{ij} \right] \, d^0V \label{eq:solid:deltaWi_u}
\end{eqnarray}

\paragraph{Internal elastic reactions due to internal stress}
Because equation~\ref{eq:solid:deltaWi_u} and equation~\ref{eq:solid:deltaWi} must be valid for arbitrary virtual displacements $\delta\hat{\T{u}}$, we can get $^i\hat{\T{k}}$ just by comparing the coefficients of those two equations.
\begin{eqnarray}
^i\hat{\T{k}}_{kl} & = & \frac{1}{2} \, \int_{^0V} \left[ \sum_{i=1}^3 \sum_{j=1}^3 \left(\T{h}_{0d_{li}} \, \T{F}_{kj} + \T{h}_{0d_{lj}} \, \T{F}_{ki}\right) \, \T{S}_{ij} \right] \, d^0V \label{eq:solid:ki}
\end{eqnarray}

\paragraph{Numerical integration}
Finally standard numerical integration schemes (e.g. Gauss-Legendre) are applied, which sum up the weighted integrand at several integration points \cite{BATHE2016}, \cite{KUEBLER2005}.
Also in this case the values of $\det{\left({^0}\T{J}\right)}$ for each integration point~$g$ are constant with respect to time, because a Total Lagrangian Formulation is used.
\begin{eqnarray}
^i\hat{\T{k}}_{kl} & \approx & \frac{1}{2} \sum_{g=1}^{N_g} \left.\left[ \sum_{i=1}^3 \sum_{j=1}^3 \left(\T{h}_{0d_{li}} \, \T{F}_{kj} + \T{h}_{0d_{lj}} \, \T{F}_{ki}\right) \, \T{S}_{ij}  \, \alpha \, \det{\left({^0}\T{J}\right)}\right]\right\vert_{g}
\end{eqnarray}
\begin{description}
\item{$\alpha$} Weighting factor
\item{$N_g$} Number of integration points
\item{$\vert_g$} Expression is evaluated at integration point $g$
\end{description}

\subsection{Displacement/Pressure (u/p-c) formulation}
\label{sec:solid:upc}
In case of nearly incompressible constitutive laws, pure displacement based formulations are not effective.
For that reason, it is necessary to introduce the hydrostatic pressure $\tilde{p}$ as an additional unknown.
For a detailed discussion see also \cite{BATHE2016} section~4.4.3.
In order to derive the u/p-c formulation for large displacements, the following modified strain energy potential
per unit volume is assumed for isotropic materials \cite{BATHE2016}:
\begin{eqnarray}
^i\mathbb{W} = ^i\bar{\mathbb{W}} - \frac{1}{2\kappa}\left(\bar{p} - \tilde{p}\right)^2 \label{eq:solid:upc:W}
\end{eqnarray}
See also \cite{BATHE2016} section~6.4.1, equation~6.136.

\paragraph{Strain energy potential per unit volume}
Within this section, $^i\mathbb{W}$ denotes strain energy potential per unit volume, whereas $^iW$ denotes overall strain energy potential.
They are related as follows:
\begin{eqnarray}
^iW = \bigcup_{e=1}^{N_e}\int_{^0V} {}^i\mathbb{W} \, d^0V
\end{eqnarray}
\begin{description}
\item[$^iW$] modified strain energy potential for the u/p-c formulation
\item[$^i\bar{W}$] strain energy potential for the pure displacement based formulation which is equal to equation~\ref{eq:solid:deltaWi}
\item[$^i\mathbb{W}$] modified strain energy potential per unit volume for the u/p-c formulation
\item[$^i\bar{\mathbb{W}}$] strain energy potential per unit volume for the pure displacement based formulation equivalent to equation~\ref{eq:solid:deltaWi}
\item[$\bar{p}$] hydrostatic pressure obtained from the pure displacement based formulation
\item[$\tilde{p}$] hydrostatic pressure interpolated from the element nodes
\item[$\kappa$] bulk modulus of the material
\end{description}
In order to enhance the principle of virtual-displacements equation~\ref{eq:solid:pdVA},
the virtual perturbation of the modified strain-energy potential $^i\mathbb{W}$ is derived
with respect to the new unknowns (displacement $\hat{\T{u}}$ and hydrostatic pressure $\hat{\T{p}}$ at element nodes).
This is equivalent to the application of the ``Hu-Washizu'' variational-principle. See also \cite{BATHE2016} section~4.4.3.
\begin{eqnarray}
\delta^i\mathbb{W} & = & \delta^i\bar{\mathbb{W}} - \frac{1}{\kappa}\left(\bar{p} - \tilde{p}\right)\left(\delta\bar{p} - \delta\tilde{p}\right) \label{eq:solid:upc:deltaW}
\end{eqnarray}
As a next step, the virtual perturbations of the hydrostatic pressure $\bar{p}$ and $\tilde{p}$ are derived.
\begin{eqnarray}
\delta\bar{p} & = & \sum_{i=1}^{3}\sum_{j=1}^{3}\frac{\partial \bar{p}}{\partial \boldsymbol{G}_{ij}}\,\delta\boldsymbol{G}_{ij} \label{eq:solid:upc:delta:p:bar} \\
\delta\tilde{p} & = & \sum_{i=1}^{N_p}\frac{\partial \tilde{p}}{\partial \hat{\T{p}}_i}\,\delta\hat{\T{p}}_i \label{eq:solid:upc:delta:p:tilde}
\end{eqnarray}
\begin{description}
\item[$N_p$] number element nodes for hydrostatic pressure
\end{description}
The displacement based hydrostatic pressure $\bar{p}$ is depending only on the strain tensor $\T{G}$
whereas the hydrostatic pressure $\tilde{p}$ is depending only on the assumed hydrostatic pressure at element nodes $\hat{\T{p}}$.
\paragraph{Reformulation in terms of the modified stress tensor}
In the following step, equation~\ref{eq:solid:upc:delta:p:bar} and equation~\ref{eq:solid:upc:delta:p:tilde} are substituted into equation~\ref{eq:solid:upc:deltaW}.
\begin{eqnarray}
\delta^i\mathbb{W} & = &\delta^i\bar{\mathbb{W}} - \frac{1}{\kappa}\left(\bar{p} - \tilde{p}\right)\left(\sum_{i=1}^{3}\sum_{j=1}^{3}\frac{\partial \bar{p}}{\partial \boldsymbol{G}_{ij}}\,\delta\boldsymbol{G}_{ij} - \sum_{i=1}^{N_p}\frac{\partial \tilde{p}}{\partial \hat{\T{p}}_i}\,\delta\hat{\T{p}}_i\right) \label{eq:solid:upc:deltaW:Gp}
\end{eqnarray}
Furthermore we can use the expression for the virtual perturbation of the displacement based strain energy from equation~\ref{eq:solid:deltaWi}.
Within this section, $\bar{\boldsymbol{S}}$ denotes the displacement based stress tensor and $\T{S}$ denotes the modified stress tensor.
\begin{eqnarray}
\delta^i\bar{\mathbb{W}} & = & \sum_{i=1}^{3}\sum_{j=1}^{3} \delta\boldsymbol{G}_{ij}\,\bar{\boldsymbol{S}}_{ij} \label{eq:solid:upc:deltaWi:bar}
\end{eqnarray}
When comparing equation~\ref{eq:solid:upc:deltaW:Gp} to equation~\ref{eq:solid:upc:deltaWi:bar}, it is obvious that the modified stress tensor $\T{S}$ may be defined as:
\begin{eqnarray}
\boldsymbol{S}_{ij} & = & \bar{\boldsymbol{S}}_{ij} - \frac{1}{\kappa}\left(\bar{p} - \tilde{p}\right)\frac{\partial \bar{p}}{\partial \boldsymbol{G}_{ij}} \label{eq:solid:upc:S}
\end{eqnarray}
The derivation of equation~\ref{eq:solid:upc:S} is specific to a particular type of constitutive law.
At the moment, the following types of constitutive laws area available for u/p-c formulations:
\begin{itemize}
\item Hookean linear elastic isotropic
\item nonlinear hyperelastic Mooney-Rivlin
\item bilinear-elastoplastic with isotropic hardening
\end{itemize}
Since the modified stress tensor is depending not only on the strain tensor $\T{G}$, but also on the hydrostatic pressure $\tilde{p}$,
specialized types of constitutive laws are required. Furthermore, it is required that those constitutive laws
must return also the condition of volumetric compatibility $\frac{1}{\kappa}\left(\bar{p} - \tilde{p}\right)$.
See also MBDyn's \emph{input-manual} for further information about constitutive laws usable for solid elements based on u/p-c formulations.
\paragraph{Using the modified stress tensor and the condition of volumetric compatibility}
Now equation~\ref{eq:solid:upc:S} and equation~\ref{eq:solid:upc:deltaWi:bar} are substituted into equation~\ref{eq:solid:upc:deltaW:Gp}.
\begin{eqnarray}
\delta^i\mathbb{W} & = & \sum_{i=1}^{3}\sum_{j=1}^{3} \delta\boldsymbol{G}_{ij}\,\boldsymbol{S}_{ij} + \frac{1}{\kappa}\left(\bar{p} - \tilde{p}\right)\sum_{i=1}^{N_p}\frac{\partial \tilde{p}}{\partial \hat{\T{p}}_i}\,\delta\hat{\T{p}}_i \label{eq:solid:upc:deltaWi:V}
\end{eqnarray}
Since we are using isoparametric shape functions $\T{g}$ in order to interpolate the hydrostatic pressure $\tilde{p}$, the following simplification can be applied:
\begin{eqnarray}
\tilde{p} & = & \sum_{i=1}^{N_p} \T{g}_i \, \hat{\T{p}}_i \\
\frac{\partial \tilde{p}}{\partial \hat{\T{p}}_i} & = & \T{g}_i
\end{eqnarray}
If the order of interpolation for the hydrostatic pressure $\hat{\T{p}}$ is too high compared to the order of interpolation for displacements $\hat{\T{u}}$,
then the resulting element may behave like an displacement based element which will not be effective \cite{BATHE2016}.
So, a lower order of interpolation is used for the hydrostatic pressure in MBDyn.
For example, linear elements are using a constant hydrostatic pressure, whereas quadratic elements
are using a linear interpolation of the hydrostatic pressure based on the pressure at corner nodes.
See also MBDyn's \emph{input-manual} for further information about the node layout.
\paragraph{Integration over the volume of all elements}
In the next step, we integrate the virtual perturbation of the modified strain energy over the element volume,
and sum up the contributions of all the elements.
\begin{eqnarray}
\delta^iW & = & \bigcup_{e=1}^{N_e}\int_{^0V} \delta^i\mathbb{W} \, d^0V \label{eq:solid:upc:deltaWi:tot}
\end{eqnarray}
Finally we can substitute equation~\ref{eq:solid:upc:deltaWi:V} into equation~\ref{eq:solid:upc:deltaWi:tot}.
\begin{eqnarray}
\delta^iW & = & \bigcup_{e=1}^{N_e}\int_{^0V} \sum_{i=1}^{3}\sum_{j=1}^{3} \delta\boldsymbol{G}_{ij}\,\boldsymbol{S}_{ij} \, d^0V
          + \bigcup_{e=1}^{N_e}\int_{^0V} \frac{1}{\kappa}\left(\bar{p} - \tilde{p}\right)\sum_{i=1}^{N_p} \T{g}_i \, \delta\hat{\T{p}}_i \, d^0V
\end{eqnarray}
In the end, we can use the same formula for the internal force vector ${}^i\hat{\T{k}}$ as equation~\ref{eq:solid:ki},
just by replacing the displacement based stress tensor $\bar{\T{S}}$ by the modified stress tensor $\T{S}$.
In addition to that, equation~\ref{eq:solid:upc:c} is required, which represents the weak form of the condition of volumetric compatibility.
\begin{eqnarray}
^i\hat{\T{k}}_{kl} & = & \frac{1}{2} \, \int_{^0V} \left[ \sum_{i=1}^3 \sum_{j=1}^3 \left(\T{h}_{0d_{li}} \, \T{F}_{kj} + \T{h}_{0d_{lj}} \, \T{F}_{ki}\right) \, \T{S}_{ij} \right] \, d^0V \\
\hat{\T{c}}_i & = &\int_{^0V} \frac{1}{\kappa}\left(\bar{p} - \tilde{p}\right) \, \T{g}_i \, d^0V \label{eq:solid:upc:c} \\
\delta^iW & = & \bigcup_{e=1}^{N_e}\sum_{k=1}^{3} \sum_{l=1}^{N_n} \delta \hat{\T{u}}_{kl} \, ^i\hat{\T{k}}_{kl}
            +  \bigcup_{e=1}^{N_e} \sum_{i=1}^{N_p} \delta\hat{\T{p}}_i \, \hat{\T{c}}_i
\end{eqnarray}
Those expressions will be used in order to formulate the global equations of motion \ref{eq:solid:global_eq1} and \ref{eq:solid:global_eq1c} in section~\ref{sec:solid:global_eq}.
\subsection{Virtual work of inertia terms and body loads}
Also accelerations $\ddot{\hat{\T{u}}}$ and virtual displacements $\delta\hat{\T{u}}$ are interpolated by the same shape functions $\T{h}$.
\begin{eqnarray}
\delta^{m}W & = & \bigcup_{e=1}^{N_e} \int_{^0V} \sum_{i=1}^3 {^0}\rho \, \delta\T{u}_i \, \left(\ddot{\T{u}}_i - \T{b}_i\right) \, d^0V \label{eq:solid:delta_Wm} \\
\delta\T{u}_i & = & \sum_{k=1}^{N_n} \delta\hat{\T{u}}_{ik} \, \T{h}_k \label{eq:solid:delta_u} \\
\ddot{\T{u}}_i & = & \sum_{k=1}^{N_n} \ddot{\hat{\T{u}}}_{ik} \, \T{h}_k \label{eq:solid:delta_ddotu}
\end{eqnarray}

\paragraph{Body loads}
In MBDyn, body loads due to gravity $\T{g}$ and rigid body kinematics are considered. See also section~\ref{sec:nodes:structural nodes:relative frame dynamics}.
As a consequence, solid elements may be used also for typical rotor-dynamic analysis.
\begin{eqnarray}
\T{b} & = & \T{g} - \ddot{\overline{\T{x}}}_0 - \overline{\T{\omega}}_0 \times \left(\overline{\T{\omega}}_0 \times \T{x} \right) - \dot{\overline{\T{\omega}}}_0 \times \T{x} - 2 \, \overline{\T{\omega}}_0 \dot{\T{u}}
\end{eqnarray}

\begin{description}
\item[$\T{g}$] Gravity.
\item[$\ddot{\overline{\T{x}}}_0$] Acceleration of the global reference frame.
\item[$\overline{\T{\omega}}_0$]  Angular velocity of the global reference frame.
\item[$\dot{\overline{\T{\omega}}}_0$]  Angular acceleration of the global reference frame.
\end{description}

\paragraph{Mass matrix and body load vector}
When equation~\ref{eq:solid:delta_u} and equation~\ref{eq:solid:delta_ddotu} are substituted into equation~\ref{eq:solid:delta_Wm},
the virtual work $\delta^mW$ can be written in terms of the consistent mass matrix $\T{M}$ and body load vector $^b\hat{\T{f}}$.
Because of the isoparametric approach, the consistent mass matrix $\T{M}$ is constant with respect to time and needs to be evaluated only once \cite{BATHE2016}, \cite{KUEBLER2005}.
In contradiction to that, the actual magnitudes of $\T{b}$ and $^b\hat{\T{f}}$ may be time dependent.
\begin{eqnarray}
\hat{\T{M}}_{kl} & = & \int_{^0V} {^0}\rho \, \T{h}_k \, \T{h}_l \, d^0V \approx \sum_{g=1}^{N_g} \left.\left[{^0}\rho \, \T{h}_k \, \T{h}_l \, \alpha \, \det{\left({^0}\T{J}\right)}\right]\right\vert_{g} \\
{^b}\hat{\T{f}}_{ik} & = & \int_{^0V} {^0}\rho \, \T{h}_k \, \T{b}_i \, d^0V \approx \sum_{g=1}^{N_g} \left.\left[{^0}\rho \, \T{h}_k \, \T{b}_i \, \alpha \, \det{\left({^0}\T{J}\right)}\right]\right\vert_{g} \\
\delta^{m}W & = & \bigcup_{e=1}^{N_e}\left(\sum_{i=1}^3\sum_{k=1}^{N_n}\sum_{l=1}^{N_n} \delta\hat{\T{u}}_{ik} \, \ddot{\hat{\T{u}}}_{il} \, \hat{\T{M}}_{kl}
 - \sum_{i=1}^3\sum_{k=1}^{N_n} \delta\hat{\T{u}}_{ik} \, {^b}\hat{\T{f}}_{ik}\right) = \bigcup_{e=1}^{N_e} \delta\hat{\T{u}}^T \, \left(\T{M} \, \ddot{\hat{\T{u}}} - {^b}\hat{\T{f}} \right)
\end{eqnarray}
\begin{description}
\item[$\hat{\T{M}}$] $N_n\times N_n$ element mass matrix in compact storage
\item[$\T{M}$] $3 N_n \times 3 N_n$ element mass matrix in redundant storage (e.g. $\T{M}_{iijk} = \hat{\T{M}}_{jk}$)
\item[${^b}\hat{\T{f}}$] Element body load vector
\end{description}

\paragraph{Lumped mass matrix}
In addition to a consistent mass matrix, also a lumped mass matrix can be used in MBDyn.
For that purpose, the diagonal of the consistent mass matrix is scaled, so that the overall mass is conserved.
\begin{eqnarray}
\hat{\T{M}}_{kk}^{\star} & = & \int_{^0V} {^0}\rho \, \T{h}_k^2 \, d^0V \approx \sum_{g=1}^{N_g} \left.\left[{^0}\rho \, \T{h}_k^2 \, \alpha \, \det{\left({^0}\T{J}\right)}\right]\right\vert_{g} \\
\hat{\T{M}}_{kk}&=&\hat{\T{M}}_{kk}^{\star} \, \frac{\,m}{\sum_{k=1}^{N_n}\hat{\T{M}}_{kk}^{\star}} \\
m&=&\int_{^0V}{^0}\rho\,d{^0V} \approx \sum_{g=1}^{N_g} \left.\left[{^0}\rho \, \alpha \, \det{\left(^0\T{J}\right)}\right]\right\vert_{g}
\end{eqnarray}
If a lumped mass matrix is used, the virtual work of the inertia terms becomes:
\begin{eqnarray}
\delta^{m}W & = & \bigcup_{e=1}^{N_e}\left(\sum_{i=1}^3\sum_{k=1}^{N_n}\delta\hat{\T{u}}_{ik} \, \ddot{\hat{\T{u}}}_{ik} \, \hat{\T{M}}_{kk}
 - \sum_{i=1}^3\sum_{k=1}^{N_n} \delta\hat{\T{u}}_{ik} \, {^b}\hat{\T{f}}_{ik}\right)
\end{eqnarray}

\subsection{Reformulation of inertia terms required for a first order DAE solver}
\label{sec:solid:global_eq}
The principle of virtual displacements equation~\ref{eq:solid:pdVA} leads to the following system of second order ordinary differential equations.
\begin{eqnarray}
\bigcup_{e=1}^{N_e} \left(\T{M} \, \ddot{\hat{\T{u}}} + {^i}\hat{\T{k}} - {^b}\hat{\T{f}} - {^e}\hat{\T{f}}\right) = \T{0} \label{eq:solid:global_eq1}
\end{eqnarray}
\paragraph{The special case of u/p-c formulations}
In case of an u/p-c formulation, an additional set of algebraic equations~\ref{eq:solid:upc:c} is required in order to determine the hydrostatic pressure $\hat{\T{p}}$.
\begin{eqnarray}
\bigcup_{e=1}^{N_e} \hat{\T{c}} = \T{0} \label{eq:solid:global_eq1c}
\end{eqnarray}
As a consequence, the whole system of equations becomes Differential Algebraic, even if it is unconstrained.
\paragraph{Implementation issues in MBDyn}
Since MBDyn does not use accelerations for the solution of equations of motion, a reformulation of the equations of motion is required for solid elements.
It is required also because rigid bodies might be present in the same model, and those rigid bodies might share common nodes with solid elements.
So, equation~\ref{eq:solid:global_eq1} cannot be implemented directly.

\paragraph{Introducing the momentum}
Since the mass matrix $\T{M}$ is constant with respect to time,
equation~\ref{eq:solid:global_eq1} can be easily reformulated as a system of two first order ODE's
using the momentum $\hat{\T{\beta}}$ as a new unknown.
This approach is strictly consistent with MBDyn's conventions for rigid bodies.
\begin{eqnarray}
\hat{\T{\beta}}=\T{M}\,\dot{\hat{\T{u}}} \label{eq:solid:beta} \\
\dot{\hat{\T{\beta}}}=\T{M}\,\ddot{\hat{\T{u}}} \label{eq:solid:betadot}
\end{eqnarray}

\begin{description}
\item[$\hat{\T{\beta}}$] Momentum at element nodes
\end{description}

With equation~\ref{eq:solid:beta} and equation~\ref{eq:solid:betadot} the global system of equations is formulated as:
\begin{eqnarray}
\bigcup_{e=1}^{N_e} \left(\hat{\T{\beta}} - \T{M} \, \dot{\hat{\T{u}}}\right) = \T{0} \label{eq:solid:global:beta_assembly} \\
-\bigcup_{e=1}^{N_e} \left(\dot{\hat{\T{\beta}}} + {^i}\hat{\T{k}} - {^b}\hat{\T{f}} - {^e}\hat{\T{f}}\right) = \T{0} \label{eq:solid:global:betadot_assembly} \\
-\bigcup_{e=1}^{N_e} \hat{\T{c}} = \T{0} \label{eq:solid:global_eq1c_assembly}
\end{eqnarray}
In equation~\ref{eq:solid:global:beta_assembly} to \ref{eq:solid:global_eq1c_assembly},
MBDyn's sign conventions are applied in order to ensure compatiblity with other elements.
Equation~\ref{eq:solid:global_eq1c_assembly} is present only in case of u/p-c based formulations.

\paragraph{Limitations}
The only remaining limitation of this approach is, that it is not possible to compute accelerations
for a model using solid elements, unless a lumped mass matrix is used for all solid elements in the model.
See the MBDyn's \emph{input-manual} for further information on how to activate a lumped mass matrix
instead of a consistent mass matrix.

\subsection{Elements for surface loads}
In order to apply surface loads at solid elements, dedicated surface elements are required.
Basically the same isoparametric approach known from section~\ref{sec:solid:interpol} is applied also to surface elements.
Instead of the determinant of the Jacobian matrix, equation~\ref{eq:surface:dA} is used to transform infinitesimal surface areas
between natural curvilinear coordinates $\T{r}$ and global Cartesian coordinates $\T{x}$.
\begin{eqnarray}
\T{x}_i & = & \sum_{k=1}^{N_n} \T{h}_k \, \hat{\T{x}}_{ik} \\
\T{u}_i & = & \sum_{k=1}^{N_n} \T{h}_k \, \hat{\T{u}}_{ik} \\
dA & = & \left\Vert\frac{\partial \T{x}}{\partial \T{r}_1} \times \frac{\partial \T{x}}{\partial \T{r}_2} \right\Vert \,d\T{r}_1 \, d\T{r}_2 \label{eq:surface:dA} \\
\frac{\partial \T{x}_i}{\partial \T{r}_1} & = &\sum_{k=1}^{N_n} \frac{\partial \T{h}_{k}}{\partial \T{r}_1} \, \hat{\T{x}}_i \\
\frac{\partial \T{x}_i}{\partial \T{r}_2} & = &\sum_{k=1}^{N_n} \frac{\partial \T{h}_{k}}{\partial \T{r}_2} \, \hat{\T{x}}_i \\
\T{r}=\begin{pmatrix} \T{r}_1 & \T{r}_2 \end{pmatrix}^T
\end{eqnarray}
\paragraph{Shape functions}
Also surface elements are based on C++ templates in order to simplify the implementation of new element types.
The following types of elements are currently implemented:
\begin{itemize}
\item Quadrangular elements with linear and quadratic shape functions \cite{BATHE2016}, \cite{DHONDT2004}
\item Triangular elements with quadratic shape functions \cite{CODEASTERR30301}
\end{itemize}
See also MBDyn's \emph{input-manual} for further information about the node layout of different surface elements.

\paragraph{Virtual work of surface loads}
In order to derive the expressions for external virtual work, surface loads must be expressed in terms of
virtual displacements $\delta\hat{\T{u}}$ at the element nodes.
\begin{eqnarray}
\delta{^e}W & = & \bigcup_{e=1}^{N_e}\int_{A} \left(\sum_{i=1}^3 \delta\T{u}_i\,{^A}\T{f}_i \right)\,dA \\
          & = & \bigcup_{e=1}^{N_e}\int_{A} \left(\sum_{i=1}^3 \sum_{k=1}^{N_n} \delta\hat{\T{u}}_{ik} \, \T{h}_{k} \, {^A}\T{f}_i \right) \, dA \\
          & = & \bigcup_{e=1}^{N_e} \sum_{i=1}^3 \sum_{k=1}^{N_n} \delta\hat{\T{u}}_{ik} \, {^e}\hat{\T{f}}_{ik}
\end{eqnarray}
\paragraph{Pressure loads}
In case of pressure loads, ${^A}\T{f}$ will always act normal to the surface.
It should be emphasized, that the surface normal- and tangent vectors must be evaluated with respect to the deformed state.
\begin{eqnarray}
{^A}\T{f} & = & -\frac{1}{\left\Vert{^A}\T{n}\right\Vert} \, {^A}\T{n} \, p \\
p & = & \sum_{k=1}^{N_n} \T{h}_{k} \, \hat{\T{p}}_k \\
{^A}\T{n} & = & \frac{\partial \T{x}}{\partial \T{r}_1} \times \frac{\partial \T{x}}{\partial \T{r}_2}
%{^A}\T{n}_{1_i} & = & \frac{\partial \T{x}_i}{\partial \T{r}_1}  = \sum_{k=1}^{N_n} \frac{\partial \T{h}_{k}}{\partial \T{r}_1} \, \hat{\T{x}}_i \\
%{^A}\T{n}_{2_i} & = & \frac{\partial \T{x}_i}{\partial \T{r}_2}  = \sum_{k=1}^{N_n} \frac{\partial \T{h}_{k}}{\partial \T{r}_2} \, \hat{\T{x}}_i \\
%dA & = & \left\Vert{^A}\T{n}^{\star}\right\Vert \,d\T{r}_1 \, d\T{r}_2= \left\Vert{^A}\T{n}_1 \times {^A}\T{n}_2 \right\Vert \,d\T{r}_1 \, d\T{r}_2
\end{eqnarray}
\begin{description}
\item[$p$] pressure at a point $\T{x}$ at the surface
\item[$\hat{\T{p}}$] pressures at the element nodes
\item[${^A}\T{n}$] outward surface normal vector at point $\T{x}$
\end{description}
Finally, the virtual work of pressure loads can be evaluated by numerical integration across the surface area.
Due to the fact, that the surface may be moved or deformed during the simulation, pressure loads will not be constant,
even if the nodal pressures are constant values. In addition to that, pressure loads do contribute to the global stiffness matrix,
and they will directly (e.g. due to the deformation of the surface) and indirectly
(e.g. due to the stress in solid elements) affect mode shapes and natural frequencies of the structures they are applied to.
\begin{eqnarray}
\delta{^e}W & = & -\bigcup_{e=1}^{N_e}\int_{A} \left(\sum_{i=1}^3 \sum_{k=1}^{N_n} \delta\hat{\T{u}}_{ik} \, \T{h}_{k} \, {^A}\T{n}_i \, \frac{p}{\left\Vert{^A}\T{n}\right\Vert}\right) \, dA \\
 & \approx & -\bigcup_{e=1}^{N_e}\sum_{g=1}^{N_g} \left.\left(\sum_{i=1}^3 \sum_{k=1}^{N_n} \delta\hat{\T{u}}_{ik} \, \T{h}_{k} \, p \, {^A}\T{n}_i \, \alpha \right)\right\vert_{g}
\end{eqnarray}
\paragraph{Surface traction's}
Surface traction's are similar to pressure loads, but they do not necessarily act normal to the surface.
Instead the initial orientation of traction loads will be specified by the user, by means of an orientation matrix $\T{R}_f$.
In that way, shear stresses may be imposed at the surface of a body.
\begin{eqnarray}
\T{e}_1^{\star} & = & \frac{\partial \T{x}}{\partial \T{r}_1} \\
\T{e}_2^{\star} & = & \frac{\partial \T{x}}{\partial \T{r}_2} \\
\T{e}_3^{\star} & = & \T{e}_1^{\star} \times \T{e}_2^{\star} \\
\T{e}_2^{\star\star} & = & \T{e}_3^{\star} \times \T{e}_1^{\star} \\
{^A}\T{R} & = & \begin{pmatrix}
\frac{1}{\left\Vert\T{e}_1^{\star}\right\Vert} \, \T{e}_1^{\star} & \frac{1}{\left\Vert\T{e}_2^{\star\star}\right\Vert} \, \T{e}_2^{\star\star} & \frac{1}{\left\Vert\T{e}_3^{\star}\right\Vert} \, \T{e}_3^{\star}
\end{pmatrix} \\
{^A}\T{f} & = & {^A}\T{R} \, {^{A_0}}\T{R}^T \, \T{R}_f \, {^A}\bar{\T{f}} \\
{^A}\bar{\T{f}}_i & = & \sum_{k=1}^{N_n} \T{h}_{k} \, {^A}\bar{\hat{\T{f}}}_{ik} \\
\delta{^e}W & = & \bigcup_{e=1}^{N_e}\int_{A} \left(\sum_{i=1}^3 \sum_{k=1}^{N_n} \delta\hat{\T{u}}_{ik} \, \T{h}_{k} \, {^A}\T{f}_i \right) \, dA \\
 & \approx & \bigcup_{e=1}^{N_e}\sum_{g=1}^{N_g} \left.\left(\sum_{i=1}^3 \sum_{k=1}^{N_n} \delta\hat{\T{u}}_{ik} \, \T{h}_{k} \, {^A}\T{f}_i \, \left\Vert \T{e}_3^{\star} \right\Vert \, \alpha \right)\right\vert_{g}
\end{eqnarray}
\begin{description}
\item[${^A}\T{R}$] relative orientation matrix at point $\T{x}$ at the deformed state
\item[${^{A_0}}\T{R}$] relative orientation matrix at point $\T{x}$ at the undeformed state
\item[$\T{R}_f$] absolute orientation matrix for the applied traction load with respect to the undeformed state
\item[${^A}\T{f}$] surface traction's at point $\T{x}$ with respect to the global reference frame
\item[${^A}\bar{\T{f}}$] surface traction's at point $\T{x}$ with respect to the local reference frame
\item[${^A}\bar{\hat{\T{f}}}$] surface traction's at element nodes
\end{description}

\subsection{Implementation notes}
Many authors of textbooks about nonlinear Finite Element Methods spent
a lot of effort to explain the consistent derivation of the tangent stiffness matrix,
since this is crucial for any Newton-like nonlinear solver \cite{WALLRAPP1998}, \cite{BATHE2016}.
In contradiction to that, the analytical derivation of a sparse tangent stiffness matrix is not required for MBDyn's solid elements,
because they are exploiting a technique called ``Automatic~Differentiation'' for Newton~based- as well as Newton~Krylov based solvers.
