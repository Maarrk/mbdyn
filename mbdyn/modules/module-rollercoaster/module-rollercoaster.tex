\documentclass[12pt,dvips,fleqn,italian]{article}
% $Header$
% $Header$
% Copyright (C) 1996-2013 Pierangelo Masarati <masarati@aero.polimi.it>
% Dipartimento di Ingegneria Aerospaziale, Politecnico di Milano
%
% Parentesi: tonde, quadre, curly, dritte, doppie e angolari.
\newcommand{\plbr}[1]{ \left( #1 \right) }
\newcommand{\sqbr}[1]{ \left[ #1 \right] }
\newcommand{\cubr}[1]{ \left\{ #1 \right\} }
\newcommand{\shbr}[1]{ \left| #1 \right| }
\newcommand{\nrbr}[1]{ \left\| #1 \right\| }
\newcommand{\anbr}[1]{ \langle #1 \rangle }

% Parentesi solo a sinistra: tonde, quadre, curly, dritte, doppie e angolari.
\newcommand{\lplbr}[1]{ \left( #1 \right. }
\newcommand{\lsqbr}[1]{ \left[ #1 \right. }
\newcommand{\lcubr}[1]{ \left\{ #1 \right. }
\newcommand{\lshbr}[1]{ \left| #1 \right. }
\newcommand{\lnrbr}[1]{ \left\| #1 \right. }
\newcommand{\lanbr}[1]{ \langle #1 \right. }

% Parentesi solo a destra: tonde, quadre, curly, dritte,doppie e angolari.
\newcommand{\rplbr}[1]{ \left. #1 \right) }
\newcommand{\rsqbr}[1]{ \left. #1 \right] }
\newcommand{\rcubr}[1]{ \left. #1 \right\} }
\newcommand{\rshbr}[1]{ \left. #1 \right| }
\newcommand{\rnrbr}[1]{ \left. #1 \right\| }
\newcommand{\ranbr}[1]{ \left. #1 \rangle }

% Vettori verticali:
\newcommand{\vvect}[2]{ \begin{array}{ #1 } #2 \end{array} }
\newcommand{\cvvect}[1]{ \begin{array}{c} #1 \end{array} }
\newcommand{\lvvect}[1]{ \begin{array}{l} #1 \end{array} }
\newcommand{\rvvect}[1]{ \begin{array}{r} #1 \end{array} }

% Vettori orizzontali:
\newcommand{\hvect}[2]{ \begin{array}{ #1 } #2 \end{array} }

% Matrici:
\newcommand{\matr}[2]{ \begin{array}{ #1 } #2 \end{array} }

% Integrali: uso \intg{inf}{sup}{arg}{dvar}
\newcommand{\intg}[4]{ \int_{#1}^{#2} {#3} \ {#4} }

% Limite: uso \limt{var}{lim}{arg}
\newcommand{\limt}[3]{ \lim_{{#1} \rightarrow {#2}} {#3}}

% LogLike functions
\newcommand{\llk}[1]{\ensuremath{\mathrm{#1}}}

\newcommand{\diag}[0]{\llk{diag}}
\newcommand{\tr}[0]{\llk{tr}}
\newcommand{\sym}[0]{\llk{sym}}
\newcommand{\skw}[0]{\llk{skw}}

\newcommand{\step}[0]{\llk{step}}
\newcommand{\imp}[0]{\llk{imp}}

\newcommand{\grad}[0]{\llk{grad}}
\newcommand{\divr}[0]{\llk{div}}
\newcommand{\rot}[0]{\llk{rot}}

% In italiano ...
\newcommand{\sca}[0]{\llk{sca}}

% first, second, etc
\newcommand{\first}[0]{1\ensuremath{^{\mathrm{st}}}}    % 1^st
\newcommand{\second}[0]{2\ensuremath{^{\mathrm{nd}}}}   % 2^nd
\newcommand{\third}[0]{3\ensuremath{^{\mathrm{rd}}}}    % 3^rd
\newcommand{\rth}[0]{\ensuremath{^{\mathrm{th}}}}       %  ^th

\newcommand{\degr}[0]{\ensuremath{^{\mathrm{o}}}}

% esponenziale
\providecommand{\e}[1]{\llk{e}^{#1}}


\usepackage{babel}

\begin{document}


\title{\textbf{Vincolo ``rollercoaster''}}
\author{Stefania Gualdi, Marco Morandini, Pierangelo Masarati \\
\small Dipartimento di Ingegneria Aerospaziale, Politecnico di Milano}
\date{}
\maketitle

\section*{Sommario}
Il vincolo impone ad un corpo di spostarsi lungo una linea e di assumere
un'orientazione imposta.
I due problemi possono essere visti separatamente.
La linea pu\`o dipendere dalla configurazione, ovvero essere rappresentata
dalla linea di riferimento che definisce una trave.


\section*{Supporto rigido}
Si consideri innanzitutto un supporto rigido, dato da una linea su cui
\`e definita una orientazione.

\subsection*{Vincolo in posizione}
Il luogo delle configurazioni che il corpo pu\`{o} assumere \`{e}
parametrizzato in funzione di un'ascissa curvilinea $ s $; 
si richiede che sia funzione continua fino alla derivata prima.
Si deve notare che, in alcuni casi, la derivata della linea pu\`o 
non essere continua in alcuni punti; il vincolo, in tali circostanze, 
richiede un trattamento particolare che verr\`a discusso a parte.
La posizione della linea \`{e} data dalle coordinate 
$ x\plbr{s}:B\in\Re\mapsto\Re^3 $,
mentre la giacitura di ogni punto \`{e} definita da una matrice di rotazione
$ R\plbr{s}:B\in\Re\mapsto{SO\plbr{\Re^3}} $.
Le derivate della linea e della giacitura sono definite da
$ x'\plbr{s}=\partial{x}/\partial{s} $ e da
$ \rho\plbr{s}\times{}=\partial{R}/\partial{s}R^T $.
Si consideri l'ipotesi aggiuntiva che l'asse 3 del sistema di riferimento
che definisce l'orientazione del vincolo sia diretto come la tangente alla
linea stessa; questa ipotesi verr\`a rimossa quando il vincolo sar\`a
applicato ad un supporto flessibile.
L'equazione di vincolo si scrive come
\begin{displaymath}
    R^T \plbr{ x_b - x\plbr{s} } \ = \ 0 ,
\end{displaymath}
la quale afferma che il punto $ x_b $ appartenente al corpo $ b $ deve
giacere sulla linea $ x\plbr{s} $ ad una generica ascissa $ s $ al momento
incognita.
La reazione vincolare che il corpo ed il vincolo si scambiano, in assenza di
attrito, \`{e} diretta nelle direzioni 1 e 2 del sistema locale.
Sia quindi $ e_i $ l'$i$-esimo versore del sistema di riferimento che
definisce la giacitura del vincolo.
Le forze agenti sul corpo $ b $ sono
\begin{displaymath}
    F \ = \ e_1 v_1 + e_2 v_2 ,
\end{displaymath}
dove $ v_i $ \`{e} la $i$-esima componente della reazione vincolare nel
sistema solidale con il vincolo.
La linearizzazione delle relazioni di vincolo risulta in
\begin{displaymath}
    R^T \plbr{ \rho \times^T \plbr{ x_b - x } - x' }\Delta{s}
    + R^T \Delta{x_b} \ = \ 0
\end{displaymath}
per quanto riguarda l'equazione di vincolo, mentre per la forza si ottiene
\begin{displaymath}
    \Delta{F} \ = \ 
        \rho \times \plbr{ e_1 v_1 + e_2 v_2 } \Delta{s}
	+ e_1 \Delta{v_1} + e_2 \Delta{v_2} .
\end{displaymath}
L'equazione di vincolo va intesa nel modo seguente: le prime due equazioni
affermano che la proiezione della distanza tra il corpo ed il vincolo nelle
direzioni normali alla linea di riferimento devono essere nulle. 
Ad una violazione di queste equazioni corrisponde la nascita di una reazione
vincolare.
La terza equazione afferma che anche la proiezione lungo la linea deve
essere nulla; ad una sua violazione per\`{o} non corrisponde una reazione
vincolare, ma essa viene usata per determinare l'ascissa curvilinea a cui il
corpo si trova.
Il sistema e significativamente lo Jacobiano si semplificano se le equazioni
di vincolo vengono scritte direttamente nel sistema globale, quindi non
premoltiplicate per la trasposta della giacitura del vincolo; in tal caso si
ottiene
\begin{displaymath}
    x_b - x\plbr{s} \ = \ 0
\end{displaymath}
e 
\begin{displaymath}
    \Delta{x_b} - x' \Delta{s} \ = \ 0 .
\end{displaymath}



\subsection*{Vincolo in rotazione}
Il vincolo in rotazione si ottiene imponendo che gli assi dei sistemi di
riferimento del corpo e del vincolo siano ortogonali a due a due.
Tale vincolo \`{e} del tutto analogo a quello che si scrive di solito tra
due corpi, solo che la matrice di rotazione del vincolo ora \`{e}
parametrizzata nell'ascissa curvilinea.
Si consideri ad esempio l'ortogonalit\`{a} tra due assi non-omonimi di corpo
e vincolo, $ e_{b1} $ e $ e_2 $.
L'equazione di vincolo \`{e}
\begin{displaymath}
    e_{b1}^T e_2 \ = \ 0 .
\end{displaymath}
Questa equazione d\`{a} origine ad una coppia che agisce attorno all'asse
ortogonale ai due che definiscono il vincolo, quindi
\begin{displaymath}
    M \ = \ e_{b1} \times e_2 m_3 .
\end{displaymath}
La loro linearizzazione d\`{a}
\begin{displaymath}
    \plbr{ e_{b1} \times e_2 }^T \theta_{\Delta{b}}
    +\plbr{ e_2 \times e_{b1} }^T \rho \Delta{s} \ = \ 0
\end{displaymath}
e
\begin{displaymath}
    \Delta{M} \ = \
        m_3 e_2 \times e_{b1} \times \theta_{\Delta{b}}
	- m_3 e_{b1} \times e_2 \times \rho \Delta{s}
	+ e_{b1} \times e_2 \Delta{m_3} .
\end{displaymath}
Quindi l'ascissa curvilinea viene risolta dal vincolo in posizione, mentre
il vincolo in rotazione si limita ad imporre, nello stesso punto, che anche
i sistemi di riferimento coincidano.



\section*{Supporto flessibile}
Si consideri ora un supporto flessibile, costituito da un modello di trave.

\subsection*{Vincolo in posizione}
Il vincolo impone che il corpo $b$ scorra lungo la linea di riferimento
della trave, e che sia forzato a seguirne l'orientazione.
Le equazioni scritte in precedenza si modificano per quanto riguarda la
descrizione del supporto e della sua legge oraria.
La posizione del supporto \`{e} definita in funzione delle posizioni
dei punti di riferimento $ x_n $ della trave 
\begin{displaymath}
    x\plbr{s} \ = \ x\plbr{x_n} \ = \ N_i\plbr{s}\plbr{x_i + f_i}
\end{displaymath}
attrverso le funzioni di forma $N_i\plbr{s}$; le posizioni nodali
$x_n$ sono date dalle posizioni dei nodi $x_i$ corrette 
da un eventuale offset rigido $f_i$.
L'orientazione della trave \`{e} definita dalla matrice di rotazione
\begin{displaymath}
    R\plbr{s} \ = \ R_{\Delta}\plbr{ N_i\plbr{s} g_i } R_r 
\end{displaymath}
in funzione della orientazione di riferimento $R_r$ e dei parametri di
rotazione nodali $g_i$ che concorrono a descrivere la perturbazione di
rotazione incrementale $R_{\Delta}$.
Assume un'importanza fondamentale, nella descrizione del vincolo,
la tangente alla linea di riferimento, $ t $.
Essa \`e data dalla derivata della posizione della trave $ l $, definita
come:
\begin{displaymath}
    l \ = \ x'\plbr{s} \ = \ N'_i\plbr{s}\plbr{x_i + f_i}
\end{displaymath}
da cui la tangente:
\begin{displaymath}
    t \ = \ \frac{l}{\nrbr{l}}
\end{displaymath}
In genere, per semplificare le linearizzazioni, \`e opportuno considerare
direttamente $ l $ al posto della tangente, moltiplicando per
$ \nrbr{l} $.

\noindent
L'equazione di vincolo in posizione rimane formalmente quella del caso
precedente:
\begin{displaymath}
    x_b - x \ = \ 0 ;
\end{displaymath}
la sua linearizzazione, tuttavia, chiama in causa anche i gradi di
libert\`{a} dell'elemento di trave.
Si ha infatti:
\begin{eqnarray*}
    \Delta{x\plbr{s}} & = &
    \frac{\partial{x\plbr{s}}}{\partial{x_n}} \Delta{x_n}
    + \frac{\partial{x\plbr{s}}}{\partial{s}} \Delta{s} \\
    & = &  
    N_i\plbr{s} \plbr{\Delta{x_i} - f_i\times\Delta{g_i}} 
    + N'_i\plbr{s} \plbr{x_i + f_i} \Delta{s} .
\end{eqnarray*}
La reazione vincolare, intesa come entit\`a vettoriale, deve essere 
ortogonale alla tangente alla linea di riferimento della trave.
Quindi, detto $ v $ il vettore delle reazioni, si ottiene:
\begin{displaymath}
    v^T t \ = \ 0
\end{displaymath}
e, per semplificare la linearizzazione, si pu\`o scrivere:
\begin{displaymath}
    v^T l \ = \ 0
\end{displaymath}
Le forze applicate dal vincolo sono:
\begin{displaymath}
    F \ = \ v
\end{displaymath}
Le rispettive linearizzazioni sono:
\begin{displaymath}
    v^T \Delta{l} + l^T \Delta{v} \ = \ 0
\end{displaymath}
e
\begin{displaymath}
    \Delta{F} \ = \ \Delta{v}
\end{displaymath}
La linearizzazione della derivata della linea di riferimento assume 
una forma particolare:
\begin{displaymath}
    \Delta{l} \ = \ N'_i\plbr{\Delta{x_i} - f_i\times\Delta{g_i}}
    + N''_i\plbr{x_i + f_i} \Delta{s}
\end{displaymath}
e, definendo $l'=N''_i\plbr{x_i + f_i}$, con implicita la sommatoria 
sugli indici,
\begin{displaymath}
    \Delta{l} \ = \ N'_i\plbr{\Delta{x_i} - f_i\times\Delta{g_i}}
    + l' \Delta{s}
\end{displaymath}
Quindi la linearizzazione di quest'ultima equazione \`e:
\begin{displaymath}
	l^T \Delta{v} 
	+ v^T N'_i \Delta{x_i} 
	+ \plbr{f_i\times{v}}^T N'_i \Delta{g_i}
	+ v^T \plbr{x_i + f_i} N''_i \Delta{s}
	\ = \ 0
\end{displaymath}



\subsection*{Vincolo in rotazione}
L'equazione di vincolo alla rotazione richiede una cura particolare.
Si possono ipotizzare tre diversi vincoli:
\begin{enumerate}
	\item giunto sferico: consente la libera rotazione attorno a tutti
	gli assi; di conseguenza non si scrivono equazioni di vincolo
	alla rotazione;
	\item manicotto ``classico'': consente la libera rotazione attorno
	all'asse del manicotto; si scrivono due sole equazioni di vincolo
	alla rotazione
	\item manicotto ``rigato'': non consente libert\`a di rotazione;
	si scrivono tre equazioni di vincolo alla rotazione.
\end{enumerate}


\paragraph*{Manicotto ``classico''}
Occorre prestare attenzione a come si definiscono i vincoli di rotazione.
Non \`e ragionevole, in questo caso, ipotizzare che l'orientazione 
del manicotto coincida con l'orientazione della trave all'ascissa
curvilinea identificata dal vincolo di posizione.
Infatti l'orientazione della trave pu\`o non avere un asse parallelo
alla tangente della linea di riferimento qualora si abbia deformazione
di scorrimento.
\`E invece ragionevole ipotizzare che un asse privilegiato del manicotto
sia orientato come la tangente alla linea di riferimento:
\begin{displaymath}
	t \ = \ \frac{l}{\nrbr{l}} ,
\end{displaymath}
con 
\begin{displaymath}
	l \ = \ x'\plbr{s}
\end{displaymath}
e che quindi gli altri due assi siano ad esso ortogonali:
\begin{eqnarray*}
	e_{b2}^T t & = & 0 \\
	e_{b3}^T t & = & 0
\end{eqnarray*}
Per semplificare le linearizzazioni, dal momento che si considera nulla
la rotazione iniziale, si pu\`o sostituire la tangente
con la derivata della linea media:
\begin{eqnarray*}
	e_{b2}^T l & = & 0 \\
	e_{b3}^T l & = & 0
\end{eqnarray*}
Queste due equazioni definiscono il vincolo alla rotazione del manicotto 
classico; le corrispondenti reazioni vincolari sono le coppie:
\begin{displaymath}
	M \ = \ m_2 e_{b2}\times{t} + m_3 e_{b3}\times{t}
\end{displaymath}
Anche qui, data l'indeterminazione delle reazioni vincolari, si pu\`o 
usare la derivata della linea media al posto della tangente:
\begin{displaymath}
	M \ = \ m_2 e_{b2}\times{l} + m_3 e_{b3}\times{l}
\end{displaymath}
Ovviamente le reazioni sono date da $M_i=\nrbr{l}m_i$.

\noindent
La linearizzazione delle equazioni di vincolo d\`a:
\begin{eqnarray*}
	0 & = & \plbr{e_{bj}\times{l}}^T \theta_{\Delta{b}}
	+ e_{bj}^T \Delta{l} \\
	& \cong & \plbr{e_{bj}\times{l}}^T \Delta{g_b}
	+ e_{bj}^T\plbr{
		N'_i \plbr{\Delta{x_i} - f_i\times\Delta{g_i}}
		+ N''_i \plbr{x_i + f_i} \Delta{s}
	} \\
	& \cong & \plbr{e_{bj}\times{l}}^T \Delta{g_b}
	+ e_{bj}^T\plbr{
		N'_i \plbr{\Delta{x_i} - f_i\times\Delta{g_i}}
		+ l' \Delta{s}
	}
\end{eqnarray*}
con $j=2,3$, dove, in accordo con l'approssimazione aggiornata-aggiornata,
si \`e confuso $\theta_{\Delta{b}}$ con $\Delta{g_b}$; 
la linearizzazione delle reazioni vincolari d\`a:
\begin{eqnarray*}
	0 & = & e_{b2}\times{l} \Delta{m_2} + e_{b3}\times{l} \Delta{m_3} 
	+ \plbr{m_2 e_{b2} + m_3 e_{b3}}\times \Delta{l} \\
	& & \mbox{} 
	+ l\times\plbr{m_2 e_{b2} + m_3 e_{b3}}\times \theta_{\Delta{b}} \\
	& \cong & e_{b2}\times{l} \Delta{m_2} + e_{b3}\times{l} \Delta{m_3} \\
	&& \mbox{} + \plbr{m_2 e_{b2} + m_3 e_{b3}}\times \plbr{
		N'_i \plbr{\Delta{x_i} - f_i\times\Delta{g_i}}
		+ N''_i \plbr{x_i + f_i} \Delta{s}
	} \\
	& & \mbox{} + l\times\plbr{m_2 e_{b2} + m_3 e_{b3}}\times \Delta{g_b}
\end{eqnarray*}

\paragraph*{Manicotto ``rigato''}
L'equazione di vincolo alla rotazione attorno alla linea di riferimento 
della trave \`e un po' pi\`u complessa, e richiede la definizione 
della rotazione relativa tra manicotto e trave.
Se la rotazione relativa \`e piccola, pu\`o essere approssimata 
con l'\emph{assiale} della matrice, ovvero la sua parte emisimmetrica; 
infatti, data una matrice di rotazione
\begin{displaymath}
    R \ = \ I + \sin\plbr{\varphi}v\times{} 
    	+ \plbr{1-\cos\plbr{\varphi}}v\times{v\times{}}
\end{displaymath}
la sua parte emisimmetrica \`e 
$\llk{skw}\plbr{R}=\sin\plbr{\varphi}v\times{}$, e, se l'angolo \`e piccolo,
essa \`e data da $\llk{skw}\plbr{R}\cong\varphi{v\times{}}$.
Si assume quindi:
\begin{displaymath}
	R \ \cong \ I + \theta\times{}
\end{displaymath}
da cui si ottiene:
\begin{displaymath}
	\theta \ = \ \llk{ax}\plbr{R_{\Delta}} 
		\ = \ \llk{ax}\plbr{R_bR^T}
\end{displaymath}
La proiezione di tale rotazione nella direzione tangente alla curva 
deve essere nulla (o pari alla rotazione iniziale):
\begin{displaymath}
	\theta^T t \ = \ 0 .
\end{displaymath}
Alla tangente si pu\`o sostituire la derivata della linea di riferimento, 
semplificando cos\`{\i} le linearizzazioni:
\begin{displaymath}
	\theta^T l \ = \ 0 .
\end{displaymath}
La linearizzazione della rotazione relativa, $\theta_{\Delta}$,
in generale dipendente dalla parametrizzazione $\llk{p}$ attraverso
la relazione $ \theta_{\Delta}=\Gamma\plbr{\llk{p}}\Delta{\llk{p}} $, 
\`e data da:
\begin{eqnarray*}
	\theta_{\Delta} & = & \Delta\llk{ax}\plbr{R_bR^T} \\
	& = & \llk{ax}\plbr{\Delta\plbr{R_bR^T}\plbr{R_bR^T}^T} \\
	& = & \llk{ax}\plbr{\theta_{\Delta{b}}\times{R_bR^TRR_b^T}
		+R_bR^T\theta_{\Delta{t}}\times{}^TRR_b^T} \\
	& = & \llk{ax}\plbr{R_bR_b^T\theta_{\Delta{b}}\times{R_bR_b^T}
		+R_bR^T\theta_{\Delta{t}}\times{}^TRR_b^T} \\
	& = & \llk{ax}\plbr{
		R_b\plbr{
			R_b^T\theta_{\Delta{b}}\times{R_b}
			-R^T\theta_{\Delta{t}}\times{R}
		}R_b^T
	} \\
	& = & R_b\llk{ax}\plbr{
		\plbr{R_b^T\theta_{\Delta{b}}}\times{}
		-\plbr{R^T\theta_{\Delta{t}}}\times{}
	} \\
	& = & R_b\llk{ax}\plbr{\plbr{
		R_b^T\theta_{\Delta{b}} - R^T\theta_{\Delta{t}}
	}\times{}} \\
	& = & R_b\plbr{R_b^T\theta_{\Delta{b}}-R^T\theta_{\Delta{t}}}
\end{eqnarray*}
o
\begin{displaymath}
	\theta_{\Delta} \ = \ \theta_{\Delta{b}}-R_bR^T\theta_{\Delta{t}}
\end{displaymath}
dove si sono sfruttate le propriet\`a:
\begin{eqnarray*}
	\Delta\llk{ax}\plbr{R} & = & \llk{ax}\plbr{\Delta{R} R^T} \\
	\Delta{R} & = & \theta\times{R} \\
	R R^T & = & I \\
	\theta\times{}^T & = & -\theta\times \\
	R \theta\times{R^T} & = & \plbr{R\theta}\times{} \\
	\llk{ax}\plbr{\theta\times{}} & = & \theta \\
	\theta_1\times{}+\theta_2\times{} & = & \plbr{\theta_1+\theta_2}\times{} \\
	\llk{ax}\plbr{\theta\times{}} & = & \theta
\end{eqnarray*}
Questo significa che la perturbazione dell'angolo relativo \`e data 
dalla differenza tra le perturbazioni degli angoli attraverso un termine 
di trasporto dato dalla rotazione relativa tra i due sistemi di riferimento.

\noindent
La coppia risultante da questa equazione di vincolo si applica semplicemente
attorno alla direzione tangente:
\begin{displaymath}
	M \ = \ m_1 t
\end{displaymath}
o, per sempliciare le linearizzazioni, si pu\`o considerare direttamente
la derivata degli spostamenti $l$, data l'indeterminazione dello scalare
$m_1$:
\begin{displaymath}
	M \ = \ m_1 l .
\end{displaymath}
Anche in questo caso il modulo del momento \`e dato da $M_1=m_1\nrbr{l}$.


\subsection*{Trave a volumi finiti}
Si consideri una formulazione del modello di trave a volumi finiti
\cite{FV-AIAA}.
Secondo tale schema, l'elemento di trave \`{e} suddiviso nettamente in
regioni a cavallo dei nodi; le forze che cadono all'interno di una regione
sono attribuite totalmente al nodo relativo.
Quindi, dette $w_i$ le funzioni di attivazione delle regioni che
costituiscono l'elemento di trave (costanti a tratti), la reazione vincolare
viene applicata come
\begin{eqnarray*}
    F_i & = & - w_i F\delta{s} , \\
    M_i & = & - w_i \plbr{ x\plbr{s} - x_i }\times F\delta{s} 
        - M\delta{s} ,
\end{eqnarray*}
dove il momento \`e definito come $ M=Hm$, a seconda dei tre modelli 
considerati, con:
\begin{enumerate}
	\item giunto sferico:
	\begin{displaymath}
		H \ = \ 0
	\end{displaymath}
	\item manicotto ``classico'':
	\begin{displaymath}
		H \ = \ \sqbr{\matr{cc}{
			e_{b2}\times{l} & e_{b3}\times{l}
		}}
	\end{displaymath}
	\item manicotto ``rigato'':
	\begin{displaymath}
		H \ = \ \sqbr{\matr{ccc}{
			l & e_{b2}\times{l} & e_{b3}\times{l}
		}}
	\end{displaymath}
\end{enumerate}
La sua linearizzazione, considerando le approssimazioni della
linearizzazione delle rotazioni ammesse dall'approccio aggiornato-aggiornato
descritto in \cite{PHD-MASARATI}, comporta i termini
\begin{eqnarray*}
    \Delta{F_i} & = & - w_i \lplbr{
        \rho \times \plbr{ e_1 v_1 + e_2 v_2 } \Delta{s}
    } \\
    & & \rplbr{ \mbox{}
	- \plbr{ e_1 v_1 + e_2 v_2 } \times N_i \Delta{g}
	+ e_1 \Delta{v_1} + e_2 \Delta{v_2}
    } , \\
    \Delta{M_i} & = & - w_i \lplbr{
        - F \times \plbr{ N_j - \delta_{ij} }\Delta{ x_j }
	+ \plbr{ x\plbr{s} - x_i }\times \Delta{F}
    } \\
    & & \rplbr{ \mbox{}
        + \Delta{H} m + H \Delta{m}
    } .
\end{eqnarray*}




\section*{Conclusioni}
Il vincolo \`{e} relativamente semplice da implementare, come verificato
scrivendo un modulo per MBDyn. 
Il problema fondamentale \`{e} disporre di posizione, direzione tangente,
orientazione e curvatura del vincolo in funzione continua e derivabile
dell'ascissa curvilinea.



\bibliographystyle{ieeetr}
\bibliography{mybib}

\pagebreak
\subsection*{Traccia del lavoro}

\begin{itemize}
\item formulazione analitica del vincolo
\item implementazione in MBDyn (volumi finiti; elementi finiti?)
\item discussione su attrito e altre problematiche collegate
\item caso prova: soluzione analitica lineare confrontata a soluzione vf e fe;
  discorso sulla convergenza e sulla modalita` di applicazione
  di forza concentrata/vf (fare modello lineare a un elemento?)
\item applicazione aeronautica: carrello a stelo
\item altre applicazioni: cenno al rollercoaster
\end{itemize}

\noindent
Soluzione lineare trave caricata trasversalmente ad un estremo e incastrata
all'altro:
\begin{eqnarray*}
	v\plbr{x} & = & \frac{x}{GA}+\frac{1}{3}\frac{x^3}{EJ} \\
	\phi\plbr{x} & = & \frac{1}{2}\frac{x^2}{EJ}
\end{eqnarray*}
Soluzone per trave lunga $L$ caricata a distanza $d<L$ dall'incastro:
\begin{eqnarray*}
	v\plbr{d} & = & \frac{d}{GA}+\frac{1}{3}\frac{d^3}{EJ} \\
	\phi\plbr{d} & = & \frac{1}{2}\frac{d^2}{EJ} \\
	v\plbr{L} & = & v\plbr{d} + \plbr{L-d}\phi\plbr{d} \\
	& = & \frac{d}{GA}+\frac{1}{6}\frac{d^2}{EJ}\plbr{3L - d} \\
	\phi\plbr{L} & = & \phi\plbr{d}
\end{eqnarray*}


\end{document}
