% MBDyn (C) is a multibody analysis code.
% http://www.mbdyn.org
%
% Copyright (C) 1996-2004
%
% Pierangelo Masarati  <masarati@aero.polimi.it>
%
% Dipartimento di Ingegneria Aerospaziale - Politecnico di Milano
% via La Masa, 34 - 20156 Milano, Italy
% http://www.aero.polimi.it
%
% Changing this copyright notice is forbidden.
%
% This program is free software; you can redistribute it and/or modify
% it under the terms of the GNU General Public License as published by
% the Free Software Foundation (version 2 of the License).
% 
%
% This program is distributed in the hope that it will be useful,
% but WITHOUT ANY WARRANTY; without even the implied warranty of
% MERCHANTABILITY or FITNESS FOR A PARTICULAR PURPOSE.  See the
% GNU General Public License for more details.
%
% You should have received a copy of the GNU General Public License
% along with this program; if not, write to the Free Software
% Foundation, Inc., 59 Temple Place, Suite 330, Boston, MA  02111-1307  USA

\documentclass[10pt,dvips]{report}

%\usepackage[pdftex]{graphicx}
\usepackage[T1]{fontenc}
\usepackage{ae,aecompl}
\usepackage{graphicx}
\usepackage{amsmath}
\usepackage[dvips,breaklinks=true,colorlinks=false]{hyperref}
\usepackage{html}

% $Header$
% Copyright (C) 1996-2013 Pierangelo Masarati <masarati@aero.polimi.it>
% Dipartimento di Ingegneria Aerospaziale, Politecnico di Milano
%
% Parentesi: tonde, quadre, curly, dritte, doppie e angolari.
\newcommand{\plbr}[1]{ \left( #1 \right) }
\newcommand{\sqbr}[1]{ \left[ #1 \right] }
\newcommand{\cubr}[1]{ \left\{ #1 \right\} }
\newcommand{\shbr}[1]{ \left| #1 \right| }
\newcommand{\nrbr}[1]{ \left\| #1 \right\| }
\newcommand{\anbr}[1]{ \langle #1 \rangle }

% Parentesi solo a sinistra: tonde, quadre, curly, dritte, doppie e angolari.
\newcommand{\lplbr}[1]{ \left( #1 \right. }
\newcommand{\lsqbr}[1]{ \left[ #1 \right. }
\newcommand{\lcubr}[1]{ \left\{ #1 \right. }
\newcommand{\lshbr}[1]{ \left| #1 \right. }
\newcommand{\lnrbr}[1]{ \left\| #1 \right. }
\newcommand{\lanbr}[1]{ \langle #1 \right. }

% Parentesi solo a destra: tonde, quadre, curly, dritte,doppie e angolari.
\newcommand{\rplbr}[1]{ \left. #1 \right) }
\newcommand{\rsqbr}[1]{ \left. #1 \right] }
\newcommand{\rcubr}[1]{ \left. #1 \right\} }
\newcommand{\rshbr}[1]{ \left. #1 \right| }
\newcommand{\rnrbr}[1]{ \left. #1 \right\| }
\newcommand{\ranbr}[1]{ \left. #1 \rangle }

% Vettori verticali:
\newcommand{\vvect}[2]{ \begin{array}{ #1 } #2 \end{array} }
\newcommand{\cvvect}[1]{ \begin{array}{c} #1 \end{array} }
\newcommand{\lvvect}[1]{ \begin{array}{l} #1 \end{array} }
\newcommand{\rvvect}[1]{ \begin{array}{r} #1 \end{array} }

% Vettori orizzontali:
\newcommand{\hvect}[2]{ \begin{array}{ #1 } #2 \end{array} }

% Matrici:
\newcommand{\matr}[2]{ \begin{array}{ #1 } #2 \end{array} }

% Integrali: uso \intg{inf}{sup}{arg}{dvar}
\newcommand{\intg}[4]{ \int_{#1}^{#2} {#3} \ {#4} }

% Limite: uso \limt{var}{lim}{arg}
\newcommand{\limt}[3]{ \lim_{{#1} \rightarrow {#2}} {#3}}

% LogLike functions
\newcommand{\llk}[1]{\ensuremath{\mathrm{#1}}}

\newcommand{\diag}[0]{\llk{diag}}
\newcommand{\tr}[0]{\llk{tr}}
\newcommand{\sym}[0]{\llk{sym}}
\newcommand{\skw}[0]{\llk{skw}}

\newcommand{\step}[0]{\llk{step}}
\newcommand{\imp}[0]{\llk{imp}}

\newcommand{\grad}[0]{\llk{grad}}
\newcommand{\divr}[0]{\llk{div}}
\newcommand{\rot}[0]{\llk{rot}}

% In italiano ...
\newcommand{\sca}[0]{\llk{sca}}

% first, second, etc
\newcommand{\first}[0]{1\ensuremath{^{\mathrm{st}}}}    % 1^st
\newcommand{\second}[0]{2\ensuremath{^{\mathrm{nd}}}}   % 2^nd
\newcommand{\third}[0]{3\ensuremath{^{\mathrm{rd}}}}    % 3^rd
\newcommand{\rth}[0]{\ensuremath{^{\mathrm{th}}}}       %  ^th

\newcommand{\degr}[0]{\ensuremath{^{\mathrm{o}}}}

% esponenziale
\providecommand{\e}[1]{\llk{e}^{#1}}


\newcommand{\kw}[1]{\texttt{#1}}
%\newcommand{\kw}[1]{\fbox{\texttt{#1}}}
\newcommand{\T}[1]{\boldsymbol{#1}}

\begin{document}

\begin{latexonly}
\title{\bf MBDyn Installation Manual \\
Version
0.1.0

}
\author{Pierangelo Masarati \vspace{5mm}\\
    \sc Dipartimento di Ingegneria Aerospaziale \\
    \sc Politecnico di Milano}
\date{Automatically generated \today}
\maketitle
\end{latexonly}

\begin{htmlonly}
\begin{center}
\textbf{\LARGE MBDyn Installation Manual}

\emph{\large Pierangelo Masarati}

\textsc{Dipartimento di Ingegneria Aerospaziale \\ Politecnico di Milano}

\today
\end{center}
\end{htmlonly}




\tableofcontents
\newpage

\chapter{Introduction}
This document describes how to download, build, install and execute
MBDyn --- MultiBody Analysis program, a suite of tools 
for multibody/multidisciplinary analysis of complex systems.

For any question or problem, to fix typos, bugs, for comments and
suggestions, please contact the Development Team
without hesitation:\vspace{10mm}\\

\noindent
\begin{tabular}{ll}
\multicolumn{2}{l}{Pierangelo Masarati,} \\
\multicolumn{2}{l}{MBDyn Development Team} \\
\multicolumn{2}{l}{Dipartimento di Ingegneria Aerospaziale} \\
\multicolumn{2}{l}{Politecnico di Milano} \\
\multicolumn{2}{l}{via La Masa 34, 20156 Milano, Italy} \\
Fax: & +39 02 2399 8334 \\
E-mail: & \htmladdnormallink{\kw{mbdyn@aero.polimi.it}}{mailto:mbdyn@aero.polimi.it} \\
Web: & \htmladdnormallink{\kw{http://www.aero.polimi.it/\~{}mbdyn/}}{http://www.aero.polimi.it/~mbdyn/}
\end{tabular}
\vspace{10mm}


\noindent
This document is also available online at \\
\htmladdnormallink{\kw{http://mbdyn.aero.polimi.it/\~{}masarati/MBDyn-input/install/}}{http://mbdyn.aero.polimi.it/~masarati/MBDyn-input/install/}



\chapter{Getting the package}
The package can be downloaded in source form from
\begin{center}
\htmladdnormallink{\kw{http://mbdyn.aero.polimi.it/\~{}mbdyn/}}{http://mbdyn.aero.polimi.it/\~{}mbdyn/}
\end{center}

Binary releases and snapshots are also available for Windows 2000/XP at 
\htmladdnormallink{\kw{http://mbdyn.aero.polimi.it/\~{}masarati/Download/mbdyn/}}{http://mbdyn.aero.polimi.it/~masarati/Download/mbdyn/},
compiled with Cygwin (see
\htmladdnormallink{\kw{http://www.cygwin.com/}}{http://www.cygwin.com/})
and thus require \kw{cygltdl-3.dll};
they are provided to Windows users to save them the burden of installing
Cygwin (very easy and straightforward, though) and to compile a package
under an unfamiliar environment.
However, no support is provided for those builds, unless problems 
are easily identifiable as related to the sources and not to the OS,
and they impact the UN*X vesion as well.

MBDyn may use a wide range of packages, if available on the host system
and correctly detected by \kw{configure}.

\section{Mathematical Utilities}
MBDyn may exploit the availability of some mathematical utilities;
neither of them is required for a basic compilation, but they may 
be useful or required for specific features.

\subsection{ATLAS}\label{sec:DOWNLOAD:NUMERICAL:ATLAS}
The Automatically Tunable Linear Algebra Subroutines are
a replacement of the standard BLAS.
They are exploited by the linear solver Umfpack
(see Section~\ref{sec:DOWNLOAD:SOLVERS:UMFPACK})
and can be used by Lapack (see Section~\ref{sec:DOWNLOAD:SOLVERS:LAPACK})
and other packages that require BLAS
(see Section~\ref{sec:DOWNLOAD:NUMERICAL:BLAS}).

\subsection{BLAS}\label{sec:DOWNLOAD:NUMERICAL:BLAS}
The Basic Linear Algebra Subroutines
can be used by Umfpack (see Section~\ref{sec:DOWNLOAD:SOLVERS:UMFPACK})
and other packages.
Specially tuned binaries for each architecture (processor type,
cache size and otehr special hardware features) are advisable;
otherwise, instead of compiling them of your own, ATLAS
(see Section~\ref{sec:DOWNLOAD:NUMERICAL:ATLAS}) are considered
a better replacement.

\subsection{Metis}


\section{Linear Solvers}
MBDyn may use a variety of linear solvers

\subsection{Naive}
The Naive solver is also native.
It is especially meant for medium size problems (between 100 and 1000
unknowns) and is essentially a sparse solver optimized for speed 
instead of memory usage.

\subsection{Y12}
The Y12 solver is also builtin.
the MBDyn Project distributes the Y12 library 
\textbf{AS IS} and \textbf{WITHOUT ANY WARRANTY} as part 
of the source code, with no copyright statement and no license.
Of course credits go to the original Authors:
Zahari Zlatev, Jerzy Wasniewski, and Kjeld Schaumburg, 
Comp.\ Sci., Math.\ Inst., University of Aarhus, 
Ny Munkegade, DK 8000 Aarhus.
This solver is recommended for moderate to large problems,
if Umfpack (see Section~\ref{sec:DOWNLOAD:SOLVERS:UMFPACK}) 
is not available.

\subsection{Umfpack}\label{sec:DOWNLOAD:SOLVERS:UMFPACK}
The Umfpack linear sparse solver library must be downloaded separately, from
\htmladdnormallink{\kw{http://www.cise.ufl.edu/research/sparse/umfpack/}}{http://www.cise.ufl.edu/research/sparse/umfpack/}.
Credit goes to Timothy A.\ Davis, University of Florida.
Umfpack is used by permission; please read its Copyright,
License and Availability note.
It is used as the standard sparse solver by MATLAB (see
\htmladdnormallink{\kw{http://www.mathworks.com/}}{http://www.mathworks.com/}).
This solver is recommended for very large problems,
and as a general purpose solver.

\subsection{Lapack}

\subsection{SuperLU (experimental)}

\subsection{TAUCS (experimental)}

\subsection{Harwell (historical)}

\subsection{Meschach (historical)}



\section{Utilities}

\section{Communication}

\section{Real-Time}




\chapter{Building}

\chapter{Installing}

\chapter{Executing}

\chapter{Troubleshooting}

\chapter{Developers}

\bibliographystyle{unsrt}
\bibliography{mybib}


\end{document}
