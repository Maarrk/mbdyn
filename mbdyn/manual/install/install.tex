% $Header$
% MBDyn (C) is a multibody analysis code.
% http://www.mbdyn.org
%
% Copyright (C) 1996-2017
%
% Pierangelo Masarati  <masarati@aero.polimi.it>
%
% Dipartimento di Ingegneria Aerospaziale - Politecnico di Milano
% via La Masa, 34 - 20156 Milano, Italy
% http://www.aero.polimi.it
%
% Changing this copyright notice is forbidden.
%
% This program is free software; you can redistribute it and/or modify
% it under the terms of the GNU General Public License as published by
% the Free Software Foundation (version 2 of the License).
% 
%
% This program is distributed in the hope that it will be useful,
% but WITHOUT ANY WARRANTY; without even the implied warranty of
% MERCHANTABILITY or FITNESS FOR A PARTICULAR PURPOSE.  See the
% GNU General Public License for more details.
%
% You should have received a copy of the GNU General Public License
% along with this program; if not, write to the Free Software
% Foundation, Inc., 59 Temple Place, Suite 330, Boston, MA  02111-1307  USA

\documentclass[10pt,dvips]{report}

%\usepackage[pdftex]{graphicx}
\usepackage[T1]{fontenc}
\usepackage{ae,aecompl}
\usepackage{graphicx}
\usepackage{amsmath}
\usepackage[dvips,breaklinks=true,colorlinks=false]{hyperref}
\usepackage{html}

% $Header$
% Copyright (C) 1996-2013 Pierangelo Masarati <masarati@aero.polimi.it>
% Dipartimento di Ingegneria Aerospaziale, Politecnico di Milano
%
% Parentesi: tonde, quadre, curly, dritte, doppie e angolari.
\newcommand{\plbr}[1]{ \left( #1 \right) }
\newcommand{\sqbr}[1]{ \left[ #1 \right] }
\newcommand{\cubr}[1]{ \left\{ #1 \right\} }
\newcommand{\shbr}[1]{ \left| #1 \right| }
\newcommand{\nrbr}[1]{ \left\| #1 \right\| }
\newcommand{\anbr}[1]{ \langle #1 \rangle }

% Parentesi solo a sinistra: tonde, quadre, curly, dritte, doppie e angolari.
\newcommand{\lplbr}[1]{ \left( #1 \right. }
\newcommand{\lsqbr}[1]{ \left[ #1 \right. }
\newcommand{\lcubr}[1]{ \left\{ #1 \right. }
\newcommand{\lshbr}[1]{ \left| #1 \right. }
\newcommand{\lnrbr}[1]{ \left\| #1 \right. }
\newcommand{\lanbr}[1]{ \langle #1 \right. }

% Parentesi solo a destra: tonde, quadre, curly, dritte,doppie e angolari.
\newcommand{\rplbr}[1]{ \left. #1 \right) }
\newcommand{\rsqbr}[1]{ \left. #1 \right] }
\newcommand{\rcubr}[1]{ \left. #1 \right\} }
\newcommand{\rshbr}[1]{ \left. #1 \right| }
\newcommand{\rnrbr}[1]{ \left. #1 \right\| }
\newcommand{\ranbr}[1]{ \left. #1 \rangle }

% Vettori verticali:
\newcommand{\vvect}[2]{ \begin{array}{ #1 } #2 \end{array} }
\newcommand{\cvvect}[1]{ \begin{array}{c} #1 \end{array} }
\newcommand{\lvvect}[1]{ \begin{array}{l} #1 \end{array} }
\newcommand{\rvvect}[1]{ \begin{array}{r} #1 \end{array} }

% Vettori orizzontali:
\newcommand{\hvect}[2]{ \begin{array}{ #1 } #2 \end{array} }

% Matrici:
\newcommand{\matr}[2]{ \begin{array}{ #1 } #2 \end{array} }

% Integrali: uso \intg{inf}{sup}{arg}{dvar}
\newcommand{\intg}[4]{ \int_{#1}^{#2} {#3} \ {#4} }

% Limite: uso \limt{var}{lim}{arg}
\newcommand{\limt}[3]{ \lim_{{#1} \rightarrow {#2}} {#3}}

% LogLike functions
\newcommand{\llk}[1]{\ensuremath{\mathrm{#1}}}

\newcommand{\diag}[0]{\llk{diag}}
\newcommand{\tr}[0]{\llk{tr}}
\newcommand{\sym}[0]{\llk{sym}}
\newcommand{\skw}[0]{\llk{skw}}

\newcommand{\step}[0]{\llk{step}}
\newcommand{\imp}[0]{\llk{imp}}

\newcommand{\grad}[0]{\llk{grad}}
\newcommand{\divr}[0]{\llk{div}}
\newcommand{\rot}[0]{\llk{rot}}

% In italiano ...
\newcommand{\sca}[0]{\llk{sca}}

% first, second, etc
\newcommand{\first}[0]{1\ensuremath{^{\mathrm{st}}}}    % 1^st
\newcommand{\second}[0]{2\ensuremath{^{\mathrm{nd}}}}   % 2^nd
\newcommand{\third}[0]{3\ensuremath{^{\mathrm{rd}}}}    % 3^rd
\newcommand{\rth}[0]{\ensuremath{^{\mathrm{th}}}}       %  ^th

\newcommand{\degr}[0]{\ensuremath{^{\mathrm{o}}}}

% esponenziale
\providecommand{\e}[1]{\llk{e}^{#1}}


\newcommand{\kw}[1]{\texttt{#1}}
%\newcommand{\kw}[1]{\fbox{\texttt{#1}}}
\newcommand{\T}[1]{\boldsymbol{#1}}

\begin{document}

\begin{latexonly}
\title{\bf MBDyn Installation Manual \\
Version
0.1.0

}
\author{Pierangelo Masarati \vspace{5mm}\\
    \sc Dipartimento di Ingegneria Aerospaziale \\
    \sc Politecnico di Milano}
\date{Automatically generated \today}
\maketitle
\end{latexonly}

\begin{htmlonly}
\begin{center}
\textbf{\LARGE MBDyn Installation Manual}

\emph{\large Pierangelo Masarati}

\textsc{Dipartimento di Ingegneria Aerospaziale \\ Politecnico di Milano}

\today
\end{center}
\end{htmlonly}




\tableofcontents
\newpage

\chapter{Introduction}
This document describes how to download, build, install and execute
MBDyn --- MultiBody Analysis program, a suite of tools 
for multibody/multidisciplinary analysis of complex systems.

For any question or problem, to fix typos, bugs, for comments and
suggestions, please contact the Development Team
without hesitation:\vspace{10mm}\\

\noindent
\begin{tabular}{ll}
\multicolumn{2}{l}{Pierangelo Masarati,} \\
\multicolumn{2}{l}{MBDyn Development Team} \\
\multicolumn{2}{l}{Dipartimento di Ingegneria Aerospaziale} \\
\multicolumn{2}{l}{Politecnico di Milano} \\
\multicolumn{2}{l}{via La Masa 34, 20156 Milano, Italy} \\
Fax: & +39 02 2399 8334 \\
E-mail: & \htmladdnormallink{\kw{mbdyn@aero.polimi.it}}{mailto:mbdyn@aero.polimi.it} \\
Web: & \htmladdnormallink{\kw{http://www.aero.polimi.it/\~{}mbdyn/}}{http://www.aero.polimi.it/~mbdyn/}
\end{tabular}
\vspace{10mm}


\noindent
This document is also available online at \\
\htmladdnormallink{\kw{http://www.aero.polimi.it/\~{}masarati/MBDyn-input/install/}}{http://www.aero.polimi.it/~masarati/MBDyn-input/install/}



\chapter{Getting the package}

\section{MBDyn}
The package can be downloaded in source form from
\begin{center}
\htmladdnormallink{\kw{http://www.aero.polimi.it/\~{}mbdyn/}}{http://www.aero.polimi.it/\~{}mbdyn/}
\end{center}

Binary releases and snapshots are also available for Windows 2000/XP at 
\htmladdnormallink{\kw{http://www.aero.polimi.it/\~{}masarati/Download/mbdyn/}}{http://www.aero.polimi.it/~masarati/Download/mbdyn/},
compiled with Cygwin (see
\htmladdnormallink{\kw{http://www.cygwin.com/}}{http://www.cygwin.com/})
and thus require \kw{cygltdl-3.dll};
they are provided to Windows users to save them the burden of installing
Cygwin (very easy and straightforward, though) and to compile a package
under an unfamiliar environment.
However, no support is provided for those builds, unless problems 
are easily identifiable as related to the sources and not to the OS,
and they impact the UN*X version as well.

MBDyn may use a wide range of packages, if available on the host system
and correctly detected by \kw{configure}.

\section{Mathematical Utilities}
MBDyn may exploit the availability of some mathematical utilities;
neither of them is required for a basic compilation, but they may 
be useful or required for specific features.

\subsection{ATLAS}\label{sec:DOWNLOAD:NUMERICAL:ATLAS}
The Automatically Tunable Linear Algebra Subroutines are
a replacement of the standard BLAS.
They are exploited by the linear solver Umfpack
(see Section~\ref{sec:DOWNLOAD:SOLVERS:UMFPACK})
and can be used by Lapack (see Section~\ref{sec:DOWNLOAD:SOLVERS:LAPACK})
and other packages that require BLAS
(see Section~\ref{sec:DOWNLOAD:NUMERICAL:BLAS}).

Note that ATLAS provides its own implementation of Lapack
but only for a limited set of functions.
This set of functions is packed into an archive called \kw{liblapack.a},
so in typical installations it may shadow complete implementations
of the Lapack package.
A useful ``trick'' to have a complete Lapack installation that exploits
ATLAS performances where available is described in 
\htmladdnormallink{\kw{http://math-atlas.sourceforge.net/errata.html\#completelp}}{http://math-atlas.sourceforge.net/errata.html\#completelp}.
The essential instructions are reported below, assuming that
\kw{\$ATLAS} and \kw{\$LAPACK} respectively are the paths to your ATLAS
and Lapack static library archive files:
\begin{verbatim}
  mkdir tmp
  cd tmp
  ar x $ATLAS/liblapack.a
  cp $LAPACK/liblapack.a $ATLAS/liblapack.a
  ar r $ATLAS/liblapack.a *.o
  cd ..
  rm -rf tmp
\end{verbatim}


\subsection{BLAS}\label{sec:DOWNLOAD:NUMERICAL:BLAS}
The Basic Linear Algebra Subroutines
can be used by Umfpack (see Section~\ref{sec:DOWNLOAD:SOLVERS:UMFPACK})
and other packages.
Specially tuned binaries for each architecture (processor type,
cache size and other special hardware features) are advisable;
otherwise, instead of compiling them of your own, ATLAS
(see Section~\ref{sec:DOWNLOAD:NUMERICAL:ATLAS}) are considered
a better replacement.


\section{Linear Solvers}
MBDyn may use a variety of linear solvers

\subsection{Naive}
The Naive solver is also native.
It is especially meant for medium size problems (between 100 and 1000
unknowns) and is essentially a sparse solver optimized for speed 
rather than memory usage.

\subsection{Y12}
The Y12 solver is also built-in.
the MBDyn Project distributes the Y12 library 
\textbf{AS IS} and \textbf{WITHOUT ANY WARRANTY} as part 
of the source code, with no copyright statement and no license.
Of course credits go to the original Authors:
Zahari Zlatev, Jerzy Wasniewski, and Kjeld Schaumburg, 
Comp.\ Sci., Math.\ Inst., University of Aarhus, 
Ny Munkegade, DK 8000 Aarhus.
This solver is recommended for moderate to large problems,
if Umfpack (see Section~\ref{sec:DOWNLOAD:SOLVERS:UMFPACK}) 
is not available.

\subsection{Umfpack}\label{sec:DOWNLOAD:SOLVERS:UMFPACK}
The Umfpack linear sparse solver library must be downloaded separately, from
\htmladdnormallink{\kw{http://www.cise.ufl.edu/research/sparse/umfpack/}}{http://www.cise.ufl.edu/research/sparse/umfpack/}.
Credit goes to Timothy A.\ Davis, University of Florida.
Umfpack is used by permission; please read its Copyright,
License and Availability note.
It is used as the standard sparse solver by MATLAB (see
\htmladdnormallink{\kw{http://www.mathworks.com/}}{http://www.mathworks.com/}).
This solver is recommended for very large problems,
and as a general purpose solver.

\subsection{Lapack}
\label{sec:DOWNLOAD:SOLVERS:LAPACK}

\subsection{SuperLU (experimental)}
\label{sec:DOWNLOAD:SOLVERS:SUPERLU}

\subsection{TAUCS (experimental)}
\label{sec:DOWNLOAD:SOLVERS:TAUCS}

\subsection{Harwell (historical)}
\label{sec:DOWNLOAD:SOLVERS:HSL}

\subsection{Meschach (historical)}
\label{sec:DOWNLOAD:SOLVERS:MESCHACH}



\section{Utilities}

\section{Communication}

\subsection{MPI}

\subsection{Metis}
Metis is a package that performs automatic domain decomposition.
It is used by the Schur solution manager to partition the model
into submodels of nearly equal computational cost and with minimal
interface size.
Its compilation is straightforward.
It can be downloaded from
\htmladdnormallink{\kw{http://www-users.cs.umn.edu/\~{}karypis/metis/metis/}}{http://www-users.cs.umn.edu/~karypis/metis/metis/}.
Its compilation for MBDyn used to require a special patch to cleanup
its namespace; now the patch is no longer required as a better namespace
separation has been operated.
No installation procedure is provided as of version 4.0; at the end
of compilation, the library \kw{libmetis.a} is available in the build tree,
and it should be copied to a directory where the loader can locate it;
the headers \kw{Lib/*.h} should be copied where the C preprocessor
can locate them.

\section{Real-Time}

\subsection{RTAI}

\section{Build Environments}

\subsection{GNU/Linux}
The preferred build environment is GNU/Linux, using gcc, g++
and either g77 or gfortran.

\subsection{Windows: CygWin}
MBDyn has been successfully compiled under CygWin.
No special requirement is known.
Special attention is required to build with some specific feature,
significantly run-time loadable modules.

\subsection{Windows: MSYS/MinGW}
MBDyn has been successfully compiled under MSYS/MinGW.
Special attention needs to be played to what version is used.

Download packages from
\htmladdnormallink{\kw{http://www.mingw.org/}}{http://www.mingw.org/}.

Required packages (in order of installation):
\begin{itemize}
\item \kw{MinGW-3.1.0-1.exe} (install wizard)
\item \kw{MSYS-1.0.10.exe} (install wizard; if not automagically done,
edit file \kw{/etc/fstab} as indicated in MSYS' documentation)
\item \kw{msysDTK-1.0.1.exe} (install wizard)
\item \kw{gcc-core-3.4.2-20040916-1.tar.gz}
	(archive; untar after changing directory into \kw{/mingw})
\item \kw{gcc-g++-3.4.2-20040916-1.tar.gz}
	(archive; untar after changing directory into \kw{/mingw})
\item \kw{gcc-g77-3.4.2-20040916-1.tar.gz}
	(archive; untar after changing directory into \kw{/mingw})
\item \kw{msys-libtool-1.5.tar.bz2}
	(archive; untar after changing directory into \kw{/})
\end{itemize}



\subsection{PowerPC: Mac OS X}
(Reported by Torsten Sadowski, blindly reported here)

You have to tell \texttt{configure} where to find \texttt{fink}
includes and libraries
\begin{verbatim}
  export LDFLAGS=-L/sw/lib
  export CPPFLAGS=-I/sw/include
\end{verbatim}




\chapter{Building}
The configuration of the MBDyn package is based on GNU's autotools
(see \htmladdnormallink{\kw{http://www.gnu.org/}}{http://www.gnu.org/}
for details).

\section{Configuring}

\subsection{Option List}
Specific options (from \kw{configure -{}-help}):
{\small
\begin{verbatim}
  --enable-debug          enable debugging (no)
  --with-debug-mode[={none|mem}]  with debug mode {none|mem} (none)
  --enable-socket-drives  enable socket drives (auto)
  --enable-runtime-loading    enable runtime loading (auto)
  --with-static-modules   build (known) modules as static (auto)
  --enable-crypt          enable crypt (deprecated) (no)
  --enable-schur          enable Schur parallel solver
                          (needs MPI and either Metis or Chaco) (auto)
  --enable-multithread    enable multithread solution (no)
  --enable-adams          enable MSC.ADAMS output (no)
  --enable-motionview     enable Altair's Motion View output (no)
  --with-tcl              with tcl interpreters (auto)
  --with-libf2c[={f2c|g2c}] with f2c library (auto)
  --with-fs[={unix|dos}]  filesystem type (unix)
  --with-mpi              with MPI support (=pmpi for profiling) (auto)
  --enable-debug-mpi      enable MPI debugging (no)
  --with-metis            with Metis model partitioning support (auto)
  --with-threads          with threads (auto)
  --with-rtai             with RTAI support (no)

       math libraries:
  --with-blas             with (C)BLAS math library (auto)
  --with-goto=lib(s)      with Goto BLAS implementation
  --with-ginac            with GiNaC support
                          (ginac-config must be in $PATH) (auto)

       linear algebra solvers (naive is enabled by default):
  --with-y12              with Y12 sparse math library (yes)
  --with-umfpack          with Umfpack math library (auto)
  --with-lapack           with LAPACK math library (auto)
  --with-hsl              with HSL (Harwell) sparse math library - historical (auto)
  --with-meschach         with Meschach math library - historical (auto)
  --with-superlu          with SuperLU math library - eXperimental (auto)

       misc security libraries:
  --with-pam              with PAM support (auto)
  --with-sasl2            with Cyrus SASL2 support (auto)

       supported features:
  --with-struct           with structural elements (yes)
  --with-elec             with electric stuff (yes)
  --with-aero             with aerodynamic stuff (yes)
  --with-aero-output={std,gauss,node}  aerodynamic output mode (auto)
  --with-hydr             with hydraulic stuff (yes)

  --with-module=<list>    build listed modules (see modules/)
\end{verbatim}
}

\subsection{Multithread Assembly/Solution}
The switch \kw{-{}-enable-multithread} enables multithreaded assembly;
it defaults to \kw{auto}, i.e.\ if the system meets the following
requirements, it is automatically enabled.
Essentially, a working \kw{-lpthread} library and a POSIX compliant
\kw{pthread.h} header must be available.
Currently, only the SuperLU sparse threaded solver is available;
it is experimental.
A threaded implementation of the built-in \kw{naive} sparse solver 
is under development.

\subsection{Schur Parallel Solver}
The Schur parallel solver is enabled by using the switch
\kw{-{}-enable-schur}.
It requires a working MPI library with the \kw{ch} driver 
and the C++ interface, and a partitioning library.
The MPI library is selected by the switch \kw{-{}-with-mpi},
which allows to specify the value \kw{pmpi} if the profiling
version of the library is to be used.
Note that recent MPI releases by default only build the profiling
version of the C++ library, so the \kw{pmpi} value is mandatory.
Currently, the only partitioning library that is supported by MBDyn
is METIS; a patch to support Chaco is being incorporated,
and may be available in future releases.
The Schur solver is not compatible with the multithreaded
assembly/solution, so if no switch is specified and the system
meets the requirements for both, the multithreaded assembly wins
over the Schur solver.
As a consequence, the schur solver should be explicitly enabled.

\subsection{Real-Time Simulator}
The real-time simulator is enabled by using the switch
\kw{-{}-with-rtai}, which essentially detects the availability
of the mandatory headers of the GNU/Linux Real-Time Application
Interface.
These headers changed between 2.4.13 and 3.X; both versions 
are automatically detected.

\chapter{Installing}
Run \kw{make install}; this essentially installs the binary,
the utilities, the man page and the dynamic modules, if any.

\chapter{Executing}
Prepare an input file and run \kw{mbdyn -f <input> -o <output>}.

\section{Regular Execution}

\section{Parallel Execution}

\section{Real-Time Execution}

\section{External Execution}
To run MBDyn as a Simulink module, use the \kw{SimulinkInterface}
that is available under \kw{contrib/}.

\chapter{Troubleshooting}

\chapter{HOWTOs}
This chapter contains mini-howtos about typical significant 
configurations of MBDyn as performed by the developers.
Feel free to contribute your own if you had to do any unusual
configuring to meet special needs.

\section{Run-Time Modules}
To enable run-time loading of modules, a powerful means to extend
MBDyn functionality, one needs to:
\begin{itemize}
\item install, and let MBDyn detect, the ltdl library, an ancillary
library of libtool that handles platform-independent run-time loading
of software modules;
\item enable run-time loading of modules; this requires to configure
MBDyn with \kw{{-}{-}enable-runtime-loading};
\item when using gcc, it is recommended to add \kw{-rdynamic} to
the \kw{LDFLAGS} environment variable; this is required to make
functions and objects provided by MBDyn available in the modules;
\item when developing a custom module, it must be placed in a subdirectory
of the directory \kw{modules/}, named \kw{module-name/}.
The module should be contained in one file, with the same name
of the directory plus the language-specific extension; this should be
	\begin{itemize}
	\item C: \kw{module-name.c};
	\item C++: \kw{module-name.cc};
	\item Fortran 77: \kw{module-name.f}.
\end{itemize}
\item the building of each module must be explicitly enabled; 
this requires to configure with \kw{{-}{-}with-module=list}, 
providing a list of the module names (without the leading \kw{module-}).
\end{itemize}
For example, to enable run-time loading and to build the \kw{wheel2} module,
use
\begin{verbatim}
    $ ./configure --enable-runtime-loading --with-module=wheel2
\end{verbatim}
Occasionally, one may want to statically build some well-known modules
into MBDyn, specifically those that implement new elements.
This requires to configure with \kw{{-}{-}with-static-modules}.
After that, the user-defined elements are available as joints.
Currently, only the \kw{wheel2} module can be statically built
into MBDyn; this feature will require some reworking.

Run-time loading of dynamic modules is known to work on any flavor
of GNU/Linux on X86/X86\_64 and on Windows using Cygwin, when compiled 
with gcc.
It may work on other architectures and with other compilers, but it has
not been tested by, nor reported to, the Developers.


\section{Schur Solver}
This section describes how the Schur parallel solver available
within the MBDyn package has been successfully compiled and executed
on a dual Athlon SMP machine.
By no means it is intended to suggest how the related packages 
should be built for other purposes, nor it may represent a replacement
for the original build and install procedures.
Please refer to the documentation available with each package
for more details or for troubleshooting.

\noindent
Software prerequisites:
\begin{itemize}
\item gcc/g++/g77 >= 3.0 (tested with gcc/g++/g77 3.2.1, 3.3.1 and 3.4.0)
\end{itemize}
Software requirements:
\begin{itemize}
\item MBDyn 1.2.1
\item mpich 1.2.5.2
\item Metis 4.0
\end{itemize}

\paragraph{MPI}
\begin{itemize}
\item download \kw{mpich.tar.gz} from
\htmladdnormallink{\kw{http://www-unix.mcs.anl.gov/mpi/}}{http://www-unix.mcs.anl.gov/mpi/}
\item untar the package in a temporary directory
\item configure the package with
\begin{verbatim}
    $ ./configure -rsh=ssh --disable-f77 --prefix=/usr/local/mpich
\end{verbatim}
\item \kw{make} the package
\item \kw{make install}
\item edit \kw{/usr/local/mpich/share/machines.<arch>} to enumerate
the max number instances of the MBDyn process you want to allow
on each machine; in my case \kw{<arch>} is LINUX (more details in
\kw{/usr/local/mpich/doc/mpichman-chp4.pdf}).
\end{itemize}

\paragraph{Metis}
\begin{itemize}
\item download XXX from 
\htmladdnormallink{\kw{http://www-users.cs.umn.edu/\~{}karypis/metis/metis/}}{http://www-users.cs.umn.edu/~karypis/metis/metis/}
\item untar it in a temporary directory
\item apply the patch \kw{metis-namespace-cleanup.patch} from \\
\htmladdnormallink{\kw{http://www.aero.polimi.it/\~{}masarati/Download/mbdyn/}}{http://www.aero.polimi.it/~masarati/Download/mbdyn/}
\item \kw{make} the package
\item ...
\item copy the library \kw{libmetis.a} in \kw{/usr/local/lib}
and the header files \kw{defs.h}, \kw{macros.h}, \kw{metis.h},
\kw{proto.h}, \kw{rename.h} and \kw{struct.h} in \\
\kw{/usr/local/include/metis}.
\end{itemize}

\paragraph{MBDyn}
\begin{itemize}
\item download \kw{mbdyn-1.2.1.tar.gz} from \\
\htmladdnormallink{\kw{http://www.aero.polimi.it/\~{}masarati/Download/mbdyn/}}{http://www.aero.polimi.it/~masarati/Download/mbdyn/}
\item untar it in a temporary directory
\item make sure the compiler can find headers from MPI and Metis
by defining
\begin{verbatim}
    CPPFLAGS="-I/usr/local/mpich/include \
        -I/usr/local/mpich/include/mpi2c++ \
        -I/usr/local/include/metis"
\end{verbatim}
\item make sure the linker can find the libraries from MPI and Metis
by defining
\begin{verbatim}
    LDFLAGS="-L/usr/local/mpich/lib \
        -L/usr/local/lib"
\end{verbatim}
\item configure MBDyn by running
\begin{verbatim}
    $ ./configure --with-mpi=pmpi --with-metis --enable-schur \
        --prefix=/usr/local
\end{verbatim}
\item \kw{make} the package
\item \kw{make install} the package
\end{itemize}

\paragraph{Run It!} \mbox{} \\
To test the system, one needs a test input file; the example 
\kw{cantilever2} from
\htmladdnormallink{\kw{http://www.aero.polimi.it/\~{}mbdyn/documentation/examples/}}{http://www.aero.polimi.it/\~{}mbdyn/documentation/examples/}
should do the trick.
The input file does not need any specific change, unless special 
features are required.
To run it on an SMP machine, simply execute
\begin{verbatim}
    $ /usr/local/mpich/bin/mpirun -np 2 mbdyn -f cantilever2 -o /tmp/ -ss
\end{verbatim}
This generates a set of files \kw{/tmp/cantilever2.0.*}
and \kw{/tmp/cantilever2.1.*} with the outputs from processes 0 and 1.

\section{Real-Time Simulator}
This section describes how the Real-Time simulator available
within the MBDyn package has been successfully compiled and executed.
By no means it is intended to suggest how the related packages 
should be built for other purposes, nor it may represent a replacement
for the original build and install procedures.
Please refer to the documentation available with each package
for more details or for troubleshooting.

\noindent
Software prerequisites:
\begin{itemize}
\item gcc/g++/g77 >= 3.0 (tested with gcc 3.3.1 and 3.4.0)
\end{itemize}
Software requirements:
\begin{itemize}
\item MBDyn 1.2.1
\item RTAI 3.0
\end{itemize}

\paragraph{RTAI}
\begin{itemize}
\item download \kw{rtai-3.0.tar.gz} from
\htmladdnormallink{\kw{http://www.rtai.org/}}{http://www.rtai.org/}
\item untar the package in a temporary directory
\item configure the package with
\begin{verbatim}
$ ./configure 
\end{verbatim}
\item \kw{make} the package
\item \kw{make install}
\end{itemize}

\paragraph{MBDyn}
\begin{itemize}
\item download \kw{mbdyn-1.2.1.tar.gz} from \\
\htmladdnormallink{\kw{http://www.aero.polimi.it/\~{}masarati/Download/mbdyn/}}{http://www.aero.polimi.it/~masarati/Download/mbdyn/}
\item untar it in a temporary directory
\item make sure the compiler can find headers from RTAI
by defining
\begin{verbatim}
    CPPFLAGS="-I/home/realtime"
\end{verbatim}
\item configure MBDyn by running
\begin{verbatim}
    $ ./configure --with-rtai
\end{verbatim}
\item \kw{make} the package
\item \kw{make install} the package
\end{itemize}

\paragraph{Run It!} \mbox{} \\
To test the system, one needs a test input file; the example 
\kw{RT-MBDyn/pendulum} from
\htmladdnormallink{\kw{http://www.aero.polimi.it/\~{}mbdyn/documentation/examples/}}{http://www.aero.polimi.it/\~{}mbdyn/documentation/examples/}
should do the trick.
Simply execute
\begin{verbatim}
    $ mbdyn -f pendulum -ss
\end{verbatim}
...

\section{Simulink Interface}
\label{sec:SimulinkInterface}
This section describes how the Simulink Interface available
within the MBDyn package has been successfully compiled and executed.
By no means it is intended to suggest how the related packages 
should be built for other purposes, nor it may represent a replacement
for the original build and install procedures.
Please refer to the documentation available with each package
for more details or for troubleshooting.
An analogous interface with Scicos \cite{SCICOS}
is described in Section~\ref{sec:ScicosInterface}.

\noindent
Software prerequisites:
\begin{itemize}
\item gcc/g++/g77 >= 3.0 (tested with gcc 3.3.1 and 3.4.0)
\item Matlab/Simulink (tested with XXX)
\end{itemize}
Software requirements:
\begin{itemize}
\item MBDyn 1.2.1
\end{itemize}

\paragraph{MBDyn}
\begin{itemize}
\item download \kw{mbdyn-1.2.1.tar.gz} from \\
\htmladdnormallink{\kw{http://www.aero.polimi.it/\~{}masarati/Download/mbdyn/}}{http://www.aero.polimi.it/~masarati/Download/mbdyn/}
\item untar it in a temporary directory
\item configure MBDyn by running
\begin{verbatim}
    $ ./configure 
\end{verbatim}
\item \kw{make} the package
\item \kw{make install} the package
\item in directory \kw{contrib/SimulinkInterface}, edit the file 
\kw{Makefile} such that the appropriate \kw{mex} compiler is used;
in my system it is \kw{/opt/matlab/bin/mex}
\item \kw{make} the package
\end{itemize}

\paragraph{Run It!} \mbox{} \\
To test the system, one needs a test input file; the example 
\kw{pendulum} in the subdirectory \kw{examples} 
should do the trick.
\begin{itemize}
\item start Matlab
\item add the subdirectory \kw{contrib/SimulinkInterface} to Matlab's 
path, by executing
\begin{verbatim}
    > path('<path/to>/contrib/SimulinkInterface', path);
\end{verbatim}
\item execute simulink by clicking on the related icon, or by running
\begin{verbatim}
    > simulink
\end{verbatim}
\item open the model by clicking \kw{File->Open} on the menubar
\item run the model by clicking \kw{Simulation->Start} on the menubar.
some times, the first run fails; we're still trying to track that down.
In case, just try again...
\item for more details on creating your own model, or on editing
the interface parameters, refer to the \kw{README} 
in \kw{contrib/SimulinkInterface}
\end{itemize}

\section{Scicos Interface}
\label{sec:ScicosInterface}
This section describes how the Scicos \cite{SCICOS} interface available
within the MBDyn package has been successfully compiled and executed.
By no means it is intended to suggest how the related packages 
should be built for other purposes, nor it may represent a replacement
for the original build and install procedures.
Please refer to the documentation available with each package
for more details or for troubleshooting.

An analogous interface with Simulink \cite{SIMULINK}
is described in Section~\ref{sec:SimulinkInterface}.

\begin{comment}

\bigskip

\noindent
Software prerequisites:
\begin{itemize}
\item gcc/g++/g77 >= 3.0 (tested with gcc 3.3.1 and 3.4.0)
\item Matlab/Simulink (tested with XXX)
\end{itemize}
Software requirements:
\begin{itemize}
\item MBDyn 1.2.1
\end{itemize}

\paragraph{MBDyn}
\begin{itemize}
\item download \kw{mbdyn-1.2.1.tar.gz} from \\
\htmladdnormallink{\kw{http://www.aero.polimi.it/\~{}masarati/Download/mbdyn/}}{http://www.aero.polimi.it/~masarati/Download/mbdyn/}
\item untar it in a temporary directory
\item configure MBDyn by running
\begin{verbatim}
    $ ./configure 
\end{verbatim}
\item \kw{make} the package
\item \kw{make install} the package
\item in directory \kw{contrib/SimulinkInterface}, edit the file 
\kw{Makefile} such that the appropriate \kw{mex} compiler is used;
in my system it is \kw{/opt/matlab/bin/mex}
\item \kw{make} the package
\end{itemize}

\paragraph{Run It!} \mbox{} \\
To test the system, one needs a test input file; the example 
\kw{pendulum} in the subdirectory \kw{examples} 
should do the trick.
\begin{itemize}
\item start Matlab
\item add the subdirectory \kw{contrib/SimulinkInterface} to Matlab's 
path, by executing
\begin{verbatim}
    > path('<path/to>/contrib/SimulinkInterface', path);
\end{verbatim}
\item execute simulink by clicking on the related icon, or by running
\begin{verbatim}
    > simulink
\end{verbatim}
\item open the model by clicking \kw{File->Open} on the menubar
\item run the model by clicking \kw{Simulation->Start} on the menubar.
some times, the first run fails; we're still trying to track that down.
In case, just try again...
\item for more details on creating your own model, or on editing
the interface parameters, refer to the \kw{README} 
in \kw{contrib/SimulinkInterface}
\end{itemize}

\end{comment}




\chapter{Developers}

\section{Prepare for Building}
Developers need to prepare the build environment by running
\begin{verbatim}
    $ ./bootstrap.sh
\end{verbatim}
This need appropriate versions of the autotools:
\begin{itemize}
\item \kw{automake}: there seems to be no special version requirement;
\item \kw{autoconf}: there seems to be no special version requirement;
\item \kw{libtool}: needs to be at least version 1.5.
\end{itemize}

\bibliographystyle{unsrt}
\bibliography{../mybib,../mbdyn}


\end{document}
