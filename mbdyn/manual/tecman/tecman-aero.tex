% MBDyn (C) is a multibody analysis code.
% http://www.mbdyn.org
%
% Copyright (C) 1996-2009
%
% Pierangelo Masarati  <masarati@aero.polimi.it>
%
% Dipartimento di Ingegneria Aerospaziale - Politecnico di Milano
% via La Masa, 34 - 20156 Milano, Italy
% http://www.aero.polimi.it
%
% Changing this copyright notice is forbidden.
%
% This program is free software; you can redistribute it and/or modify
% it under the terms of the GNU General Public License as published by
% the Free Software Foundation (version 2 of the License).
% 
%
% This program is distributed in the hope that it will be useful,
% but WITHOUT ANY WARRANTY; without even the implied warranty of
% MERCHANTABILITY or FITNESS FOR A PARTICULAR PURPOSE.  See the
% GNU General Public License for more details.
%
% You should have received a copy of the GNU General Public License
% along with this program; if not, write to the Free Software
% Foundation, Inc., 59 Temple Place, Suite 330, Boston, MA  02111-1307  USA
%
% Alessandro Fumagalli <fumagalli@aero.polimi.it> is the author of this document

Aerodynamic elements apply aerodynamic forces to structural nodes.

This section is not intended 
to give details about the aerodynamic models adopted but mainly 
discuss the computation of the contributions to the Jacobian 
matrix of the aerodynamic elements.

\paragraph{Files.} \
It is implemented in files\\
\begin{tabular}{l}
\texttt{mbdyn/aero/aeroelem.h} \\
\texttt{mbdyn/aero/aeroelem.cc} 
\end{tabular}

\section{Linearization of Sectional Aerodynamic Forces and Moments}
MBDyn's built-in 2D aerodynamics computes
aerodynamic forces $\tilde{\T{f}}_{a/\xi}$
and moments $\tilde{\T{c}}_{a/\xi}$ per unit span,
in a relative frame at station $\xi$,
based on the instantaneous value of linear and angular velocity
boundary conditions, respectively $\tilde{\T{v}}$ and $\tilde{\T{\omega}}$,
expressed in the same relative frame, namely
\begin{subequations}
\begin{align}
	\tilde{\T{f}}_{a/\xi}
	&=
	\tilde{\T{f}}_{a/\xi}\plbr{\tilde{\T{v}}, \tilde{\T{\omega}}, \xi}
	\\
	\tilde{\T{c}}_{a/\xi}
	&=
	\tilde{\T{c}}_{a/\xi}\plbr{\tilde{\T{v}}, \tilde{\T{\omega}}, \xi}
	.
\end{align}
\end{subequations}

The boundary conditions are computed by projecting the \emph{effective}
linear and angular velocity at station $\xi$,
respectively $\T{v}\plbr{\xi}$ and $\T{\omega}\plbr{\xi}$,
in the reference frame of the aerodynamic forces,
namely
\begin{subequations}
\begin{align}
	\tilde{\T{v}}\plbr{\xi}
	&=
	\TT{R}^T\plbr{\xi} \T{v}\plbr{\xi}
	\\
	\tilde{\T{\omega}}\plbr{\xi}
	&=
	\TT{R}^T\plbr{\xi} \T{\omega}\plbr{\xi}
	,
\end{align}
\end{subequations}
where $\TT{R}\plbr{\xi}$ is the matrix that expresses
the local orientation of the aerodynamic section at station $\xi$.

In detail, the effective linear velocity at an arbitrary station
is the combination of the absolute velocity resulting from the kinematics
of the model, of an airstream velocity that may depend
on the absolute location of a reference point and on time,
and of a contribution resulting from an inflow model, namely
\begin{align}
	\T{v}\plbr{\xi}
	&=
	\T{v}_{\text{kin}}\plbr{\xi}
	+ \T{v}_{\infty}\plbr{\T{x}\plbr{\xi}, t}
	+ \T{v}_{\text{inflow}}\plbr{\T{x}\plbr{\xi}}
	.
\end{align}
It is assumed that the last two contributions do not depend
on the state of the problem, or only depend on it in a loose manner,
and thus do not directly participate in the linearization
of the aerodynamic forces and moments.

Their linearization yields
\begin{subequations}
\begin{align}
	\delta\tilde{\T{v}}
	&=
	\TT{R}^T\plbr{\xi} \plbr{
		\delta\T{v}_{\text{kin}}\plbr{\xi}
		+ \T{v}\plbr{\xi} \times \T{\theta}_{\delta}\plbr{\xi}
	}
	\\
	\delta\tilde{\T{\omega}}
	&=
	\TT{R}^T\plbr{\xi} \plbr{
		\delta\T{\omega}\plbr{\xi}
		+ \T{\omega}\plbr{\xi} \times \T{\theta}_{\delta}\plbr{\xi}
	}
	,
\end{align}
\end{subequations}
where
\begin{align}
	\T{\theta}_{\delta}\plbr{\xi}
	&=
	\text{ax}\plbr{\delta\TT{R}\plbr{\xi} \TT{R}^T\plbr{\xi}}
	.
\end{align}

By means of numerical integration, the force and moment per unit span
are integrated into discrete contributions to the force and the moment
applied to the appropriate node equilibrium.
This is usually done by multiplying each force and moment per unit span
contribution by an appropriate reference length coefficient.
As a consequence, in the following force and moment contributions
will be considered, namely
\begin{subequations}
\begin{align}
	\tilde{\T{f}}_a
	&=
	\tilde{\T{f}}_a\plbr{\tilde{\T{v}}, \tilde{\T{\omega}}, \xi}
	\\
	\tilde{\T{c}}_a
	&=
	\tilde{\T{c}}_a\plbr{\tilde{\T{v}}, \tilde{\T{\omega}}, \xi}
	.
\end{align}
\end{subequations}
As already mentioned, each weighted sectional force and moment contribution
is applied to the appropriate node after projection in the global reference
frame.
The contribution to the force and moment equilibrium of the $n$-th node is
\begin{subequations}
	\label{eq:aero:fm-contrib}
\begin{align}
	\Delta\T{f}_n
	&=
	\TT{R}\plbr{\xi} \tilde{\T{f}}_a
	\label{eq:aero:fm-contrib:f}
	\\
	\Delta\T{c}_n
	&=
	\TT{R}\plbr{\xi} \tilde{\T{c}}_a
	+ \plbr{\T{x}\plbr{\xi} - \T{x}_n} \times \Delta\T{f}_n
	\label{eq:aero:fm-contrib:m}
	.
\end{align}
\end{subequations}
Their linearization yields
\begin{subequations}
\begin{align}
	\delta\Delta\T{f}_n
	&=
	- \Delta\T{f}_n \times \T{\theta}_{\delta}\plbr{\xi}
	+ \TT{R}\plbr{\xi} \delta\tilde{\T{f}}_a
	\\
	\delta\Delta\T{c}_n
	&=
	- \Delta\T{c}_n \times \T{\theta}_{\delta}\plbr{\xi}
	+ \TT{R}\plbr{\xi} \delta\tilde{\T{c}}_a
	- \Delta\T{f}_n \times \plbr{
		\delta\T{x}\plbr{\xi}
		- \delta\T{x}_n
	}
	\nonumber \\
	&\hphantom{= } \mbox{}
	+ \plbr{
		\T{x}\plbr{\xi}
		- \T{x}_n
	} \times \plbr{
		- \Delta\T{f}_n \times \T{\theta}_{\delta}\plbr{\xi}
		+ \TT{R}\plbr{\xi} \delta\tilde{\T{f}}_a
	}
	,
\end{align}
\end{subequations}
which can be summarized as
\begin{align}
	\cubr{\cvvect{
		\delta\Delta\T{f}_n \\
		\delta\Delta\T{c}_n
	}}
	&=
	\sqbr{\cvvect{
		- \Delta\T{f}_n \times{} \\
		- \Delta\T{c}_n \times{}
		- \plbr{\T{x}\plbr{\xi} - \T{x}_n} \times \Delta\T{f}_n \times{}
	}} \T{\theta}_{\delta}\plbr{\xi}
	\nonumber \\
	& \hphantom{= } \mbox{}
	+ \sqbr{\cvvect{
		\TT{0} \\
		- \Delta\T{f}_n \times{}
	}} \plbr{
		\delta\T{x}\plbr{\xi}
		- \delta\T{x}_n
	}
	\nonumber \\
	& \hphantom{= } \mbox{}
	+ \sqbr{\matr{cc}{
		\TT{I} & \TT{0} \\
		\plbr{\T{x}\plbr{\xi} - \T{x}_n} \times{} & \TT{I}
	}} \cubr{\cvvect{
		\TT{R}\plbr{\xi} \delta\tilde{\T{f}}_a \\
		\TT{R}\plbr{\xi} \delta\tilde{\T{c}}_a
	}}
	.
\end{align}

It is assumed that the Jacobian matrix of the sectional force and moment
with respect to the linear and angular velocity
is either available or can be computed by numerical differentiation.
The resulting force and moment perturbation is
\begin{align}
	\cubr{\cvvect{
		\delta\tilde{\T{f}}_a \\
		\delta\tilde{\T{c}}_a
	}} &= \sqbr{\matr{cc}{
		\tilde{\T{f}}_{a/\tilde{\T{v}}} & \tilde{\T{f}}_{a/\tilde{\T{\omega}}} \\
		\tilde{\T{c}}_{a/\tilde{\T{v}}} & \tilde{\T{c}}_{a/\tilde{\T{\omega}}}
	}} \cubr{\cvvect{
		\delta\tilde{\T{v}} \\
		\delta\tilde{\T{\omega}}
	}}
	\label{eq:aero:jac}
	.
\end{align}
The computation of the matrix of Eq.~(\ref{eq:aero:jac})
is delegated to the \texttt{AeroData} class.

Each type of element determines how the sectional force and moment
contributions are applied to the nodes, and how the sectional
boundary conditions at each section are computed from the kinematics
of the nodes.



\section{Numerical Linearization of Sectional Forces}
Consider an arbitrary submatrix of the Jacobian matrix
of Eq.~(\ref{eq:aero:jac}), $\TT{J}=\T{p}_{/\T{q}}$.
Its generic element, the $c$-th component of $\T{p}$
derived by the $r$-th component of $\T{q}$, is
\begin{align}
	J_{rc} &= \frac{\partial\T{p}_r}{\partial\T{q}_c}
	.
\end{align}
A forward difference approach is used, namely
\begin{align}
	J_{rc} &\cong \frac{
		\T{p}_r\plbr{\T{q} + \Delta q \T{e}_c}
		- \T{p}_r\plbr{\T{q}}
	}{
		\Delta q
	}
	,
\end{align}
where $\T{e}_c$ is the unit vector along the $c$-th component,
and $\Delta q$ is a suitably chosen perturbation.
Alternatively, a centered difference approach can be used, namely
\begin{align}
	J_{rc} &\cong \frac{
		\T{p}_r\plbr{\T{q} + \Delta q \T{e}_c}
		- \T{p}_r\plbr{\T{q} - \Delta q \T{e}_c}
	}{
		2 \Delta q
	}
	.
\end{align}
The perturbation is
\begin{align}
	\Delta q &= \varepsilon \nrbr{\T{q}} + \nu
	.
\end{align}
Since the boundary condition $\T{q}$ is perturbed in order to determine
an equivalent perturbation of angle of attack, a resolution of few tenth
of degree is deemed sufficient.
As a consequence, $\varepsilon>0$ must be a ``small'' number
that, in case $\T{e}_c$ is orthogonal to $\T{q}$,
yields an angle of the order of a tenth of a degree.
The default value is $\varepsilon=10^{-3}$.
However, in order to avoid divisions by too small numbers,
the perturbation is corrected by another ``small'' parameter,
$\nu>0$.
The default value is $\nu=10^{-9}$.




\section{Aerodynamic Forces with Internal States}
Consider the case of a model of the aerodynamic forces
that requires the use of internal states $\T{q}$, namely
\begin{subequations}
\begin{align}
	\tilde{\T{f}}_a
	&=
	\tilde{\T{f}}_a\plbr{\tilde{\T{v}},\tilde{\T{\omega}},\T{q}}
	\\
	\tilde{\T{c}}_a
	&=
	\tilde{\T{c}}_a\plbr{\tilde{\T{v}},\tilde{\T{\omega}},\T{q}}
	\\
	\T{0}
	&=
	\T{g}\plbr{\tilde{\T{v}},\tilde{\T{\omega}},\T{q}, \dot{\T{q}}}
	\label{eq:aero:states:g}
	.
\end{align}
\end{subequations}
Usually, the dynamic model of the aerodynamics is differential,
and thus its perturbation yields a linearized state-space system,
\begin{subequations}
\begin{align}
	\delta\T{g}
	&=
	\T{g}_{/\tilde{\T{v}}} \delta\tilde{\T{v}}
	+ \T{g}_{/\tilde{\T{\omega}}} \delta\tilde{\T{\omega}}
	+ \T{g}_{/\T{q}} \delta\T{q}
	+ \T{g}_{/\dot{\T{q}}} \delta\dot{\T{q}}
	\label{eq:aero:states:delta-g}
	\\
	\delta \tilde{\T{f}}_a
	&=
	\tilde{\T{f}}_{a/\tilde{\T{v}}} \delta\tilde{\T{v}}
	+ \tilde{\T{f}}_{a/\tilde{\T{\omega}}} \delta\tilde{\T{\omega}}
	+ \tilde{\T{f}}_{a/\T{q}} \delta\T{q}
	\label{eq:aero:states:delta-f}
	\\
	\delta \tilde{\T{c}}_a
	&=
	\tilde{\T{c}}_{a/\tilde{\T{v}}} \delta\tilde{\T{v}}
	+ \tilde{\T{c}}_{a/\tilde{\T{\omega}}} \delta\tilde{\T{\omega}}
	+ \tilde{\T{c}}_{a/\T{q}} \delta\T{q}
	\label{eq:aero:states:delta-c}
	.
\end{align}
\end{subequations}
In those cases, the underlying aerodynamic model has
to deal with $\T{g}$, Eq.~(\ref{eq:aero:states:g}),
and its perturbation $\delta\T{g}$, Eq.~(\ref{eq:aero:states:g}).

However, the aerodynamic elements have to:
\begin{enumerate}

\item provide the underlying aerodynamic model the Jacobian submatrices
required to compute $\delta\tilde{\T{v}}$ and $\delta\tilde{\T{\omega}}$
from the perturbations of the nodal position, orientation,
and linear and angular velocity that are needed to deal with $\delta\T{g}$,
Eq.~(\ref{eq:aero:states:g});

\item account for the contribution of the Jacobian matrices
$\delta\tilde{\T{f}}_{a/\T{q}}$ and $\delta\tilde{\T{f}}_{a/\dot{\T{q}}}$
respectively required by Eqs.~(\ref{eq:aero:states:delta-f})
and~(\ref{eq:aero:states:delta-c}).
\end{enumerate}




\section{Aerodynamic body}
The \texttt{aerodynamic body} element applies aerodynamic forces
to the structural node it is connected to,
based on the relative velocity between an aerodynamic surface attached 
to the node and the airstream.

The boundary conditions are related to the rigid body motion
of the node, so
\begin{subequations}
\begin{align}
	\TT{R}\plbr{\xi}
	&=
	\TT{R}_n \TT{R}_a \TT{R}_t\plbr{\xi}
	\\
	\T{b}\plbr{\xi}
	&=
	\TT{R}_n\plbr{
		\tilde{\T{b}}_0
		+ \TT{R}_a \plbr{
			b\plbr{\xi} \T{e}_1
			+ \xi \T{e}_3
		}
	}
	\\
	\T{\omega}\plbr{\xi}
	&=
	\T{\omega}_n
	\\
	\T{v}_{\text{kin}}\plbr{\xi}
	&=
	\T{v}_n + \T{\omega}_n \times \T{b}\plbr{\xi}
	,
\end{align}
\end{subequations}
where
$\TT{R}_n$ is the orientation of the node,
$\TT{R}_a$ is the relative orientation of the aerodynamics
with respect to the node,
$\TT{R}_t$ is the pretwist matrix,
$\T{v}_n$ is the absolute velocity of the node,
$\T{\omega}_n$ is the absolute angular velocity of the node,
$\tilde{\T{b}}_0$ is an offset between the node and the reference location
of the aerodynamic body,
and $b\plbr{\xi}$ is the chordwise location of the point
where the boundary conditions are evaluated.

Their linearization is straightforward:
\begin{subequations}
\begin{align}
	\T{\theta}_{\delta}\plbr{\xi}
	&=
	\T{\theta}_{n\delta}
	\\
	\delta\T{b}\plbr{\xi}
	&=
	- \T{b}\plbr{\xi}\times\T{\theta}_{n\delta}
	\\
	\delta\T{\omega}\plbr{\xi}
	&=
	\delta\T{\omega}_n
	\\
	\delta\T{v}_{\text{kin}}\plbr{\xi}
	&=
	\delta\T{v}_n
	- \T{b}\plbr{\xi}\times\delta\T{\omega}_n
	- \T{\omega}_n \times \T{b}\plbr{\xi} \times \T{\theta}_{n\delta}
	.
\end{align}
\end{subequations}
Eq.~(\ref{eq:aero:fm-contrib:m}) can be rewritten as
\begin{align}
	\Delta\T{c}_n
	&=
	\TT{R}\plbr{\xi}\plbr{
		\tilde{\T{c}}_a
		+ \tilde{\T{o}}\plbr{\xi} \times \tilde{\T{f}}_a
	} ,
\end{align}
where
$\tilde{\T{o}}\plbr{\xi}=\TT{R}^T\plbr{\xi}\plbr{\T{x}\plbr{\xi} - \T{x}_n}$,
the offset between the point where the force is applied and the node, 
in the reference frame of the node, does not depend on the kinematics
of the system, since the body is rigid.
Its linearization yields
\begin{align}
	\delta\Delta\T{c}_n
	&=
	- \Delta\T{c}_n \times \T{\theta}_{\delta}\plbr{\xi}
	+ \TT{R}\plbr{\xi} \plbr{
		\delta\tilde{\T{c}}_a
		+ \tilde{\T{o}}\plbr{\xi} \times \delta\tilde{\T{f}}_a
	}
	,
\end{align}
since $\delta\tilde{\T{o}}\plbr{\xi}\equiv{0}$.
So the linearized force and moment is
\begin{align}
	\cubr{\cvvect{
		\delta\Delta\T{f}_n \\
		\delta\Delta\T{c}_n
	}}
	&=
	- \sqbr{\cvvect{
		\Delta\T{f}_n \times{} \\
		\Delta\T{c}_n \times{}
	}} \T{\theta}_{n\delta}
	+ \sqbr{\matr{cc}{
		\TT{I} & \TT{0} \\
		\T{o}\plbr{\xi}\times{} & \TT{I}
	}} \cubr{\cvvect{
		\TT{R}\plbr{\xi} \delta\tilde{\T{f}}_a \\
		\TT{R}\plbr{\xi} \delta\tilde{\T{c}}_a
	}}
	.
\end{align}
with $\T{o}\plbr{\xi}=\TT{R}\plbr{\xi} \tilde{\T{o}}\plbr{\xi}$.

The linearization of the boundary conditions yields
\begin{subequations}
\begin{align}
	\cubr{\cvvect{
		\delta\tilde{\T{v}}\plbr{\xi} \\
		\delta\tilde{\T{\omega}}\plbr{\xi}
	}}
	&=
	\sqbr{\matr{cc}{
		\TT{R}^T\plbr{\xi} & \TT{0} \\
		\TT{0} & \TT{R}^T\plbr{\xi}
	}} \lplbr{
		\sqbr{\cvvect{
			\T{v}\plbr{\xi}\times{}
				- \T{\omega}_n\times\T{b}\plbr{\xi}\times{}
			\\
			\T{\omega}_n\times{}
		}} \T{\theta}_{n\delta}
	} \nonumber \\
	& \hphantom{= } \rplbr{
		\mbox{} + \sqbr{\matr{cc}{
			\TT{I} & -\T{b}\plbr{\xi}\times{} \\
			\TT{0} & \TT{I}
		}} \cubr{\cvvect{
			\delta\T{v}_n \\
			\delta\T{\omega}_n
		}}
	}
	\\
	&\equu 
	\sqbr{\matr{cc}{
		\TT{R}^T\plbr{\xi} & \TT{0} \\
		\TT{0} & \TT{R}^T\plbr{\xi}
	}} \lplbr{
		\sqbr{\cvvect{
			\plbr{\T{v}\plbr{\xi} + \T{b}\plbr{\xi}\times \T{\omega}_n}\times{}
			\\
			\T{0}
		}} \delta\T{g}_n
	} \nonumber \\
	& \hphantom{= } \rplbr{
		\mbox{} + \sqbr{\matr{cc}{
			\TT{I} & -\T{b}\plbr{\xi}\times{} \\
			\TT{0} & \TT{I}
		}} \cubr{\cvvect{
			\delta\dot{\T{x}}_n \\
			\delta\dot{\T{g}}_n
		}}
	}
	.
\end{align}
\end{subequations}
After defining
\begin{subequations}
\begin{align}
	\T{f}_{a/\tilde{\T{v}}} &= \TT{R}\plbr{\xi} \tilde{\T{f}}_{a/\tilde{\T{v}}} \TT{R}^T\plbr{\xi} \\
	\T{f}_{a/\tilde{\T{\omega}}} &= \TT{R}\plbr{\xi} \tilde{\T{f}}_{a/\tilde{\T{\omega}}} \TT{R}^T\plbr{\xi} \\
	\T{c}_{a/\tilde{\T{v}}} &= \TT{R}\plbr{\xi} \tilde{\T{c}}_{a/\tilde{\T{v}}} \TT{R}^T\plbr{\xi} \\
	\T{c}_{a/\tilde{\T{\omega}}} &= \TT{R}\plbr{\xi} \tilde{\T{c}}_{a/\tilde{\T{\omega}}} \TT{R}^T\plbr{\xi}
	,
\end{align}
\end{subequations}
the linearization becomes
\begin{align}
	\sqbr{\matr{c}{
		\T{f}_{a/\tilde{\T{v}}}
		\\
		\T{o}\plbr{\xi}\times\T{f}_{a/\tilde{\T{v}}} + \T{c}_{a/\tilde{\T{v}}}
	}} \delta\dot{\T{x}}_n
	\nonumber \\
	+ \sqbr{\matr{c}{
		\T{f}_{a/\tilde{\T{\omega}}} - \T{f}_{a/\tilde{\T{v}}} \T{b}\plbr{\xi}\times{}
		\\
		\T{o}\plbr{\xi} \times \T{f}_{a/\tilde{\T{\omega}}}
		+ \T{c}_{a/\tilde{\T{\omega}}}
		- \plbr{
			\T{o}\plbr{\xi} \times \T{f}_{a/\tilde{\T{v}}}
			+ \T{c}_{a/\tilde{\T{v}}}
		}
		\T{b}\plbr{\xi}\times{}
	}} \delta\dot{\T{g}}_n
	\nonumber \\
	+ \sqbr{\matr{c}{
		\T{f}_{a/\tilde{\T{v}}} \plbr{\T{v}\plbr{\xi} + \T{b}\plbr{\xi} \times \T{\omega}_n} \times{} \\
		\plbr{
			\T{o}\plbr{\xi} \times \T{f}_{a/\tilde{\T{v}}}
			+ \T{c}_{a/\tilde{\T{v}}}
		} \plbr{\T{v}\plbr{\xi} + \T{b}\plbr{\xi} \times \T{\omega}_n} \times{}
	}} \delta\T{g}_n
	\nonumber \\
	- \sqbr{\cvvect{
		\Delta\T{f}_n \times{} \\
		\Delta\T{c}_n \times{} \\
	}} \delta\T{g}_n
	\equu
	\cubr{\cvvect{
		\delta\Delta\T{f}_n \\
		\delta\Delta\T{c}_n
	}}
	.
\end{align}





\section{Aerodynamic Beam (3 Nodes)}
The \texttt{aerodynamic beam3} element applies aerodynamic forces
to the nodes of a three node structural beam element. 

The aerodynamic forces/moments acting on each node 
are computed based on the relative velocity of a set of locations
along the beam and the airstream.
The kinematic quantities of the beam are computed based
on an interpolation of the kinematics of the three nodes.

\paragraph{Kinematics Interpolation.}
\emph{Note: this part is common to all elements that use the
three-node beam discretization and interpolation model.}
The generic field variable $\T{p}\plbr{x}$ is interpolated
using parabolic functions related to the value of the field 
variable at three locations that in general can be offset
from the nodes,
\begin{align}
	\T{p}\plbr{\xi}
	&=
	\sum_{i=1,2,3} N_i\plbr{\xi} \T{p}_i
	,
\end{align}
with
\begin{subequations}
\begin{align}
	N_1 &= \frac{1}{2} \xi \plbr{\xi - 1}
	\\
	N_2 &= 1 - \xi^2
	\\
	N_3 &= \frac{1}{2} \xi \plbr{\xi + 1}
	.
\end{align}
\end{subequations}

Orientation at an arbitrary location $\xi$:
\begin{align}
	\TT{R}\plbr{\xi}
	&=
	\TT{R}_{2_a} \TT{R}\plbr{\T{\theta}\plbr{\xi}}
	\\
	\T{\theta}\plbr{\xi}
	&=
	\sum_{i=1,3} N_i\plbr{\xi} \T{\theta}_{2\leftarrow i}
	= N_1\plbr{\xi} \T{\theta}_{2\leftarrow 1} + N_3\plbr{\xi} \T{\theta}_{2\leftarrow 3}
	\\
	\T{\theta}_{2\leftarrow i}
	&=
	\text{ax}\plbr{\text{exp}^{-1}\plbr{\TT{R}_{2_a}^T \TT{R}_{i_a}}}
	.
\end{align}
Orientation is dealt with specially, given its special nature.
The orientation of the mid node is used as a reference,
and the orientation parameters that express the relative orientation
between each of the end nodes and the mid node are interpolated.
The interpolated orientation parameters are used to compute
the interpolated relative orientation matrix, which is then
pre-multiplied by the orientation matrix of the mid node.
Summation in this case occurs on $i=1,3$ only because by definition
$\T{\theta}_{2\leftarrow 2}=\text{ax}(\text{exp}^{-1}((\TT{I}))\equiv\T{0}$.

Since the Euler-Rodrigues orientation parameters are used
(the so-caller `rotation vector'),
the magnitude of the relative orientation between each end node
and the mid node must be limited (formally, to $\pi$,
but it should be less for accuracy).

Position at an arbitrary location $\xi$:
\begin{align}
	\T{x}\plbr{\xi}
	&=
	\sum_{i=1,2,3} N_i\plbr{\xi} \plbr{
		\T{x}_i + \T{o}_i
	}
	,
\end{align}
with $\T{o}_i=\TT{R}_i \tilde{\T{o}}_i$.

Angular velocity at an arbitrary location $\xi$:
\begin{align}
	\T{\omega}\plbr{\xi}
	&=
	\sum_{i=1,2,3} N_i\plbr{\xi} \T{\omega}_i
	.
\end{align}

Velocity at an arbitrary location $\xi$:
\begin{align}
	\T{v}_{\text{kin}}\plbr{\xi}
	&=
	\sum_{i=1,2,3} N_i\plbr{\xi} \plbr{
		\T{v}_i
		+ \T{\omega}_i \times \T{o}_i
	}
	.
\end{align}

\paragraph{Perturbation of Interpolated Kinematics.}
Orientation perturbation at an arbitrary location $\xi$:
\begin{align}
	\delta\T{\theta}\plbr{\xi}
	&=
	\sum_{i=1,3} N_i\plbr{\xi} \delta\T{\theta}_{2\leftarrow i}
	\\
	\delta\T{\theta}_{2\leftarrow i}
	&=
	\TT{\Gamma}^{-1}\plbr{\T{\theta}_{2\leftarrow i}} \T{\theta}_{(2\leftarrow i) \delta}
	\\
	\T{\theta}_{(2\leftarrow i) \delta}
	&=
	\TT{R}_{2_a}^T \plbr{
		\T{\theta}_{i\delta}
		- \T{\theta}_{2\delta}
	}
	\\
	\T{\theta}_{\delta}\plbr{\xi}
	&=
	\T{\theta}_{2\delta}
	+ \sum_{i=1,3} \TT{R}_{2_a} \TT{\Gamma}\plbr{\T{\theta}\plbr{\xi}}
		N_i\plbr{\xi} \TT{\Gamma}^{-1}\plbr{\T{\theta}_{2\leftarrow i}}
		\TT{R}_{2_a}^T \plbr{\T{\theta}_{i\delta} - \T{\theta}_{2\delta}}
	\nonumber \\
	&= \plbr{
		\TT{I}
		- \TT{R}_{2_a} \TT{\Gamma}\plbr{\T{\theta}\plbr{\xi}}
			\sum_{i=1,3} N_i\plbr{\xi} \TT{\Gamma}^{-1}\plbr{\T{\theta}_{2\leftarrow i}}
			\TT{R}_{2_a}^T
	} \T{\theta}_{2\delta}
	\nonumber \\
	& \hphantom{= } \mbox{}
	+ \TT{R}_{2_a} \TT{\Gamma}\plbr{\T{\theta}\plbr{\xi}}
		N_1\TT{\Gamma}^{-1}\plbr{\T{\theta}_{2\leftarrow 1}}
		\TT{R}_{2_a}^T \T{\theta}_{1\delta}
	\nonumber \\
	& \hphantom{= } \mbox{}
	+ \TT{R}_{2_a} \TT{\Gamma}\plbr{\T{\theta}\plbr{\xi}}
		N_3\TT{\Gamma}^{-1}\plbr{\T{\theta}_{2\leftarrow 3}}
		\TT{R}_{2_a}^T \T{\theta}_{3\delta}
	\nonumber \\
	&=
	\sum_{i=1,2,3} \TT{\Theta}_i\plbr{\xi} \T{\theta}_{i\delta}
	,
\end{align}
with
\begin{subequations}
\begin{align}
	\TT{\Theta}_i\plbr{\xi}
	&=
	\TT{R}_{2_a} \TT{\Gamma}\plbr{\T{\theta}\plbr{\xi}}
		N_i\TT{\Gamma}^{-1}\plbr{\T{\theta}_{2\leftarrow i}}
		\TT{R}_{2_a}^T
		& i=\text{1 and 3}
	\\
	\TT{\Theta}_2\plbr{\xi}
	&=
	\TT{I} - \TT{\Theta}_1\plbr{\xi} - \TT{\Theta}_3\plbr{\xi}
\end{align}
\end{subequations}
playing the role of shape functions.

In fact, note that, when $N_1=1$ and $N_3=0$, then
$\TT{\Theta}_1=\TT{I}$, $\TT{\Theta}_2=\TT{\Theta}_3=\TT{0}$
and $\T{\theta}_{\delta}\plbr{\xi}=\T{\theta}_{1\delta}$,
while, when $N_1=0$ and $N_3=1$, then
$\TT{\Theta}_1=\TT{\Theta}_2=\TT{0}$, $\TT{\Theta}_3=\TT{I}$
and $\T{\theta}_{\delta}\plbr{\xi}=\T{\theta}_{3\delta}$.
Finally, when $N_1=N_3=0$, then
$\TT{\Theta}_1=\TT{\Theta}_3=\TT{0}$, $\TT{\Theta}_2=\TT{I}$
and $\T{\theta}_{\delta}\plbr{\xi}=\T{\theta}_{2\delta}$.
Moreover, $\sum_{i=1,2,3}\TT{\Theta}_i\plbr{\xi}=\TT{I} \ \forall\xi$.

Position perturbation at an arbitrary location $\xi$:
\begin{align}
	\delta\T{x}\plbr{\xi}
	&=
	\sum_{i=1,2,3} N_i\plbr{\xi} \plbr{
		\delta\T{x}_i + \T{\theta}_{i\delta}\times\T{o}_i
	}
	.
\end{align}

Angular velocity perturbation at an arbitrary location $\xi$:
\begin{align}
	\delta\T{\omega}\plbr{\xi}
	&=
	\sum_{i=1,2,3} N_i \delta\T{\omega}_i
	\nonumber \\
	&\equu
	\sum_{i=1,2,3} N_i \plbr{
		\delta\dot{\T{g}}_i
		- \T{\omega}_i \times \delta\T{g}_i
	}
	.
\end{align}

Velocity perturbation at an arbitrary location $\xi$:
\begin{align}
	\delta\T{v}_{\text{kin}}\plbr{\xi}
	&=
	\sum_{i=1,2,3} N_i\plbr{\xi} \plbr{
		\delta\T{v}_i
		+ \delta\T{\omega}_i \times \T{o}_i
		+ \T{\omega}_i \times \delta\T{o}_i
	}
	\nonumber \\
	&= 
	\sum_{i=1,2,3} N_i\plbr{\xi} \plbr{
		\delta\T{v}_i
		- \T{o}_i \times \delta\T{\omega}_i
		- \T{\omega}_i \times \T{o}_i \times \T{\theta}_{i\delta}
	}
	\nonumber \\
	&\equu
	\sum_{i=1,2,3} N_i\plbr{\xi} \plbr{
		\delta\dot{\T{x}}_i
		- \T{o}_i \times \delta\dot{\T{g}}_i
		- \plbr{\T{\omega}_i \times \T{o}_i} \times \delta\T{g}_i
	}
	.
\end{align}


\paragraph{Boundary Conditions Perturbation.}
Angular velocity perturbation:
\begin{align}
	\delta\tilde{\T{\omega}}\plbr{\xi}
	&= \TT{R}^T\plbr{\xi} \sum_{i=1,2,3}\plbr{
		N_i\plbr{\xi} \delta\T{\omega}_i
		+ \T{\omega}\plbr{\xi} \times \TT{\Theta}_i\plbr{\xi} \T{\theta}_{i\delta}
	}
	\nonumber \\
	&\equu
	\TT{R}^T\plbr{\xi} \plbr{
		N_i\plbr{\xi} \delta\dot{\T{g}}_i
		+ \plbr{
			\T{\omega}\plbr{\xi} \times \TT{\Theta}_i\plbr{\xi}
			- N_i\plbr{\xi} \T{\omega}_i \times{}
		} \delta\T{g}_i
	}
	.
\end{align}

Velocity perturbation:
\begin{align}
	\delta\tilde{\T{v}}\plbr{\xi}
	&=
	\TT{R}^T\plbr{\xi} \lplbr{
		N_i\plbr{\xi} \delta\T{v}_i
		- N_i\plbr{\xi} \T{o}_i \times \delta\T{\omega}_i
	}
	\nonumber \\
	& \hphantom{= \TT{R}^T\plbr{\xi}(}
	\rplbr{
		\mbox{} + \plbr{
			\T{v}\plbr{\xi} \times \TT{\Theta}_i\plbr{\xi}
			- N_i\plbr{\xi} \T{\omega}_i \times \T{o}_i \times{}
		} \T{\theta}_{i\delta}
	}
	\nonumber \\
	&\equu
	\TT{R}^T\plbr{\xi} \lplbr{
		N_i\plbr{\xi} \delta\dot{\T{x}}_i
		- N_i\plbr{\xi} \T{o}_i \times \delta\dot{\T{g}}_i
	}
	\nonumber \\
	& \hphantom{= \TT{R}^T\plbr{\xi}(}
	\rplbr{
		\mbox{} + \plbr{
			\T{v}\plbr{\xi} \times \TT{\Theta}_i\plbr{\xi}
			- N_i\plbr{\xi} \plbr{\T{\omega}_i \times \T{o}_i} \times{}
		} \delta\T{g}_i
	}
	.
\end{align}
They can be summarized as
\begin{align}
	\cubr{\cvvect{
		\delta\tilde{\T{v}}\plbr{\xi} \\
		\delta\tilde{\T{\omega}}\plbr{\xi}
	}}
	&=
	\sqbr{\matr{cc}{
		\TT{R}^T\plbr{\xi} & \TT{0} \\
		\TT{0} & \TT{R}^T\plbr{\xi}
	}} \sum_{i=1,2,3} \lplbr{
		N_i\plbr{\xi} \sqbr{\matr{cc}{
			\TT{I} & - \T{o}_i \times{} \\
			\TT{0} & \TT{I}
		}} \cubr{\cvvect{
			\delta\T{v}_i \\
			\delta\T{\omega}_i
		}}
	}
	\nonumber \\
	& \hphantom{= } \mbox{}
	+ \rplbr{
		\sqbr{\cvvect{
			\T{v}\plbr{\xi} \times \TT{\Theta}_i\plbr{\xi}
			- N_i\plbr{\xi} \T{\omega}_i \times \T{o}_i \times{} \\
			\T{\omega}\plbr{\xi} \times \TT{\Theta}_i\plbr{\xi}
		}} \T{\theta}_{i\delta}
	}
	\nonumber \\
	&\equu
	\sqbr{\matr{cc}{
		\TT{R}^T\plbr{\xi} & \TT{0} \\
		\TT{0} & \TT{R}^T\plbr{\xi}
	}} \sum_{i=1,2,3} \lplbr{
		N_i\plbr{\xi} \sqbr{\matr{cc}{
			\TT{I} & - \T{o}_i \times{} \\
			\TT{0} & \TT{I}
		}} \cubr{\cvvect{
			\delta\dot{\T{x}}_i \\
			\delta\dot{\T{g}}_i
		}}
	}
	\nonumber \\
	& \hphantom{= } \mbox{}
	+ \rplbr{
		\sqbr{\cvvect{
			\T{v}\plbr{\xi} \times \TT{\Theta}_i\plbr{\xi}
			- N_i\plbr{\xi} \plbr{\T{\omega}_i \times \T{o}_i} \times{} \\
			\T{\omega}\plbr{\xi} \times \TT{\Theta}_i\plbr{\xi}
			- N_i\plbr{\xi} \T{\omega}_i \times{}
		}} \delta\T{g}_i
	}
	.
\end{align}

\paragraph{Contribution to Jacobian Matrix.}
After defining
\begin{subequations}
\begin{align}
	\T{f}_{a/\tilde{\T{v}}} &= \TT{R}\plbr{\xi} \tilde{\T{f}}_{a/\tilde{\T{v}}} \TT{R}^T\plbr{\xi} \\
	\T{f}_{a/\tilde{\T{\omega}}} &= \TT{R}\plbr{\xi} \tilde{\T{f}}_{a/\tilde{\T{\omega}}} \TT{R}^T\plbr{\xi} \\
	\T{c}_{a/\tilde{\T{v}}} &= \TT{R}\plbr{\xi} \tilde{\T{c}}_{a/\tilde{\T{v}}} \TT{R}^T\plbr{\xi} \\
	\T{c}_{a/\tilde{\T{\omega}}} &= \TT{R}\plbr{\xi} \tilde{\T{c}}_{a/\tilde{\T{\omega}}} \TT{R}^T\plbr{\xi} \\
	\TT{B}_{\tilde{\T{v}}} &= \T{v}\plbr{\xi} \times \TT{\Theta}_i\plbr{\xi}
		- N_i\plbr{\xi} \plbr{\T{\omega}_i \times \T{o}_i} \times{} \\
	\TT{B}_{\tilde{\T{\omega}}} &= \T{\omega}\plbr{\xi} \times \TT{\Theta}_i\plbr{\xi}
		- N_i\plbr{\xi} \T{\omega}_i \times{} \\
	\T{d}_n\plbr{\xi} &= \T{x}\plbr{\xi} - \T{x}_n
	,
\end{align}
\end{subequations}
the contribution of the $i$-th node's motion to the equilibrium
of the $n$-th node ($i=1,2,3$, $n=1,2,3$) is
\begin{align}
	N_i\plbr{\xi} \sqbr{\matr{c}{
		\T{f}_{a/\tilde{\T{v}}} \\
		\T{d}_n\plbr{\xi} \times \T{f}_{a/\tilde{\T{v}}}
		+ \T{c}_{a/\tilde{\T{v}}}
	}} \delta\dot{\T{x}}_i
	\nonumber \\
%
	\mbox{}
	+ N_i\plbr{\xi} \sqbr{\matr{c}{
		\T{f}_{a/\tilde{\T{\omega}}}
		- \T{f}_{a/\tilde{\T{v}}} \T{o}_i \times{} \\
		\T{d}_n\plbr{\xi} \times \T{f}_{a/\tilde{\T{\omega}}}
		+ \T{c}_{a/\tilde{\T{\omega}}}
		- \plbr{
			\T{d}_n\plbr{\xi} \times \T{f}_{a/\tilde{\T{v}}}
			+ \T{c}_{a/\tilde{\T{v}}}
		} \T{o}_i \times{}
	}} \delta\dot{\T{g}}_i
	\nonumber \\
%
	\mbox{}
	+ \sqbr{\matr{c}{
		\T{f}_{a/\tilde{\T{v}}} \TT{B}_{\tilde{\T{v}}}
			+ \T{f}_{a/\tilde{\T{\omega}}} \TT{B}_{\tilde{\T{\omega}}} \\
		\T{d}_n\plbr{\xi} \times \plbr{
			\T{f}_{a/\tilde{\T{v}}} \TT{B}_{\tilde{\T{v}}}
				+ \T{f}_{a/\tilde{\T{\omega}}} \TT{B}_{\tilde{\T{\omega}}}
		}
		+ \T{c}_{a/\tilde{\T{v}}} \TT{B}_{\tilde{\T{v}}}
			+ \T{c}_{a/\tilde{\T{\omega}}} \TT{B}_{\tilde{\T{\omega}}}
	}} \delta\T{g}_i
	\nonumber \\
%
	\mbox{}
	+ \sqbr{\matr{c}{
		\TT{0} \\
		- \plbr{
			N_i - \delta_{ni}
		} \Delta\T{f}_n \times{}
	}} \delta\T{x}_i
	\nonumber \\
%
	\mbox{}
	+ \sqbr{\matr{c}{
		- \Delta\T{f}_n \times \TT{\Theta}_i\plbr{\xi} \\
		- \plbr{
			\T{d}_n\plbr{\xi} \times \Delta\T{f}_n \times{}
			+ \Delta\T{c}_n \times{}
		} \TT{\Theta}_i\plbr{\xi}
		+ \Delta\T{f}_n \times N_i \T{o}_i \times{}
	}} \delta\T{g}_i
	\nonumber \\
%
	\equu
	\cubr{\cvvect{
		\delta\Delta\T{f}_n \\
		\delta\Delta\T{c}_n
	}}
	,
\end{align}
where $\delta_{ni}$ is Dirac's delta, which is 1 when $i=n$, and 0 otherwise.




\section{Aerodynamic Beam (2 Nodes)}
The \texttt{aerodynamic beam2} element applies aerodynamic forces
to the nodes of a two node structural beam element. 

The aerodynamic forces/moments acting on each node 
are computed based on the relative velocity of a set of locations
along the beam and the airstream.
The kinematic quantities of the beam are computed based
on an interpolation of the kinematics of the three nodes.

\paragraph{Kinematics Interpolation.}
\emph{Note: this part is common to all elements that use the
two-node beam discretization and interpolation model.}
The generic field variable $\T{p}\plbr{x}$ is interpolated
using linear functions related to the value of the field 
variable at two locations that in general can be offset
from the nodes,
\begin{align}
	\T{p}\plbr{\xi}
	&=
	\sum_{i=1,2} N_i\plbr{\xi} \T{p}_i
	,
\end{align}
with
\begin{subequations}
\label{eq:aero:beam2:N}
\begin{align}
	N_1 &= \frac{1}{2} \plbr{1 - \xi}
	\\
	N_2 &= \frac{1}{2} \plbr{1 + \xi}
	.
\end{align}
\end{subequations}

Orientation at an arbitrary location $\xi$:
\begin{align}
	\TT{R}\plbr{\xi}
	&=
	\TT{R}_{\text{mid}} \text{exp}\plbr{\T{\theta}\plbr{\xi}\times{}}
	\\
	\TT{R}_{\text{mid}}
	&=
	\TT{R}_{1_a} \text{exp}\plbr{\overline{\T{\theta}}\times{}}
	= \TT{R}_{2_a} \text{exp}\plbr{\overline{\T{\theta}}\times{}}^T
	\\
	\overline{\T{\theta}}
	&=
	\frac{1}{2} \text{ax}\plbr{\text{exp}^{-1}\plbr{\TT{R}_{1_a}^T \TT{R}_{2_a}}}
	\\
	\T{\theta}\plbr{\xi}
	&=
	\sum_{i=1,2} N_i\plbr{\xi} \T{\theta}_{\text{mid}\leftarrow i}
	= N_1\plbr{\xi} \T{\theta}_{\text{mid}\leftarrow 1}
		+ N_2\plbr{\xi} \T{\theta}_{\text{mid}\leftarrow 2}
	\\
	\T{\theta}_{\text{mid}\leftarrow 1}
	&= -\overline{\T{\theta}}
	\\
	\T{\theta}_{\text{mid}\leftarrow 2}
	&=
	\overline{\T{\theta}}
	.
\end{align}
Orientation is dealt with specially, given its special nature.
The orientation of the mid point is used as a reference,
and the orientation parameters that express the relative orientation
between each of the end nodes and the mid point are interpolated.
The interpolated orientation parameters are used to compute
the interpolated relative orientation matrix, which is then
pre-multiplied by the orientation matrix of the mid point.

Since the Euler-Rodrigues parametrization is used
(the so-called `rotation vector'),
the magnitude of the relative orientation between each end node
must be limited (formally, to $\pi$, but it should be less for accuracy).

Note that since the generic orientation $\TT{R}(\xi)$
is the result of a sequence of orientations
about a common axis $\overline{\T{\theta}}$,
it can be conveniently rewritten as
\begin{align}
	\TT{R}\plbr{\xi}
	&= \TT{R}_{1_a} \text{exp}\plbr{\overline{\T{\theta}}\times{}}
		\text{exp}\plbr{\plbr{\plbr{N_2\plbr{\xi} - N_1\plbr{\xi}} \overline{\T{\theta}}}\times{}}
	\nonumber \\
	&= \TT{R}_{1_a} 
		\text{exp}\plbr{\plbr{\plbr{1 + N_2\plbr{\xi} - N_1\plbr{\xi}} \overline{\T{\theta}}}\times{}}
	.
\end{align}
In fact, when $\xi=-1$, $N_1=1$ and $N_2=0$,
then $\TT{R}\plbr{\xi}=\TT{R}_{1_a}$,
while when $\xi=1$, $N_1=0$ and $N_2=1$,
then $\TT{R}\plbr{\xi}=\TT{R}_{2_a}$.

Position at an arbitrary location $\xi$:
\begin{align}
	\T{x}\plbr{\xi}
	&=
	\sum_{i=1,2} N_i\plbr{\xi} \plbr{
		\T{x}_i + \T{o}_i
	}
	,
\end{align}
with $\T{o}_i=\TT{R}_i \tilde{\T{o}}_i$.

Angular velocity at an arbitrary location $\xi$:
\begin{align}
	\T{\omega}\plbr{\xi}
	&=
	\sum_{i=1,2} N_i\plbr{\xi} \T{\omega}_i
	.
\end{align}

Velocity at an arbitrary location $\xi$:
\begin{align}
	\T{v}_{\text{kin}}\plbr{\xi}
	&=
	\sum_{i=1,2} N_i\plbr{\xi} \plbr{
		\T{v}_i
		+ \T{\omega}_i \times \T{o}_i
	}
	.
\end{align}

\paragraph{Perturbation of Interpolated Kinematics.}
Orientation perturbation at an arbitrary location $\xi$:
\begin{align}
	\T{\theta}_{\delta}\plbr{\xi}
	&=
	\frac{1 + N_2\plbr{\xi} - N_1\plbr{\xi}}{2} \T{\theta}_{2\delta}
	+ \frac{1 + N_1\plbr{\xi} - N_2\plbr{\xi}}{2} \T{\theta}_{1\delta}
	\\
	&=
	\sum_{i=1,2} \mathcal{N}_i\plbr{\xi} \T{\theta}_{i\delta}
	,
\end{align}
with
\begin{subequations}
\begin{align}
	\mathcal{N}_i\plbr{\xi}
	&=
	\frac{1 + N_i\plbr{\xi} - N_{3-i}\plbr{\xi}}{2}
	\label{eq:aero:beam2:NN}
	.
\end{align}
\end{subequations}
Note, however, that according to Eqs.~(\ref{eq:aero:beam2:N}),
the shape functions of Eq.~(\ref{eq:aero:beam2:NN})
are $\mathcal{N}_i\plbr{\xi}=N_i\plbr{\xi}$.

Position perturbation at an arbitrary location $\xi$:
\begin{align}
	\delta\T{x}\plbr{\xi}
	&=
	\sum_{i=1,2} N_i\plbr{\xi} \plbr{
		\delta\T{x}_i + \T{\theta}_{i\delta}\times\T{o}_i
	}
	.
\end{align}

Angular velocity perturbation at an arbitrary location $\xi$:
\begin{align}
	\delta\T{\omega}\plbr{\xi}
	&=
	\sum_{i=1,2} N_i \delta\T{\omega}_i
	\nonumber \\
	&\equu
	\sum_{i=1,2} N_i \plbr{
		\delta\dot{\T{g}}_i
		- \T{\omega}_i \times \delta\T{g}_i
	}
	.
\end{align}

Velocity perturbation at an arbitrary location $\xi$:
\begin{align}
	\delta\T{v}_{\text{kin}}\plbr{\xi}
	&=
	\sum_{i=1,2} N_i\plbr{\xi} \plbr{
		\delta\T{v}_i
		+ \delta\T{\omega}_i \times \T{o}_i
		+ \T{\omega}_i \times \delta\T{o}_i
	}
	\nonumber \\
	&= 
	\sum_{i=1,2} N_i\plbr{\xi} \plbr{
		\delta\T{v}_i
		- \T{o}_i \times \delta\T{\omega}_i
		- \T{\omega}_i \times \T{o}_i \times \T{\theta}_{i\delta}
	}
	\nonumber \\
	&\equu
	\sum_{i=1,2} N_i\plbr{\xi} \plbr{
		\delta\dot{\T{x}}_i
		- \T{o}_i \times \delta\dot{\T{g}}_i
		- \plbr{\T{\omega}_i \times \T{o}_i} \times \delta\T{g}_i
	}
	.
\end{align}


\paragraph{Boundary Conditions Perturbation.}
Angular velocity perturbation:
\begin{align}
	\delta\tilde{\T{\omega}}\plbr{\xi}
	&= \TT{R}^T\plbr{\xi} \sum_{i=1,2} \plbr{
		N_i\plbr{\xi} \delta\T{\omega}_i
		+ \T{\omega}\plbr{\xi} \times \mathcal{N}_i\plbr{\xi} \T{\theta}_{i\delta}
	}
	\nonumber \\
	&\equu
	\TT{R}^T\plbr{\xi} \sum_{i=1,2} \plbr{
		N_i\plbr{\xi} \delta\dot{\T{g}}_i
		+ \plbr{
			\mathcal{N}_i \T{\omega}\plbr{\xi}
			- N_i\plbr{\xi} \T{\omega}_i
		} \times \delta\T{g}_i
	}
	\nonumber \\
	&=
	\TT{R}^T\plbr{\xi} \sum_{i=1,2} N_i\plbr{\xi} \plbr{
		\delta\dot{\T{g}}_i
		+ \plbr{
			\T{\omega}\plbr{\xi}
			- \T{\omega}_i
		} \times \delta\T{g}_i
	}
	.
\end{align}

Velocity perturbation:
\begin{align}
	\delta\tilde{\T{v}}\plbr{\xi}
	&=
	\TT{R}^T\plbr{\xi} \lplbr{
		N_i\plbr{\xi} \delta\T{v}_i
		- N_i\plbr{\xi} \T{o}_i \times \delta\T{\omega}_i
	}
	\nonumber \\
	& \hphantom{= \TT{R}^T\plbr{\xi}(}
	\rplbr{
		\mbox{} + \plbr{
			\mathcal{N}_i\plbr{\xi} \T{v}\plbr{\xi} \times{}
			- N_i\plbr{\xi} \T{\omega}_i \times \T{o}_i \times{}
		} \T{\theta}_{i\delta}
	}
	\nonumber \\
	&\equu
	\TT{R}^T\plbr{\xi} \lplbr{
		N_i\plbr{\xi} \delta\dot{\T{x}}_i
		- N_i\plbr{\xi} \T{o}_i \times \delta\dot{\T{g}}_i
	}
	\nonumber \\
	& \hphantom{= \TT{R}^T\plbr{\xi}(}
	\rplbr{
		\mbox{} + \plbr{
			\mathcal{N}_i\plbr{\xi} \T{v}\plbr{\xi}
			- N_i\plbr{\xi} \T{\omega}_i \times \T{o}_i
		} \times \delta\T{g}_i
	}
	\nonumber \\
	&=
	\TT{R}^T\plbr{\xi} N_i\plbr{\xi} \plbr{
		\delta\dot{\T{x}}_i
		- \T{o}_i \times \delta\dot{\T{g}}_i
		+ \plbr{
			\T{v}\plbr{\xi}
			- \T{\omega}_i \times \T{o}_i
		} \times \delta\T{g}_i
	}
	.
\end{align}
They can be summarized as
\begin{align}
	\cubr{\cvvect{
		\delta\tilde{\T{v}}\plbr{\xi} \\
		\delta\tilde{\T{\omega}}\plbr{\xi}
	}}
	&=
	\sqbr{\matr{cc}{
		\TT{R}^T\plbr{\xi} & \TT{0} \\
		\TT{0} & \TT{R}^T\plbr{\xi}
	}} \sum_{i=1,2,3} \lplbr{
		N_i\plbr{\xi} \sqbr{\matr{cc}{
			\TT{I} & - \T{o}_i \times{} \\
			\TT{0} & \TT{I}
		}} \cubr{\cvvect{
			\delta\T{v}_i \\
			\delta\T{\omega}_i
		}}
	}
	\nonumber \\
	& \hphantom{= } \mbox{}
	+ \rplbr{
		\sqbr{\cvvect{
			\mathcal{N}_i\plbr{\xi} \T{v}\plbr{\xi} \times{}
			- N_i\plbr{\xi} \T{\omega}_i \times \T{o}_i \times{} \\
			\mathcal{N}_i\plbr{\xi} \T{\omega}\plbr{\xi} \times{}
		}} \T{\theta}_{i\delta}
	}
	\nonumber \\
	&\equu
	\sqbr{\matr{cc}{
		\TT{R}^T\plbr{\xi} & \TT{0} \\
		\TT{0} & \TT{R}^T\plbr{\xi}
	}} \sum_{i=1,2,3} \lplbr{
		N_i\plbr{\xi} \sqbr{\matr{cc}{
			\TT{I} & - \T{o}_i \times{} \\
			\TT{0} & \TT{I}
		}} \cubr{\cvvect{
			\delta\dot{\T{x}}_i \\
			\delta\dot{\T{g}}_i
		}}
	}
	\nonumber \\
	& \hphantom{= } \mbox{}
	+ \rplbr{
		\sqbr{\cvvect{
			\plbr{
				\mathcal{N}_i\plbr{\xi} \T{v}\plbr{\xi}
				- N_i\plbr{\xi} \T{\omega}_i \times \T{o}_i
			} \times{} \\
			\plbr{
				\mathcal{N}_i\plbr{\xi} \T{\omega}\plbr{\xi}
				- N_i\plbr{\xi} \T{\omega}_i
			} \times{}
		}} \delta\T{g}_i
	}
	\nonumber \\
	&=
	\sqbr{\matr{cc}{
		\TT{R}^T\plbr{\xi} & \TT{0} \\
		\TT{0} & \TT{R}^T\plbr{\xi}
	}} \sum_{i=1,2,3} N_i\plbr{\xi} \lplbr{
		\sqbr{\matr{cc}{
			\TT{I} & - \T{o}_i \times{} \\
			\TT{0} & \TT{I}
		}} \cubr{\cvvect{
			\delta\dot{\T{x}}_i \\
			\delta\dot{\T{g}}_i
		}}
	}
	\nonumber \\
	& \hphantom{= } \mbox{}
	+ \rplbr{
		\sqbr{\cvvect{
			\plbr{
				\T{v}\plbr{\xi}
				- \T{\omega}_i \times \T{o}_i
			} \times{} \\
			\plbr{
				\T{\omega}\plbr{\xi}
				- \T{\omega}_i
			} \times{}
		}} \delta\T{g}_i
	}
	.
\end{align}

\paragraph{Contribution to Jacobian Matrix.}
After defining
\begin{subequations}
\begin{align}
	\T{f}_{a/\tilde{\T{v}}} &= \TT{R}\plbr{\xi} \tilde{\T{f}}_{a/\tilde{\T{v}}} \TT{R}^T\plbr{\xi} \\
	\T{f}_{a/\tilde{\T{\omega}}} &= \TT{R}\plbr{\xi} \tilde{\T{f}}_{a/\tilde{\T{\omega}}} \TT{R}^T\plbr{\xi} \\
	\T{c}_{a/\tilde{\T{v}}} &= \TT{R}\plbr{\xi} \tilde{\T{c}}_{a/\tilde{\T{v}}} \TT{R}^T\plbr{\xi} \\
	\T{c}_{a/\tilde{\T{\omega}}} &= \TT{R}\plbr{\xi} \tilde{\T{c}}_{a/\tilde{\T{\omega}}} \TT{R}^T\plbr{\xi} \\
	\TT{B}_{\tilde{\T{v}}} &= \plbr{
		\mathcal{N}_i\plbr{\xi} \T{v}\plbr{\xi}
		- N_i\plbr{\xi} \T{\omega}_i \times \T{o}_i
	} \times{}
	\nonumber \\
	&= N_i\plbr{\xi} \plbr{
		\T{v}\plbr{\xi}
		- \T{\omega}_i \times \T{o}_i
	} \times{}
	\\
	\TT{B}_{\tilde{\T{\omega}}} &= \plbr{
		\mathcal{N}_i\plbr{\xi} \T{\omega}\plbr{\xi}
		- N_i\plbr{\xi} \T{\omega}_i
	} \times{}
	\nonumber \\
	&= N_i\plbr{\xi} \plbr{
		\T{\omega}\plbr{\xi}
		- \T{\omega}_i
	} \times{}
	\\
	\T{d}_n\plbr{\xi} &= \T{x}\plbr{\xi} - \T{x}_n
	,
\end{align}
\end{subequations}
the contribution of the $i$-th node's motion to the equilibrium
of the $n$-th node ($i=1,2$, $n=1,2$) is
\begin{align}
	N_i\plbr{\xi} \sqbr{\matr{c}{
		\T{f}_{a/\tilde{\T{v}}} \\
		\T{d}_n\plbr{\xi} \times \T{f}_{a/\tilde{\T{v}}}
		+ \T{c}_{a/\tilde{\T{v}}}
	}} \delta\dot{\T{x}}_i
	\nonumber \\
%
	\mbox{}
	+ N_i\plbr{\xi} \sqbr{\matr{c}{
		\T{f}_{a/\tilde{\T{\omega}}}
		- \T{f}_{a/\tilde{\T{v}}} \T{o}_i \times{} \\
		\T{d}_n\plbr{\xi} \times \T{f}_{a/\tilde{\T{\omega}}}
		+ \T{c}_{a/\tilde{\T{\omega}}}
		- \plbr{
			\T{d}_n\plbr{\xi} \times \T{f}_{a/\tilde{\T{v}}}
			+ \T{c}_{a/\tilde{\T{v}}}
		} \T{o}_i \times{}
	}} \delta\dot{\T{g}}_i
	\nonumber \\
%
	\mbox{}
	+ \sqbr{\matr{c}{
		\T{f}_{a/\tilde{\T{v}}} \TT{B}_{\tilde{\T{v}}}
			+ \T{f}_{a/\tilde{\T{\omega}}} \TT{B}_{\tilde{\T{\omega}}} \\
		\T{d}_n\plbr{\xi} \times \plbr{
			\T{f}_{a/\tilde{\T{v}}} \TT{B}_{\tilde{\T{v}}}
				+ \T{f}_{a/\tilde{\T{\omega}}} \TT{B}_{\tilde{\T{\omega}}}
		}
		+ \T{c}_{a/\tilde{\T{v}}} \TT{B}_{\tilde{\T{v}}}
			+ \T{c}_{a/\tilde{\T{\omega}}} \TT{B}_{\tilde{\T{\omega}}}
	}} \delta\T{g}_i
	\nonumber \\
%
	\mbox{}
	+ \sqbr{\matr{c}{
		\TT{0} \\
		- \plbr{
			N_i - \delta_{ni}
		} \Delta\T{f}_n \times{}
	}} \delta\T{x}_i
	\nonumber \\
%
	\mbox{}
	+ \sqbr{\matr{c}{
		- \mathcal{N}_i\plbr{\xi} \Delta\T{f}_n \times{} \\
		- \mathcal{N}_i\plbr{\xi} \plbr{
			\T{d}_n\plbr{\xi} \times \Delta\T{f}_n \times{}
			+ \Delta\T{c}_n \times{}
		}
		+ \Delta\T{f}_n \times N_i \T{o}_i \times{}
	}} \delta\T{g}_i
	\nonumber \\
%
	\equu
	\cubr{\cvvect{
		\delta\Delta\T{f}_n \\
		\delta\Delta\T{c}_n
	}}
	,
\end{align}
where $\delta_{ni}$ is Dirac's delta, which is 1 when $i=n$, and 0 otherwise.





