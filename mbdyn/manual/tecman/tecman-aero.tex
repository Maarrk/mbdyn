% MBDyn (C) is a multibody analysis code.
% http://www.mbdyn.org
%
% Copyright (C) 1996-2014
%
% Pierangelo Masarati  <masarati@aero.polimi.it>
%
% Dipartimento di Ingegneria Aerospaziale - Politecnico di Milano
% via La Masa, 34 - 20156 Milano, Italy
% http://www.aero.polimi.it
%
% Changing this copyright notice is forbidden.
%
% This program is free software; you can redistribute it and/or modify
% it under the terms of the GNU General Public License as published by
% the Free Software Foundation (version 2 of the License).
% 
%
% This program is distributed in the hope that it will be useful,
% but WITHOUT ANY WARRANTY; without even the implied warranty of
% MERCHANTABILITY or FITNESS FOR A PARTICULAR PURPOSE.  See the
% GNU General Public License for more details.
%
% You should have received a copy of the GNU General Public License
% along with this program; if not, write to the Free Software
% Foundation, Inc., 59 Temple Place, Suite 330, Boston, MA  02111-1307  USA
%
% Alessandro Fumagalli <fumagalli@aero.polimi.it> is the author of this document

Aerodynamic elements apply aerodynamic forces to structural nodes.

This section is not intended 
to give details about the aerodynamic models adopted but mainly 
discuss the computation of the contributions to the Jacobian 
matrix of the aerodynamic elements.

\paragraph{Files.} \
2D aerodnamic elements are implemented in files \\
\begin{tabular}{l}
\texttt{mbdyn/aero/aeroelem.h} \\
\texttt{mbdyn/aero/aeroelem.cc} 
\end{tabular}

2D aerodynamic models are implemented in files \\
\begin{tabular}{l}
\texttt{mbdyn/aero/aerodata.h} \\
\texttt{mbdyn/aero/aerodata.cc} \\
\texttt{mbdyn/aero/aerodata\_impl.h} \\
\texttt{mbdyn/aero/aerodata\_impl.cc} 
\end{tabular}

\section{Linearization of 2D Aerodynamic Forces and Moments}
MBDyn's built-in 2D aerodynamics computes
aerodynamic forces $\tilde{\T{f}}_{a/\xi}$
and moments $\tilde{\T{c}}_{a/\xi}$ per unit span,
in a relative frame at station $\xi$,
based on the instantaneous value of linear and angular velocity
boundary conditions, respectively $\tilde{\T{v}}$ and $\tilde{\T{\omega}}$,
expressed in the same relative frame, namely
\begin{subequations}
\begin{align}
	\tilde{\T{f}}_{a/\xi}
	&=
	\tilde{\T{f}}_{a/\xi}\plbr{\tilde{\T{v}}, \tilde{\T{\omega}}, \xi}
	\\
	\tilde{\T{c}}_{a/\xi}
	&=
	\tilde{\T{c}}_{a/\xi}\plbr{\tilde{\T{v}}, \tilde{\T{\omega}}, \xi}
	.
\end{align}
\end{subequations}

The boundary conditions are computed by projecting the \emph{effective}
linear and angular velocity at station $\xi$,
respectively $\T{v}\plbr{\xi}$ and $\T{\omega}\plbr{\xi}$,
in the reference frame of the aerodynamic forces,
namely
\begin{subequations}
\begin{align}
	\tilde{\T{v}}\plbr{\xi}
	&=
	\TT{R}^T\plbr{\xi} \T{v}\plbr{\xi}
	\\
	\tilde{\T{\omega}}\plbr{\xi}
	&=
	\TT{R}^T\plbr{\xi} \T{\omega}\plbr{\xi}
	,
\end{align}
\end{subequations}
where $\TT{R}\plbr{\xi}$ is the matrix that expresses
the local orientation of the aerodynamic section at station $\xi$.

In detail, the effective linear velocity at an arbitrary station
is the combination of the absolute velocity resulting from the kinematics
of the model, of an airstream velocity that may depend
on the absolute location of a reference point and on time,
and of a contribution resulting from an inflow model, namely
\begin{align}
	\T{v}\plbr{\xi}
	&=
	\T{v}_{\text{kin}}\plbr{\xi}
	+ \T{v}_{\infty}\plbr{\T{x}\plbr{\xi}, t}
	+ \T{v}_{\text{inflow}}\plbr{\T{x}\plbr{\xi}}
	.
\end{align}
It is assumed that the last two contributions do not depend
on the state of the problem, or only depend on it in a loose manner,
and thus do not directly participate in the linearization
of the aerodynamic forces and moments.

Their linearization yields
\begin{subequations}
\begin{align}
	\delta\tilde{\T{v}}
	&=
	\TT{R}^T\plbr{\xi} \plbr{
		\delta\T{v}_{\text{kin}}\plbr{\xi}
		+ \T{v}\plbr{\xi} \times \T{\theta}_{\delta}\plbr{\xi}
	}
	\\
	\delta\tilde{\T{\omega}}
	&=
	\TT{R}^T\plbr{\xi} \plbr{
		\delta\T{\omega}\plbr{\xi}
		+ \T{\omega}\plbr{\xi} \times \T{\theta}_{\delta}\plbr{\xi}
	}
	,
\end{align}
\end{subequations}
where
\begin{align}
	\T{\theta}_{\delta}\plbr{\xi}
	&=
	\text{ax}\plbr{\delta\TT{R}\plbr{\xi} \TT{R}^T\plbr{\xi}}
	.
\end{align}

By means of numerical integration, the force and moment per unit span
are integrated into discrete contributions to the force and the moment
applied to the appropriate node equilibrium.
This is usually done by multiplying each force and moment per unit span
contribution by an appropriate reference length coefficient.
As a consequence, in the following force and moment contributions
will be considered, namely
\begin{subequations}
\begin{align}
	\tilde{\T{f}}_a
	&=
	\tilde{\T{f}}_a\plbr{\tilde{\T{v}}, \tilde{\T{\omega}}, \xi}
	\\
	\tilde{\T{c}}_a
	&=
	\tilde{\T{c}}_a\plbr{\tilde{\T{v}}, \tilde{\T{\omega}}, \xi}
	.
\end{align}
\end{subequations}
As already mentioned, each weighted sectional force and moment contribution
is applied to the appropriate node after projection in the global reference
frame.
The contribution to the force and moment equilibrium of the $n$-th node is
\begin{subequations}
	\label{eq:aero:fm-contrib}
\begin{align}
	\Delta\T{f}_n
	&=
	\TT{R}\plbr{\xi} \tilde{\T{f}}_a
	\label{eq:aero:fm-contrib:f}
	\\
	\Delta\T{c}_n
	&=
	\TT{R}\plbr{\xi} \tilde{\T{c}}_a
	+ \plbr{\T{x}\plbr{\xi} - \T{x}_n} \times \Delta\T{f}_n
	\label{eq:aero:fm-contrib:m}
	.
\end{align}
\end{subequations}
Their linearization yields
\begin{subequations}
\begin{align}
	\delta\Delta\T{f}_n
	&=
	- \Delta\T{f}_n \times \T{\theta}_{\delta}\plbr{\xi}
	+ \TT{R}\plbr{\xi} \delta\tilde{\T{f}}_a
	\\
	\delta\Delta\T{c}_n
	&=
	- \Delta\T{c}_n \times \T{\theta}_{\delta}\plbr{\xi}
	+ \TT{R}\plbr{\xi} \delta\tilde{\T{c}}_a
	- \Delta\T{f}_n \times \plbr{
		\delta\T{x}\plbr{\xi}
		- \delta\T{x}_n
	}
	\nonumber \\
	&\hphantom{= } \mbox{}
	+ \plbr{
		\T{x}\plbr{\xi}
		- \T{x}_n
	} \times \plbr{
		- \Delta\T{f}_n \times \T{\theta}_{\delta}\plbr{\xi}
		+ \TT{R}\plbr{\xi} \delta\tilde{\T{f}}_a
	}
	,
\end{align}
\end{subequations}
which can be summarized as
\begin{align}
	\cubr{\cvvect{
		\delta\Delta\T{f}_n \\
		\delta\Delta\T{c}_n
	}}
	&=
	\sqbr{\cvvect{
		- \Delta\T{f}_n \times{} \\
		- \Delta\T{c}_n \times{}
		- \plbr{\T{x}\plbr{\xi} - \T{x}_n} \times \Delta\T{f}_n \times{}
	}} \T{\theta}_{\delta}\plbr{\xi}
	\nonumber \\
	& \hphantom{= } \mbox{}
	+ \sqbr{\cvvect{
		\TT{0} \\
		- \Delta\T{f}_n \times{}
	}} \plbr{
		\delta\T{x}\plbr{\xi}
		- \delta\T{x}_n
	}
	\nonumber \\
	& \hphantom{= } \mbox{}
	+ \sqbr{\matr{cc}{
		\TT{I} & \TT{0} \\
		\plbr{\T{x}\plbr{\xi} - \T{x}_n} \times{} & \TT{I}
	}} \cubr{\cvvect{
		\TT{R}\plbr{\xi} \delta\tilde{\T{f}}_a \\
		\TT{R}\plbr{\xi} \delta\tilde{\T{c}}_a
	}}
	.
\end{align}

It is assumed that the Jacobian matrix of the sectional force and moment
with respect to the linear and angular velocity
is either available or can be computed by numerical differentiation.
The resulting force and moment perturbation is
\begin{align}
	\cubr{\cvvect{
		\delta\tilde{\T{f}}_a \\
		\delta\tilde{\T{c}}_a
	}} &= \sqbr{\matr{cc}{
		\tilde{\T{f}}_{a/\tilde{\T{v}}} & \tilde{\T{f}}_{a/\tilde{\T{\omega}}} \\
		\tilde{\T{c}}_{a/\tilde{\T{v}}} & \tilde{\T{c}}_{a/\tilde{\T{\omega}}}
	}} \cubr{\cvvect{
		\delta\tilde{\T{v}} \\
		\delta\tilde{\T{\omega}}
	}}
	\label{eq:aero:jac}
	.
\end{align}
The computation of the matrix of Eq.~(\ref{eq:aero:jac})
is delegated to the \texttt{AeroData} class.

Each type of element determines how the sectional force and moment
contributions are applied to the nodes, and how the sectional
boundary conditions at each section are computed from the kinematics
of the nodes.



\section{Numerical Linearization of Sectional Forces}
Consider an arbitrary submatrix of the Jacobian matrix
of Eq.~(\ref{eq:aero:jac}), $\TT{J}=\T{p}_{/\T{q}}$.
Its generic element, the $c$-th component of $\T{p}$
derived by the $r$-th component of $\T{q}$, is
\begin{align}
	J_{rc} &= \frac{\partial\T{p}_r}{\partial\T{q}_c}
	.
\end{align}
A forward difference approach is used, namely
\begin{align}
	J_{rc} &\cong \frac{
		\T{p}_r\plbr{\T{q} + \Delta q \T{e}_c}
		- \T{p}_r\plbr{\T{q}}
	}{
		\Delta q
	}
	,
\end{align}
where $\T{e}_c$ is the unit vector along the $c$-th component,
and $\Delta q$ is a suitably chosen perturbation.
Alternatively, a centered difference approach can be used, namely
\begin{align}
	J_{rc} &\cong \frac{
		\T{p}_r\plbr{\T{q} + \Delta q \T{e}_c}
		- \T{p}_r\plbr{\T{q} - \Delta q \T{e}_c}
	}{
		2 \Delta q
	}
	.
\end{align}
The perturbation is
\begin{align}
	\Delta q &= \varepsilon \nrbr{\T{q}} + \nu
	.
\end{align}
Since the boundary condition $\T{q}$ is perturbed in order to determine
an equivalent perturbation of angle of attack, a resolution of few tenth
of degree is deemed sufficient.
As a consequence, $\varepsilon>0$ must be a ``small'' number
that, in case $\T{e}_c$ is orthogonal to $\T{q}$,
yields an angle of the order of a tenth of a degree.
The default value is $\varepsilon=10^{-3}$.
However, in order to avoid divisions by too small numbers,
the perturbation is corrected by another ``small'' parameter,
$\nu>0$.
The default value is $\nu=10^{-9}$.




\section{Aerodynamic Forces with Internal States}
Consider the case of a model of the aerodynamic forces
that requires the use of internal states $\T{q}$, namely
\begin{subequations}
\begin{align}
	\tilde{\T{f}}_a
	&=
	\tilde{\T{f}}_a\plbr{\tilde{\T{v}},\tilde{\T{\omega}},\T{q}}
	\\
	\tilde{\T{c}}_a
	&=
	\tilde{\T{c}}_a\plbr{\tilde{\T{v}},\tilde{\T{\omega}},\T{q}}
	\\
	\T{0}
	&=
	\T{g}\plbr{\tilde{\T{v}},\tilde{\T{\omega}},\T{q}, \dot{\T{q}}}
	\label{eq:aero:states:g}
	.
\end{align}
\end{subequations}
Usually, the dynamic model of the aerodynamics is differential,
and thus its perturbation yields a linearized state-space system,
\begin{subequations}
\begin{align}
	\delta\T{g}
	&=
	\T{g}_{/\tilde{\T{v}}} \delta\tilde{\T{v}}
	+ \T{g}_{/\tilde{\T{\omega}}} \delta\tilde{\T{\omega}}
	+ \T{g}_{/\T{q}} \delta\T{q}
	+ \T{g}_{/\dot{\T{q}}} \delta\dot{\T{q}}
	\label{eq:aero:states:delta-g}
	\\
	\delta \tilde{\T{f}}_a
	&=
	\tilde{\T{f}}_{a/\tilde{\T{v}}} \delta\tilde{\T{v}}
	+ \tilde{\T{f}}_{a/\tilde{\T{\omega}}} \delta\tilde{\T{\omega}}
	+ \tilde{\T{f}}_{a/\T{q}} \delta\T{q}
	\label{eq:aero:states:delta-f}
	\\
	\delta \tilde{\T{c}}_a
	&=
	\tilde{\T{c}}_{a/\tilde{\T{v}}} \delta\tilde{\T{v}}
	+ \tilde{\T{c}}_{a/\tilde{\T{\omega}}} \delta\tilde{\T{\omega}}
	+ \tilde{\T{c}}_{a/\T{q}} \delta\T{q}
	\label{eq:aero:states:delta-c}
	,
\end{align}
\end{subequations}
where, with respect to $\delta\T{g}$, the terms $\delta\tilde{\T{v}}$
and $\delta\tilde{\T{\omega}}$ play the role of the input, while $\delta\T{q}$
plays the role of the state;
$\delta\tilde{\T{f}}_a$, $\delta\tilde{\T{c}}_a$ play the role of the output.

In those cases, the underlying aerodynamic model has
to deal with $\T{g}$, Eq.~(\ref{eq:aero:states:g}),
and its perturbation $\delta\T{g}$, Eq.~(\ref{eq:aero:states:g}).

However, the aerodynamic elements have to:
\begin{enumerate}

\item provide the underlying aerodynamic model the Jacobian submatrices
required to compute $\delta\tilde{\T{v}}$ and $\delta\tilde{\T{\omega}}$
from the perturbations of the nodal position, orientation,
and linear and angular velocity that are needed to deal with $\delta\T{g}$,
Eq.~(\ref{eq:aero:states:g});

\item account for the contribution of the Jacobian matrices
$\delta\tilde{\T{f}}_{a/\T{q}}$ and $\delta\tilde{\T{c}}_{a/\T{q}}$
respectively required by Eqs.~(\ref{eq:aero:states:delta-f})
and~(\ref{eq:aero:states:delta-c}).

\end{enumerate}




\section{Aerodynamic body}
The \texttt{aerodynamic body} element applies aerodynamic forces
to the structural node it is connected to,
based on the relative velocity between an aerodynamic surface attached 
to the node and the airstream.

The boundary conditions are related to the rigid body motion
of the node, so
\begin{subequations}
\begin{align}
	\TT{R}\plbr{\xi}
	&=
	\TT{R}_n \TT{R}_a \TT{R}_t\plbr{\xi}
	\\
	\T{b}\plbr{\xi}
	&=
	\TT{R}_n\plbr{
		\tilde{\T{b}}_0
		+ \TT{R}_a \plbr{
			b\plbr{\xi} \T{e}_1
			+ \xi \T{e}_3
		}
	}
	\\
	\T{\omega}\plbr{\xi}
	&=
	\T{\omega}_n
	\\
	\T{v}_{\text{kin}}\plbr{\xi}
	&=
	\T{v}_n + \T{\omega}_n \times \T{b}\plbr{\xi}
	,
\end{align}
\end{subequations}
where
$\TT{R}_n$ is the orientation of the node,
$\TT{R}_a$ is the relative orientation of the aerodynamics
with respect to the node,
$\TT{R}_t$ is the pretwist matrix,
$\T{v}_n$ is the absolute velocity of the node,
$\T{\omega}_n$ is the absolute angular velocity of the node,
$\tilde{\T{b}}_0$ is an offset between the node and the reference location
of the aerodynamic body,
and $b\plbr{\xi}$ is the chordwise location of the point
where the boundary conditions are evaluated.

Their linearization is straightforward:
\begin{subequations}
\begin{align}
	\T{\theta}_{\delta}\plbr{\xi}
	&=
	\T{\theta}_{n\delta}
	\\
	\delta\T{b}\plbr{\xi}
	&=
	- \T{b}\plbr{\xi}\times\T{\theta}_{n\delta}
	\\
	\delta\T{\omega}\plbr{\xi}
	&=
	\delta\T{\omega}_n
	\\
	\delta\T{v}_{\text{kin}}\plbr{\xi}
	&=
	\delta\T{v}_n
	- \T{b}\plbr{\xi}\times\delta\T{\omega}_n
	- \T{\omega}_n \times \T{b}\plbr{\xi} \times \T{\theta}_{n\delta}
	.
\end{align}
\end{subequations}
Eq.~(\ref{eq:aero:fm-contrib:m}) can be rewritten as
\begin{align}
	\Delta\T{c}_n
	&=
	\TT{R}\plbr{\xi}\plbr{
		\tilde{\T{c}}_a
		+ \tilde{\T{o}}\plbr{\xi} \times \tilde{\T{f}}_a
	} ,
\end{align}
where
$\tilde{\T{o}}\plbr{\xi}=\TT{R}^T\plbr{\xi}\plbr{\T{x}\plbr{\xi} - \T{x}_n}$,
the offset between the point where the force is applied and the node, 
in the reference frame of the node, does not depend on the kinematics
of the system, since the body is rigid.
Its linearization yields
\begin{align}
	\delta\Delta\T{c}_n
	&=
	- \Delta\T{c}_n \times \T{\theta}_{\delta}\plbr{\xi}
	+ \TT{R}\plbr{\xi} \plbr{
		\delta\tilde{\T{c}}_a
		+ \tilde{\T{o}}\plbr{\xi} \times \delta\tilde{\T{f}}_a
	}
	,
\end{align}
since $\delta\tilde{\T{o}}\plbr{\xi}\equiv{0}$.
So the linearized force and moment is
\begin{align}
	\cubr{\cvvect{
		\delta\Delta\T{f}_n \\
		\delta\Delta\T{c}_n
	}}
	&=
	- \sqbr{\cvvect{
		\Delta\T{f}_n \times{} \\
		\Delta\T{c}_n \times{}
	}} \T{\theta}_{n\delta}
	+ \sqbr{\matr{cc}{
		\TT{I} & \TT{0} \\
		\T{o}\plbr{\xi}\times{} & \TT{I}
	}} \cubr{\cvvect{
		\TT{R}\plbr{\xi} \delta\tilde{\T{f}}_a \\
		\TT{R}\plbr{\xi} \delta\tilde{\T{c}}_a
	}}
	.
\end{align}
with $\T{o}\plbr{\xi}=\TT{R}\plbr{\xi} \tilde{\T{o}}\plbr{\xi}$.

The linearization of the boundary conditions yields
\begin{subequations}
\begin{align}
	\cubr{\cvvect{
		\delta\tilde{\T{v}}\plbr{\xi} \\
		\delta\tilde{\T{\omega}}\plbr{\xi}
	}}
	&=
	\sqbr{\matr{cc}{
		\TT{R}^T\plbr{\xi} & \TT{0} \\
		\TT{0} & \TT{R}^T\plbr{\xi}
	}} \lplbr{
		\sqbr{\cvvect{
			\T{v}\plbr{\xi}\times{}
				- \T{\omega}_n\times\T{b}\plbr{\xi}\times{}
			\\
			\T{\omega}_n\times{}
		}} \T{\theta}_{n\delta}
	} \nonumber \\
	& \hphantom{= } \rplbr{
		\mbox{} + \sqbr{\matr{cc}{
			\TT{I} & -\T{b}\plbr{\xi}\times{} \\
			\TT{0} & \TT{I}
		}} \cubr{\cvvect{
			\delta\T{v}_n \\
			\delta\T{\omega}_n
		}}
	}
	\\
	&\equu 
	\sqbr{\matr{cc}{
		\TT{R}^T\plbr{\xi} & \TT{0} \\
		\TT{0} & \TT{R}^T\plbr{\xi}
	}} \lplbr{
		\sqbr{\cvvect{
			\plbr{\T{v}\plbr{\xi} + \T{b}\plbr{\xi}\times \T{\omega}_n}\times{}
			\\
			\T{0}
		}} \delta\T{g}_n
	} \nonumber \\
	& \hphantom{= } \rplbr{
		\mbox{} + \sqbr{\matr{cc}{
			\TT{I} & -\T{b}\plbr{\xi}\times{} \\
			\TT{0} & \TT{I}
		}} \cubr{\cvvect{
			\delta\dot{\T{x}}_n \\
			\delta\dot{\T{g}}_n
		}}
	}
	.
\end{align}
\end{subequations}
After defining
\begin{subequations}
\begin{align}
	\T{f}_{a/\tilde{\T{v}}} &= \TT{R}\plbr{\xi} \tilde{\T{f}}_{a/\tilde{\T{v}}} \TT{R}^T\plbr{\xi} \\
	\T{f}_{a/\tilde{\T{\omega}}} &= \TT{R}\plbr{\xi} \tilde{\T{f}}_{a/\tilde{\T{\omega}}} \TT{R}^T\plbr{\xi} \\
	\T{c}_{a/\tilde{\T{v}}} &= \TT{R}\plbr{\xi} \tilde{\T{c}}_{a/\tilde{\T{v}}} \TT{R}^T\plbr{\xi} \\
	\T{c}_{a/\tilde{\T{\omega}}} &= \TT{R}\plbr{\xi} \tilde{\T{c}}_{a/\tilde{\T{\omega}}} \TT{R}^T\plbr{\xi}
	,
\end{align}
\end{subequations}
the linearization becomes
\begin{align}
	\sqbr{\matr{c}{
		\T{f}_{a/\tilde{\T{v}}}
		\\
		\T{o}\plbr{\xi}\times\T{f}_{a/\tilde{\T{v}}} + \T{c}_{a/\tilde{\T{v}}}
	}} \delta\dot{\T{x}}_n
	\nonumber \\
	+ \sqbr{\matr{c}{
		\T{f}_{a/\tilde{\T{\omega}}} - \T{f}_{a/\tilde{\T{v}}} \T{b}\plbr{\xi}\times{}
		\\
		\T{o}\plbr{\xi} \times \T{f}_{a/\tilde{\T{\omega}}}
		+ \T{c}_{a/\tilde{\T{\omega}}}
		- \plbr{
			\T{o}\plbr{\xi} \times \T{f}_{a/\tilde{\T{v}}}
			+ \T{c}_{a/\tilde{\T{v}}}
		}
		\T{b}\plbr{\xi}\times{}
	}} \delta\dot{\T{g}}_n
	\nonumber \\
	+ \sqbr{\matr{c}{
		\T{f}_{a/\tilde{\T{v}}} \plbr{\T{v}\plbr{\xi} + \T{b}\plbr{\xi} \times \T{\omega}_n} \times{} \\
		\plbr{
			\T{o}\plbr{\xi} \times \T{f}_{a/\tilde{\T{v}}}
			+ \T{c}_{a/\tilde{\T{v}}}
		} \plbr{\T{v}\plbr{\xi} + \T{b}\plbr{\xi} \times \T{\omega}_n} \times{}
	}} \delta\T{g}_n
	\nonumber \\
	- \sqbr{\cvvect{
		\Delta\T{f}_n \times{} \\
		\Delta\T{c}_n \times{} \\
	}} \delta\T{g}_n
	\equu
	\cubr{\cvvect{
		\delta\Delta\T{f}_n \\
		\delta\Delta\T{c}_n
	}}
	.
\end{align}





\section{Aerodynamic Beam (3 Nodes)}
The \texttt{aerodynamic beam3} element applies aerodynamic forces
to the nodes of a three node structural beam element. 

The aerodynamic forces/moments acting on each node 
are computed based on the relative velocity of a set of locations
along the beam and the airstream.
The kinematic quantities of the beam are computed based
on an interpolation of the kinematics of the three nodes.

\paragraph{Kinematics Interpolation.}
\emph{Note: this part is common to all elements that use the
three-node beam discretization and interpolation model.}
The generic field variable $\T{p}\plbr{x}$ is interpolated
using parabolic functions related to the value of the field 
variable at three locations that in general can be offset
from the nodes,
\begin{align}
	\T{p}\plbr{\xi}
	&=
	\sum_{i=1,2,3} N_i\plbr{\xi} \T{p}_i
	,
\end{align}
with
\begin{subequations}
\begin{align}
	N_1 &= \frac{1}{2} \xi \plbr{\xi - 1}
	\\
	N_2 &= 1 - \xi^2
	\\
	N_3 &= \frac{1}{2} \xi \plbr{\xi + 1}
	.
\end{align}
\end{subequations}

Orientation at an arbitrary location $\xi$:
\begin{align}
	\TT{R}\plbr{\xi}
	&=
	\TT{R}_{2_a} \TT{R}\plbr{\T{\theta}\plbr{\xi}}
	\\
	\T{\theta}\plbr{\xi}
	&=
	\sum_{i=1,3} N_i\plbr{\xi} \T{\theta}_{2\leftarrow i}
	= N_1\plbr{\xi} \T{\theta}_{2\leftarrow 1} + N_3\plbr{\xi} \T{\theta}_{2\leftarrow 3}
	\\
	\T{\theta}_{2\leftarrow i}
	&=
	\text{ax}\plbr{\text{exp}^{-1}\plbr{\TT{R}_{2_a}^T \TT{R}_{i_a}}}
	.
\end{align}
Orientation is dealt with specially, given its special nature.
The orientation of the mid node is used as a reference,
and the orientation parameters that express the relative orientation
between each of the end nodes and the mid node are interpolated.
The interpolated orientation parameters are used to compute
the interpolated relative orientation matrix, which is then
pre-multiplied by the orientation matrix of the mid node.
Summation in this case occurs on $i=1,3$ only because by definition
$\T{\theta}_{2\leftarrow 2}=\text{ax}(\text{exp}^{-1}((\TT{I}))\equiv\T{0}$.

Since the Euler-Rodrigues orientation parameters are used
(the so-caller `rotation vector'),
the magnitude of the relative orientation between each end node
and the mid node must be limited (formally, to $\pi$,
but it should be less for accuracy).

Position at an arbitrary location $\xi$:
\begin{align}
	\T{x}\plbr{\xi}
	&=
	\sum_{i=1,2,3} N_i\plbr{\xi} \plbr{
		\T{x}_i + \T{o}_i
	}
	,
\end{align}
with $\T{o}_i=\TT{R}_i \tilde{\T{o}}_i$.

Angular velocity at an arbitrary location $\xi$:
\begin{align}
	\T{\omega}\plbr{\xi}
	&=
	\sum_{i=1,2,3} N_i\plbr{\xi} \T{\omega}_i
	.
\end{align}

Velocity at an arbitrary location $\xi$:
\begin{align}
	\T{v}_{\text{kin}}\plbr{\xi}
	&=
	\sum_{i=1,2,3} N_i\plbr{\xi} \plbr{
		\T{v}_i
		+ \T{\omega}_i \times \T{o}_i
	}
	.
\end{align}

\paragraph{Perturbation of Interpolated Kinematics.}
Orientation perturbation at an arbitrary location $\xi$:
\begin{align}
	\delta\T{\theta}\plbr{\xi}
	&=
	\sum_{i=1,3} N_i\plbr{\xi} \delta\T{\theta}_{2\leftarrow i}
	\\
	\delta\T{\theta}_{2\leftarrow i}
	&=
	\TT{\Gamma}^{-1}\plbr{\T{\theta}_{2\leftarrow i}} \T{\theta}_{(2\leftarrow i) \delta}
	\\
	\T{\theta}_{(2\leftarrow i) \delta}
	&=
	\TT{R}_{2_a}^T \plbr{
		\T{\theta}_{i\delta}
		- \T{\theta}_{2\delta}
	}
	\\
	\T{\theta}_{\delta}\plbr{\xi}
	&=
	\T{\theta}_{2\delta}
	+ \sum_{i=1,3} \TT{R}_{2_a} \TT{\Gamma}\plbr{\T{\theta}\plbr{\xi}}
		N_i\plbr{\xi} \TT{\Gamma}^{-1}\plbr{\T{\theta}_{2\leftarrow i}}
		\TT{R}_{2_a}^T \plbr{\T{\theta}_{i\delta} - \T{\theta}_{2\delta}}
	\nonumber \\
	&= \plbr{
		\TT{I}
		- \TT{R}_{2_a} \TT{\Gamma}\plbr{\T{\theta}\plbr{\xi}}
			\sum_{i=1,3} N_i\plbr{\xi} \TT{\Gamma}^{-1}\plbr{\T{\theta}_{2\leftarrow i}}
			\TT{R}_{2_a}^T
	} \T{\theta}_{2\delta}
	\nonumber \\
	& \hphantom{= } \mbox{}
	+ \TT{R}_{2_a} \TT{\Gamma}\plbr{\T{\theta}\plbr{\xi}}
		N_1\TT{\Gamma}^{-1}\plbr{\T{\theta}_{2\leftarrow 1}}
		\TT{R}_{2_a}^T \T{\theta}_{1\delta}
	\nonumber \\
	& \hphantom{= } \mbox{}
	+ \TT{R}_{2_a} \TT{\Gamma}\plbr{\T{\theta}\plbr{\xi}}
		N_3\TT{\Gamma}^{-1}\plbr{\T{\theta}_{2\leftarrow 3}}
		\TT{R}_{2_a}^T \T{\theta}_{3\delta}
	\nonumber \\
	&=
	\sum_{i=1,2,3} \TT{\Theta}_i\plbr{\xi} \T{\theta}_{i\delta}
	,
\end{align}
with
\begin{subequations}
\begin{align}
	\TT{\Theta}_i\plbr{\xi}
	&=
	\TT{R}_{2_a} \TT{\Gamma}\plbr{\T{\theta}\plbr{\xi}}
		N_i\TT{\Gamma}^{-1}\plbr{\T{\theta}_{2\leftarrow i}}
		\TT{R}_{2_a}^T
		& i=\text{1 and 3}
	\\
	\TT{\Theta}_2\plbr{\xi}
	&=
	\TT{I} - \TT{\Theta}_1\plbr{\xi} - \TT{\Theta}_3\plbr{\xi}
\end{align}
\end{subequations}
playing the role of shape functions.

In fact, note that, when $N_1=1$ and $N_3=0$, then
$\TT{\Theta}_1=\TT{I}$, $\TT{\Theta}_2=\TT{\Theta}_3=\TT{0}$
and $\T{\theta}_{\delta}\plbr{\xi}=\T{\theta}_{1\delta}$,
while, when $N_1=0$ and $N_3=1$, then
$\TT{\Theta}_1=\TT{\Theta}_2=\TT{0}$, $\TT{\Theta}_3=\TT{I}$
and $\T{\theta}_{\delta}\plbr{\xi}=\T{\theta}_{3\delta}$.
Finally, when $N_1=N_3=0$, then
$\TT{\Theta}_1=\TT{\Theta}_3=\TT{0}$, $\TT{\Theta}_2=\TT{I}$
and $\T{\theta}_{\delta}\plbr{\xi}=\T{\theta}_{2\delta}$.
Moreover, $\sum_{i=1,2,3}\TT{\Theta}_i\plbr{\xi}=\TT{I} \ \forall\xi$.

Position perturbation at an arbitrary location $\xi$:
\begin{align}
	\delta\T{x}\plbr{\xi}
	&=
	\sum_{i=1,2,3} N_i\plbr{\xi} \plbr{
		\delta\T{x}_i + \T{\theta}_{i\delta}\times\T{o}_i
	}
	.
\end{align}

Angular velocity perturbation at an arbitrary location $\xi$:
\begin{align}
	\delta\T{\omega}\plbr{\xi}
	&=
	\sum_{i=1,2,3} N_i \delta\T{\omega}_i
	\nonumber \\
	&\equu
	\sum_{i=1,2,3} N_i \plbr{
		\delta\dot{\T{g}}_i
		- \T{\omega}_i \times \delta\T{g}_i
	}
	.
\end{align}

Velocity perturbation at an arbitrary location $\xi$:
\begin{align}
	\delta\T{v}_{\text{kin}}\plbr{\xi}
	&=
	\sum_{i=1,2,3} N_i\plbr{\xi} \plbr{
		\delta\T{v}_i
		+ \delta\T{\omega}_i \times \T{o}_i
		+ \T{\omega}_i \times \delta\T{o}_i
	}
	\nonumber \\
	&= 
	\sum_{i=1,2,3} N_i\plbr{\xi} \plbr{
		\delta\T{v}_i
		- \T{o}_i \times \delta\T{\omega}_i
		- \T{\omega}_i \times \T{o}_i \times \T{\theta}_{i\delta}
	}
	\nonumber \\
	&\equu
	\sum_{i=1,2,3} N_i\plbr{\xi} \plbr{
		\delta\dot{\T{x}}_i
		- \T{o}_i \times \delta\dot{\T{g}}_i
		- \plbr{\T{\omega}_i \times \T{o}_i} \times \delta\T{g}_i
	}
	.
\end{align}


\paragraph{Boundary Conditions Perturbation.}
Angular velocity perturbation:
\begin{align}
	\delta\tilde{\T{\omega}}\plbr{\xi}
	&= \TT{R}^T\plbr{\xi} \sum_{i=1,2,3}\plbr{
		N_i\plbr{\xi} \delta\T{\omega}_i
		+ \T{\omega}\plbr{\xi} \times \TT{\Theta}_i\plbr{\xi} \T{\theta}_{i\delta}
	}
	\nonumber \\
	&\equu
	\TT{R}^T\plbr{\xi} \plbr{
		N_i\plbr{\xi} \delta\dot{\T{g}}_i
		+ \plbr{
			\T{\omega}\plbr{\xi} \times \TT{\Theta}_i\plbr{\xi}
			- N_i\plbr{\xi} \T{\omega}_i \times{}
		} \delta\T{g}_i
	}
	.
\end{align}

Velocity perturbation:
\begin{align}
	\delta\tilde{\T{v}}\plbr{\xi}
	&=
	\TT{R}^T\plbr{\xi} \lplbr{
		N_i\plbr{\xi} \delta\T{v}_i
		- N_i\plbr{\xi} \T{o}_i \times \delta\T{\omega}_i
	}
	\nonumber \\
	& \hphantom{= \TT{R}^T\plbr{\xi}(}
	\rplbr{
		\mbox{} + \plbr{
			\T{v}\plbr{\xi} \times \TT{\Theta}_i\plbr{\xi}
			- N_i\plbr{\xi} \T{\omega}_i \times \T{o}_i \times{}
		} \T{\theta}_{i\delta}
	}
	\nonumber \\
	&\equu
	\TT{R}^T\plbr{\xi} \lplbr{
		N_i\plbr{\xi} \delta\dot{\T{x}}_i
		- N_i\plbr{\xi} \T{o}_i \times \delta\dot{\T{g}}_i
	}
	\nonumber \\
	& \hphantom{= \TT{R}^T\plbr{\xi}(}
	\rplbr{
		\mbox{} + \plbr{
			\T{v}\plbr{\xi} \times \TT{\Theta}_i\plbr{\xi}
			- N_i\plbr{\xi} \plbr{\T{\omega}_i \times \T{o}_i} \times{}
		} \delta\T{g}_i
	}
	.
\end{align}
They can be summarized as
\begin{align}
	\cubr{\cvvect{
		\delta\tilde{\T{v}}\plbr{\xi} \\
		\delta\tilde{\T{\omega}}\plbr{\xi}
	}}
	&=
	\sqbr{\matr{cc}{
		\TT{R}^T\plbr{\xi} & \TT{0} \\
		\TT{0} & \TT{R}^T\plbr{\xi}
	}} \sum_{i=1,2,3} \lplbr{
		N_i\plbr{\xi} \sqbr{\matr{cc}{
			\TT{I} & - \T{o}_i \times{} \\
			\TT{0} & \TT{I}
		}} \cubr{\cvvect{
			\delta\T{v}_i \\
			\delta\T{\omega}_i
		}}
	}
	\nonumber \\
	& \hphantom{= } \mbox{}
	+ \rplbr{
		\sqbr{\cvvect{
			\T{v}\plbr{\xi} \times \TT{\Theta}_i\plbr{\xi}
			- N_i\plbr{\xi} \T{\omega}_i \times \T{o}_i \times{} \\
			\T{\omega}\plbr{\xi} \times \TT{\Theta}_i\plbr{\xi}
		}} \T{\theta}_{i\delta}
	}
	\nonumber \\
	&\equu
	\sqbr{\matr{cc}{
		\TT{R}^T\plbr{\xi} & \TT{0} \\
		\TT{0} & \TT{R}^T\plbr{\xi}
	}} \sum_{i=1,2,3} \lplbr{
		N_i\plbr{\xi} \sqbr{\matr{cc}{
			\TT{I} & - \T{o}_i \times{} \\
			\TT{0} & \TT{I}
		}} \cubr{\cvvect{
			\delta\dot{\T{x}}_i \\
			\delta\dot{\T{g}}_i
		}}
	}
	\nonumber \\
	& \hphantom{= } \mbox{}
	+ \rplbr{
		\sqbr{\cvvect{
			\T{v}\plbr{\xi} \times \TT{\Theta}_i\plbr{\xi}
			- N_i\plbr{\xi} \plbr{\T{\omega}_i \times \T{o}_i} \times{} \\
			\T{\omega}\plbr{\xi} \times \TT{\Theta}_i\plbr{\xi}
			- N_i\plbr{\xi} \T{\omega}_i \times{}
		}} \delta\T{g}_i
	}
	.
\end{align}

\paragraph{Contribution to Jacobian Matrix.}
After defining
\begin{subequations}
\begin{align}
	\T{f}_{a/\tilde{\T{v}}} &= \TT{R}\plbr{\xi} \tilde{\T{f}}_{a/\tilde{\T{v}}} \TT{R}^T\plbr{\xi} \\
	\T{f}_{a/\tilde{\T{\omega}}} &= \TT{R}\plbr{\xi} \tilde{\T{f}}_{a/\tilde{\T{\omega}}} \TT{R}^T\plbr{\xi} \\
	\T{c}_{a/\tilde{\T{v}}} &= \TT{R}\plbr{\xi} \tilde{\T{c}}_{a/\tilde{\T{v}}} \TT{R}^T\plbr{\xi} \\
	\T{c}_{a/\tilde{\T{\omega}}} &= \TT{R}\plbr{\xi} \tilde{\T{c}}_{a/\tilde{\T{\omega}}} \TT{R}^T\plbr{\xi} \\
	\TT{B}_{\tilde{\T{v}}} &= \T{v}\plbr{\xi} \times \TT{\Theta}_i\plbr{\xi}
		- N_i\plbr{\xi} \plbr{\T{\omega}_i \times \T{o}_i} \times{} \\
	\TT{B}_{\tilde{\T{\omega}}} &= \T{\omega}\plbr{\xi} \times \TT{\Theta}_i\plbr{\xi}
		- N_i\plbr{\xi} \T{\omega}_i \times{} \\
	\T{d}_n\plbr{\xi} &= \T{x}\plbr{\xi} - \T{x}_n
	,
\end{align}
\end{subequations}
the contribution of the $i$-th node's motion to the equilibrium
of the $n$-th node ($i=1,2,3$, $n=1,2,3$) is
\begin{align}
	N_i\plbr{\xi} \sqbr{\matr{c}{
		\T{f}_{a/\tilde{\T{v}}} \\
		\T{d}_n\plbr{\xi} \times \T{f}_{a/\tilde{\T{v}}}
		+ \T{c}_{a/\tilde{\T{v}}}
	}} \delta\dot{\T{x}}_i
	\nonumber \\
%
	\mbox{}
	+ N_i\plbr{\xi} \sqbr{\matr{c}{
		\T{f}_{a/\tilde{\T{\omega}}}
		- \T{f}_{a/\tilde{\T{v}}} \T{o}_i \times{} \\
		\T{d}_n\plbr{\xi} \times \T{f}_{a/\tilde{\T{\omega}}}
		+ \T{c}_{a/\tilde{\T{\omega}}}
		- \plbr{
			\T{d}_n\plbr{\xi} \times \T{f}_{a/\tilde{\T{v}}}
			+ \T{c}_{a/\tilde{\T{v}}}
		} \T{o}_i \times{}
	}} \delta\dot{\T{g}}_i
	\nonumber \\
%
	\mbox{}
	+ \sqbr{\matr{c}{
		\T{f}_{a/\tilde{\T{v}}} \TT{B}_{\tilde{\T{v}}}
			+ \T{f}_{a/\tilde{\T{\omega}}} \TT{B}_{\tilde{\T{\omega}}} \\
		\T{d}_n\plbr{\xi} \times \plbr{
			\T{f}_{a/\tilde{\T{v}}} \TT{B}_{\tilde{\T{v}}}
				+ \T{f}_{a/\tilde{\T{\omega}}} \TT{B}_{\tilde{\T{\omega}}}
		}
		+ \T{c}_{a/\tilde{\T{v}}} \TT{B}_{\tilde{\T{v}}}
			+ \T{c}_{a/\tilde{\T{\omega}}} \TT{B}_{\tilde{\T{\omega}}}
	}} \delta\T{g}_i
	\nonumber \\
%
	\mbox{}
	+ \sqbr{\matr{c}{
		\TT{0} \\
		- \plbr{
			N_i - \delta_{ni}
		} \Delta\T{f}_n \times{}
	}} \delta\T{x}_i
	\nonumber \\
%
	\mbox{}
	+ \sqbr{\matr{c}{
		- \Delta\T{f}_n \times \TT{\Theta}_i\plbr{\xi} \\
		- \plbr{
			\T{d}_n\plbr{\xi} \times \Delta\T{f}_n \times{}
			+ \Delta\T{c}_n \times{}
		} \TT{\Theta}_i\plbr{\xi}
		+ \Delta\T{f}_n \times N_i \T{o}_i \times{}
	}} \delta\T{g}_i
	\nonumber \\
%
	\equu
	\cubr{\cvvect{
		\delta\Delta\T{f}_n \\
		\delta\Delta\T{c}_n
	}}
	,
\end{align}
where $\delta_{ni}$ is Dirac's delta, which is 1 when $i=n$, and 0 otherwise.




\section{Aerodynamic Beam (2 Nodes)}
The \texttt{aerodynamic beam2} element applies aerodynamic forces
to the nodes of a two node structural beam element. 

The aerodynamic forces/moments acting on each node 
are computed based on the relative velocity of a set of locations
along the beam and the airstream.
The kinematic quantities of the beam are computed based
on an interpolation of the kinematics of the three nodes.

\paragraph{Kinematics Interpolation.}
\emph{Note: this part is common to all elements that use the
two-node beam discretization and interpolation model.}
The generic field variable $\T{p}\plbr{x}$ is interpolated
using linear functions related to the value of the field 
variable at two locations that in general can be offset
from the nodes,
\begin{align}
	\T{p}\plbr{\xi}
	&=
	\sum_{i=1,2} N_i\plbr{\xi} \T{p}_i
	,
\end{align}
with
\begin{subequations}
\label{eq:aero:beam2:N}
\begin{align}
	N_1 &= \frac{1}{2} \plbr{1 - \xi}
	\\
	N_2 &= \frac{1}{2} \plbr{1 + \xi}
	.
\end{align}
\end{subequations}

Orientation at an arbitrary location $\xi$:
\begin{align}
	\TT{R}\plbr{\xi}
	&=
	\TT{R}_{\text{mid}} \text{exp}\plbr{\T{\theta}\plbr{\xi}\times{}}
	\\
	\TT{R}_{\text{mid}}
	&=
	\TT{R}_{1_a} \text{exp}\plbr{\overline{\T{\theta}}\times{}}
	= \TT{R}_{2_a} \text{exp}\plbr{\overline{\T{\theta}}\times{}}^T
	\\
	\overline{\T{\theta}}
	&=
	\frac{1}{2} \text{ax}\plbr{\text{exp}^{-1}\plbr{\TT{R}_{1_a}^T \TT{R}_{2_a}}}
	\\
	\T{\theta}\plbr{\xi}
	&=
	\sum_{i=1,2} N_i\plbr{\xi} \T{\theta}_{\text{mid}\leftarrow i}
	= N_1\plbr{\xi} \T{\theta}_{\text{mid}\leftarrow 1}
		+ N_2\plbr{\xi} \T{\theta}_{\text{mid}\leftarrow 2}
	\\
	\T{\theta}_{\text{mid}\leftarrow 1}
	&= -\overline{\T{\theta}}
	\\
	\T{\theta}_{\text{mid}\leftarrow 2}
	&=
	\overline{\T{\theta}}
	.
\end{align}
Orientation is dealt with specially, given its special nature.
The orientation of the mid point is used as a reference,
and the orientation parameters that express the relative orientation
between each of the end nodes and the mid point are interpolated.
The interpolated orientation parameters are used to compute
the interpolated relative orientation matrix, which is then
pre-multiplied by the orientation matrix of the mid point.

Since the Euler-Rodrigues parametrization is used
(the so-called `rotation vector'),
the magnitude of the relative orientation between each end node
must be limited (formally, to $\pi$, but it should be less for accuracy).

Note that since the generic orientation $\TT{R}(\xi)$
is the result of a sequence of orientations
about a common axis $\overline{\T{\theta}}$,
it can be conveniently rewritten as
\begin{align}
	\TT{R}\plbr{\xi}
	&= \TT{R}_{1_a} \text{exp}\plbr{\overline{\T{\theta}}\times{}}
		\text{exp}\plbr{\plbr{\plbr{N_2\plbr{\xi} - N_1\plbr{\xi}} \overline{\T{\theta}}}\times{}}
	\nonumber \\
	&= \TT{R}_{1_a} 
		\text{exp}\plbr{\plbr{\plbr{1 + N_2\plbr{\xi} - N_1\plbr{\xi}} \overline{\T{\theta}}}\times{}}
	.
\end{align}
In fact, when $\xi=-1$, $N_1=1$ and $N_2=0$,
then $\TT{R}\plbr{\xi}=\TT{R}_{1_a}$,
while when $\xi=1$, $N_1=0$ and $N_2=1$,
then $\TT{R}\plbr{\xi}=\TT{R}_{2_a}$.

Position at an arbitrary location $\xi$:
\begin{align}
	\T{x}\plbr{\xi}
	&=
	\sum_{i=1,2} N_i\plbr{\xi} \plbr{
		\T{x}_i + \T{o}_i
	}
	,
\end{align}
with $\T{o}_i=\TT{R}_i \tilde{\T{o}}_i$.

Angular velocity at an arbitrary location $\xi$:
\begin{align}
	\T{\omega}\plbr{\xi}
	&=
	\sum_{i=1,2} N_i\plbr{\xi} \T{\omega}_i
	.
\end{align}

Velocity at an arbitrary location $\xi$:
\begin{align}
	\T{v}_{\text{kin}}\plbr{\xi}
	&=
	\sum_{i=1,2} N_i\plbr{\xi} \plbr{
		\T{v}_i
		+ \T{\omega}_i \times \T{o}_i
	}
	.
\end{align}

\paragraph{Perturbation of Interpolated Kinematics.}
Orientation perturbation at an arbitrary location $\xi$:
\begin{align}
	\T{\theta}_{\delta}\plbr{\xi}
	&=
	\frac{1 + N_2\plbr{\xi} - N_1\plbr{\xi}}{2} \T{\theta}_{2\delta}
	+ \frac{1 + N_1\plbr{\xi} - N_2\plbr{\xi}}{2} \T{\theta}_{1\delta}
	\\
	&=
	\sum_{i=1,2} \mathcal{N}_i\plbr{\xi} \T{\theta}_{i\delta}
	,
\end{align}
with
\begin{subequations}
\begin{align}
	\mathcal{N}_i\plbr{\xi}
	&=
	\frac{1 + N_i\plbr{\xi} - N_{3-i}\plbr{\xi}}{2}
	\label{eq:aero:beam2:NN}
	.
\end{align}
\end{subequations}
Note, however, that according to Eqs.~(\ref{eq:aero:beam2:N}),
the shape functions of Eq.~(\ref{eq:aero:beam2:NN})
are $\mathcal{N}_i\plbr{\xi}=N_i\plbr{\xi}$.

Position perturbation at an arbitrary location $\xi$:
\begin{align}
	\delta\T{x}\plbr{\xi}
	&=
	\sum_{i=1,2} N_i\plbr{\xi} \plbr{
		\delta\T{x}_i + \T{\theta}_{i\delta}\times\T{o}_i
	}
	.
\end{align}

Angular velocity perturbation at an arbitrary location $\xi$:
\begin{align}
	\delta\T{\omega}\plbr{\xi}
	&=
	\sum_{i=1,2} N_i \delta\T{\omega}_i
	\nonumber \\
	&\equu
	\sum_{i=1,2} N_i \plbr{
		\delta\dot{\T{g}}_i
		- \T{\omega}_i \times \delta\T{g}_i
	}
	.
\end{align}

Velocity perturbation at an arbitrary location $\xi$:
\begin{align}
	\delta\T{v}_{\text{kin}}\plbr{\xi}
	&=
	\sum_{i=1,2} N_i\plbr{\xi} \plbr{
		\delta\T{v}_i
		+ \delta\T{\omega}_i \times \T{o}_i
		+ \T{\omega}_i \times \delta\T{o}_i
	}
	\nonumber \\
	&= 
	\sum_{i=1,2} N_i\plbr{\xi} \plbr{
		\delta\T{v}_i
		- \T{o}_i \times \delta\T{\omega}_i
		- \T{\omega}_i \times \T{o}_i \times \T{\theta}_{i\delta}
	}
	\nonumber \\
	&\equu
	\sum_{i=1,2} N_i\plbr{\xi} \plbr{
		\delta\dot{\T{x}}_i
		- \T{o}_i \times \delta\dot{\T{g}}_i
		- \plbr{\T{\omega}_i \times \T{o}_i} \times \delta\T{g}_i
	}
	.
\end{align}


\paragraph{Boundary Conditions Perturbation.}
Angular velocity perturbation:
\begin{align}
	\delta\tilde{\T{\omega}}\plbr{\xi}
	&= \TT{R}^T\plbr{\xi} \sum_{i=1,2} \plbr{
		N_i\plbr{\xi} \delta\T{\omega}_i
		+ \T{\omega}\plbr{\xi} \times \mathcal{N}_i\plbr{\xi} \T{\theta}_{i\delta}
	}
	\nonumber \\
	&\equu
	\TT{R}^T\plbr{\xi} \sum_{i=1,2} \plbr{
		N_i\plbr{\xi} \delta\dot{\T{g}}_i
		+ \plbr{
			\mathcal{N}_i \T{\omega}\plbr{\xi}
			- N_i\plbr{\xi} \T{\omega}_i
		} \times \delta\T{g}_i
	}
	\nonumber \\
	&=
	\TT{R}^T\plbr{\xi} \sum_{i=1,2} N_i\plbr{\xi} \plbr{
		\delta\dot{\T{g}}_i
		+ \plbr{
			\T{\omega}\plbr{\xi}
			- \T{\omega}_i
		} \times \delta\T{g}_i
	}
	.
\end{align}

Velocity perturbation:
\begin{align}
	\delta\tilde{\T{v}}\plbr{\xi}
	&=
	\TT{R}^T\plbr{\xi} \lplbr{
		N_i\plbr{\xi} \delta\T{v}_i
		- N_i\plbr{\xi} \T{o}_i \times \delta\T{\omega}_i
	}
	\nonumber \\
	& \hphantom{= \TT{R}^T\plbr{\xi}(}
	\rplbr{
		\mbox{} + \plbr{
			\mathcal{N}_i\plbr{\xi} \T{v}\plbr{\xi} \times{}
			- N_i\plbr{\xi} \T{\omega}_i \times \T{o}_i \times{}
		} \T{\theta}_{i\delta}
	}
	\nonumber \\
	&\equu
	\TT{R}^T\plbr{\xi} \lplbr{
		N_i\plbr{\xi} \delta\dot{\T{x}}_i
		- N_i\plbr{\xi} \T{o}_i \times \delta\dot{\T{g}}_i
	}
	\nonumber \\
	& \hphantom{= \TT{R}^T\plbr{\xi}(}
	\rplbr{
		\mbox{} + \plbr{
			\mathcal{N}_i\plbr{\xi} \T{v}\plbr{\xi}
			- N_i\plbr{\xi} \T{\omega}_i \times \T{o}_i
		} \times \delta\T{g}_i
	}
	\nonumber \\
	&=
	\TT{R}^T\plbr{\xi} N_i\plbr{\xi} \plbr{
		\delta\dot{\T{x}}_i
		- \T{o}_i \times \delta\dot{\T{g}}_i
		+ \plbr{
			\T{v}\plbr{\xi}
			- \T{\omega}_i \times \T{o}_i
		} \times \delta\T{g}_i
	}
	.
\end{align}
They can be summarized as
\begin{align}
	\cubr{\cvvect{
		\delta\tilde{\T{v}}\plbr{\xi} \\
		\delta\tilde{\T{\omega}}\plbr{\xi}
	}}
	&=
	\sqbr{\matr{cc}{
		\TT{R}^T\plbr{\xi} & \TT{0} \\
		\TT{0} & \TT{R}^T\plbr{\xi}
	}} \sum_{i=1,2} \lplbr{
		N_i\plbr{\xi} \sqbr{\matr{cc}{
			\TT{I} & - \T{o}_i \times{} \\
			\TT{0} & \TT{I}
		}} \cubr{\cvvect{
			\delta\T{v}_i \\
			\delta\T{\omega}_i
		}}
	}
	\nonumber \\
	& \hphantom{= } \mbox{}
	+ \rplbr{
		\sqbr{\cvvect{
			\mathcal{N}_i\plbr{\xi} \T{v}\plbr{\xi} \times{}
			- N_i\plbr{\xi} \T{\omega}_i \times \T{o}_i \times{} \\
			\mathcal{N}_i\plbr{\xi} \T{\omega}\plbr{\xi} \times{}
		}} \T{\theta}_{i\delta}
	}
	\nonumber \\
	&\equu
	\sqbr{\matr{cc}{
		\TT{R}^T\plbr{\xi} & \TT{0} \\
		\TT{0} & \TT{R}^T\plbr{\xi}
	}} \sum_{i=1,2} \lplbr{
		N_i\plbr{\xi} \sqbr{\matr{cc}{
			\TT{I} & - \T{o}_i \times{} \\
			\TT{0} & \TT{I}
		}} \cubr{\cvvect{
			\delta\dot{\T{x}}_i \\
			\delta\dot{\T{g}}_i
		}}
	}
	\nonumber \\
	& \hphantom{= } \mbox{}
	+ \rplbr{
		\sqbr{\cvvect{
			\plbr{
				\mathcal{N}_i\plbr{\xi} \T{v}\plbr{\xi}
				- N_i\plbr{\xi} \T{\omega}_i \times \T{o}_i
			} \times{} \\
			\plbr{
				\mathcal{N}_i\plbr{\xi} \T{\omega}\plbr{\xi}
				- N_i\plbr{\xi} \T{\omega}_i
			} \times{}
		}} \delta\T{g}_i
	}
	\nonumber \\
	&=
	\sqbr{\matr{cc}{
		\TT{R}^T\plbr{\xi} & \TT{0} \\
		\TT{0} & \TT{R}^T\plbr{\xi}
	}} \sum_{i=1,2} N_i\plbr{\xi} \lplbr{
		\sqbr{\matr{cc}{
			\TT{I} & - \T{o}_i \times{} \\
			\TT{0} & \TT{I}
		}} \cubr{\cvvect{
			\delta\dot{\T{x}}_i \\
			\delta\dot{\T{g}}_i
		}}
	}
	\nonumber \\
	& \hphantom{= } \mbox{}
	+ \rplbr{
		\sqbr{\cvvect{
			\plbr{
				\T{v}\plbr{\xi}
				- \T{\omega}_i \times \T{o}_i
			} \times{} \\
			\plbr{
				\T{\omega}\plbr{\xi}
				- \T{\omega}_i
			} \times{}
		}} \delta\T{g}_i
	}
	.
\end{align}

i\paragraph{Contribution to Jacobian Matrix.}
After defining
\begin{subequations}
\begin{align}
	\T{f}_{a/\tilde{\T{v}}} &= \TT{R}\plbr{\xi} \tilde{\T{f}}_{a/\tilde{\T{v}}} \TT{R}^T\plbr{\xi} \\
	\T{f}_{a/\tilde{\T{\omega}}} &= \TT{R}\plbr{\xi} \tilde{\T{f}}_{a/\tilde{\T{\omega}}} \TT{R}^T\plbr{\xi} \\
	\T{c}_{a/\tilde{\T{v}}} &= \TT{R}\plbr{\xi} \tilde{\T{c}}_{a/\tilde{\T{v}}} \TT{R}^T\plbr{\xi} \\
	\T{c}_{a/\tilde{\T{\omega}}} &= \TT{R}\plbr{\xi} \tilde{\T{c}}_{a/\tilde{\T{\omega}}} \TT{R}^T\plbr{\xi} \\
	\TT{B}_{\tilde{\T{v}}} &= \plbr{
		\mathcal{N}_i\plbr{\xi} \T{v}\plbr{\xi}
		- N_i\plbr{\xi} \T{\omega}_i \times \T{o}_i
	} \times{}
	\nonumber \\
	&= N_i\plbr{\xi} \plbr{
		\T{v}\plbr{\xi}
		- \T{\omega}_i \times \T{o}_i
	} \times{}
	\\
	\TT{B}_{\tilde{\T{\omega}}} &= \plbr{
		\mathcal{N}_i\plbr{\xi} \T{\omega}\plbr{\xi}
		- N_i\plbr{\xi} \T{\omega}_i
	} \times{}
	\nonumber \\
	&= N_i\plbr{\xi} \plbr{
		\T{\omega}\plbr{\xi}
		- \T{\omega}_i
	} \times{}
	\\
	\T{d}_n\plbr{\xi} &= \T{x}\plbr{\xi} - \T{x}_n
	,
\end{align}
\end{subequations}
the contribution of the $i$-th node's motion to the equilibrium
of the $n$-th node ($i=1,2$, $n=1,2$) is
\begin{align}
	N_i\plbr{\xi} \sqbr{\matr{c}{
		\T{f}_{a/\tilde{\T{v}}} \\
		\T{d}_n\plbr{\xi} \times \T{f}_{a/\tilde{\T{v}}}
		+ \T{c}_{a/\tilde{\T{v}}}
	}} \delta\dot{\T{x}}_i
	\nonumber \\
%
	\mbox{}
	+ N_i\plbr{\xi} \sqbr{\matr{c}{
		\T{f}_{a/\tilde{\T{\omega}}}
		- \T{f}_{a/\tilde{\T{v}}} \T{o}_i \times{} \\
		\T{d}_n\plbr{\xi} \times \T{f}_{a/\tilde{\T{\omega}}}
		+ \T{c}_{a/\tilde{\T{\omega}}}
		- \plbr{
			\T{d}_n\plbr{\xi} \times \T{f}_{a/\tilde{\T{v}}}
			+ \T{c}_{a/\tilde{\T{v}}}
		} \T{o}_i \times{}
	}} \delta\dot{\T{g}}_i
	\nonumber \\
%
	\mbox{}
	+ \sqbr{\matr{c}{
		\T{f}_{a/\tilde{\T{v}}} \TT{B}_{\tilde{\T{v}}}
			+ \T{f}_{a/\tilde{\T{\omega}}} \TT{B}_{\tilde{\T{\omega}}} \\
		\T{d}_n\plbr{\xi} \times \plbr{
			\T{f}_{a/\tilde{\T{v}}} \TT{B}_{\tilde{\T{v}}}
				+ \T{f}_{a/\tilde{\T{\omega}}} \TT{B}_{\tilde{\T{\omega}}}
		}
		+ \T{c}_{a/\tilde{\T{v}}} \TT{B}_{\tilde{\T{v}}}
			+ \T{c}_{a/\tilde{\T{\omega}}} \TT{B}_{\tilde{\T{\omega}}}
	}} \delta\T{g}_i
	\nonumber \\
%
	\mbox{}
	+ \sqbr{\matr{c}{
		\TT{0} \\
		- \plbr{
			N_i - \delta_{ni}
		} \Delta\T{f}_n \times{}
	}} \delta\T{x}_i
	\nonumber \\
%
	\mbox{}
	+ \sqbr{\matr{c}{
		- \mathcal{N}_i\plbr{\xi} \Delta\T{f}_n \times{} \\
		- \mathcal{N}_i\plbr{\xi} \plbr{
			\T{d}_n\plbr{\xi} \times \Delta\T{f}_n \times{}
			+ \Delta\T{c}_n \times{}
		}
		+ \Delta\T{f}_n \times N_i \T{o}_i \times{}
	}} \delta\T{g}_i
	\nonumber \\
%
	\equu
	\cubr{\cvvect{
		\delta\Delta\T{f}_n \\
		\delta\Delta\T{c}_n
	}}
	,
\end{align}
where $\delta_{ni}$ is Dirac's delta, which is 1 when $i=n$, and 0 otherwise.

\section{Unsteady aerodynamics model}
% MBDyn (C) is a multibody analysis code.
% http://www.mbdyn.org
%
% Copyright (C) 1996-2015
%
% Pierangelo Masarati  <masarati@aero.polimi.it>
%
% Dipartimento di Ingegneria Aerospaziale - Politecnico di Milano
% via La Masa, 34 - 20156 Milano, Italy
% http://www.aero.polimi.it
%
% Changing this copyright notice is forbidden.
%
% This program is free software; you can redistribute it and/or modify
% it under the terms of the GNU General Public License as published by
% the Free Software Foundation (version 2 of the License).
% 
%
% This program is distributed in the hope that it will be useful,
% but WITHOUT ANY WARRANTY; without even the implied warranty of
% MERCHANTABILITY or FITNESS FOR A PARTICULAR PURPOSE.  See the
% GNU General Public License for more details.
%
% You should have received a copy of the GNU General Public License
% along with this program; if not, write to the Free Software
% Foundation, Inc., 59 Temple Place, Suite 330, Boston, MA  02111-1307  USA
%
% Mattia Mattaboni <mattaboni@aero.polimi.it> is the author of this document

\emph{(Author: Mattia Mattaboni)} \\
The 2D unsteady aerodynamics loads are computed implementing
the Theodorsen theory in state-space form using
the Wagner approximation of the Theodorsen function.
\begin{subequations}
\begin{align}
	\tilde{\T{f}}_a &= 
		\frac{1}{2} \rho \tilde{\T{v}}^T \tilde{\T{v}} c
		\tilde{\T{c}}_{f_{a}}
		\plbr{ \T{y} 
		\plbr{ \tilde{\T{v}}, \tilde{\T{\omega}}, \T{q}}, U_\infty}\\
	\tilde{\T{c}}_a &= 
		\frac{1}{2} \rho \tilde{\T{v}}^T \tilde{\T{v}} c^2
		\tilde{\T{c}}_{c_{a}} 
		\plbr{ \T{y} 
		\plbr{ \tilde{\T{v}}, \tilde{\T{\omega}}, \T{q}}, U_\infty}\\
	\T{0} &=
		\T{g}\plbr{\tilde{\T{v}},\tilde{\T{\omega}},\T{q}, \dot{\T{q}}} = 
		\dot{\T{q}} - \T{A}\plbr{U_\infty} \T{q} - \T{B}\plbr{U_\infty}
 		\T{u}\plbr{\tilde{\T{v}}, \tilde{\T{\omega}}},
\end{align}
\end{subequations}
where $\rho$ is the air density, $c$ the airfoil chord and $\T{y}$ is 
\begin{equation}
	\T{y} = \TT{C}\plbr{U_\infty} \T{q} + \TT{D} \plbr{U_\infty}
 		\T{u}\plbr{\tilde{\T{v}}, \tilde{\T{\omega}}}.
\end{equation}

Matrix $\TT{A}$ is
\begin{equation}
	\footnotesize
	\TT{A} = \left[
		\begin{array}{cccccc}
		0 & 1 & 0 & 0 & 0 & 0\\
		-b_1 b_2 \plbr{ \frac{2U_\infty}{c}}^2 & -\plbr{ b_1 + b_2} \plbr{\frac{2 U_\infty}{c}}&
 		0 & 0 & 0 & 0 \\
		0 & 0 & -2\omega_\text{PD} & -\omega_\text{PD}^2 & 0 & 0\\
		0 & 0 & 1 & 0 & 0 & 0\\
		0 & 0 & 0 & 0 &-2\omega_\text{PD} & -\omega_\text{PD}^2 \\
		0 & 0 & 0 & 0 & 1 & 0
		\end{array}
		\right],
\end{equation}
where $A_1$, $A_2$, $b_1$ and $b_2$ are the coefficients of the Theodorsen
function approximation (Table~\ref{tab:aero:unsteady:theodorsen-coeffs}
from \cite{BIELAWA92} and \cite{LEISHMAN-2006}),
$\omega_\text{PD}$ is the frequency of the pseudo-derivative algorithm
\begin{align}
	\mathcal{L}\plbr{\dot{f}}
	&= s \mathcal{L}\plbr{f} - f\plbr{0}
	\cong \frac{s}{\plbr{1 + s/\omega_\text{PD}}^2} \mathcal{L}\plbr{f}
\end{align}
(where $f\plbr{0}$ can be neglected under broad assumptions),
and
\begin{equation}
%U_\infty = \sqrt{\tilde{\T{v}}^T\tilde{\T{v}}}.
U_\infty = \sqrt{ \tilde{v}_x^2 +\tilde{v}_y^2}.
\end{equation}

\begin{table}
\centering
\caption{Coefficients of the Wagner indicial response approximation
of Theodorsen's function (\cite{BIELAWA92,LEISHMAN-2006})}
\label{tab:aero:unsteady:theodorsen-coeffs}
\begin{tabular}{r|r@{.}lr@{.}l}
& \multicolumn{2}{c}{approx.~1} & \multicolumn{2}{c}{approx.~2} \\
\hline
$A_1$ 	& 0&165		& 0&165 \\
$A_2$ 	& 0&335		& 0&335 \\
$b_1$ 	& 0&0455	& 0&041 \\
$b_2$	& 0&3		& 0&32
\end{tabular}
\end{table}

Matrix $\T{B}$ is
\begin{equation}
	\T{B} = \tilde{\T{B}} \T{T}_u \plbr{U_\infty},
\end{equation}
where
\begin{equation}
	\tilde{\T{B}} = \left[
		\begin{array}{ccc}
			0 & 0 & 0 \\
			1 & 0 & 0 \\
			0 & 1 & 0 \\
			0 & 0 & 0 \\
			0 & 0 & 1 \\
			0 & 0 & 0 
		\end{array}
		\right],
\end{equation}
and
\begin{equation}
	\T{T}_u = \left[
		\begin{array}{c}
		\begin{array}{cc}
		0 & 1
		\end{array}\\ 
		\left[
		\begin{array}{cc}
			1 & -\frac{d_{1/4}}{U_\infty} \\
			1 & -\frac{d_{3/4}}{U_\infty} 
		\end{array}
		\right]^{-1}
		\end{array}
		\right] = \left[ 
		\begin{array}{c}
		\begin{array}{cc}
		0 & 1
		\end{array}\\
		\frac{1}{d_{3/4}-d_{1/4}} \left[  
		\begin{array}{cc}
			d_{3/4} & -{d_{1/4}} \\
			U_\infty & -U_\infty
		\end{array}
		\right]
		\end{array}
		\right].
\end{equation}
\begin{equation}
	\T{B} = \left[
		\begin{array}{cc}
		0 & 0 \\
		0 & 1 \\
		\frac{1}{d_{3/4}-d_{1/4}} d_{3/4} & 
		-\frac{1}{d_{3/4}-d_{1/4}} d_{1/4}\\ 
		0 & 0 \\
		\frac{1}{d_{3/4}-d_{1/4}} U_\infty & 
		-\frac{1}{d_{3/4}-d_{1/4}} U_\infty \\ 
		0 & 0
		\end{array}
		\right]
\end{equation}
		
		

\begin{equation}
	\T{u} = \cubr{\cvvect{
		\tan^{-1} \plbr{ \frac{-V_y - \omega_z d_{1/4}}{V_x}}\\
		\tan^{-1} \plbr{ \frac{-V_y - \omega_z d_{3/4}}{V_x}}\\
	}}
\end{equation}
		
Matrix $\T{C}$ is
\begin{equation}
	\T{C} = \left[
		\begin{array}{cccccc}
		\left(A_1+A_2\right)b_1 b_2 \left( \frac{2 U_\infty}{c}\right)^2 &
		 \left(A_1  b_1 + A_2 b_2 \right) \left( \frac{2 U_\infty}{c}\right) & 0 & 0 &0 & 0 \\
		0 & 0 & \omega_\text{PD}^2 & 0 & 0 & 0 \\
		0 & 0 & 0 & 0 & \omega_\text{PD}^2 & 0 \\
		0 & 0 & 0 & 0 & 0 & 0 
		\end{array}
		\right].
\end{equation}
Matrix $\T{D}$ is
\begin{equation}
	\T{D} = \tilde{\T{D}} \T{T}_u \plbr{U_\infty},
\end{equation}
where
\begin{equation}
	\tilde{\T{D}} = \left[
		\begin{array}{ccc}
		\left( 1 - A_1 - A_2 \right) & 0 & 0 \\
		0 & 0 & 0 \\
		0 & 0 & 0 \\
		0 & 0 & 1
		\end{array}
		\right].
\end{equation}
\begin{equation}
	\T{D} = \left[
		\begin{array}{cc}
		0 & \left( 1 -A_1 - A_2 \right) \\
		0 & 0 \\ 
		0 & 0 \\ 
		\frac{1}{d_{3/4}-d_{1/4}} U_\infty & 
		-\frac{1}{d_{3/4}-d_{1/4}} U_\infty \\ 
		\end{array}
		\right].
\end{equation}
	
\begin{equation}
	\tilde{\T{c}}_{f_{a}} = \cubr{\cvvect{
		-c_d \\
		c_l \\
		0
	}}
	= \T{c}_{f_a}^\text{lookup}\plbr{\T{y}} + \tilde{\T{T}}_{f_a} \T{y}.
\end{equation}
\begin{equation}
	\tilde{\T{c}}_{c_{a}} = \left\{
		\begin{array}{c}
		0 \\ 0 \\ c_{m}
		\end{array}
		\right\} = 
		\T{c}_{c_a}^\text{lookup}\plbr{\T{y}} + \tilde{\T{T}}_{c_a} \T{y}.
\end{equation}
The second part is the noncirculatory effect.

\begin{equation}
	\left\{ 
	\begin{array}{c}
	c_l^{NC} \\ c_m^{NC}
	\end{array} 
	\right\}=  \T{T}_1 \T{T}_2 \plbr{U_\infty} \T{y} = \tilde{\T{T}} \T{y},
\end{equation}
where
\begin{equation}
	\T{T}_1 = \left[
		\begin{array}{cc}
		 1 & 0 \\
		-\frac{a+\frac{1}{2}}{2} & 1
		\end{array}
		\right],
\end{equation}
and
\begin{equation}
	\T{T}_2 = \frac{C_{l_\alpha}}{2} \frac{c}{2}\left[
		\begin{array}{cccc} 
		0 & \frac{1}{U_\infty} & -\frac{c a}{2 U_\infty^2} & 0\\
		0 & \frac{a}{2 U_\infty} & -\frac{c}{4 U_\infty^2}\left( \frac{1}{8} + a^2\right) & -\frac{1}{4U_\infty}
		\end{array}
		\right].
\end{equation}
		
	
\begin{equation}
	\tilde{\T{T}} = \frac{C_{l_\alpha}}{2} \frac{c}{2} \left[
		\begin{array}{cccc}
		0 & \frac{1}{U_\infty} & -\frac{c a}{2 U_\infty^2} & 0\\
		0 &
		-\frac{1}{4 U_\infty} &
		\plbr{ \frac{ c a}{8 U_\infty^2} - \frac{c}{32 U_\infty^2} }& 
		-\frac{1}{4 U_\infty}
		\end{array}	
		\right]
\end{equation}

\begin{equation}
	\tilde{\T{T}}_{f_a} = \frac{C_{l_\alpha}}{2} \frac{c}{2} \left[
		\begin{array}{cccc}
		0 & \frac{1}{U_\infty} & -\frac{c a}{2 U_\infty^2} & 0\\
		0 & 0 & 0 & 0\\
		0 & 0 & 0 & 0
		\end{array}
		\right]
\end{equation}
\begin{equation}
	\tilde{\T{T}}_{c_a} = \frac{C_{l_\alpha}}{2} \frac{c}{2} \left[
		\begin{array}{cccc}
		0 & 0 & 0 & 0\\
		0 & 0 & 0 & 0\\
		0 &
		-\frac{1}{4 U_\infty} &
		\plbr{ \frac{ c a}{8 U_\infty^2} - \frac{c}{32 U_\infty^2} }& 
		-\frac{1}{4 U_\infty}
		\end{array}
		\right]
\end{equation}

Finally,
\begin{equation}
	\tilde{\T{c}}_{f_{a}} = 
		\T{c}_{f_a}^\text{lookup}\plbr{y_1} +
		\cubr{\cvvect{
			0 \\
			\frac{c_{l_\alpha}}{2} \frac{c}{2 U_\infty} \plbr{ y_2 - \frac{c a}{2 U_\infty} y_3} \\ 
			0
		}}.
\end{equation}
\begin{equation}
	\tilde{\T{c}}_{c_{a}} = 
		\T{c}_{c_a}^\text{lookup}\plbr{y_1} + \left\{
		\begin{array}{c}
		0 \\ 0\\
		\frac{c_{l_\alpha}}{2} \frac{c}{2 U_\infty}
		 \plbr{ -\frac{1}{4} y_2 +
		\plbr{\frac{ c a}{8 U_\infty} - \frac{c}{32 U_\infty} } y_3 
		-\frac{1}{4} y_4}
		\end{array}
		\right\}.
\end{equation}

\subsection{Perturbation of the Equations}

\subsubsection{Perturbation of $\T{g}$ with respect to $\tilde{\T{v}}$}
The perturbation of $\T{g}$ is
%\begin{equation}
%\T{g}_{/\tilde{\T{v}}} = 
%	-\T{A}_{/U_\infty} {U_\infty}_{/\tilde{\T{v}}} \T{q}	
%	-\T{B}_{/U_\infty} {U_\infty}_{/\tilde{\T{v}}} \T{u}
%	-\T{B} \T{u}_{/\tilde{\T{v}}}.
%\end{equation}
\begin{equation}\label{eq:g_v}
\T{g}_{/\tilde{\T{v}}} = 
	-\left( \T{A}_{/U_\infty} \T{q}	+ \T{B}_{/U_\infty} \T{u} \right) 
	{U_\infty}_{/\tilde{\T{v}}}
	-\T{B} \T{u}_{/\tilde{\T{v}}}.
\end{equation}
where, according to the previous definition of the matrices,
the derivatives are
\begin{equation}
\T{A}_{/U_\infty} = \left[
	\begin{array}{cccccc}
	0 & 0 & 0 & 0 & 0 & 0\\
	-b_1 b_2 \frac{8}{c^2} U_\infty & -\left( b_1 + b_2 \right) \frac{2}{c} & 0 & 0 & 0 & 0\\
	0 & 0 & 0 & 0 & 0 & 0\\
	0 & 0 & 0 & 0 & 0 & 0\\
	0 & 0 & 0 & 0 & 0 & 0\\
	0 & 0 & 0 & 0 & 0 & 0
	\end{array}
	\right]
\end{equation}
	
\begin{equation}
\T{B}_{/U_\infty} = \left[
	\begin{array}{cc}
	0 & 0 \\
	0 & 0 \\
	0 & 0 \\
	0 & 0 \\
	\frac{1}{d_{3/4} - d_{1/4}} & 	-\frac{1}{d_{3/4} - d_{1/4}} \\
	0 & 0 
	\end{array}
	\right]
\end{equation}

%\begin{equation}
%	{U_\infty}_{/\tilde{v}_x} = \frac{\tilde{v}_x}{U_\infty},
%\end{equation}
%\begin{equation}
%	{U_\infty}_{/\tilde{v}_y} = \frac{\tilde{v}_y}{U_\infty},
%\end{equation}
%\begin{equation}
%	{U_\infty}_{/\tilde{v}_z} = 0,
%\end{equation}
\begin{equation}
	{U_\infty}_{/\tilde{\T{v}}} = \left[
	\begin{array}{ccc}
		\frac{\tilde{v}_x}{U_\infty} &
		\frac{\tilde{v}_y}{U_\infty} &
		0
	\end{array}
	\right]
\end{equation}

\begin{equation}
	\T{u}_{/\tilde{\T{v}}} = \left[
	\begin{array}{ccc}
	\frac{\tilde{v}_y + \tilde{\omega}_z d_{1/4}}{\tilde{v}_x^2 + \tilde{v}_y^2 + 
		\tilde{\omega}_z^2 d_{1/4}^2 + 2 \tilde{v}_y \tilde{\omega}_z d_{1/4}} &
	\frac{-\tilde{v}_x}{\tilde{v}_x^2 + \tilde{v}_y^2 + 
		\tilde{\omega}_z^2 d_{1/4}^2 + 2 \tilde{v}_y \tilde{\omega}_z d_{1/4}} &
	0\\
	\frac{\tilde{v}_y + \tilde{\omega}_z d_{3/4}}{\tilde{v}_x^2 + \tilde{v}_y^2 + 
		\tilde{\omega}_z^2 d_{3/4}^2 + 2 \tilde{v}_y \tilde{\omega}_z d_{3/4}} &
	\frac{-\tilde{v}_x}{\tilde{v}_x^2 + \tilde{v}_y^2 + 
		\tilde{\omega}_z^2 d_{3/4}^2 + 2 \tilde{v}_y \tilde{\omega}_z d_{3/4}} &
	0
	\end{array}
	\right]
\end{equation}
The explicit computation of each term of Eq.~\ref{eq:g_v} yields
\begin{align}
&\plbr{\T{A}_{U_\infty} \T{q} + \T{B}_{U_\infty} \T{u} } {U_\infty}_{/\tilde{v}}
	= \left\{
		\begin{array}{c}
			0 \\
			-b_1 b_2 \frac{8}{c^2} U_\infty q_1 - \plbr{b_1+b_2} \frac{2}{c} q_2 \\
			0 \\
			0 \\
			\frac{u_1-u_2}{d_{3/4}-d_{1_/4}}\\
			0 
		\end{array}\right\}
		\left\{
		\begin{array}{ccc}
			\frac{\tilde{v}_x}{U_\infty} &
			\frac{\tilde{v}_x}{U_\infty} &
			0
		\end{array}\right\} \\
	&= \left[
		\begin{array}{ccc}
			0 & 0 & 0\\
			\plbr{-b_1 b_2 \frac{8}{c^2} U_\infty q_1 - \plbr{b_1+b_2} \frac{2}{c} q_2} \frac{\tilde{v}_x}{U_\infty} &
			\plbr{-b_1 b_2 \frac{8}{c^2} U_\infty q_1 - \plbr{b_1+b_2} \frac{2}{c} q_2} \frac{\tilde{v}_y}{U_\infty} & 0\\
			0 & 0 & 0\\
			0 & 0 & 0\\
			\plbr{ \frac{u_1-u_2}{d_{3/4}-d_{1_/4}}} \frac{\tilde{v}_x}{U_\infty} &
			\plbr{ \frac{u_1-u_2}{d_{3/4}-d_{1_/4}}} \frac{\tilde{v}_y}{U_\infty} & 0\\
			0 & 0 & 0
		\end{array} \right]
\end{align}
for the first, while the second yields
\begin{equation}
\T{B} \T{u}_{/\tilde{\T{v}}} = \left[
	\begin{array}{ccc}
		0 & 0 & 0 \\
		\T{B} \T{u}_{/\tilde{\T{v}}}\plbr{2,1} &
		\T{B} \T{u}_{/\tilde{\T{v}}}\plbr{2,2} & 0 \\
		\T{B} \T{u}_{/\tilde{\T{v}}}\plbr{3,1} &
		\T{B} \T{u}_{/\tilde{\T{v}}}\plbr{3,2} & 0 \\
		0 & 0 & 0 \\
		\T{B} \T{u}_{/\tilde{\T{v}}}\plbr{5,1} &
		\T{B} \T{u}_{/\tilde{\T{v}}}\plbr{5,2} & 0 \\
		0 & 0 & 0 
	\end{array} \right]
\end{equation}
The non-null terms are
\begin{equation}
\T{B} \T{u}_{/\tilde{\T{v}}}\plbr{2,1}  = \T{u}_{/\tilde{\T{v}}}\plbr{2,1}
\end{equation}
\begin{equation}
\T{B} \T{u}_{/\tilde{\T{v}}}\plbr{2,2}  = \T{u}_{/\tilde{\T{v}}}\plbr{2,2}
\end{equation}
\begin{equation}
\T{B} \T{u}_{/\tilde{\T{v}}}\plbr{3,1}  = \T{B}\plbr{3,1}\T{u}_{/\tilde{\T{v}}}\plbr{1,1}+
					\T{B}\plbr{3,2}\T{u}_{/\tilde{\T{v}}}\plbr{2,1}
\end{equation}
\begin{equation}
\T{B} \T{u}_{/\tilde{\T{v}}}\plbr{3,2}  = \T{B}\plbr{3,1}\T{u}_{/\tilde{\T{v}}}\plbr{1,2}+
					\T{B}\plbr{3,2}\T{u}_{/\tilde{\T{v}}}\plbr{2,2}
\end{equation}
\begin{equation}
\T{B} \T{u}_{/\tilde{\T{v}}}\plbr{3,1}  = \T{B}\plbr{5,1}\T{u}_{/\tilde{\T{v}}}\plbr{1,1}+
					\T{B}\plbr{5,2}\T{u}_{/\tilde{\T{v}}}\plbr{2,1}
\end{equation}
\begin{equation}
\T{B} \T{u}_{/\tilde{\T{v}}}\plbr{3,2}  = \T{B}\plbr{5,1}\T{u}_{/\tilde{\T{v}}}\plbr{1,2}+
					\T{B}\plbr{5,2}\T{u}_{/\tilde{\T{v}}}\plbr{2,2}
\end{equation}
So, the Jacobian matrix $\T{g}_{/\tilde{\T{v}}}$ is
\begin{equation}
\T{g}_{/\tilde{\T{v}}} = \left[
	\begin{array}{ccc}
		0 & 0 & 0 \\
		\T{g}_{/\tilde{\T{v}}}\plbr{2,1} &
		\T{g}_{/\tilde{\T{v}}}\plbr{2,2} & 0 \\
		\T{g}_{/\tilde{\T{v}}}\plbr{3,1} &
		\T{g}_{/\tilde{\T{v}}}\plbr{3,2} & 0 \\
		0 & 0 & 0 \\
		\T{g}_{/\tilde{\T{v}}}\plbr{5,1} &
		\T{g}_{/\tilde{\T{v}}}\plbr{5,2} & 0 \\
		0 & 0 & 0 
	\end{array} \right]
	,
\end{equation}
where the matrix elements are
\begin{equation}
\T{g}_{/\tilde{\T{v}}}\plbr{2,1}  = 
-\plbr{-b_1 b_2 \frac{8}{c^2} U_\infty q_1 - \plbr{b_1+b_2} \frac{2}{c} q_2} \frac{\tilde{v}_x}{U_\infty} 
-\T{u}_{/\tilde{\T{v}}}\plbr{2,1}
\end{equation}
\begin{equation}
\T{g}_{/\tilde{\T{v}}}\plbr{2,2}  = 
-\plbr{-b_1 b_2 \frac{8}{c^2} U_\infty q_1 - \plbr{b_1+b_2} \frac{2}{c} q_2} \frac{\tilde{v}_y}{U_\infty} 
-\T{u}_{/\tilde{\T{v}}}\plbr{2,2}
\end{equation}
\begin{equation}
\T{g}_{/\tilde{\T{v}}}\plbr{3,1}  = -\plbr{\T{B}\plbr{3,1}\T{u}_{/\tilde{\T{v}}}\plbr{1,1}+
					\T{B}\plbr{3,2}\T{u}_{/\tilde{\T{v}}}\plbr{2,1}}
\end{equation}
\begin{equation}
\T{g}_{/\tilde{\T{v}}}\plbr{3,2}  = -\plbr{\T{B}\plbr{3,1}\T{u}_{/\tilde{\T{v}}}\plbr{1,2}+
					\T{B}\plbr{3,2}\T{u}_{/\tilde{\T{v}}}\plbr{2,2}}
\end{equation}
\begin{equation}
\T{g}_{\tilde{\T{v}}}\plbr{3,1}  = 
-\plbr{ \frac{u_1-u_2}{d_{3/4}-d_{1_/4}}} \frac{\tilde{v}_x}{U_\infty}
-\plbr{\T{B}\plbr{5,1}\T{u}_{/\tilde{\T{v}}}\plbr{1,1}+
					\T{B}\plbr{5,2}\T{u}_{/\tilde{\T{v}}}\plbr{2,1}}
\end{equation}
\begin{equation}
\T{g}_{\tilde{\T{v}}}\plbr{3,2}  = 
-\plbr{ \frac{u_1-u_2}{d_{3/4}-d_{1_/4}}} \frac{\tilde{v}_y}{U_\infty}
-\plbr{\T{B}\plbr{5,1}\T{u}_{/\tilde{\T{v}}}\plbr{1,2}+
					\T{B}\plbr{5,2}\T{u}_{/\tilde{\T{v}}}\plbr{2,2}}
\end{equation}

	
			


\subsubsection{Perturbation of $\T{g}$ with respect to $\tilde{\T{\omega}}$}
The perturbation of $\T{g}$ yields
\begin{equation}
\T{g}_{/\tilde{\T{\omega}}} = 
	-\T{B} \T{u}_{/\tilde{\T{\omega}}}.
\end{equation}
Where
\begin{equation}
	\T{u}_{/\tilde{\T{\omega}}} = \left[
	\begin{array}{ccc}
	0 & 0 &
	\frac{-\tilde{v}_x d_{1/4}}{\tilde{v}_x^2 + \tilde{v}_y^2 + 
		\tilde{\omega}_z^2 d_{1/4}^2 + 2 \tilde{v}_y \tilde{\omega}_z d_{1/4}} \\
	0 & 0 &
	\frac{-\tilde{v}_x d_{3/4}}{\tilde{v}_x^2 + \tilde{v}_y^2 + 
		\tilde{\omega}_z^2 d_{3/4}^2 + 2 \tilde{v}_y \tilde{\omega}_z d_{3/4}} 
	\end{array}
	\right]
\end{equation}
Thus,
\begin{equation}
	\T{g}_{/\tilde{\T{\omega}}} = \left[
		\begin{array}{ccc}
			0 & 0 & 0\\
			0 & 0 & \T{g}_{/\tilde{\T{\omega}}}\plbr{2,3} \\
			0 & 0 & \T{g}_{/\tilde{\T{\omega}}}\plbr{3,3} \\
			0 & 0 & 0\\
			0 & 0 & \T{g}_{/\tilde{\T{\omega}}}\plbr{5,3} \\
			0 & 0 & 0
		\end{array} \right]
\end{equation}
where
\begin{equation}
	\T{g}_{/\tilde{\T{\omega}}}\plbr{2,3} = 
		-\T{u}_{/\tilde{\T{\omega}}}\plbr{2,3}
\end{equation} 
\begin{equation}
	\T{g}_{/\tilde{\T{\omega}}}\plbr{5,3} = -\plbr{
		\T{B}\plbr{3,1}\T{u}_{/\tilde{\T{\omega}}}\plbr{1,3} + 
		\T{B}\plbr{3,2}\T{u}_{/\tilde{\T{\omega}}}\plbr{2,3}}
\end{equation} 
\begin{equation}
	\T{g}_{/\tilde{\T{\omega}}}\plbr{5,3} = -\plbr{
		\T{B}\plbr{5,1}\T{u}_{/\tilde{\T{\omega}}}\plbr{1,3} + 
		\T{B}\plbr{5,2}\T{u}_{/\tilde{\T{\omega}}}\plbr{2,3}}
\end{equation} 
			
	


\subsubsection{Perturbation of $\T{g}$ with respect to $\T{q}$}
The perturbation of $\T{g}$ yields
\begin{equation}
\T{g}_{/{\T{q}}} = -\T{A}
\end{equation}
\subsubsection{Perturbation of $\T{g}$ with respect to $\T{\dot{q}}$}
The perturbation of $\T{g}$ yields
\begin{equation}
\T{g}_{/{\T{\dot{q}}}} = \T{I}
\end{equation}

\subsection{Perturbation of the aerodynamic forces}

\subsubsection{Perturbation of $\T{\tilde{f}}_a$ with respect to $\tilde{\T{v}}$}


\begin{equation}
\tilde{\T{f}}_{a/\tilde{\T{v}}} = 
	\rho c \tilde{\T{c}}_{f_{a}} \left[
		\begin{array}{ccc}
			\tilde{v}_x & \tilde{v}_y & \tilde{v}_z
		\end{array} \right] +
	\frac{1}{2} \rho \tilde{\T{v}}^T \tilde{\T{v}} c
	\plbr{ \tilde{\T{c}}_{{f_{a}}_{/\T{y}}} \T{y}_{/\tilde{\T{v}}} + 
	\tilde{\T{c}}_{{f_{a}}_{/U_{\infty}}} {U_\infty}_{/\tilde{\T{v}}} } 
\end{equation}
where
\begin{equation}
\T{y}_{/\tilde{\T{v}}} = \plbr{
	\T{C}_{/U_{\infty}} \T{q} + \T{D}_{/U_{\infty}} \T{u}} {U_\infty}_{/\tilde{\T{v}}}
	+ \T{D} \T{u}_{/\tilde{\T{v}}}
\end{equation}


\begin{equation}
\T{C}_{/U_\infty} = \left[
	\begin{array}{cccccc}
	\plbr{ A_1 + A_2} b_1 b_2 \frac{8}{c^2} U_\infty & \plbr{ A_1 b_1 + A_2 b_2} \frac{2}{c} & 0 & 0 & 0 & 0\\
	0 & 0 & 0 & 0 & 0 & 0\\
	0 & 0 & 0 & 0 & 0 & 0\\
	0 & 0 & 0 & 0 & 0 & 0\\
	\end{array}
	\right]
\end{equation}

\begin{equation}
\T{D}_{/U_\infty} = \left[
	\begin{array}{cc}
	0 & 0 \\
	0 & 0 \\
	0 & 0 \\
	\frac{1}{d_{3/4} - d_{1/4}} & 	-\frac{1}{d_{3/4} - d_{1/4}} \\
	\end{array}
	\right]
\end{equation}

\begin{equation}
\tilde{\T{c}}_{{f_{a}}_{/U_{\infty}}} = \left\{
	\begin{array}{c}
		0 \\
		\frac{C_{l_{\alpha}}}{2} \frac{c}{2} \frac{1}{U_\infty^2} 
		\plbr{ -y_2 + \frac{c a}{U_\infty} y_3} \\
		0
	\end{array} \right\}
\end{equation}

\begin{equation}
\tilde{\T{c}}_{{f_{a}}_{/\T{y}}} = \left[
	\begin{array}{cccc}
		\frac{ d c_d^{\text{lookup}}}{d \alpha} & 0 & 0 & 0 \\
		\frac{ d c_l^{\text{lookup}}}{d \alpha} & \frac{C_{l_{\alpha}}}{2} \frac{c}{2 U_\infty}&
			-\frac{c_{l_{\alpha}}}{2} \frac{c^2 a}{4 U_\infty^2} & 0 \\
		0 & 0 & 0 & 0
	\end{array} \right]
\end{equation}

\subsubsection{Perturbation of $\T{\tilde{f}}_a$ with respect to $\tilde{\T{\omega}}$}
\begin{equation}
\T{\tilde{f}}_{a/\tilde{\T{\omega}}} = 
	\frac{1}{2} \rho \tilde{\T{v}}^T \tilde{\T{v}} c
	\plbr{ \tilde{\T{c}}_{{f_{a}}_{/\T{y}}} \T{y}_{/\tilde{\T{\omega}}} } 
\end{equation}
where
\begin{equation}
\T{y}_{/\tilde{\T{\omega}}} = 
	\T{D} \T{u}_{/\tilde{\T{\omega}}}
\end{equation}
\subsubsection{Perturbation of $\T{\tilde{f}}_a$ with respect to $\T{q}$}
\begin{equation}
\T{\tilde{f}}_{a/\T{q}} = 
	\frac{1}{2} \rho \tilde{\T{v}}^T \tilde{\T{v}} c
	\plbr{ \tilde{\T{c}}_{{f_{a}}_{/\T{y}}} \T{y}_{/\T{q}} } 
\end{equation}
where
\begin{equation}
\T{y}_{/\T{q}} = \T{C} 
\end{equation}

\subsubsection{Perturbation of $\T{\tilde{f}}_a$ with respect to $\T{\dot{q}}$}
\begin{equation}
\T{\tilde{f}}_{a/\T{\dot{q}}} = \T{0}
\end{equation}

\subsection{Perturbation of the aerodynamic moments}
\subsubsection{Perturbation of $\T{\tilde{c}}_a$ with respect to $\tilde{\T{v}}$}
		
\begin{equation}
\T{\tilde{c}}_{a/\tilde{\T{v}}} = 
	\rho c^2 \tilde{\T{c}}_{c_{a}} \left[
		\begin{array}{ccc}
			\tilde{v}_x & \tilde{v}_y & \tilde{v}_z
		\end{array} \right] +
	\frac{1}{2} \rho \tilde{\T{v}}^T \tilde{\T{v}} c^2
	\plbr{ \tilde{\T{c}}_{{c_{a}}_{/\T{y}}} \T{y}_{/\tilde{\T{v}}} + 
	\tilde{\T{c}}_{{c_{a}}_{/U_{\infty}}} {U_\infty}_{/\tilde{\T{v}}} } 
\end{equation}

\begin{equation}
\tilde{\T{c}}_{{c_{a}}_{/U_{\infty}}} = \left\{
	\begin{array}{c}
		0 \\
		0 \\
		\frac{c_{l_{\alpha}}}{2} \frac{c}{2} \frac{1}{4 U_\infty^2} 
		\plbr{ y_2 - \plbr{ \frac{c a}{U_\infty} - \frac{c}{4 U_\infty} } y_3 +y_4}
	\end{array} \right\}
\end{equation}

\begin{equation}
\tilde{\T{c}}_{{c_{a}}_{/\T{y}}} = \left[
	\begin{array}{cccc}
		0 & 0 & 0 & 0 \\
		0 & 0 & 0 & 0 \\
		\frac{ d c_m^{\text{lookup}}}{d \alpha} & 
			-\frac{1}{4} \frac{c_{l_{\alpha}}}{2} \frac{c}{2 U_\infty} &
			\frac{c_{l_{\alpha}}}{2} \frac{c}{2 U_\infty} \plbr{ \frac{c a}{8 U_\infty} -\frac{c}{32 U_\infty}} &
			-\frac{1}{4} \frac{c_{l_{\alpha}}}{2} \frac{c}{2 U_\infty} 
	\end{array} \right]
\end{equation}
\subsubsection{Perturbation of $\T{\tilde{c}}_a$ with respect to $\tilde{\T{\omega}}$}
\begin{equation}
\T{\tilde{c}}_{a/\tilde{\T{\omega}}} = 
	\frac{1}{2} \rho \tilde{\T{v}}^T \tilde{\T{v}} c^2
	\plbr{ \tilde{\T{c}}_{{c_{a}}_{/\T{y}}} \T{y}_{/\tilde{\T{\omega}}} } 
\end{equation}
\subsubsection{Perturbation of $\T{\tilde{c}}_a$ with respect to $\T{q}$}
\begin{equation}
\T{\tilde{c}}_{a/\T{q}} = 
	\frac{1}{2} \rho \tilde{\T{v}}^T \tilde{\T{v}} c^2
	\plbr{ \tilde{\T{c}}_{{c_{a}}_{/\T{y}}} \T{y}_{/\T{q}} } 
\end{equation}
\subsubsection{Perturbation of $\T{\tilde{c}}_a$ with respect to $\T{\dot{q}}$}
\begin{equation}
\T{\tilde{c}}_{a/\T{\dot{q}}} = \T{0}
\end{equation}

\subsection{Finite difference version}

A reduced state-space model, with just 2 states instead of 6 states, can be
used deleting the pseudo-derivative algorithm and computing the derivate using
the backward finite diffenence.

The set of equations that descibes the unsteady aerodynamic loads is still the same:
\begin{subequations}
\begin{align}
	\tilde{\T{f}}_a &= 
		\frac{1}{2} \rho \tilde{\T{v}}^T \tilde{\T{v}} c
		\tilde{\T{c}}_{f_{a}}
		\plbr{ y 
		\plbr{ \tilde{\T{v}}, \tilde{\T{\omega}}, \T{q}}, U_\infty}\\
	\tilde{\T{c}}_a &= 
		\frac{1}{2} \rho \tilde{\T{v}}^T \tilde{\T{v}} c^2
		\tilde{\T{c}}_{c_{a}} 
		\plbr{ y 
		\plbr{ \tilde{\T{v}}, \tilde{\T{\omega}}, \T{q}}, U_\infty}\\
	\T{0} &=
		\T{g}\plbr{\tilde{\T{v}},\tilde{\T{\omega}},\T{q}, \dot{\T{q}}} = 
		\dot{\T{q}} - \T{A}\plbr{U_\infty} \T{q} - \T{B}
 		u\plbr{\tilde{\T{v}}, \tilde{\T{\omega}}},
\end{align}
\end{subequations}
where $\rho$ is the air density, $c$ the airfoil chord and $\T{y}$ is defined as:
\begin{equation}
	y = \TT{C}\plbr{U_\infty} \T{q} + \TT{D} 
 		u\plbr{\tilde{\T{v}}, \tilde{\T{\omega}}}.
\end{equation}

In this case the matrix $\TT{A}$ is:
\begin{equation}
	\footnotesize
	\TT{A} = \left[
		\begin{array}{cc}
		0 & 1 \\
		-b_1 b_2 \plbr{ \frac{2U_\infty}{c}}^2 & -\plbr{ b_1 + b_2} \plbr{\frac{2 U_\infty}{c}}
		\end{array}
		\right],
\end{equation}
where $A_1$, $A_2$, $b_1$ and $b_2$ are the coefficients of the Theodorsen
function approximation (Table~\ref{tab:aero:unsteady:theodorsen-coeffs}
from \cite{BIELAWA92} and \cite{LEISHMAN-2006}),

The matrix $\TT{B}$ is simply:
\begin{equation}
	\TT{B} = \left[
		\begin{array}{c}
			0  \\
			1 
		\end{array}
		\right],
\end{equation}
Whereas the input of the system $u$ is the angle of attack computed
at the 3/4 chord point:
\begin{equation}
	u = \tan^{-1} \plbr{ \frac{-V_y - \omega_z d_{3/4}}{V_x}}
\end{equation}

The $\TT{C}$ matrix is:
\begin{equation}
	\T{C} = \left[
		\begin{array}{cc}
		\left(A_1+A_2\right)b_1 b_2 \left( \frac{2 U_\infty}{c}\right)^2 &
		 \left(A_1  b_1 + A_2 b_2 \right) \left( \frac{2 U_\infty}{c}\right) 
		\end{array}
		\right].
\end{equation}
and the $\TT{D}$ matrix:
\begin{equation}
	\T{D} = \left[
		\begin{array}{cc}
		\left( 1 -A_1 - A_2 \right) 
		\end{array}
		\right].
\end{equation}
Using the otuput $\T{y}$ of this SISO state-space model as input 
for the lookup table the circulatory part of the unsteady loads is
computed.

The non-circulatory part is computed starting from the angle of attack
computed at 1/4 chord ($u_1$) and 3/4 chord ($u_2$):
\begin{equation}
\left\{
	\begin{array}{c}
		\alpha_{pivot} \\
		\dot{\alpha}
	\end{array}
\right\} = 
\left[
	\begin{array}{cc}
		1 & -\frac{d_{1/4}}{U_\infty}\\
		1 & -\frac{d_{3/4}}{U_\infty}
	\end{array}
\right]^{-1} 
\left\{
	\begin{array}{c}
		u_1 \\
		u_2
	\end{array}
\right\} = \frac{1}{d_{3/4}-d_{1/4}} 
\left[
	\begin{array}{cc}
		d_{3/4} & -d_{1/4}\\
		U_\infty & -U_\infty
	\end{array}
\right]
\left\{
	\begin{array}{c}
		u_1 \\
		u_2
	\end{array}
\right\}
\end{equation}
where:
\begin{equation}
	\T{u} = \cubr{\cvvect{
		\tan^{-1} \plbr{ \frac{-V_y - \omega_z d_{1/4}}{V_x}}\\
		\tan^{-1} \plbr{ \frac{-V_y - \omega_z d_{3/4}}{V_x}}\\
	}}
\end{equation}

Thus, the aerodynamic coefficients result:
\begin{equation}
	\tilde{\T{c}}_{f_{a}} = 
		\T{c}_{f_a}^\text{lookup}\plbr{y} +
		\cubr{\cvvect{
			0 \\
			\frac{c_{l_\alpha}}{2} \frac{c}{2 U_\infty} \plbr{ \dot{\alpha}_{pivot} - \frac{c a}{2 U_\infty} \ddot{\alpha}} \\ 
			0
		}}.
\end{equation}
\begin{equation}
	\tilde{\T{c}}_{c_{a}} = 
		\T{c}_{c_a}^\text{lookup}\plbr{y} + \left\{
		\begin{array}{c}
		0 \\ 0\\
		\frac{c_{l_\alpha}}{2} \frac{c}{2 U_\infty}
		 \plbr{ -\frac{1}{4} \dot{\alpha}_{pivot} +
		\plbr{\frac{ c a}{8 U_\infty} - \frac{c}{32 U_\infty} } \ddot{\alpha}
		-\frac{1}{4} \dot{\alpha}}
		\end{array}
		\right\}.
\end{equation}
In order to compute the aerodynamic coefficients $\dot{\alpha}_{pivot}$ and $\ddot{\alpha}$ are
necessary and they can be computed using the backward finite difference:
\begin{equation}
\left\{
	\begin{array}{c}
		\dot{\alpha}_{pivot} \\
		\ddot{\alpha}
	\end{array}
\right\} = 
\frac{1}{\Delta t} \left(
\left\{
	\begin{array}{c}
		\alpha_{pivot} \\
		\dot{\alpha}
	\end{array}
\right\}_k - 
\left\{
	\begin{array}{c}
		\alpha_{pivot} \\
		\dot{\alpha}
	\end{array}
\right\}_{k-1}
\right)
\end{equation} 
\subsection{Perturbation of the equations}

\subsubsection{Perturbation of $\T{g}$ with respect to $\tilde{\T{v}}$}
The perturbation of $\T{g}$ is:
%\begin{equation}
%\T{g}_{/\tilde{\T{v}}} = 
%	-\T{A}_{/U_\infty} {U_\infty}_{/\tilde{\T{v}}} \T{q}	
%	-\T{B}_{/U_\infty} {U_\infty}_{/\tilde{\T{v}}} \T{u}
%	-\T{B} \T{u}_{/\tilde{\T{v}}}.
%\end{equation}
\begin{equation}\label{eq:g_v_2}
\T{g}_{/\tilde{\T{v}}} = 
	- \T{A}_{/U_\infty} \T{q} 
	{U_\infty}_{/\tilde{\T{v}}}
	-\T{B} u_{/\tilde{\T{v}}}.
\end{equation}
where, accordingly with the previous definition of the matrices the derivatives are:
\begin{equation}
\T{A}_{/U_\infty} = \left[
	\begin{array}{cccccc}
	0 & 0\\
	-b_1 b_2 \frac{8}{c^2} U_\infty & -\left( b_1 + b_2 \right) \frac{2}{c}
	\end{array}
	\right]
\end{equation}
	
\begin{equation}
	{U_\infty}_{/\tilde{\T{v}}} = \left[
	\begin{array}{ccc}
		\frac{\tilde{v}_x}{U_\infty} &
		\frac{\tilde{v}_y}{U_\infty} &
		0
	\end{array}
	\right]
\end{equation}

\begin{equation}
	u_{/\tilde{\T{v}}} = \left[
	\begin{array}{ccc}
	\frac{\tilde{v}_y + \tilde{\omega}_z d_{3/4}}{\tilde{v}_x^2 + \tilde{v}_y^2 + 
		\tilde{\omega}_z^2 d_{3/4}^2 + 2 \tilde{v}_y \tilde{\omega}_z d_{3/4}} &
	\frac{-\tilde{v}_x}{\tilde{v}_x^2 + \tilde{v}_y^2 + 
		\tilde{\omega}_z^2 d_{3/4}^2 + 2 \tilde{v}_y \tilde{\omega}_z d_{3/4}} &
	0
	\end{array}
	\right]
\end{equation}
We can now explicitly perform the computation of each term of Eq.~\ref{eq:g_v_2}:
\begin{align}
&\T{A}_{U_\infty} \T{q}  {U_\infty}_{/\tilde{v}}
	= \left\{
		\begin{array}{c}
			0 \\
			-b_1 b_2 \frac{8}{c^2} U_\infty q_1 - \plbr{b_1+b_2} \frac{2}{c} q_2 \\
		\end{array}\right\}
		\left\{
		\begin{array}{ccc}
			\frac{\tilde{v}_x}{U_\infty} &
			\frac{\tilde{v}_x}{U_\infty} &
			0
		\end{array}\right\} \\
	&= \left[
		\begin{array}{ccc}
			0 & 0 & 0\\
			\plbr{-b_1 b_2 \frac{8}{c^2} U_\infty q_1 - \plbr{b_1+b_2} \frac{2}{c} q_2} \frac{\tilde{v}_x}{U_\infty} &
			\plbr{-b_1 b_2 \frac{8}{c^2} U_\infty q_1 - \plbr{b_1+b_2} \frac{2}{c} q_2} \frac{\tilde{v}_y}{U_\infty} & 0\\
		\end{array} \right]
\end{align}
wheres the second is:
\begin{equation}
\T{B} u_{/\tilde{\T{v}}} = \left[
	\begin{array}{ccc}
		0 & 0 & 0 \\
		u_{/\tilde{\T{v}}}\plbr{1,1} &
		u_{/\tilde{\T{v}}}\plbr{1,2} & 0 \\
	\end{array} \right]
\end{equation}
So, the jacobian matrix $\T{g}_{/\tilde{\T{v}}}$ results:
\begin{equation}
\T{g}_{/\tilde{\T{v}}} = \left[
	\begin{array}{ccc}
		0 & 0 & 0 \\
		\T{g}_{/\tilde{\T{v}}}\plbr{2,1} &
		\T{g}_{/\tilde{\T{v}}}\plbr{2,2} & 0 \\
	\end{array} \right]
\end{equation}
where the matrix elements are:
\begin{equation}
\T{g}_{/\tilde{\T{v}}}\plbr{2,1}  = 
-\plbr{-b_1 b_2 \frac{8}{c^2} U_\infty q_1 - \plbr{b_1+b_2} \frac{2}{c} q_2} \frac{\tilde{v}_x}{U_\infty} 
-u_{/\tilde{\T{v}}}\plbr{1,1}
\end{equation}
\begin{equation}
\T{g}_{/\tilde{\T{v}}}\plbr{2,2}  = 
-\plbr{-b_1 b_2 \frac{8}{c^2} U_\infty q_1 - \plbr{b_1+b_2} \frac{2}{c} q_2} \frac{\tilde{v}_y}{U_\infty} 
-u_{/\tilde{\T{v}}}\plbr{1,2}
\end{equation}


\subsubsection{Perturbation of $\T{g}$ with respect to $\tilde{\T{\omega}}$}
The perturbation of $\T{g}$ is:
	

\begin{equation}
\T{g}_{/\tilde{\T{\omega}}} = 
	-\T{B} u_{/\tilde{\T{\omega}}}.
\end{equation}
Where:
\begin{equation}
	u_{/\tilde{\T{\omega}}} = \left[
	\begin{array}{ccc}
	0 & 0 &
	\frac{-\tilde{v}_x d_{3/4}}{\tilde{v}_x^2 + \tilde{v}_y^2 + 
		\tilde{\omega}_z^2 d_{3/4}^2 + 2 \tilde{v}_y \tilde{\omega}_z d_{3/4}} 
	\end{array}
	\right]
\end{equation}
Thus, it results:
\begin{equation}
	\T{g}_{/\tilde{\T{\omega}}} = \left[
		\begin{array}{ccc}
			0 & 0 & 0\\
			0 & 0 & -u_{/\tilde{\T{\omega}}}\plbr{1,3} \\
		\end{array} \right]
\end{equation}
			

\subsubsection{Perturbation of $\T{g}$ with respect to $\T{q}$}
The perturbation of $\T{g}$ is simply:
\begin{equation}
\T{g}_{/{\T{q}}} = -\T{A}
\end{equation}
\subsubsection{Perturbation of $\T{g}$ with respect to $\T{\dot{q}}$}
The perturbation of $\T{g}$ is simply:
\begin{equation}
\T{g}_{/{\T{\dot{q}}}} = \T{I}
\end{equation}

\subsection{Perturbation of the aerodynamic forces}

\subsubsection{Perturbation of $\T{\tilde{f}}_a$ with respect to $\tilde{\T{v}}$}


\begin{equation}
\T{\tilde{f}}_{a/\tilde{\T{v}}} = 
	\rho c \tilde{\T{c}}_{f_{a}} \left[
		\begin{array}{ccc}
			\tilde{v}_x & \tilde{v}_y & \tilde{v}_z
		\end{array} \right] +
	\frac{1}{2} \rho \tilde{\T{v}}^T \tilde{\T{v}} c
	\plbr{ \tilde{\T{c}}_{{f_{a}}_{/y}} y_{/\tilde{\T{v}}} + 
	\tilde{\T{c}}_{{f_{a}}_{/U_{\infty}}} {U_\infty}_{/\tilde{\T{v}}} } 
\end{equation}
where:
\begin{equation}
y_{/\tilde{\T{v}}} = 
	\T{C}_{/U_{\infty}} \T{q}{U_\infty}_{/\tilde{\T{v}}}
	+ \T{D} u_{/\tilde{\T{v}}}
\end{equation}


\begin{equation}
\T{C}_{/U_\infty} = \left[
	\begin{array}{cccccc}
	\plbr{ A_1 + A_2} b_1 b_2 \frac{8}{c^2} U_\infty & \plbr{ A_1 b_1 + A_2 b_2} \frac{2}{c}\\
	\end{array}
	\right]
\end{equation}

\begin{equation}
\tilde{\T{c}}_{{f_{a}}_{/U_{\infty}}} = \left\{
	\begin{array}{c}
		0 \\
		\frac{C_{l_{\alpha}}}{2} \frac{c}{2} \frac{1}{U_\infty^2} 
		\plbr{ -\dot{\alpha}_{pivot} + \frac{c a}{U_\infty} \ddot{\alpha}} \\
		0
	\end{array} \right\}
\end{equation}

\begin{equation}
\tilde{\T{c}}_{{f_{a}}_{/y}} = \left\{
	\begin{array}{cccc}
		\frac{ d c_d^{\text{lookup}}}{d \alpha}\\
		\frac{ d c_l^{\text{lookup}}}{d \alpha}\\
		0
	\end{array} \right\}
\end{equation}
where the dependence of $\dot{\alpha}_{pivot}$, $\dot{\alpha}$ and $\ddot{\alpha}$ is 
neglected.

\subsubsection{Perturbation of $\T{\tilde{f}}_a$ with respect to $\tilde{\T{\omega}}$}
\begin{equation}
\T{\tilde{f}}_{a/\tilde{\T{\omega}}} = 
	\frac{1}{2} \rho \tilde{\T{v}}^T \tilde{\T{v}} c
	\plbr{ \tilde{\T{c}}_{{f_{a}}_{/\T{y}}} y_{/\tilde{\T{\omega}}} } 
\end{equation}
where:
\begin{equation}
y_{/\tilde{\T{\omega}}} = 
	\T{D} u_{/\tilde{\T{\omega}}}
\end{equation}
\subsubsection{Perturbation of $\T{\tilde{f}}_a$ with respect to $\T{q}$}
\begin{equation}
\T{\tilde{f}}_{a/\T{q}} = 
	\frac{1}{2} \rho \tilde{\T{v}}^T \tilde{\T{v}} c
	\plbr{ \tilde{\T{c}}_{{f_{a}}_{/y}} \T{y}_{/\T{q}} } 
\end{equation}
where:
\begin{equation}
y_{/\T{q}} = \T{C} 
\end{equation}

\subsubsection{Perturbation of $\T{\tilde{f}}_a$ with respect to $\T{\dot{q}}$}
\begin{equation}
\T{\tilde{f}}_{a/\T{\dot{q}}} = \T{0}
\end{equation}

\subsection{Perturbation of the aerodynamic moments}
\subsubsection{Perturbation of $\T{\tilde{c}}_a$ with respect to $\tilde{\T{v}}$}
		
\begin{equation}
\T{\tilde{c}}_{a/\tilde{\T{v}}} = 
	\rho c^2 \tilde{\T{c}}_{c_{a}} \left[
		\begin{array}{ccc}
			\tilde{v}_x & \tilde{v}_y & \tilde{v}_z
		\end{array} \right] +
	\frac{1}{2} \rho \tilde{\T{v}}^T \tilde{\T{v}} c^2
	\plbr{ \tilde{\T{c}}_{{c_{a}}_{/y}} y_{/\tilde{\T{v}}} + 
	\tilde{\T{c}}_{{c_{a}}_{/U_{\infty}}} {U_\infty}_{/\tilde{\T{v}}} } 
\end{equation}

\begin{equation}
\tilde{\T{c}}_{{c_{a}}_{/U_{\infty}}} = \left\{
	\begin{array}{c}
		0 \\
		0 \\
		\frac{c_{l_{\alpha}}}{2} \frac{c}{2} \frac{1}{4 U_\infty^2} 
		\plbr{ \dot{\alpha}_{pivot} - \plbr{ \frac{c a}{U_\infty} - \frac{c}{4 U_\infty} } \ddot{\alpha} + \dot{\alpha}}
	\end{array} \right\}
\end{equation}

\begin{equation}
\tilde{\T{c}}_{{c_{a}}_{/y}} = \left[
	\begin{array}{cccc}
		0\\
		0\\
		\frac{ d c_m^{\text{lookup}}}{d \alpha} 
	\end{array} \right]
\end{equation}
where again the dependence of $\dot{\alpha}_{pivot}$, $\dot{\alpha}$ and $\ddot{\alpha}$ is 
neglected.
\subsubsection{Perturbation of $\T{\tilde{c}}_a$ with respect to $\tilde{\T{\omega}}$}
\begin{equation}
\T{\tilde{c}}_{a/\tilde{\T{\omega}}} = 
	\frac{1}{2} \rho \tilde{\T{v}}^T \tilde{\T{v}} c^2
	\plbr{ \tilde{\T{c}}_{{c_{a}}_{/y}} y_{/\tilde{\T{\omega}}} } 
\end{equation}
\subsubsection{Perturbation of $\T{\tilde{c}}_a$ with respect to $\T{q}$}
\begin{equation}
\T{\tilde{c}}_{a/\T{q}} = 
	\frac{1}{2} \rho \tilde{\T{v}}^T \tilde{\T{v}} c^2
	\plbr{ \tilde{\T{c}}_{{c_{a}}_{/y}} y_{/\T{q}} } 
\end{equation}
\subsubsection{Perturbation of $\T{\tilde{c}}_a$ with respect to $\T{\dot{q}}$}
\begin{equation}
\T{\tilde{c}}_{a/\T{\dot{q}}} = \T{0}
\end{equation}



\section{Rotor}
% MBDyn (C) is a multibody analysis code.
% http://www.mbdyn.org
%
% Copyright (C) 1996-2010
%
% Pierangelo Masarati  <masarati@aero.polimi.it>
%
% Dipartimento di Ingegneria Aerospaziale - Politecnico di Milano
% via La Masa, 34 - 20156 Milano, Italy
% http://www.aero.polimi.it
%
% Changing this copyright notice is forbidden.
%
% This program is free software; you can redistribute it and/or modify
% it under the terms of the GNU General Public License as published by
% the Free Software Foundation (version 2 of the License).
% 
%
% This program is distributed in the hope that it will be useful,
% but WITHOUT ANY WARRANTY; without even the implied warranty of
% MERCHANTABILITY or FITNESS FOR A PARTICULAR PURPOSE.  See the
% GNU General Public License for more details.
%
% You should have received a copy of the GNU General Public License
% along with this program; if not, write to the Free Software
% Foundation, Inc., 59 Temple Place, Suite 330, Boston, MA  02111-1307  USA
%
% Mattia Mattaboni <mattaboni@aero.polimi.it> and
% Pierangelo Masarati <masarati@aero.polimi.it> authored this document.

\paragraph{Definitions.} \
Axis 3 is the rotor's axis.
$\tilde{\T{v}}$ is the composition of the velocity of the `aircraft' node
and of the airstream speed, if any, projected in the reference frame
of the `aircraft' node, namely
\begin{align}
	\tilde{\T{v}}
	&=
	\TT{R}_\text{craft}^T \plbr{
		- \T{v}_\text{craft}
		+ \T{v}_\infty
	}
	.
\end{align}
%
\begin{subequations}
\begin{align}
	v_{12}
	&=
	\sqrt{v_1^2 + v_2^2}
	\\
	v
	&=
	\sqrt{v_1^2 + v_2^2 + v_3^2}
	= \sqrt{\tilde{\T{v}}^T \tilde{\T{v}}}
	\\
	\sin\alpha_d
	&=
	-v_3/v
	\\
	\cos\alpha_d
	&=
	v_{12}/v
	\\
	\psi_0
	&=
	\tan^{-1}\plbr{\frac{v_2}{v_1}}
	\\
	v_\text{tip}
	&=
	\Omega R
	\\
	\mu
	&=
	\cos\alpha_d \frac{v}{v_\text{tip}}
\end{align}
\end{subequations}
Note: $v \ge 0$ and $v_{12} \ge 0$ by definition;
as a consequence, $\cos\alpha_d \ge 0$,
while the sign of $\sin\alpha_d$ depends on whether
the flow related to the absolute motion of the rotor
enters the disk from above (> 0) or from below (< 0).
$v_\text{tip} > 0$ by construction
($\Omega = \nrbr{\T{\omega}}$, and no induced velocity
is computed if $\Omega$ is below a threshold).
As a consequence, $\mu \ge 0$.


\paragraph{Ground effect.} \
If defined, according to \cite{CHEESEMAN-BENNETT-1955},
\begin{subequations}
\begin{align}
	u_\text{IGE}
	&=
	k_\text{GE} u_\text{OGE}
	\\
	k
	&=
	1 - \frac{1}{z^2}
	\\
	z
	&= \max\plbr{\frac{h}{R}, \frac{1}{4}}
\end{align}
\end{subequations}
where $h$ is the component along axis 3 of the `ground' node
of distance between the `aircraft' and the `ground' node,
assuming the `aircraft' node is located at the hub center.

\paragraph{Reference Induced Velocity.} \
The reference induced velocity $u$ is computed by solving the implicit problem
\begin{align}
	f
	&=
	% \frac{u}{v_\text{tip}}
	\lambda_u
	- \frac{C_t}{2\sqrt{\mu^2 + \lambda^2}}
	=
	0
	,
\end{align}
with
\begin{subequations}
\begin{align}
	C_t
	&=
	\frac{T}{\rho \pi R^4 \Omega^2}
	\\
	\lambda_u
	&=
	\frac{u}{v_\text{tip}}
	\\
	\lambda
	&=
	\frac{v \sin\alpha_d + u}{v_\text{tip}}
	=
	\mu \tan\alpha_d
	+ \lambda_u
	;
\end{align}
\end{subequations}
$T$ is the component of the aerodynamic force of the rotor
along the shaft axis.
The value of $\lambda_u$ is initialized using the reference induced velocity
$u$ at the previous step/iteration.
Only when $u=0$ and $C_t\neq 0$, $u$ is initialized using its nominal value
in hover,
\begin{align}
	u
	&=
	\text{sign}\plbr{T} \sqrt{\frac{\nrbr{T}}{2 \rho A}}
	.
\end{align}
The problem is solved by means of a local Newton iteration.
The Jacobian of the problem is
\begin{align}
	\frac{\partial f}{\partial \lambda_u}
	&=
	1 + \frac{C_t}{2 \plbr{\mu^2 + \lambda^2}^{3/2}} \lambda
	.
\end{align}
The solution,
\begin{align}
	\Delta\lambda_u
	&=
	- \plbr{\frac{\partial f}{\partial \lambda_u}}^{-1} f
	,
\end{align}
is added to $\lambda_u$ as
$\lambda_u += \eta \Delta\lambda_u$,
where $0 < \eta \le 1$ is an optional relaxation factor
(by default, $\eta=1$).

\paragraph{Corrections.} \
The reference induced velocity is corrected by separately correcting
the inflow and advance parameters, namely
\begin{subequations}
\begin{align}
	\lambda^*
	&=
	\frac{\lambda}{k_\text{H}^2}
	\\
	\mu^*
	&=
	\frac{\mu}{k_\text{FF}}
\end{align}
\end{subequations}
(by default, $k_\text{H}=1$ and $k_\text{FF}=1$).
The reference induced velocity is then recomputed as
\begin{align}
	u^*
	&=
	\plbr{1 - \rho} k_\text{GE} v_\text{tip}
		\frac{C_t}{2 \sqrt{\mu^{*^2} + \lambda^{*^2}}}
	+ \rho u^*_\text{prev}
	,
\end{align}
where $0 \le \rho < 1$ is a memory factor (by default, $\rho=0$).

Note: in principle, multiple solutions for $\lambda_u$ are possible.
However, only one solution is physical.
Currently, no strategy is put in place to ensure that only the physical
solution is considered.


\subsection{Uniform Inflow Model}

The induced velocity is equal to its reference value, $u^*$, everywhere.


\subsection{Glauert Model}

In forward flight (when $\mu > 0.15$) the inflow over the rotor disk can 
be approximated by:
\begin{align}
\lambda_i &= \lambda_0 \left( 1 + k_x \frac{x}{R} + k_y \frac{y}{R} \right)
\\
&= \lambda_0 \left( 1 + k_x r \cos{\psi} + k_y r \sin{\psi} \right),
\end{align}
where the mean induced velocity $\lambda_0$ is computed as shown in the
previous section,
while $r=\sqrt{x^2 + y^2}/R$ is the nondimensional radius.

In literature a lot of expressions for the $k_x$ and $k_y$ coefficients have
been proposed by different authors, as summarized in table \ref{tab:GlauertCoeff}.
Up to now in MBDyn the following expressions are
implemented:
\begin{align}
k_x &=  \frac{4}{3} \left( 1 - 1.8 \mu^2 \right) \tan{\frac{\chi}{2}}
\\
k_y &= 0,
\end{align}
\begin{table}[h]
\centering
\caption{Glauert inflow model (source: Leishman \cite{LEISHMAN-2006})}\label{tab:GlauertCoeff}
\begin{tabular}{l|c|c}
\textbf{Author(s)} & $k_x$ & $k_y$\\
\hline
Coleman et al. (1945)	&	
$\tan{\frac{\chi}{2}}$	&	$0$\\
Drees (1949)		&	
$\frac{4}{3}\frac{\left( 1 - \cos{\chi} -1.8 \mu^2\right)}{\sin{\chi}} $	&	$-2 \mu$\\
Payne (1959)		&
$\frac{4}{3} ( \mu/ \lambda/ (1.2 + \mu/\lambda))$	&	$0$\\
White \& Blake (1979)	&
$\sqrt{2} \sin{\chi}$	&	$0$\\
Pitt \& Peters (1981)	&
$\frac{15 \pi}{23} \tan{\frac{\chi}{2}}$	&	$0$\\
Howlett (1981)		&
$ \sin^2{\chi}$	&	$0$\\
\end{tabular}
\end{table}

\textbf{FIXME: \`e diversa dalle espressioni riportate sul Leishman!}

where $\chi$ is the wake skew angle:
\begin{equation}
\chi = \tan^{-1}\left( \frac{\mu}{\lambda} \right).
\end{equation}
\textbf{TODO: check, dovrebbe essere equivalente all'espressione di Leishman ma
sarebbe meglio verificare!!!}

Note: the Glauert inflow model exactly matches the uniform inflow model
when the advance ratio is null (in hover).
In fact, when $\mu = 0$ then $\chi = k_x = 0$.
The inflow is thus uniform over the rotor disk, and equal to $\lambda_0$.

\subsection{Mangler-Squire Model}

The Mangler-Squire model is developed under the high speed assumption
and it should be used only for advance ratio grater than 0.1.

In the original formulation the induced velocity is computed as:
\begin{equation}
\lambda_i = \left( \frac{ 2 C_T}{\mu} \right) \left[ 
\frac{c_0}{2} - \sum_{n=1}^{\infty} c_n(r,\alpha_d) \cos{n \psi} \right],
\end{equation}
since the advance ratio $\mu$ appears in the denominator this expression is
not valid in hover.
Bramwell \cite{BRAMWELL-1976} proposed a different expression for the induced 
velocity:
\begin{equation}
\lambda_i = 4 \lambda_0 \left[ 
\frac{c_0}{2} - \sum_{n=1}^{\infty} c_n(r,\alpha_d) \cos{n \psi} \right],
\end{equation}
where $\lambda_0$ is the mean inflow computed as shown before. In this way
the Mangler-Squire inflow model makes sense also in hover. MBDyn uses the latter
version.

The expression of the $c_n$ coefficients depends on the form of the 
rotor disk loading. Mangler and Squire developed the theory for two fundamental
forms: Type I (elliptical loading) and Type III (a loading that vanishes at the 
edges and at the center of the disk). The total loading is finally obtained by
a linear combination of Type I and Type III loadings (see \cite{LEISHMAN-2006}).

MBDyn uses just a Type III loading, and the resulting expressions for the
$c_n$ coefficients are:
\begin{equation}
c_0 = \frac{15}{8} \eta \left( 1-\eta^2 \right),
\end{equation}
where $\eta = \sqrt{ 1 - r^2}$,
and $r=\sqrt{x^2 + y^2}/R$ is the nondimensional radius.
\begin{equation}
c_1 = -\frac{ 15 \pi}{256} 
\left( 5 - 9 \eta^2 \right)
\left[ 
\left(1-\eta^2\right)
\left( \frac{1 - \sin{\alpha_d}}{1 + \sin{\alpha_d}} \right)
\right]^{\frac{1}{2}},
\end{equation}
\begin{equation}
c_3 = \frac{ 45 \pi}{256} 
\left[ 
\left(1-\eta^2\right)
\left( \frac{1 - \sin{\alpha_d}}{1 + \sin{\alpha_d}} \right)
\right]^{\frac{3}{2}},
\end{equation}
and $c_n = 0$ for odd values of $n \ge 0$.

For even values:
\begin{equation}
c_n = (-1)^{\frac{n}{2}-1} 
\frac{15}{8}
\left[ 
{\frac{\eta + n}{n^2 -1}} \dot {\frac{9 \eta^2 + n^2 -6}{n^2 -9}} + 
{\frac{ 3 \eta}{n^2 -9}} 
\right]
\left[ 
\left( \frac{1 - \eta}{1 + \eta} \right)
\left( \frac{1 - \sin{\alpha_d}}{1 + \sin{\alpha_d}} \right)
\right]^{\frac{n}{2}},
\end{equation}

Note: the version proposed by Bramwell \cite{BRAMWELL-1976}
makes sense also in hover but gives different
results with respect to the uniform inflow and the Glauert inflow models.
Let assume $\alpha_d = \pi/2$, it follows that $c_n = 0$ for $n \ge 1$.
Therefore the induced velocity does not depend on the azimuthal position 
$\psi$ but only on the radial position $r$, so the inflow is not uniform, 
but the mean inflow is still $\lambda_0$.

\subsection{Dynamic Inflow Model}

The dynamic inflow model implemented in MBDyn has been developed by Pitt
and Peters \cite{PITT}. 
Here the model is just briefly described together with the MBDyn implementation
peculiarities. 

The inflow dynamics is represented by a simple first-order linear model: 
\begin{equation}\label{eq:PP}
\TT{M} \dot{\T{\lambda}} + \TT{L}^{-1} \T{\lambda} =
 \T{c},
\end{equation}
where $\T{\lambda}$ is a vector with 3 elements:
\begin{equation}
\T{\lambda} = 
\left[ 
\begin{array}{c}
\lambda_0 \\
\lambda_s \\
\lambda_c
\end{array}
\right],
\end{equation}
the induced velocity on the rotor disk is finally obtained as 
function of the azimuthal angle $\psi$ and the non dimensional
radial position $r = \frac{\sqrt{x^2 + y^2}}{R}$ using the
following equation:
\begin{equation}
u_{ind}(r, \psi) = \Omega R ( \lambda_0 + \lambda_s r \sin{\psi} +
\lambda_c r \cos{\psi} ).
\end{equation}
The right-hand term in equation \ref{eq:PP} contains the thrust,
roll moment and pitch moment coefficients:
\begin{equation}
\T{c} = \left[
\begin{array}{c}
C_T \\
C_L \\
C_M 
\end{array}
\right] = \left[
\begin{array}{c}
\frac{T}{\rho A \Omega^2 R^2} \\
\frac{L}{\rho A \Omega^2 R^3} \\
\frac{M}{\rho A \Omega^2 R^3} 
\end{array}
\right],
\end{equation}
while $\T{\lambda}$ is derived with respect a non-dimensional
time $\Omega t$:
\begin{equation}
\dot{\T{\lambda}} = \frac{ d \T{\lambda}}{ d (\Omega t)}.
\end{equation}

Equation \ref{eq:PP} could be rewritten as:
\begin{equation}
\TT{M} \dot{\T{\lambda}} + \Omega \TT{L}^{-1} \T{\lambda} =
\Omega \T{c},
\end{equation}
where now the dot represents the (dimensional) time derivative. The latter is the 
form implemented in MBDyn.

Matrix $\TT{L}$ is defined as:
\begin{equation}
\TT{L} = \tilde{\TT{L}} \cdot \TT{K} = 
\left[
\begin{array}{ccc}
\cfrac{1}{2} & 0 & \cfrac{15 \pi}{64} \tan{\cfrac{\chi}{2}} \\
 0 & -\cfrac{4}{2 \cos^2{\cfrac{\chi}{2}}} & 0 \\
\cfrac{15 \pi}{64} \tan{\cfrac{\chi}{2}} & 0 &
-\cfrac{4 ( 2 \cos^2{\cfrac{\chi}{2}} - 1 ) }{2 \cos^2{\cfrac{\chi}{2}}}
\end{array}
\right] \left[
\begin{array}{ccc}
\cfrac{1}{V_T} & & \\
& \cfrac{1}{V_m} & \\
& & \cfrac{1}{V_m}
\end{array}
\right],
\end{equation}
where $\chi$ is wake skew angle defined as:
\begin{equation}
\chi = \tan^{-1}{\frac{\mu}{\lambda}},
\end{equation}
where 
\begin{equation}
\mu = \frac{ V_\infty \cos{\alpha_d}}{\Omega R},
\end{equation}
and
\begin{equation}
\lambda = \frac{ V_\infty \sin{\alpha_d}}{\Omega R} + 
\frac{u_{ind}^{0}}{\Omega R},
\end{equation}
where $u_{ind}^{0}$ is the steady uniform induced velocity
computed as described in the uniform inflow model section.
The elements in matrix $\TT{K}$ matrix are respectively:
\begin{equation}
V_T = \sqrt{\lambda^2 + \mu^2},
\end{equation}
and
\begin{equation}
V_m = \frac{\mu^2 + \lambda \left( \lambda + \cfrac{u_{ind}^{0}}{\Omega R} \right)}
{\sqrt{\lambda^2 + \mu^2}}
\end{equation}


\textbf{Note: matrix $\TT{L}$ is slightly different from matrix $\TT{L}$
in Pitt and Peters' paper \cite{PITT} because here the
first column elements are divided by $V_T$ while the second and
third columns elements by $V_m$, whereas in matrix $\TT{L}$
an unique value $v$ is used for all the elements in the matrix.
The $V_m$ term corresponds to the \emph{steady lift mass-flow parameter}
defined in the Pitt and Peters paper, while the $V_T$ corresponds to
the \emph{no lift mass-flow parameter}, because 
$\overline{\lambda} + \overline{\nu} = \lambda$ and 
$\overline{\nu} = u_{ind}^{0}/(\Omega R)$.
Moreover in the Pitt and Peters work the elements are function of
$\alpha = \tan^{-1}{\frac{\lambda}{\mu}}$, the relation between $\alpha$ and
$\chi$ is:
\begin{equation}
\alpha = \frac{\pi}{2} - \chi.
\end{equation}
So,
\begin{equation}
\sqrt{\frac{1-\sin{\alpha}}{1+\sin{\alpha}}} = \tan{\frac{\chi}{2}},
\end{equation}
\begin{equation}
\sin{\alpha} = 2 \cos^{2} {\frac{\chi}{2}}
\end{equation}
That means that the only difference in the MBDyn implementation is
related to matrix $\TT{K}$}.

In MBDyn the inversion of the $\TT{L}$ matrix is formulated analytically;
matrix $\TT{L}^*$ is defined as:
\begin{equation}
\TT{L}^* = \Omega \TT{L}^{-1} = \Omega \left[
\begin{array}{ccc}
\cfrac{l_{33}}{l_{11}l_{33}-l_{13}l_{31}} & 0 & -\cfrac{l_{13}}{l_{11}l_{33}-l_{13}l_{31}} \\
0 & \cfrac{1}{l_{22}} & 0 \\
-\cfrac{l_{31}}{l_{11}l_{33}-l_{13}l_{31}} & 0 & \cfrac{l_{11}}{l_{11}l_{33}-l_{13}l_{31}}
\end{array}
\right].
\end{equation}

%Finally the $\TT{M}$ matrix is defined as:
%\begin{equation}\label{eq:Mdef}
%\TT{M} = \left[
%\begin{array}{ccc}
%\frac{8}{3 \pi} & 0  & 0 \\
%0 & -\frac{16}{45 \pi} & 0 \\
%0 & 0 & -\frac{16}{45 \pi}
%\end{array}
%\right].
%\end{equation}
%
%\textbf{Note: this choice of the $\TT{M}$ matrix corresponds
%to the \emph{uncorrected} M-matrix in the Pitt and Peters paper,
%even if they suggest a mixed form between the \emph{uncorrected}
%and \emph{corrected} matrix where in the first column of the $\TT{M}$
%and $\TT{L}$ matrices the \emph{corrected} values are used, while
%the \emph{uncorrected} values are used for the second and third columns.
%Following this approach the $M_{11}$ element in matrix \ref{eq:Mdef} 
%should be replaced by $\frac{128}{75 \pi}$}.

Finally, matrix $\TT{M}$ is defined as:
\begin{equation}\label{eq:Mdef}
\TT{M} = \left[
\begin{array}{ccc}
\cfrac{128}{75 \pi} & 0  & 0 \\
0 & -\cfrac{16}{45 \pi} & 0 \\
0 & 0 & -\cfrac{16}{45 \pi}
\end{array}
\right].
\end{equation}

\textbf{Note: this choice of matrix $\TT{M}$ corresponds
to the mixed \emph{uncorrected}-\emph{corrected} (following
the authors' nomenclature) M-matrix proposed by Pitt and Peters 
in their paper.}







