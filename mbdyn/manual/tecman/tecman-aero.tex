% MBDyn (C) is a multibody analysis code.
% http://www.mbdyn.org
%
% Copyright (C) 1996-2009
%
% Pierangelo Masarati  <masarati@aero.polimi.it>
%
% Dipartimento di Ingegneria Aerospaziale - Politecnico di Milano
% via La Masa, 34 - 20156 Milano, Italy
% http://www.aero.polimi.it
%
% Changing this copyright notice is forbidden.
%
% This program is free software; you can redistribute it and/or modify
% it under the terms of the GNU General Public License as published by
% the Free Software Foundation (version 2 of the License).
% 
%
% This program is distributed in the hope that it will be useful,
% but WITHOUT ANY WARRANTY; without even the implied warranty of
% MERCHANTABILITY or FITNESS FOR A PARTICULAR PURPOSE.  See the
% GNU General Public License for more details.
%
% You should have received a copy of the GNU General Public License
% along with this program; if not, write to the Free Software
% Foundation, Inc., 59 Temple Place, Suite 330, Boston, MA  02111-1307  USA
%
% Alessandro Fumagalli <fumagalli@aero.polimi.it> is the author of this document

Aerodynamic elements apply aerodynamic forces to structural nodes.

\section{Aerodynamic Body}
The \texttt{aerodynamic body} element applies aerodynamic forces
to the structural node it is connected to,
based on the relative velocity between an aerodynamic surface attached 
to the node and the airstrem.

This section is not intended 
to give details about the aerodynamic models adopted but mainly 
discuss the computation of the contributions to the Jacobian 
matrix of the aerodynamic elements.

\paragraph{Files.} \
It is implemented in files\\
\begin{tabular}{l}
\texttt{mbdyn/aero/aeroelem.h} \\
\texttt{mbdyn/aero/aeroelem.cc} 
\end{tabular}

\paragraph{Aerodynamic body}

The aerodynamic forces and moments acting on the attached node are:
\begin{subequations}
\begin{align}
	\T{f} &= \T{R}_{loc} \T{f}_a\\ 
	\T{c} &= \T{R}_{loc} \T{c}_a + \T{x}_r\times\T{f}_a 
\end{align}
\end{subequations}
where $\T{R}_{loc} = \T{R}_n\T{R}_a\T{R}_t$ is the orientation 
matrix from the local frame of the aerodynamics to the global, 
while $\T{f}_a$ and $\T{c}_a$ are the aerodynamic force and moment 
respectively in the reference frame of the aerodynamics. $\T{R}_n$
is the orientation matrix of the node, $\T{R}_a$ is the (constant) 
relative orientation matrix of the aerodynamics frame with respect 
to the node frame while $\T{R}_t$ is a constant rotation matrix 
related to the twist of the aerodynamic surface. $\T{x}_r = \T{R}_n\T{b}$ 
is the offset of the aerodynamic surface with respet to the node; 
$\T{b}$ is constant.

is the relative offset of the aerodynamic surface from the node.

The aerodynamic forces and moments are computed, depending on the method, 
by means of a function which takes as arguments a velocity vector
in the reference frame of the aerodynamics:
\begin{equation}
	\sqbr{\matr{c}{\T{f_a}\\\T{c_a}}} = \T{g}\plbr{\T{V}_{a}}
\end{equation}
where $\T{V}_{a} = \sqbr{\T{v}_a\ \T{\omega}_a}$ includes both the 
linear and angular velocity in the reference frame of the aerodynamics:
\begin{equation}
	\T{V}_{a} = \sqbr{\matr{c}{
		\T{R}^T_{loc} \plbr{\T{v}_n + \T{\omega}_n\times\T{x}_r - \T{v}_{\infty} - \T{v}_{ind}}\\
		\T{R}^T_{loc} \T{\omega}_n
		}}
\end{equation}
The total airstream velocity in the aerodynamic frame of reference 
includes also the free stream velocity $\T{v}_{\infty}$ and, eventually, 
the induced velocity $\T{v}_{ind}$. These two components will not be 
considered in the following since they do not contribute to the 
Jacobian matrix of the problem. 
The perturbation of the aerodynamic force in the global reference frame
yields:
\begin{equation}
	\delta\plbr{\T{f}} = \T{\theta_{\delta}}\times\T{R}_{loc}\T{f}_a + 
				\T{R}_{loc}\frac{\partial\T{f}_a}{\partial\T{V}_a}
				\delta\T{V}_a
\end{equation}
The $3\times6$ matrix $\frac{\partial\T{f}_a}{\partial\T{V}_a}$, 
named $\T{J_f}_a$ in the
following, is computed numerically using a finite difference methodology
which is described below.

The perturbation of the aerodynamic moment in the global reference frame
yields:
\begin{equation}
	\delta\plbr{\T{c}} = \T{\theta_{\delta}}\times\T{R}_{loc}\T{c}_a + 
				\T{R}_{loc}\frac{\partial\T{c}_a}{\partial\T{V}_a}
				\delta\T{V}_a + \delta \T{x}_r\times\T{f} + 
				\T{x}_r\times\delta\T{f}
\end{equation}
The $3\times6$ matrix $\frac{\partial\T{c}_a}{\partial\T{V}_a}$, 
named $\T{J_c}_a$ in the
following, is computed numerically using a finite difference methodology.

The Jacobian matrix of the aerodynamic forces and moments in the reference 
frame of the aerodynamics is computed computed numerically, using a finite 
difference methodology involving a numerical perturbation relative velocity
in the reference frame of the aerodynamics. In details, each componenet
of the $6\times6$ matrix $\T{J}_a$ is computed as:
\begin{equation}
	{J_a}_{ij} = \frac{g_i(\T{V}_a + \Delta\T{V}_a) 
		- g_i(\T{V}_a)}{\Delta {V_a}_j}
\end{equation}
The perturbation of the velocity, $\Delta\T{V}_a$ in this case, plays
a crucial role. The procedure here adopted consists in obtaining this
perturbation based on the norm of the linear velocity $\T{v}_a$ for what 
concerns the perturbation of the linear velocity components, based on
the norm of the angular velocity $\T{\omega}_a$ for what concerns 
the angular velocity components. Thus:
\begin{equation}
	\Delta {V_a}_j = \Bigl\{\matr{c}{
		\varepsilon\lvert\T{v}_a\lvert\text{ if $j = 1,2,3$}\\
		\varepsilon\lvert\T{\omega}_a\lvert\text{ if $j = 4,5,6$}
		}
\end{equation}
To summarize the procedure, it can be helpful to report the algorithm
actually implemented:
\begin{verbatim}
for(i = 1; i <= 6; i++)	{
    deltaVa = V0a; 
				
    if (i <= 3)	{
        delta = norm(V0a) * epsilon;
    } else		{
        delta = norm(W0a) * epsilon;
    }
    deltaVa(i)= V0a(i) + delta;
					
    Fa = GetForces(deltaVa);
	
    for(j = 1; j <= 6; j++)	{
        Ja(j,i) = (Fa(j) - Fa0(j)) / delta 
    }
}
\end{verbatim}

For easyness of the subsequent notation, the $6\times6$ matrix $\T{J}_a$ is 
divided into four sub-matrices as:
\begin{equation}
	\T{J}_a = \sqbr{\matr{c}{\T{J_f}_a\\\T{J_c}_a}}
	=\sqbr{\matr{cc}{\T{J}_{a11} & \T{J}_{a12}\\ \T{J}_{a21} & \T{J}_{a22}}}
\end{equation}

To correctly espress the Jacobian matrix contributions, the perturbation 
$\delta\T{V}_a$ and $\delta \T{x}_r$ must be expressed in terms of 
the node coordinates and velocities, namely:
\begin{eqnarray}
	\delta\T{V}_a &=& \delta\sqbr{\matr{c}{
		\T{R}^T_{loc} \plbr{\T{v}_n + \T{\omega}_n\times\T{x}_r}\\
		\T{R}^T_{loc} \T{\omega}_n
		}} \\
\nonumber&=&\sqbr{\matr{c}{
	\T{R}^T_{loc}\plbr{
			\T{v}_n + \T{\omega}_n\times\T{x}_r 
			- \T{v}_{\infty} - \T{v}_{ind}} \times \T{\theta}_\delta+ 
			\T{R}^T_{loc}\plbr{
				\delta\T{v}_n + \delta\T{\omega}_n\times\T{x}_r +
				\T{\omega}_n\times\delta\T{x}_r}\\
	\T{R}^T_{loc}\T{\omega}_n \times\T{\theta}_\delta+ \T{R}^T_{loc}\T{\omega}_n
	}}
\end{eqnarray}
and $\delta\T{x}_r = \T{\theta}_\delta\times\T{R}_n\T{b}$.

After the algebra, the perturbation of the aerodynamic forces  
in the global reference frame reads:
\begin{eqnarray}\label{eq:deltaF}
\nonumber \delta\T{f} = 
\nonumber	&-& \T{R}_{loc}\T{f}_a\times\T{\theta}_\delta \\
\nonumber	&+& \T{R}_{loc}\T{J}_{a11}\T{R}^T_{loc}\T{v}_r\times\T{\theta}_\delta\\
\nonumber	&+& \T{R}_{loc}\T{J}_{a11}\T{R}^T_{loc}\delta\T{v}_n\\
\nonumber	&-& \T{R}_{loc}\T{J}_{a11}\T{R}^T_{loc}
			\T{x}_r\times\delta\T{\omega}_n\\
\nonumber	&+& \T{R}_{loc}\T{J}_{a11}\T{R}^T_{loc}
			\T{\omega}_n\times\T{x}_r\times\T{\theta}_\delta\\
\nonumber	&+& \T{R}_{loc}\T{J}_{a12}\T{R}^T_{loc}\T{\omega}_n
			\times\T{\theta}_\delta\\
		&+& \T{R}_{loc}\T{J}_{a12}\T{R}^T_{loc}\delta\T{\omega}_n
\end{eqnarray}
while the perturbation of the aerodynamic moments yields:

\begin{eqnarray}
\nonumber \delta\T{c} = 
\nonumber	&-& \T{R}_{loc}\T{c}_a\times\T{\theta}_\delta \\
\nonumber	&+& \T{R}_{loc}\T{J}_{a21}\T{R}^T_{loc}\T{v}_r\times\T{\theta}_\delta\\
\nonumber	&+& \T{R}_{loc}\T{J}_{a21}\T{R}^T_{loc}\delta\T{v}_n\\
\nonumber	&-& \T{R}_{loc}\T{J}_{a21}\T{R}^T_{loc}
			\T{x}_r\times\delta\T{\omega}_n\\
\nonumber	&+& \T{R}_{loc}\T{J}_{a21}\T{R}^T_{loc}
			\T{\omega}_n\times\T{x}_r\times\T{\theta}_\delta\\
\nonumber	&+& \T{R}_{loc}\T{J}_{a22}\T{R}^T_{loc}\T{\omega}_n
			\times\T{\theta}_\delta\\
\nonumber	&+& \T{R}_{loc}\T{J}_{a22}\T{R}^T_{loc}\delta\T{\omega}_n\\
\nonumber 	&+& \T{f}\times\T{x}_r\times\T{\theta}_\delta\\
		&+& \T{x}_r \times \delta\T{f}
\end{eqnarray}
where the last term, which involves $\delta\T{f}$, is the same of
Eq.~(\ref{eq:deltaF}).


