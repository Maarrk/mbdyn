% MBDyn (C) is a multibody analysis code.
% http://www.mbdyn.org
%
% Copyright (C) 1996-2014
%
% Pierangelo Masarati  <masarati@aero.polimi.it>
%
% Dipartimento di Ingegneria Aerospaziale - Politecnico di Milano
% via La Masa, 34 - 20156 Milano, Italy
% http://www.aero.polimi.it
%
% Changing this copyright notice is forbidden.
%
% This program is free software; you can redistribute it and/or modify
% it under the terms of the GNU General Public License as published by
% the Free Software Foundation (version 2 of the License).
% 
%
% This program is distributed in the hope that it will be useful,
% but WITHOUT ANY WARRANTY; without even the implied warranty of
% MERCHANTABILITY or FITNESS FOR A PARTICULAR PURPOSE.  See the
% GNU General Public License for more details.
%
% You should have received a copy of the GNU General Public License
% along with this program; if not, write to the Free Software
% Foundation, Inc., 59 Temple Place, Suite 330, Boston, MA  02111-1307  USA
%
% Mattia Mattaboni <mattaboni@aero.polimi.it> and
% Pierangelo Masarati <masarati@aero.polimi.it> authored this document.

\paragraph{Definitions.} \
Axis 3 is the rotor's axis.
$\tilde{\T{v}}$ is the composition of the velocity of the `aircraft' node
and of the airstream speed, if any, projected in the reference frame
of the `aircraft' node, namely
\begin{align}
	\tilde{\T{v}}
	&=
	\TT{R}_\text{craft}^T \plbr{
		- \T{v}_\text{craft}
		+ \T{v}_\infty
	}
	.
\end{align}
%
\begin{subequations}
\begin{align}
	v_{12}
	&=
	\sqrt{v_1^2 + v_2^2}
	\\
	v
	&=
	\sqrt{v_1^2 + v_2^2 + v_3^2}
	= \sqrt{\tilde{\T{v}}^T \tilde{\T{v}}}
	\\
	\sin\alpha_d
	&=
	-v_3/v
	\\
	\cos\alpha_d
	&=
	v_{12}/v
	\\
	\psi_0
	&=
	\tan^{-1}\plbr{\frac{v_2}{v_1}}
	\\
	v_\text{tip}
	&=
	\Omega R
	\\
	\mu
	&=
	\cos\alpha_d \frac{v}{v_\text{tip}}
\end{align}
\end{subequations}
Note: $v \ge 0$ and $v_{12} \ge 0$ by definition;
as a consequence, $\cos\alpha_d \ge 0$,
while the sign of $\sin\alpha_d$ depends on whether
the flow related to the absolute motion of the rotor
enters the disk from above (> 0) or from below (< 0).
$v_\text{tip} > 0$ by construction
($\Omega = \nrbr{\T{\omega}}$, and no induced velocity
is computed if $\Omega$ is below a threshold).
As a consequence, $\mu \ge 0$.


\paragraph{Ground effect.} \
If defined, according to \cite{CHEESEMAN-BENNETT-1955},
\begin{subequations}
\begin{align}
	u_\text{IGE}
	&=
	k_\text{GE} u_\text{OGE}
	\\
	k
	&=
	1 - \frac{1}{z^2}
	\\
	z
	&= \max\plbr{\frac{h}{R}, \frac{1}{4}}
\end{align}
\end{subequations}
where $h$ is the component along axis 3 of the `ground' node
of distance between the `aircraft' and the `ground' node,
assuming the `aircraft' node is located at the hub center.

\paragraph{Reference Induced Velocity.} \
The reference induced velocity $u$ is computed by solving the implicit problem
\begin{align}
	f
	&=
	% \frac{u}{v_\text{tip}}
	\lambda_u
	- \frac{C_t}{2\sqrt{\mu^2 + \lambda^2}}
	=
	0
	,
\end{align}
with
\begin{subequations}
\begin{align}
	C_t
	&=
	\frac{T}{\rho \pi R^4 \Omega^2}
	\\
	\lambda_u
	&=
	\frac{u}{v_\text{tip}}
	\\
	\lambda
	&=
	\frac{v \sin\alpha_d + u}{v_\text{tip}}
	=
	\mu \tan\alpha_d
	+ \lambda_u
	;
\end{align}
\end{subequations}
$T$ is the component of the aerodynamic force of the rotor
along the shaft axis.
The value of $\lambda_u$ is initialized using the reference induced velocity
$u$ at the previous step/iteration.
Only when $u=0$ and $C_t\neq 0$, $u$ is initialized using its nominal value
in hover,
\begin{align}
	u
	&=
	\text{sign}\plbr{T} \sqrt{\frac{\nrbr{T}}{2 \rho A}}
	.
\end{align}
The problem is solved by means of a local Newton iteration.
The Jacobian of the problem is
\begin{align}
	\frac{\partial f}{\partial \lambda_u}
	&=
	1 + \frac{C_t}{2 \plbr{\mu^2 + \lambda^2}^{3/2}} \lambda
	.
\end{align}
The solution,
\begin{align}
	\Delta\lambda_u
	&=
	- \plbr{\frac{\partial f}{\partial \lambda_u}}^{-1} f
	,
\end{align}
is added to $\lambda_u$ as
$\lambda_u += \eta \Delta\lambda_u$,
where $0 < \eta \le 1$ is an optional relaxation factor
(by default, $\eta=1$).

\paragraph{Corrections.} \
The reference induced velocity is corrected by separately correcting
the inflow and advance parameters, namely
\begin{subequations}
\begin{align}
	\lambda^*
	&=
	\frac{\lambda}{k_\text{H}^2}
	\\
	\mu^*
	&=
	\frac{\mu}{k_\text{FF}}
\end{align}
\end{subequations}
(by default, $k_\text{H}=1$ and $k_\text{FF}=1$).
The reference induced velocity is then recomputed as
\begin{align}
	u^*
	&=
	\plbr{1 - \rho} k_\text{GE} v_\text{tip}
		\frac{C_t}{2 \sqrt{\mu^{*^2} + \lambda^{*^2}}}
	+ \rho u^*_\text{prev}
	,
\end{align}
where $0 \le \rho < 1$ is a memory factor (by default, $\rho=0$).

Note: in principle, multiple solutions for $\lambda_u$ are possible.
However, only one solution is physical.
Currently, no strategy is put in place to ensure that only the physical
solution is considered.


\subsection{Uniform Inflow Model}

The induced velocity is equal to its reference value, $u^*$, everywhere.


\subsection{Glauert Model}

In forward flight (when $\mu > 0.15$) the inflow over the rotor disk can 
be approximated by:
\begin{align}
\lambda_i &= \lambda_0 \left( 1 + k_x \frac{x}{R} + k_y \frac{y}{R} \right)
\\
&= \lambda_0 \left( 1 + k_x r \cos{\psi} + k_y r \sin{\psi} \right),
\end{align}
where the mean induced velocity $\lambda_0$ is computed as shown in the
previous section,
while $r=\sqrt{x^2 + y^2}/R$ is the nondimensional radius.

In literature a lot of expressions for the $k_x$ and $k_y$ coefficients have
been proposed by different authors, as summarized in table \ref{tab:GlauertCoeff}.
Up to now in MBDyn the following expressions are
implemented:
\begin{align}
k_x &=  \frac{4}{3} \left( 1 - 1.8 \mu^2 \right) \tan{\frac{\chi}{2}}
\\
k_y &= 0,
\end{align}
\begin{table}[h]
\centering
\caption{Glauert inflow model (source: Leishman \cite{LEISHMAN-2006})}\label{tab:GlauertCoeff}
\begin{tabular}{l|c|c}
\textbf{Author(s)} & $k_x$ & $k_y$\\
\hline
Coleman et al. (1945)	&	
$\tan{\frac{\chi}{2}}$	&	$0$\\
Drees (1949)		&	
$\frac{4}{3}\frac{\left( 1 - \cos{\chi} -1.8 \mu^2\right)}{\sin{\chi}} $	&	$-2 \mu$\\
Payne (1959)		&
$\frac{4}{3} ( \mu/ \lambda/ (1.2 + \mu/\lambda))$	&	$0$\\
White \& Blake (1979)	&
$\sqrt{2} \sin{\chi}$	&	$0$\\
Pitt \& Peters (1981)	&
$\frac{15 \pi}{23} \tan{\frac{\chi}{2}}$	&	$0$\\
Howlett (1981)		&
$ \sin^2{\chi}$	&	$0$\\
\end{tabular}
\end{table}

\textbf{FIXME: \`e diversa dalle espressioni riportate sul Leishman!}

where $\chi$ is the wake skew angle:
\begin{equation}
\chi = \tan^{-1}\left( \frac{\mu}{\lambda} \right).
\end{equation}
\textbf{TODO: check, dovrebbe essere equivalente all'espressione di Leishman ma
sarebbe meglio verificare!!!}

Note: the Glauert inflow model exactly matches the uniform inflow model
when the advance ratio is null (in hover).
In fact, when $\mu = 0$ then $\chi = k_x = 0$.
The inflow is thus uniform over the rotor disk, and equal to $\lambda_0$.

\subsection{Mangler-Squire Model}

The Mangler-Squire model is developed under the high speed assumption
and it should be used only for advance ratio grater than 0.1.

In the original formulation the induced velocity is computed as:
\begin{equation}
\lambda_i = \left( \frac{ 2 C_T}{\mu} \right) \left[ 
\frac{c_0}{2} - \sum_{n=1}^{\infty} c_n(r,\alpha_d) \cos{n \psi} \right],
\end{equation}
since the advance ratio $\mu$ appears in the denominator this expression is
not valid in hover.
Bramwell \cite{BRAMWELL-1976} proposed a different expression for the induced 
velocity:
\begin{equation}
\lambda_i = 4 \lambda_0 \left[ 
\frac{c_0}{2} - \sum_{n=1}^{\infty} c_n(r,\alpha_d) \cos{n \psi} \right],
\end{equation}
where $\lambda_0$ is the mean inflow computed as shown before. In this way
the Mangler-Squire inflow model makes sense also in hover. MBDyn uses the latter
version.

The expression of the $c_n$ coefficients depends on the form of the 
rotor disk loading. Mangler and Squire developed the theory for two fundamental
forms: Type I (elliptical loading) and Type III (a loading that vanishes at the 
edges and at the center of the disk). The total loading is finally obtained by
a linear combination of Type I and Type III loadings (see \cite{LEISHMAN-2006}).

MBDyn uses just a Type III loading, and the resulting expressions for the
$c_n$ coefficients are:
\begin{equation}
c_0 = \frac{15}{8} \eta \left( 1-\eta^2 \right),
\end{equation}
where $\eta = \sqrt{ 1 - r^2}$,
and $r=\sqrt{x^2 + y^2}/R$ is the nondimensional radius.
\begin{equation}
c_1 = -\frac{ 15 \pi}{256} 
\left( 5 - 9 \eta^2 \right)
\left[ 
\left(1-\eta^2\right)
\left( \frac{1 - \sin{\alpha_d}}{1 + \sin{\alpha_d}} \right)
\right]^{\frac{1}{2}},
\end{equation}
\begin{equation}
c_3 = \frac{ 45 \pi}{256} 
\left[ 
\left(1-\eta^2\right)
\left( \frac{1 - \sin{\alpha_d}}{1 + \sin{\alpha_d}} \right)
\right]^{\frac{3}{2}},
\end{equation}
and $c_n = 0$ for odd values of $n \ge 0$.

For even values:
\begin{equation}
c_n = (-1)^{\frac{n}{2}-1} 
\frac{15}{8}
\left[ 
{\frac{\eta + n}{n^2 -1}} \dot {\frac{9 \eta^2 + n^2 -6}{n^2 -9}} + 
{\frac{ 3 \eta}{n^2 -9}} 
\right]
\left[ 
\left( \frac{1 - \eta}{1 + \eta} \right)
\left( \frac{1 - \sin{\alpha_d}}{1 + \sin{\alpha_d}} \right)
\right]^{\frac{n}{2}},
\end{equation}

Note: the version proposed by Bramwell \cite{BRAMWELL-1976}
makes sense also in hover but gives different
results with respect to the uniform inflow and the Glauert inflow models.
Let assume $\alpha_d = \pi/2$, it follows that $c_n = 0$ for $n \ge 1$.
Therefore the induced velocity does not depend on the azimuthal position 
$\psi$ but only on the radial position $r$, so the inflow is not uniform, 
but the mean inflow is still $\lambda_0$.

\subsection{Dynamic Inflow Model}

The dynamic inflow model implemented in MBDyn has been developed by Pitt
and Peters \cite{PITT}. 
Here the model is just briefly described together with the MBDyn implementation
peculiarities. 

The inflow dynamics is represented by a simple first-order linear model: 
\begin{equation}\label{eq:PP}
\TT{M} \dot{\T{\lambda}} + \TT{L}^{-1} \T{\lambda} =
 \T{c},
\end{equation}
where $\T{\lambda}$ is a vector with 3 elements:
\begin{equation}
\T{\lambda} = 
\left[ 
\begin{array}{c}
\lambda_0 \\
\lambda_s \\
\lambda_c
\end{array}
\right],
\end{equation}
the induced velocity on the rotor disk is finally obtained as 
function of the azimuthal angle $\psi$ and the non dimensional
radial position $r = \frac{\sqrt{x^2 + y^2}}{R}$ using the
following equation:
\begin{equation}
u_{ind}(r, \psi) = \Omega R ( \lambda_0 + \lambda_s r \sin{\psi} +
\lambda_c r \cos{\psi} ).
\end{equation}
The right-hand term in equation \ref{eq:PP} contains the thrust,
roll moment and pitch moment coefficients:
\begin{equation}
\T{c} = \left[
\begin{array}{c}
C_T \\
C_L \\
C_M 
\end{array}
\right] = \left[
\begin{array}{c}
\frac{T}{\rho A \Omega^2 R^2} \\
\frac{L}{\rho A \Omega^2 R^3} \\
\frac{M}{\rho A \Omega^2 R^3} 
\end{array}
\right],
\end{equation}
while $\T{\lambda}$ is derived with respect a non-dimensional
time $\Omega t$:
\begin{equation}
\dot{\T{\lambda}} = \frac{ d \T{\lambda}}{ d (\Omega t)}.
\end{equation}

Equation \ref{eq:PP} could be rewritten as:
\begin{equation}
\TT{M} \dot{\T{\lambda}} + \Omega \TT{L}^{-1} \T{\lambda} =
\Omega \T{c},
\end{equation}
where now the dot represents the (dimensional) time derivative. The latter is the 
form implemented in MBDyn.

Matrix $\TT{L}$ is defined as:
\begin{equation}
\TT{L} = \tilde{\TT{L}} \cdot \TT{K} = 
\left[
\begin{array}{ccc}
\cfrac{1}{2} & 0 & \cfrac{15 \pi}{64} \tan{\cfrac{\chi}{2}} \\
 0 & -\cfrac{4}{2 \cos^2{\cfrac{\chi}{2}}} & 0 \\
\cfrac{15 \pi}{64} \tan{\cfrac{\chi}{2}} & 0 &
-\cfrac{4 ( 2 \cos^2{\cfrac{\chi}{2}} - 1 ) }{2 \cos^2{\cfrac{\chi}{2}}}
\end{array}
\right] \left[
\begin{array}{ccc}
\cfrac{1}{V_T} & & \\
& \cfrac{1}{V_m} & \\
& & \cfrac{1}{V_m}
\end{array}
\right],
\end{equation}
where $\chi$ is wake skew angle defined as:
\begin{equation}
\chi = \tan^{-1}{\frac{\mu}{\lambda}},
\end{equation}
where 
\begin{equation}
\mu = \frac{ V_\infty \cos{\alpha_d}}{\Omega R},
\end{equation}
and
\begin{equation}
\lambda = \frac{ V_\infty \sin{\alpha_d}}{\Omega R} + 
\frac{u_{ind}^{0}}{\Omega R},
\end{equation}
where $u_{ind}^{0}$ is the steady uniform induced velocity
computed as described in the uniform inflow model section.
The elements in matrix $\TT{K}$ matrix are respectively:
\begin{equation}
V_T = \sqrt{\lambda^2 + \mu^2},
\end{equation}
and
\begin{equation}
V_m = \frac{\mu^2 + \lambda \left( \lambda + \cfrac{u_{ind}^{0}}{\Omega R} \right)}
{\sqrt{\lambda^2 + \mu^2}}
\end{equation}


\textbf{Note: matrix $\TT{L}$ is slightly different from matrix $\TT{L}$
in Pitt and Peters' paper \cite{PITT} because here the
first column elements are divided by $V_T$ while the second and
third columns elements by $V_m$, whereas in matrix $\TT{L}$
an unique value $v$ is used for all the elements in the matrix.
The $V_m$ term corresponds to the \emph{steady lift mass-flow parameter}
defined in the Pitt and Peters paper, while the $V_T$ corresponds to
the \emph{no lift mass-flow parameter}, because 
$\overline{\lambda} + \overline{\nu} = \lambda$ and 
$\overline{\nu} = u_{ind}^{0}/(\Omega R)$.
Moreover in the Pitt and Peters work the elements are function of
$\alpha = \tan^{-1}{\frac{\lambda}{\mu}}$, the relation between $\alpha$ and
$\chi$ is:
\begin{equation}
\alpha = \frac{\pi}{2} - \chi.
\end{equation}
So,
\begin{equation}
\sin{\alpha} = 
\cos{\chi} = 
\cos{\plbr{\frac{\chi}{2}+\frac{\chi}{2}}}=
\cos^{2}{\frac{\chi}{2}} - \sin^{2}{\frac{\chi}{2}} = 
2 \cos^{2} {\frac{\chi}{2}}-1,
\end{equation}
\begin{equation}
\sqrt{\frac{1-\sin{\alpha}}{1+\sin{\alpha}}} = 
\sqrt{ \frac{2 - 2 \cos^{2}{\frac{\chi}{2}}}{2 \cos^{2}{\frac{\chi}{2}}} } = 
\sqrt{ \frac{ 2 \sin^{2}{\frac{\chi}{2}}}{2 \cos^{2}{\frac{\chi}{2}}} } = 
\tan{\frac{\chi}{2}}.
\end{equation}
That means that the only difference in the MBDyn implementation is
related to matrix $\TT{K}$}.

In MBDyn the inversion of the $\TT{L}$ matrix is formulated analytically;
matrix $\TT{L}^*$ is defined as:
\begin{equation}
\TT{L}^* = \Omega \TT{L}^{-1} = \Omega \left[
\begin{array}{ccc}
\cfrac{l_{33}}{l_{11}l_{33}-l_{13}l_{31}} & 0 & -\cfrac{l_{13}}{l_{11}l_{33}-l_{13}l_{31}} \\
0 & \cfrac{1}{l_{22}} & 0 \\
-\cfrac{l_{31}}{l_{11}l_{33}-l_{13}l_{31}} & 0 & \cfrac{l_{11}}{l_{11}l_{33}-l_{13}l_{31}}
\end{array}
\right].
\end{equation}

%Finally the $\TT{M}$ matrix is defined as:
%\begin{equation}\label{eq:Mdef}
%\TT{M} = \left[
%\begin{array}{ccc}
%\frac{8}{3 \pi} & 0  & 0 \\
%0 & -\frac{16}{45 \pi} & 0 \\
%0 & 0 & -\frac{16}{45 \pi}
%\end{array}
%\right].
%\end{equation}
%
%\textbf{Note: this choice of the $\TT{M}$ matrix corresponds
%to the \emph{uncorrected} M-matrix in the Pitt and Peters paper,
%even if they suggest a mixed form between the \emph{uncorrected}
%and \emph{corrected} matrix where in the first column of the $\TT{M}$
%and $\TT{L}$ matrices the \emph{corrected} values are used, while
%the \emph{uncorrected} values are used for the second and third columns.
%Following this approach the $M_{11}$ element in matrix \ref{eq:Mdef} 
%should be replaced by $\frac{128}{75 \pi}$}.

Finally, matrix $\TT{M}$ is defined as:
\begin{equation}\label{eq:Mdef}
\TT{M} = \left[
\begin{array}{ccc}
\cfrac{128}{75 \pi} & 0  & 0 \\
0 & -\cfrac{16}{45 \pi} & 0 \\
0 & 0 & -\cfrac{16}{45 \pi}
\end{array}
\right].
\end{equation}

\textbf{Note: this choice of matrix $\TT{M}$ corresponds
to the mixed \emph{uncorrected}-\emph{corrected} (following
the authors' nomenclature) M-matrix proposed by Pitt and Peters 
in their paper.}






