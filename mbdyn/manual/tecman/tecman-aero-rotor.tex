% MBDyn (C) is a multibody analysis code.
% http://www.mbdyn.org
%
% Copyright (C) 1996-2009
%
% Pierangelo Masarati  <masarati@aero.polimi.it>
%
% Dipartimento di Ingegneria Aerospaziale - Politecnico di Milano
% via La Masa, 34 - 20156 Milano, Italy
% http://www.aero.polimi.it
%
% Changing this copyright notice is forbidden.
%
% This program is free software; you can redistribute it and/or modify
% it under the terms of the GNU General Public License as published by
% the Free Software Foundation (version 2 of the License).
% 
%
% This program is distributed in the hope that it will be useful,
% but WITHOUT ANY WARRANTY; without even the implied warranty of
% MERCHANTABILITY or FITNESS FOR A PARTICULAR PURPOSE.  See the
% GNU General Public License for more details.
%
% You should have received a copy of the GNU General Public License
% along with this program; if not, write to the Free Software
% Foundation, Inc., 59 Temple Place, Suite 330, Boston, MA  02111-1307  USA
%
% Mattia Mattaboni <mattaboni@aero.polimi.it> is the author of this document

Axis 3 is the rotor's axis.
$\T{v}$ is the composition of the velocity of the aircraft node
and of the airstream speed, if any, projected in the reference frame
of the aircraft, namely
\begin{align}
	\T{v}
	&=
	\TT{R}_\text{craft}^T \plbr{
		- \T{v}_\text{craft}
		+ \T{v}_\infty
	}
	.
\end{align}
Definitions:
\begin{subequations}
\begin{align}
	v_{12}
	&=
	\sqrt{v_1^2 + v_2^2}
	\\
	v
	&=
	\sqrt{v_1^2 + v_2^2 + v_3^2}
	= \sqrt{\T{v}^T \T{v}}
	\\
	\sin\alpha_d
	&=
	-v_3/v
	\\
	\cos\alpha_d
	&=
	v_{12}/v
	\\
	\psi_0
	&=
	\tan^{-1}\plbr{\frac{v_2}{v_1}}
	\\
	v_\text{tip}
	&=
	\Omega R
	\\
	\mu
	&=
	\cos\alpha_d \frac{v}{v_\text{tip}}
\end{align}
\end{subequations}
Note: $v \ge 0$ and $v_{12} \ge 0$ by definition;
as a consequence, $\cos\alpha_d \ge 0$,
while the sign of $\sin\alpha_d$ depends on whether
the flow related to the absolute motion of the rotor
enters the disk from above (> 0) or from below (< 0).
$v_\text{tip} > 0$ by construction
($\Omega = \nrbr{\T{\omega}}$, and no induced velocity
is computed if $\Omega$ is below a threshold).
As a consequence, $\mu \ge 0$.


Ground effect (if defined):
\begin{subequations}
\begin{align}
	u_\text{IGE}
	&=
	k_\text{GE} u_\text{OGE}
	\\
	k
	&=
	1 - \frac{1}{z^2}
	\\
	z
	&= \max\plbr{\frac{h}{R}, \frac{1}{4}}
\end{align}
\end{subequations}
where $h$ is the distance of the aircraft node from the ground node
(along axis 3 of the ground node).

The reference induced velocity $u$ is computed by solving the implicit problem
\begin{align}
	f
	&=
	% \frac{u}{v_\text{tip}}
	\lambda_u
	- \frac{C_t}{2\sqrt{\mu^2 + \lambda^2}}
	=
	0
	,
\end{align}
with
\begin{subequations}
\begin{align}
	\lambda_u
	&=
	\frac{u}{v_\text{tip}}
	\\
	\lambda
	&=
	\frac{v \sin\alpha_d + u}{v_\text{tip}}
	=
	\mu \tan\alpha_d
	+ \lambda_u
	.
\end{align}
\end{subequations}
The value of $\lambda_u$ is initialized using the reference induced velocity
$u$ at the previous step/iteration.
Only when $u=0$ and $C_t\neq 0$, $u$ is initialized using its nominal value
in hover,
\begin{align}
	u
	&=
	\text{sign}\plbr{T} \sqrt{\frac{\nrbr{T}}{2 \rho A}}
	.
\end{align}

The problem is solved by means of a local Newton iteration.
The Jacobian of the problem is
\begin{align}
	\frac{\partial f}{\partial \lambda_u}
	&=
	1 + \frac{C_t}{2 \plbr{\mu^2 + \lambda^2}^{3/2}} \lambda
	.
\end{align}
The solution,
\begin{align}
	\Delta\lambda_u
	&=
	- \plbr{\frac{\partial f}{\partial \lambda_u}}^{-1} f
	,
\end{align}
is added to $\lambda_u$ as
$\lambda_u = \lambda_u + \eta \Delta\lambda_u$,
where $0 < \eta \le 1$ is an optional relaxation factor.

Corrections:
the reference induced velocity is corrected by separately correcting
the inflow and advance parameters, namely
\begin{subequations}
\begin{align}
	\lambda^*
	&=
	\frac{\lambda}{k_\text{H}^2}
	\\
	\mu^*
	&=
	\frac{\mu}{k_\text{FF}}
	.
\end{align}
\end{subequations}
The reference induced velocity is then recomputed as
\begin{align}
	u^*
	&=
	\plbr{1 - \rho} k_\text{GE} v_\text{tip}
		\frac{C_t}{2 \sqrt{\mu^{*^2} + \lambda^{*^2}}}
	+ \rho u^*_\text{prev}
	,
\end{align}
where $0 \le \rho < 1$ is a memory factor.

Note: in principle, multiple solutions for $\lambda_u$ are possible.
However, only one solution is physical.
Currently, no strategy is put in place to ensure that only the physical
solution is considered.

