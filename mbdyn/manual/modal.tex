\chapter{Modal Element FEM File Format}
\label{sec:APP:EL:STRUCT:JOINT:MODAL:FORMAT}

This section describes the format of the FEM input to the \kw{modal}
joint of MBDyn, as resulting from the \kw{femgen} utility.
The usage of that utility has been already detailed 
in Section~\ref{sec:EL:STRUCT:JOINT:MODAL}; in short, it processes
binary output from NASTRAN, as defined by means of appropriate
ALTER files provided with MBDyn sources, into a file that is suitable
for direct input in MBDyn.
Since it is essentially a plain ASCII file, it is straightforward
to generate it from analogous results obtained with a different 
FEM software, from experiments or manually generated from analytical
or numerical models of any kind.

The format is:
{\small
\begin{verbatim}
<comments>
** RECORD GROUP 1,<any comment; "HEADER">
<comment>
<REV> <NNODES> <NNORMAL> <NATTACHMENT> <NCONSTRAINT> <NREJECTED>
<comments; NMODES = NNORMAL + NATTACHMENT + NCONSTRAINT - NREJECTED>
** RECORD GROUP 2,<any comment; "FINITE ELEMENT NODE LIST">
<FEMLABEL> [...<NNODES>]
<comments; FEM label list: <NNODES> integers>
** RECORD GROUP 3,<any comment; "INITIAL MODAL DISPLACEMENTS">
<MODEDISP> [..<NMODES>]
<comments; initial mode displacements: <NMODES> reals>
** RECORD GROUP 4,<any comment; "INITIAL MODAL VELOCITIES">
<MODEVEL> [..<NMODES>]
<comments; initial mode velocities: <NMODES> reals>
** RECORD GROUP 5,<any comment; "NODAL X COORDINATES">
<FEMX> [...<NNODES>]
<comments; FEM node X coordinates>
** RECORD GROUP 6,<any comment; "NODAL Y COORDINATES">
<FEMY> [...<NNODES>]
<comments; FEM node Y coordinates>
** RECORD GROUP 7,<any comment; "NODAL Z COORDINATES">
<FEMZ> [...<NNODES>]
<comments; FEM node Z coordinates>
** RECORD GROUP 8,<any comment; "NON-ORTHOGONALIZED MODE SHAPES">
<comment; NORMAL MODE SHAPE # 1>
    <FEM1 X> <FEM1 Y> <FEM1 Z> <FEM1 RX> <FEM1 RY> <FEM1 RZ>
    <... NNODES>
<comment; NORMAL MODE SHAPE # 2>
    <FEM1 X> <FEM1 Y> <FEM1 Z> <FEM1 RX> <FEM1 RY> <FEM1 RZ>
    <... NNODES>
<... NMODES>
<comments; for each MODE, for each NODE: modal displacements/rotations>
** RECORD GROUP 9,<any comment; "MODAL MASS MATRIX. COLUMN-MAJOR FORM">
<M_1_1>      <...> <M_1_NMODES>
<...>
<M_NMODES_1> <...> <M_NMODES_NMODES>
<comments; the modal mass matrix in column-major (symmetric?)>
** RECORD GROUP 10,<any comment; "MODAL STIFFNESS MATRIX. COLUMN-MAJOR FORM">
<K_1_1>      <...> <K_1_NMODES>
<...>
<K_NMODES_1> <...> <K_NMODES_NMODES>
<comments; the modal stiffness matrix in column-major (symmetric?)>
** RECORD GROUP 11,<any comment; "DIAGONAL OF LUMPED MASS MATRIX">
<M_1_X> <M_1_Y> <M_1_Z> <M_1_RX> <M_1_RY> <M_1_RZ>
<...>
<M_NNODES_X> <...> <M_NNODES_RZ>
<comments; the lumped diagonal mass matrix of the FEM model>
\end{verbatim}
}

Although the format loosely requires that no more than 6 numerical values 
appear on a single line, MBDyn is very forgiving about this and can parse
the input regardless of the formatting within each block.

An arbitrary number of comment lines may appera within blocks;
the beginning of a block is marked
\begin{verbatim}
** RECORD GROUP <DD>
\end{verbatim}
where the number \kw{<DD>} indicates what block is being read.
The size of each block, i.e.\ the number of values that are expected,
is defined based on the header block, so MBDyn should be able to detect
incomplete or mis-formatted files.

As an example, a very simple, hand-made FEM model file is presented below.
It models a FEM model made of three aligned nodes, where inertia 
is only associated to the mid-node.
As a consequence, the three mode shapes must be interpreted as static
shapes, since the modal mass matrix is null.
Note that each line is prefixed with a two-digit line number 
that is not part of the iput file.
Also, for readability, all comments are prefixed by \kw{**} 
like the mandatory \kw{** RECORD GROUP} lines, although not required 
by the format of the file.

{\small
\begin{verbatim}
01  ** MBDyn MODAL DATA FILE
02  ** NODE SET "ALL" 
03    
04    
05  ** RECORD GROUP 1, HEADER
06  **   REVISION,  NODE,  NORMAL, ATTACHMENT, CONSTRAINT, REJECTED MODES.
07        REV0         3         3         0         0         0
08  **
09  ** RECORD GROUP 2, FINITE ELEMENT NODE LIST
10       1001 1002 1003
11  
12  **
13  ** RECORD GROUP 3, INITIAL MODAL DISPLACEMENTS
14   0 0 0
15  **
16  ** RECORD GROUP 4, INITIAL MODALVELOCITIES
17   0 0 0
18  **
19  ** RECORD GROUP 5, NODAL X COORDINATES
20   0
21   0
22   0
23  **
24  ** RECORD GROUP 6, NODAL Y COORDINATES
25  -2.
26   0
27   2.
28  **
29  ** RECORD GROUP 7, NODAL Z COORDINATES
30   0
31   0
32   0
33  **
34  ** RECORD GROUP 8, MODE SHAPES
35  **    NORMAL MODE SHAPE #  1
36  0 0 1 0 0 0
37  0 0 0 0 0 0
38  0 0 1 0 0 0
39  **    NORMAL MODE SHAPE #  2
40  1 0 0 0 0 0
41  0 0 0 0 0 0
42  1 0 0 0 0 0
43  **    NORMAL MODE SHAPE #  3
44  0 1 0 0 0 0
45  0 0 0 0 0 0
46  0 -1 0 0 0 0
47  **
48  ** RECORD GROUP 9, MODAL MASS MATRIX
49  0 0 0
50  0 0 0
51  0 0 0
52  **
53  ** RECORD GROUP 10, MODAL STIFFNESS MATRIX
54  1 0   0
55  0 1e2 0
56  0 0   1e6
57  **
58  ** RECORD GROUP 11, DIAGONAL OF LUMPED MASS MATRIX
59  0 0 0 0 0 0
60  1 1 1 1 1 1
61  0 0 0 0 0 0
\end{verbatim}
}
