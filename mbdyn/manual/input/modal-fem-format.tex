% $Header$
% MBDyn (C) is a multibody analysis code.
% http://www.mbdyn.org
%
% Copyright (C) 1996-2012
%
% Pierangelo Masarati  <masarati@aero.polimi.it>
%
% Dipartimento di Ingegneria Aerospaziale - Politecnico di Milano
% via La Masa, 34 - 20156 Milano, Italy
% http://www.aero.polimi.it
%
% Changing this copyright notice is forbidden.
%
% This program is free software; you can redistribute it and/or modify
% it under the terms of the GNU General Public License as published by
% the Free Software Foundation (version 2 of the License).
% 
%
% This program is distributed in the hope that it will be useful,
% but WITHOUT ANY WARRANTY; without even the implied warranty of
% MERCHANTABILITY or FITNESS FOR A PARTICULAR PURPOSE.  See the
% GNU General Public License for more details.
%
% You should have received a copy of the GNU General Public License
% along with this program; if not, write to the Free Software
% Foundation, Inc., 59 Temple Place, Suite 330, Boston, MA  02111-1307  USA

\chapter{Modal Element Data}

\section{FEM File Format}
\label{sec:APP:EL:STRUCT:JOINT:MODAL:FORMAT}

This section describes the format of the FEM input to the \kw{modal}
joint of MBDyn.
This data can be obtained, for example:
\begin{itemize}
\item from Code Aster
(\htmladdnormallink{\texttt{http://www.code-aster.org/}}{http://www.code-aster.org/}),
as discussed in Section~\ref{sec:APP:EL:STRUCT:JOINT:MODAL:ASTER};
\item from NASTRAN
(\htmladdnormallink{\texttt{http://www.mscsoftware.com/}}{http://www.mscsoftware.com/})
output, using the \texttt{femgen} utility,
as detailed in Appendix~\ref{sec:APP:EL:STRUCT:JOINT:MODAL:NASTRAN}
(in short, it processes binary output from NASTRAN, as defined by means
of appropriate ALTER files provided with MBDyn sources, into a file
that is suitable for direct input in MBDyn);
\item by manually crafting the output of your favourite FEM analysis:
since it is essentially a plain ASCII file, it can be generated
in a straightforward manner from analogous results obtained with
almost any FEM software, from experiments or manually generated
from analytical or numerical models of any kind.
\end{itemize}

The format is:
{\small
%\begin{verbatim}
\begin{Verbatim}[commandchars=\\\{\}]
<comments>
** \kw{RECORD GROUP 1},<any comment to EOL; "HEADER">
<comment>
\bnt{REV} \bnt{NNODES} \bnt{NNORMAL} \bnt{NATTACHED} \bnt{NCONSTRAINT} \bnt{NREJECTED}
<comments; \nt{NMODES} = \nt{NNORMAL} + \nt{NATTACHED} + \nt{NCONSTRAINT} - \nt{NREJECTED}>
** \kw{RECORD GROUP 2},<any comment to EOL; "FINITE ELEMENT NODE LIST">
\bnt{FEMLABEL} [...\nt{NNODES}]
<comments; FEM label list: \nt{NNODES} integers>
** \kw{RECORD GROUP 3},<any comment to EOL; "INITIAL MODAL DISPLACEMENTS">
\bnt{MODEDISP} [...\nt{NMODES}]
<comments; initial mode displacements: \nt{NMODES} reals>
** \kw{RECORD GROUP 4},<any comment to EOL; "INITIAL MODAL VELOCITIES">
\bnt{MODEVEL} [...\nt{NMODES}]
<comments; initial mode velocities: \nt{NMODES} reals>
** \kw{RECORD GROUP 5},<any comment to EOL; "NODAL X COORDINATES">
\bnt{FEM_X} [...\nt{NNODES}]
<comments; FEM node X coordinates>
** \kw{RECORD GROUP 6},<any comment to EOL; "NODAL Y COORDINATES">
\bnt{FEM_Y} [...\nt{NNODES}]
<comments; FEM node Y coordinates>
** \kw{RECORD GROUP 7},<any comment to EOL; "NODAL Z COORDINATES">
\bnt{FEM_Z} [...\nt{NNODES}]
<comments; FEM node Z coordinates>
** \kw{RECORD GROUP 8},<any comment to EOL; "NON-ORTHOGONALIZED MODE SHAPES">
<comment; NORMAL MODE SHAPE # 1>
    \bnt{FEM1_X} \bnt{FEM1_Y} \bnt{FEM1_Z} \bnt{FEM1_RX} \bnt{FEM1_RY} \bnt{FEM1_RZ}
    [...\nt{NNODES}]
<comment; NORMAL MODE SHAPE # 2>
    \bnt{FEM2_X} \bnt{FEM2_Y} \bnt{FEM2_Z} \bnt{FEM2_RX} \bnt{FEM2_RY} \bnt{FEM2_RZ}
    [...\nt{NNODES}]
[...\nt{NMODES}]
<comments; for each \nt{MODE}, for each \nt{NODE}: modal displacements/rotations>
** \kw{RECORD GROUP 9},<any comment to EOL; "MODAL MASS MATRIX. COLUMN-MAJOR FORM">
\bnt{M_1_1}      [...] \bnt{M_1_NMODES}
[...]
\bnt{M_NMODES_1} [...] \bnt{M_NMODES_NMODES}
<comments; the modal mass matrix in column-major (symmetric?)>
** \kw{RECORD GROUP 10},<any comment to EOL; "MODAL STIFFNESS MATRIX. COLUMN-MAJOR FORM">
\bnt{K_1_1}      [...] \bnt{K_1_NMODES}
[...]
\bnt{K_NMODES_1} [...] \bnt{K_NMODES_NMODES}
<comments; the modal stiffness matrix in column-major (symmetric?)>
** \kw{RECORD GROUP 11},<any comment to EOL; "DIAGONAL OF LUMPED MASS MATRIX">
\bnt{M_1_X} \bnt{M_1_Y} \bnt{M_1_Z} \bnt{M_1_RX} \bnt{M_1_RY} \bnt{M_1_RZ}
[...]
\bnt{M_NNODES_X} [...] \bnt{M_NNODES_RZ}
<comments; the lumped diagonal mass matrix of the FEM model>
** \kw{RECORD GROUP 12},<any comment to EOL; "GLOBAL INERTIA PROPERTIES">
\bnt{MASS}
\bnt{X_CG} \bnt{Y_CG} \bnt{Z_CG}
\bnt{J_XX} \bnt{J_XY} \bnt{J_XZ}
\bnt{J_YX} \bnt{J_YY} \bnt{J_YZ}
\bnt{J_ZX} \bnt{J_ZY} \bnt{J_ZZ}
<comments; the global inertia properties of the modal element>
** \kw{RECORD GROUP 13},<any comment to EOL; "MODAL DAMPING MATRIX. COLUMN-MAJOR FORM">
\bnt{C_1_1}      [...] \bnt{C_1_NMODES}
[...]
\bnt{C_NMODES_1} [...] \bnt{C_NMODES_NMODES}
<comments; the modal damping matrix in column-major (symmetric?)>
\end{Verbatim}
%\end{verbatim}
}

An arbitrary number of comment lines may appear
where \texttt{<comments[...]>} is used;
only one comment line must appear where
where \texttt{<comment[...]>} is used.

The beginning of a record is marked
%\begin{verbatim}
\begin{Verbatim}[commandchars=\\\{\}]
** \kw{RECORD GROUP} \bnt{RID}
\end{Verbatim}
%\end{verbatim}
where the number \nt{RID} indicates what record is being read.
The size of each record, i.e.\ the number of values that are expected,
is defined based on the header record, so MBDyn should be able to detect
incomplete or mis-formatted files.

The records contain:
\begin{itemize}
\item \kw{RECORD GROUP 1}, a.k.a.\ the ``header'', contains a summary
of the contents of the file:
	\begin{itemize}
	\item \nt{REV} is a string that indicates the revision number;
	it is currently ignored;
	\item \nt{NNODES} is the number of (exposed) FEM nodes 
	in the FEM model;
	\item \nt{NNORMAL} is the number of normal modes;
	\item \nt{NATTACHED} is the number of ``attached'',
	i.e.\ static, modes;
	\item \nt{NCONSTRAINT} is the number of constraint modes;
	\item \nt{NREJECTED} is the number of rejected modes.
	\end{itemize}
Currently, the number of available modes is computed as
\begin{displaymath}
	\nt{NMODES} = \nt{NNORMAL} + \nt{NATTACHED}
		+ \nt{NCONSTRAINT} - \nt{NREJECTED}
\end{displaymath}
because modes are treated in a generalized manner,
so there is no need to consider the different types of shapes
in a specific manner.
Typically, one should set all numbers to 0, except for
\nt{NNORMAL} which should be set to the number of modes
actually present in the data set.
Remember that MBDyn can still select a subset of the available 
modes to be used in the analysis, so that there is no need 
to further edit this file.

\item \kw{RECORD GROUP 2} contains a listing of the \nt{NNODES} labels
of the (exposed) FEM nodes.
The labels can be any string of text, and are separated by blanks
(as intended by the \texttt{isspace(3)} C standard library function).
A label cannot thus contain any amount of whitespace.

\item \kw{RECORD GROUP 3} contains the initial values of the \nt{NMODES}
modal unknowns  (optional; set to zero if omitted);

\item \kw{RECORD GROUP 4} contains the initial values of the \nt{NMODES}
modal unknowns derivatives (optional; set to zero if omitted);

\item \kw{RECORD GROUP 5} contains the $X$ component of the position
of the \nt{NNODES} FEM nodes in the reference frame attached 
to the \kw{modal} node (or to the \kw{origin node}, if given).

\item \kw{RECORD GROUP 6} contains the $Y$ component of the data above;

\item \kw{RECORD GROUP 7} contains the $Z$ component of the data above;

\item \kw{RECORD GROUP 8} contains the non-orthogonalized components 
of the \nt{NMODES} modes; for each mode, the three components 
$X$, $Y$, $Z$ of the modal displacement, and the three components
$RX$, $RY$, $RZ$ of the linearized modal rotations are listed;
each mode shape is separated by a comment line, which typically is
\begin{verbatim}
**    NORMAL MODE SHAPE #  <N>
\end{verbatim}
for readability;

\item \kw{RECORD GROUP 9} contains the modal mass matrix,
i.e.\ a square, $\nt{NMODES} \times \nt{NMODES}$ matrix that contains
the reduced mass $\sqbr{m}$ resulting from the multiplication
\begin{equation}
	\label{eq:modal-mass-constraint}
	\sqbr{m} = \cubr{X}^T \sqbr{M} \cubr{X}
\end{equation}
When only normal modes are used, it is diagonal.
It can be semi-definite positive, or even zero, if a partially 
or entirely static model is considered.

\item \kw{RECORD GROUP 10} contains the modal stiffness matrix,
i.e.\ a square, $\nt{NMODES} \times \nt{NMODES}$ matrix that contains
the reduced stiffness $\sqbr{k}$ resulting from the multiplication
\begin{equation}
	\sqbr{k} = \cubr{X}^T \sqbr{K} \cubr{X}
\end{equation}
When only normal modes are used, it is diagonal; in that case,
the diagonal contains the modal mass times the square 
of the eigenvalues, i.e.\ $k_{ii} = \omega_i^2 m_{ii}$.
It should be definite positive; in fact, rigid degrees of freedom
that would make it semi-definite should rather be modeled by combining
separate modal submodels by way of multibody connections, so that
the multibody capability to handle finite relative displacements
and rotations is exploited.
Note however that the positive-definiteness of the generalized
stiffness matrix is not a requirement.

\item \kw{RECORD GROUP 11} contains the lumped inertia matrix
associated to the \nt{NNODES} (exposed) FEM nodes;
for each node, the $X$, $Y$, $Z$, $RX$, $RY$ and $RY$ inertia
coefficients are listed (they can be zero).
The nodal mass coefficients $Y$ and $Z$ should be equal
to coefficient $X$.
The resulting diagonal matrix should satisfy the constraint
illustrated in Equation~\ref{eq:modal-mass-constraint}.

\item \kw{RECORD GROUP 12} contains the global inertia properties
of the modal element: the total mass, the three components 
of the position of the center of mass in the local frame
of the FEM model, and the $3\times{3}$ inertia matrix
with respect to the center of mass, which is supposed
to be symmetric and positive definite (or semi-definite,
if the model is static).

\item \kw{RECORD GROUP 13} contains the modal damping matrix,
i.e.\ a square, $\nt{NMODES} \times \nt{NMODES}$ matrix that contains
the reduced damping $\sqbr{c}$ resulting from the multiplication
\begin{equation}
	\sqbr{c} = \cubr{X}^T \sqbr{C} \cubr{X}
\end{equation}
It can be semi-definite positive, or even zero, if no damping is considered.
This record is optional.
If given, it is used in the analysis unless overridden by any damping
specification indicated in the modal joint element definition.

\end{itemize}

Note that \kw{RECORD GROUP 11} and \kw{RECORD GROUP 12}
used to be mutually exclusive in earlier versions.
The reason for accepting \kw{RECORD GROUP 12} format,
regardless of \kw{RECORD GROUP 11} presence, is related
to the fact that in many applications the FEM nodal inertia
may not available and, at the same time, a zero-order approximation
of the inertia is acceptable.
The reason for allowing both is that when \kw{RECORD GROUP 11}
is present, its data used to be used to compute the rigid-body
motion invariants that can be provided by \kw{RECORD GROUP 12}.
This might represent an unacceptable approximation.
\kw{RECORD GROUP 12} allows to replace those invariants
by a better estimate, when available.

Both \kw{RECORD GROUP 11} and \kw{RECORD GROUP 12} may be absent.
This case only makes sense when a zero-order approximation
of the inertia is acceptable and the rigid-body motion
of the modal element is not allowed, i.e.\ the element is clamped
to the ground (see the keyword \kw{clamped}
in Section~\ref{sec:EL:STRUCT:JOINT:MODAL} for details).


Although the format loosely requires that no more than 6 numerical values 
appear on a single line, MBDyn is very forgiving about this and can parse
the input regardless of the formatting within each record.
However this liberality may lead to inconsistent or unexpected parsing
behavior, so be careful.



\subsection{Example: Dynamic Model}
\label{sec:APP:EL:STRUCT:JOINT:MODAL:DYNAMIC-MODEL}
As an example, a very simple, hand-made FEM model file is presented below.
It models a FEM model made of three aligned nodes, where inertia 
is evenly distributed.
Note that each line is prefixed with a two-digit line number 
that is not part of the input file.
Also, for readability, all comments are prefixed by ``\texttt{**}'', in analogy
with the mandatory ``\texttt{** RECORD GROUP}'' lines, although not strictly
required by the format of the file.

{\small
\begin{verbatim}
01  ** MBDyn MODAL DATA FILE
02  ** NODE SET "ALL" 
03    
04    
05  ** RECORD GROUP 1, HEADER
06  **   REVISION,  NODE,  NORMAL, ATTACHMENT, CONSTRAINT, REJECTED MODES.
07        REV0         3         2         0         0         0
08  **
09  ** RECORD GROUP 2, FINITE ELEMENT NODE LIST
10       1001 1002 1003
11  
12  **
13  ** RECORD GROUP 3, INITIAL MODAL DISPLACEMENTS
14   0 0
15  **
16  ** RECORD GROUP 4, INITIAL MODALVELOCITIES
17   0 0
18  **
19  ** RECORD GROUP 5, NODAL X COORDINATES
20   0
21   0
22   0
23  **
24  ** RECORD GROUP 6, NODAL Y COORDINATES
25  -2.
26   0
27   2.
28  **
29  ** RECORD GROUP 7, NODAL Z COORDINATES
30   0
31   0
32   0
33  **
34  ** RECORD GROUP 8, MODE SHAPES
35  **    NORMAL MODE SHAPE #  1
36  0 0 1 0 0 0
37  0 0 0 0 0 0
38  0 0 1 0 0 0
39  **    NORMAL MODE SHAPE #  2
40  1 0 0 0 0 0
41  0 0 0 0 0 0
42  1 0 0 0 0 0
43  **
44  ** RECORD GROUP 9, MODAL MASS MATRIX
45  2 0
46  0 2
47  **
48  ** RECORD GROUP 10, MODAL STIFFNESS MATRIX
49  1 0
50  0 1e2
51  **
52  ** RECORD GROUP 11, DIAGONAL OF LUMPED MASS MATRIX
53  1 1 1 1 1 1
54  1 1 1 1 1 1
55  1 1 1 1 1 1
\end{verbatim}
}

The corresponding global inertia properties are:
\begin{align}
	m		&= 3 \\
	\T{x}_{CM}	&= \cubr{\cvvect{
		0 \\
		0 \\
		0
	}} \\
	\T{J}		&= \sqbr{\matr{ccc}{
		11 &  0 &  0 \\
		 0 &  3 &  0 \\
		 0 &  0 & 11
	}}
\end{align}


\subsection{Example: Static Model}
As an example, a very simple, hand-made FEM model file is presented below.
It models a FEM model made of three aligned nodes, where inertia 
is only associated to the mid-node.
As a consequence, the three mode shapes must be interpreted as static
shapes, since the modal mass matrix is null.
Note that each line is prefixed with a two-digit line number 
that is not part of the input file.
Also, for readability, all comments are prefixed by ``\texttt{**}'', in analogy
with the mandatory ``\texttt{** }\kw{RECORD GROUP}'' lines, although not strictly 
required by the format of the file.

\noindent
{\small
\begin{verbatim}
01  ** MBDyn MODAL DATA FILE
02  ** NODE SET "ALL" 
03    
04    
05  ** RECORD GROUP 1, HEADER
06  **   REVISION,  NODE,  NORMAL, ATTACHMENT, CONSTRAINT, REJECTED MODES.
07        REV0         3         2         0         0         0
08  **
09  ** RECORD GROUP 2, FINITE ELEMENT NODE LIST
10       1001 1002 1003
11  
12  **
13  ** RECORD GROUP 3, INITIAL MODAL DISPLACEMENTS
14   0 0
15  **
16  ** RECORD GROUP 4, INITIAL MODALVELOCITIES
17   0 0
18  **
19  ** RECORD GROUP 5, NODAL X COORDINATES
20   0
21   0
22   0
23  **
24  ** RECORD GROUP 6, NODAL Y COORDINATES
25  -2.
26   0
27   2.
28  **
29  ** RECORD GROUP 7, NODAL Z COORDINATES
30   0
31   0
32   0
33  **
34  ** RECORD GROUP 8, MODE SHAPES
35  **    NORMAL MODE SHAPE #  1
36  0 0 1 0 0 0
37  0 0 0 0 0 0
38  0 0 1 0 0 0
39  **    NORMAL MODE SHAPE #  2
40  1 0 0 0 0 0
41  0 0 0 0 0 0
42  1 0 0 0 0 0
43  **
44  ** RECORD GROUP 9, MODAL MASS MATRIX
45  0 0
46  0 0
47  **
48  ** RECORD GROUP 10, MODAL STIFFNESS MATRIX
49  1 0
50  0 1e2
51  **
52  ** RECORD GROUP 12, GLOBAL INERTIA
53   3
54   0  0  0
55  11  0  0
56   0  3  0
57   0  0 11
\end{verbatim}
}

The same global inertia properties of the model 
in Section~\ref{sec:APP:EL:STRUCT:JOINT:MODAL:DYNAMIC-MODEL} have been used;
as a result, the two models show the same rigid body dynamics behavior,
but the dynamic model also shows deformable body dynamic behavior, while
the static one only behaves statically when straining is involved.





\section{Code Aster Procedure}
\label{sec:APP:EL:STRUCT:JOINT:MODAL:ASTER}

The \nt{fem\_data\_file} can be generated by Code Aster, using
the macro provided in the \texttt{cms.py} file, 
which is located in directory \texttt{contrib/CodeAster/cms/} 
of the distribution, and is installed in directory
\texttt{\$PREFIX/share/py/}.

Preparing the input for Code Aster essentially requires to prepare
the bulk of the mesh either manually or using some meshing tools
(e.g.\ \texttt{gmsh}), as illustrated in the related documentation
and tutorials, and then writing a simple script in \texttt{python}
(\htmladdnormallink{\texttt{http://www.python.org/}}{http://www.python.org/})
that defines the execution procedure.

To produce the data for MBDyn's modal element, a specific macro 
needs to be invoked during the execution of Aster.
The macro instructs Aster about how the solution procedure
needs to be tailored to produce the desired results, and then
generates the \texttt{.fem} file with all the requested data.

The steps of the procedure are as follows:
\begin{enumerate} % procedure

\item prepare a Code Aster input model (the \texttt{.mail} file),
containing the nodes, the connections and at least a few groups
of nodes, as indicated in the following;

\item prepare a Code Aster command file (the \texttt{.comm} file),
following, for example, one of the given templates;

\item as soon as the model, the materials, the element properties
and any optional boundary constrains are defined,
call the \kw{CMS} macro with the appropriate parameters.
The syntax of the macro is:
%\begin{verbatim}
\begin{Verbatim}[commandchars=\\\{\}]
    \kw{CMS}(
        \kw{MAILLAGE} = \bnt{mesh_concept} ,
        [ \kw{INTERFACE} = \bnt{interface_node_set_name} , ]
        \kw{EXPOSED} = _F( \{ \kw{GROUP_NO} = \bnt{exposed_node_set_name} | \kw{TOUT} = \kw{'OUI'} \} ),
        \kw{MODELE} = \bnt{model_concept} ,
        \kw{CARA_ELEM} = \bnt{element_properties_concept} ,
        \kw{CHAM_MATER} = \bnt{material_properties_concept} ,
        [ \kw{CHAR_MECA} = \bnt{boundary_conditions_concept} ,
        [ \kw{OPTIONS} = _F(
            [ \kw{NMAX_FREQ} = \bnt{maximum_number_of_frequencies} ]
        ), ]
        \kw{OUT} = _F(
            \kw{TYPE} = \{ \kw{'MBDYN'} \} ,
                [ \kw{PRECISION} = \bnt{number_of_digits} , ]
                [ \kw{DIAG_MASS} = \{ \kw{'OUI'} | \kw{'NON'} \} , ]
                [ \kw{RIGB_MASS} = \{ \kw{'OUI'} | \kw{'NON'} \} , ]
                \kw{FICHIER} = \bnt{output_file_name}
        )
    );
\end{Verbatim}
%\end{verbatim}
where:
\begin{itemize}
\item \kw{MAILLAGE} is the name of a python object
	returned by \kw{LIRE\_MAILLAGE};
\item \kw{INTERFACE} is a string containing the name of the group of nodes,
	defined in the mesh file, that will be used as interface
	to add Craig-Bampton shapes to the base
	(if omitted, only normal modes are used);
\item \kw{EXPOSED} allows two formats:
	\begin{itemize}
	\item \kw{GROUP\_NO} is a string containing the name
		of the group of nodes, defined in the mesh file,
		that will be exposed, i.e.\ that will appear
		in the \texttt{.fem} file generated by this macro;
	\item \kw{TOUT} is a string that must be set to \kw{'OUI'},
		indicating that all nodes will be exposed
		(technically speaking, all nodes involved
		in at least one connection);
	\end{itemize}
\item \kw{MODELE} is the name of a python object
	returned by \kw{AFFE\_MODELE};
\item \kw{CARA\_ELEM} is the name of a python object
	returned by \kw{AFFE\_CARA\_ELEM};
\item \kw{CHAM\_MATER} is the name of a python object
	returned by \kw{AFFE\_MATERIAU};
\item \kw{CHAR\_MECA} is the name of a python object
	returned by \kw{AFFE\_CHAR\_MECA} (optional);
\item \kw{OPTIONS} allows a set of generic options to be passed:
	\begin{itemize}
	\item \kw{NMAX\_FREQ} is the number of frequencies
		expected from the eigenanalysis, and thus the number
		of normal modes that will be added to the modal base;
		the smallest frequencies are used;
	\end{itemize}
\item \kw{OUT} describes the parameters related to the type of output
	the macro generates; right now, only output in the format 
	specified in this section is available:
	\begin{itemize}
	\item \kw{TYPE} indicates the type of output the macro
		is requested to generate; it must be \kw{'MBDYN'};
	\item \kw{FICHIER} (``file'' in French) is the name
		of the file where the generated data must be written;
		it must be a fully qualified path, otherwise
		it is not clear where the file is actually generated;
	\item \kw{PRECISION} contains the number of digits to be used
		when writing the data (for example, \kw{PRECISION}\texttt{ = 8}
		results in using the format specifier \texttt{\%16.8e});
	\item \kw{DIAG\_MASS} requests the output of the diagonal
		of the mass matrix (\texttt{RECORD GROUP 11});
		this option really makes
		sense only if all nodes are output (namely, if the nodes
		set passed to \kw{EXPOSED} consists in all nodes);
	\item \kw{DIAG\_MASS} requests the output of the rigid-body
		inertia properties (\texttt{RECORD GROUP 12});
	\end{itemize}
\end{itemize}
the macro returns a python object consisting in the generalized model
it used to generate the file, as returned by \kw{MACR\_ELEM\_DYNA}.

\item prepare a \texttt{.export} file, either following the examples
or using \texttt{astk};

\item run Aster.

\end{enumerate} % procedure
Annotated examples are provided in directory \texttt{contrib/CodeAster/cms/}.



\section{NASTRAN Procedure}
\label{sec:APP:EL:STRUCT:JOINT:MODAL:NASTRAN}

The \nt{fem\_data\_file} can be generated using NASTRAN,
by means of the \kw{ALTER} cards provided in directory \texttt{etc/modal.d/}
of the distribution.

The steps of the procedure are as follows:
\begin{enumerate} % procedure

\item prepare a NASTRAN input card deck, made of bulk data,
eigenanalysis properties data and/or static loads (details about 
this phase are currently missing and will be provided in future 
releases of the manual).

\item complete the NASTRAN input file by putting some specific
\kw{ALTER} cards.
In detail:
\begin{enumerate}
\item the file \texttt{MBDyn\_NASTRAN\_alter\_1.nas} contains
	\kw{ALTER} definitions for static solutions; appropriate
	loading subcases for each solution must be provided
	in the case control and in the bulk data sections
	of the input file;

	\emph{FIXME: I don't know how to use static shapes only}.

\item the file \texttt{MBDyn\_NASTRAN\_alter\_2.nas} contains
	\kw{ALTER} definitions for eigenanalysis solutions;
	an appropriate eigenanalysis method, with the related
	data card must be provided in the case control 
	and in the bulk data sections of the input file;

	\emph{FIXME: I don't know how to use this together
		with static shapes; I only get the normal mode shapes,
		even if the matrices are complete}.

\item the file \texttt{MBDyn\_NASTRAN\_alter\_3.nas} contains
	\kw{ALTER} definitions for eigenanalysis solutions;
	an appropriate eigenanalysis method, with the related
	data card must be provided in the case control 
	and in the bulk data sections of the input file;

	\emph{Note: this works; see} \texttt{tests/modal/beam.README}.

\end{enumerate}
Exactly one of these files must be included at the very top 
of the NASTRAN input file;
they already include the appropriate \kw{SOL} statement, so the
input file must begin with
\begin{verbatim}
$ Replace '#' below with number that matches your needs
INCLUDE 'MBDyn_NASTRAN_alter_#.nas'
CEND
$... any other executive control and bulk data card
\end{verbatim}
The static solution of case (a: \kw{SOL 101}) and the eigensolution
of case (b: \kw{SOL 103}) need to be performed in sequence; 
if only the eigensolution is to be used, the \kw{ALTER} file 
of case (c: \kw{SOL 103}) must be used.
The static solution of case (a) generates a binary file \texttt{mbdyn.stm};
the eigensolutions of cases (b--c) generate two binary files, 
\texttt{mbdyn.mat} and \texttt{mbdyn.tab}, which, in case (b), include
the static solutions as well.
The \kw{ALTER} currently included in the MBDyn distribution work 
correctly only with the following \kw{PARAM} data card:
\begin{verbatim}
PARAM,POST,-1 
\end{verbatim}


\item Run NASTRAN.


\item Run the tool \texttt{femgen}, which transforms the above binary files
into the \nt{fem\_data\_file}.
The file name is currently requested as terminal input; this is the name 
of the file that will be used in the input model for \texttt{mbdyn}.
Conventionally, the \texttt{.fem} extension is used.

\end{enumerate} % procedure



\section{Procedures for Other FEA Software}
Currently, no (semi-)automatic procedures are available to extract modal
element data from other FEA software.
If you need to use any other software, and possibly a free software,
you'll need to design your own procedure, which may go from a very simple
script (shell, awk, python) for handling of textual output, to a fully
featured modeling and translation interface.
If you want to share what you developed with the MBDyn project, feel free
to submit code, procedures or just ideas and discuss them on the
\htmladdnormallink{\texttt{mbdyn-users@mbdyn.org}}{mailto:mbdyn-users@mbdyn.org}
mailing list.

