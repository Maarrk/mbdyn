% $Header$
% MBDyn (C) is a multibody analysis code.
% http://www.mbdyn.org
%
% Copyright (C) 1996-2008
%
% Pierangelo Masarati  <masarati@aero.polimi.it>
%
% Dipartimento di Ingegneria Aerospaziale - Politecnico di Milano
% via La Masa, 34 - 20156 Milano, Italy
% http://www.aero.polimi.it
%
% Changing this copyright notice is forbidden.
%
% This program is free software; you can redistribute it and/or modify
% it under the terms of the GNU General Public License as published by
% the Free Software Foundation (version 2 of the License).
% 
%
% This program is distributed in the hope that it will be useful,
% but WITHOUT ANY WARRANTY; without even the implied warranty of
% MERCHANTABILITY or FITNESS FOR A PARTICULAR PURPOSE.  See the
% GNU General Public License for more details.
%
% You should have received a copy of the GNU General Public License
% along with this program; if not, write to the Free Software
% Foundation, Inc., 59 Temple Place, Suite 330, Boston, MA  02111-1307  USA

\chapter{Nodes}\label{sec:NODES}
The \kw{nodes} section is enclosed in the cards:
\begin{verbatim}
    begin: nodes;
        ...
    end: nodes;
\end{verbatim}
Every node card has the following format:
\begin{verbatim}
    <card> :: = <node_type> : <node_label> ,
        <arglist>
        [ , output , { yes | no } ]
        [ , <extra arglist> ] ;
\end{verbatim}
where \kw{node\_type} is one of the following:
\begin{itemize}
    \item \kw{structural}
    \item \kw{electric}
    \item \kw{abstract}
    \item \kw{parameter}
    \item \kw{hydraulic}
\end{itemize}
The data manager reads the node type and the label and checks for
duplication. If the node is not defined yet, the proper read function is
called, which parses the rest of the card and constructs the node.
The nodes are read as follows:




\section{Structural Node}\label{sec:NODE:STRUCTURAL}
The historical 6 dof structural node can be \kw{static},
\kw{dynamic}, \kw{modal} or \kw{dummy}.
The \kw{static} keyword means no inertia is related to that node, 
so it must be appropriately constrained or attached to elastic elements.
Static nodes are useful when there is no need to apply inertia
to them, thus saving 6 degrees of freedom.
The \kw{dynamic} keyword means inertia can be attached to the node, 
so it provides linear and angular momenta degrees of freedom, 
and automatically generates the so-called \kw{automatic structural}
elements.
The \kw{modal} node is a regular \kw{dynamic} node that must be used 
to describe the rigid reference motion of a \kw{modal} joint.
See Section~\ref{sec:EL:STRUCT:JOINT:MODAL} for further details.
\begin{verbatim}
    <arglist> ::= { static | dynamic | modal } ,
        (Vec3)              <absolute_position> ,
        (OrientationMatrix) <absolute_orientation_matrix> ,
            [ orientation description , <orientation_description> , ]
        (Vec3)              <absolute_velocity> ,
        (Vec3)              <absolute_angular_velocity>
        [ , assembly
          , (scalar) <position_initial_stiffness>
          , (scalar) <velocity_initial_stiffness>
          , (flag)   <omega_rotates?> ]
    <orientation_description> ::=
        { euler 123 | orientation vector | orientation matrix }
\end{verbatim}
The \kw{omega\_rotates} parameter determines whether 
the initial angular velocity should follow or not the node 
as it is rotated by the initial assembly procedure; it may assume 
the values \kw{yes}/\kw{no} or 0/nonzero.

\noindent
The \kw{dummy} structural node has been added to ease the visualization of
the kinematics of arbitrary points of the system. 
It owns no dofs, and it must be attached to another node, possibly with an
offset and a relative orientation.
Two dummy structural node types are available.
\begin{verbatim}
    <arglist> ::= dummy , <base_node> , <type> , <dummy_node_data>
\end{verbatim}
The stiffness parameters, and the \kw{omega\_rotates} flag, 
override the default values. 
They are optional, but they must be supplied all together if at least
one is to be input. \\
The \kw{omega\_rotates} flag can take values \kw{yes} or \kw{no};
a numerical value of 0 (no) or 1 (yes) is supported for backwards
compatibility, but its use is deprecated.

Only the \kw{dynamic} and the \kw{modal} node types allow
the optional output keyword
\kw{accelerations} after the standard node output parameters:
\begin{verbatim}
    <extra arglist> ::= accelerations
\end{verbatim}
to enable the output of the linear and angular accelerations of the node.
Since this is a postprocessing, it is not required by the regular analysis,
so it should be enabled only when strictly required.
Also note that accelerations may be inaccurate, since they are reconstructed
from the momentum and the momenta moment derivatives through the inertia
associated to the node.
Accelerations output can be controlled by means of the
\hyperref{\kw{default output} statement}{\kw{default output} statement (see Section~}{)}{sec:CONTROLDATA:DEFAULTOUTPUT}.

If the \kw{static} model type is used in the control data block,
all dynamic structural nodes are actually treated as static.
This is a shortcut to ease running static analyses without the need
to modify each node of a dynamic model.

Dummy nodes take the label of the \kw{base\_node} they are attached to, 
followed by the type of dummy node, possibly followed by specific data.
The following dummy types are available:

\begin{itemize}

    \item offset:
    \begin{verbatim}
    <type> ::= offset
    <dummy_node_data> ::=
        (Vec3)              <relative_offset> ,
        (OrientationMatrix) <relative_orientation_matrix>
            [ orientation description , <orientation_description> , ]
    \end{verbatim}
    It outputs the configuration of a point offset from the base node.
            
    \item relative frame:
    \begin{verbatim}
    <type> ::= relative frame
    <dummy_node_data> ::= <reference_node>
        [ , position , <reference_offset> ]
        [ , orientation , <reference_orientation_matrix> ]
            [ orientation description , <orientation_description> , ]
        [ , pivot node , <pivot_node>
            [ position , <pivot_offset> ]
            [ orientation , <pivot_orientation_matrix> ] ]
    \end{verbatim}
    It outputs the configuration of the base node in the frame defined
    by the \kw{reference\_node}, optionally offset 
    by \kw{reference\_offset} and with relative orientation 
    \kw{reference\_orientation\_matrix}.

    If a \kw{pivot\_node} is given, the relative frame motion
    is transformed as if it were expressed in the reference frame
    of the pivot node, optionally offset by \kw{pivot\_offset}
    and with relative orientation \kw{pivot\_orientation\_matrix}.

    Examples:
    \begin{verbatim}
    set: real Omega = 1.;
    structural: 1, static, null, eye, null, 0.,0.,Omega;
    structural: 1000, dummy, 1, offset, 1.,0.,0., eye;
    structural: 1001, dummy, 1, relative frame, 1000;
    structural: 2000, dynamic,
        0.,0.,1.,
        1, 1.,0.,0., 2, 0.,0.,1.,
        null,
        null,
        accelerations;
    \end{verbatim}

\end{itemize}

\paragraph{Output}
\label{sec:NODE:STRUCTURAL:OUTPUT}
Structural nodes generate two kinds of output files. 
The first refers to the kinematics of the node; its extension is \kw{.mov},
and for each time step it contains one row for each node whose output is
required.
The rows contain: \vspace{2mm} \\
\begin{tabular}{lp{140mm}}
	\hline
	1      & the label of the node \\
	2--4   & the three components of the position of the node \\
	5--7   & the three Euler angles that define the orientation of the node \\
	8--10  & the three components of the velocity of the node \\
	11--13 & the three components of the angular velocity of the node \\
	\hline
	14--16 & the three components of the linear acceleration
		of the \kw{dynamic} nodes (optional) \\
	17--19 & the three components of the angular acceleration
		of the \kw{dynamic} nodes (optional) \\
	\hline
\end{tabular}\vspace{2mm}\\
All the quantities are expressed in the global frame, except for
the dummy \kw{relative frame} node type, whose quantities are,
by definition, in the relative frame.

\emph{Note: actually, the angles denoted as ``Euler angles'' 
are the three angles that describe a rotation made of a sequence
of three steps: first, a rotation about global axis 1,
followed by a rotation about axis 2 of the frame resulting from
the previous rotation, concluded by a rotation about axis 3
of the frame resulting from the two previous rotations.
To consistently transform this set of parameters into some other
representation, see the tools
\kw{eu2rot(1)}, \kw{rot2eu(1)}, \kw{rot2eup(1)}, \kw{rot2phi(1)}.
The functions that compute the relationship between an orientation
matrix and the set of three angles and vice versa are
\kw{MatR2EulerAngles()} and \kw{EulerAngles2MatR()}, in \kw{matvec3.h}.
}

\noindent
The second output file refers only to dynamic nodes, and contains their
inertia; its extension is \kw{.ine}.
For each time step, it contains information about the inertia of all the
nodes whose output is required.
Notice that more than one inertia body can be attached to one node; the
information in this file refers to the sum of all the inertia referring to
the node.
The rows contain: \vspace{2mm} \\
\begin{tabular}{lp{140mm}}
        \hline
	1	& the label of the node \\
	2--4	& item the three components of the momentum
		in the absolute reference frame \\
	5--7	& item the three components of the momenta moment
		in the absolute reference frame,
		with respect to the coordinates of the node, 
		thus to a moving frame \\
    	8--10	& the three components of the derivative of the momentum \\
    	11--13	& the three components of the derivative of the momentum moment \\
	\hline
\end{tabular}\vspace{2mm}\\

\paragraph{Private Data}
\label{sec:NODE:STRUCTURAL:PRIV}
The following data is available:
\begin{enumerate}
\item[] (all structural nodes)
\item \kw{"X[1]"} position in global direction 1
\item \kw{"X[2]"} position in global direction 2
\item \kw{"X[3]"} position in global direction 3
\item \kw{"x[1]"} position in direction 1, in the reference frame of the node
\item \kw{"x[2]"} position in direction 2, in the reference frame of the node
\item \kw{"x[3]"} position in direction 3, in the reference frame of the node
\item \kw{"Phi[1]"} orientation vector in global direction 1
\item \kw{"Phi[2]"} orientation vector in global direction 2
\item \kw{"Phi[3]"} orientation vector in global direction 3
\item \kw{"XP[1]"} velocity in global direction 1
\item \kw{"XP[2]"} velocity in global direction 2
\item \kw{"XP[3]"} velocity in global direction 3
\item \kw{"xP[1]"} velocity in direction 1, in the reference frame of the node
\item \kw{"xP[2]"} velocity in direction 2, in the reference frame of the node
\item \kw{"xP[3]"} velocity in direction 3, in the reference frame of the node
\item \kw{"Omega[1]"} angular velocity in global direction 1
\item \kw{"Omega[2]"} angular velocity in global direction 2
\item \kw{"Omega[3]"} angular velocity in global direction 3
\item \kw{"omega[1]"} angular velocity in direction 1, in the reference frame of the node
\item \kw{"omega[2]"} angular velocity in direction 2, in the reference frame of the node
\item \kw{"omega[3]"} angular velocity in direction 3, in the reference frame of the node
\item \kw{"E[1]"} Cardan angle 1 (about global direction 1)
\item \kw{"E[2]"} Cardan angle 2 (about global direction 2)
\item \kw{"E[3]"} Cardan angle 3 (about global direction 3)
\item \kw{"PE[0]"} Euler parameter 0
\item \kw{"PE[1]"} Euler parameter 1
\item \kw{"PE[2]"} Euler parameter 2
\item \kw{"PE[3]"} Euler parameter 3
\item[] (dynamic nodes only)
\item \kw{"XPP[1]"} acceleration in global direction 1
\item \kw{"XPP[2]"} acceleration in global direction 2
\item \kw{"XPP[3]"} acceleration in global direction 3
\item \kw{"xPP[1]"} acceleration in direction 1, in the reference frame of the node
\item \kw{"xPP[2]"} acceleration in direction 2, in the reference frame of the node
\item \kw{"xPP[3]"} acceleration in direction 3, in the reference frame of the node
\item \kw{"OmegaP[1]"} angular acceleration in global direction 1
\item \kw{"OmegaP[2]"} angular acceleration in global direction 2
\item \kw{"OmegaP[3]"} angular acceleration in global direction 3
\item \kw{"omegaP[1]"} angular acceleration in direction 1, in the reference frame of the node
\item \kw{"omegaP[2]"} angular acceleration in direction 2, in the reference frame of the node
\item \kw{"omegaP[3]"} angular acceleration in direction 3, in the reference frame of the node
\end{enumerate}

Note: Euler parameters actually do not take into account 
the whole orientation of a node, since they are post-processed
from the orientation matrix.
As a consequence, they only parametrize the minimum norm orientation
that yields the current orientation matrix of the node.
The same applies to the orientation vector $\varphi$.

Note: if accelerations are requested using the \kw{string} form,
their computation is enabled even if it was not explicitly enabled
when the node was instantiated.
However, if the \kw{index} form is used, their computation must have
already been explicitly enabled.


\section{Electric Node}
\begin{verbatim}
    <arglist> ::= value , <initial_value> 
        [ , derivative , <derivative_initial_value> ]
\end{verbatim}
\emph{Note: the keywords \kw{value} and \kw{derivative}
have been introduced recently; \kw{value} is not mandatory,
resulting in a warning while \kw{derivative} is required.
The same applies to the \kw{abstract node} 
and to the \kw{hydraulic node}; the latter is an algebraic
node, so only \kw{value} is allowed.
}





\section{Abstract Node}
\begin{verbatim}
    <arglist> ::= { algebraic | differential } ,
        value , <initial_value>
        [ , derivative , <derivative_initial_value> ]
\end{verbatim}
\emph{
	Note: abstract nodes are ancestors of all scalar node types.
	Many \kw{genel} and \kw{electric} elements can be connected
	to \kw{abstract} nodes as well, since they directly use
	the ancestor class. 
}

\paragraph{Output}
The value of abstract nodes is output with file extension \kw{.abs}; for
each time step the output of the required nodes is written.
The format of each row is
\begin{itemize}
    \item the label of the node
    \item the value of the node
    \item the value of the node derivative, when \kw{differential}
\end{itemize}



\section{Hydraulic Node}
\begin{verbatim}
    <arglist> ::= value , <initial_value>
\end{verbatim}



\section{Parameter Node}
\begin{verbatim}
    <arglist> ::= { <initial_value>
        | element 
        | sample and hold, (node_dof) <signal> , <sample_time>
        | beam strain gage , <y> , <z> }
\end{verbatim}
The parameter node is derived from the class scalar algebraic node, but it
is used in a rather peculiar way: it doesn't own any degree of freedom,
so it does not participate in the solution, but is used as a sort of
placeholder for those elements that require to be connected to a scalar node
that is not otherwise significant to the analysis.
Thanks to the availability of the \kw{parameter} node, these elements
do not need be reformulated with a grounded node, while the parameter
node value can be changed during the solution by means of a proper driving 
force.

\paragraph{Element}
When the argument list starts with the keyword \kw{element}, the parameter
node expects to be bound to an element, and to access bulk element data 
(see the \hyperref{\kwnd{bind} statement}{\kw{bind} statement, Section~}{}{sec:EL:BIND}).

\paragraph{Sample and Hold}
When the argument list starts with the keyword \kw{sample and hold},
followed by a \kw{NodeDof} and a sample time,
the parameter node contains the value of the input signal, namely
the value of the node, for the duration of the \kw{sample\_time}.
This may be useful to preserve the value of some signal across
time steps.

\paragraph{Beam Strain Gage}
When the argument list starts with the keyword \kw{beam strain gage},
followed by the coordinates of a point on the section of a beam,
the \kw{parameter} node expects to be bound to a \kw{beam} element,
and to access the measure of the axial strain at point \kw{x}, \kw{y}
in the section plane as a combination of section strains and curvatures:
\begin{equation}
	\varepsilon = \nu_x
		+ \mathtt{z} \cdot \kappa_y 
		- \mathtt{y} \cdot \kappa_z ,
\end{equation}
where
\begin{itemize}
\item $\nu_x$ is the axial strain of the beam;
\item $\kappa_y$ is the bending curvature of the beam about the $y$ axis;
\item $\kappa_z$ is the bending curvature of the beam about the $z$ axis.
\end{itemize}
The span-wise location of the point where the strain is evaluated
is set in the 
\hyperref{\kwnd{bind} statement}{\kw{bind} statement (see Section~}{)}{sec:EL:BIND}.

\emph{Note: measuring strains by means of derivatives of interpolated
positions and orientations may lead to inaccurate results.}





\section{Miscellaneous}

\subsection{Output}
\label{sec:NODE:MISC:OUTPUT}
There is an extra card, that is used to modify the output behavior of nodes:  
\begin{verbatim}
    <card> ::= output : <node_type> , <node_list> ;
    <node_list> ::= { <node_label>  [ , <node_list> ] 
        | range , <node_start_label> , <node_end_label> }
\end{verbatim}
\kw{node\_item} is a valid node type that can be read in the \kw{nodes}
module.
In case the keyword \kw{range} is used, all existing nodes comprised
between \kw{<node\_start\_label>} and \kw{<node\_end\_label>}
are set; missing ones are silently ignored.

\noindent
{\em
   Note: if a node should never (\kw{no}) or always (\kw{yes}) be output,
   its output flag should be set directly on the node card. 
   The global behavior of all the nodes of a type can be set from the 
   \kw{control data} block by adding the node type to the item list in the 
   \kw{default output} card. 
   Then, the specific output flag of sets of nodes can be altered by means 
   of the \kw{output} card in the \kw{nodes} block. 
   This allows high flexibility in the selection of the desired output. 
   The same remarks apply to the output of the elements.
}

{\em
    Note: for dynamic structural nodes, the optional keyword \kw{accelerations}
    can be used right after the \kw{node\_type} to enable the output
    of the accelerations.
}

