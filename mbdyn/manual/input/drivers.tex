% $Header$
% MBDyn (C) is a multibody analysis code.
% http://www.mbdyn.org
%
% Copyright (C) 1996-2017
%
% Pierangelo Masarati  <masarati@aero.polimi.it>
%
% Dipartimento di Ingegneria Aerospaziale - Politecnico di Milano
% via La Masa, 34 - 20156 Milano, Italy
% http://www.aero.polimi.it
%
% Changing this copyright notice is forbidden.
%
% This program is free software; you can redistribute it and/or modify
% it under the terms of the GNU General Public License as published by
% the Free Software Foundation (version 2 of the License).
% 
%
% This program is distributed in the hope that it will be useful,
% but WITHOUT ANY WARRANTY; without even the implied warranty of
% MERCHANTABILITY or FITNESS FOR A PARTICULAR PURPOSE.  See the
% GNU General Public License for more details.
%
% You should have received a copy of the GNU General Public License
% along with this program; if not, write to the Free Software
% Foundation, Inc., 59 Temple Place, Suite 330, Boston, MA  02111-1307  USA

\chapter{Drivers}\label{sec:DRIVERS}
The \kw{drivers} section is enclosed by the cards:
%\begin{verbatim}
\begin{Verbatim}[commandchars=\\\{\}]
    \kw{begin} : \kw{drivers} ;
        # ...
    \kw{end} : \kw{drivers} ;
\end{Verbatim}
%\end{verbatim}
Every \kw{driver} card has the format:
%\begin{verbatim}
\begin{Verbatim}[commandchars=\\\{\}]
    \bnt{card} ::= \bnt{driver_type} : \bnt{arglist} ;
\end{Verbatim}
%\end{verbatim}
At present the only type \nt{driver\_type} of drivers supported is \kw{file}.

\section{File Drivers}
The \kw{file} drivers are defined by the statement
%\begin{verbatim}
\begin{Verbatim}[commandchars=\\\{\}]
    \kw{file} : \bnt{file_arglist} ;
\end{Verbatim}
%\end{verbatim}
A comprehensive family of \kw{file} drivers is available;
their syntax is:
%\begin{verbatim}
\begin{Verbatim}[commandchars=\\\{\}]
    \bnt{file_arglist} ::= \bnt{label} , \bnt{normal_arglist}
\end{Verbatim}
%\end{verbatim}
The following \kw{file} drivers are supported:



\subsection{Fixed Step}
%\begin{verbatim}
\begin{Verbatim}[commandchars=\\\{\}]
    \bnt{normal_arglist} ::= \kw{fixed step} , 
        \{ \kw{count} | \bnt{steps_number} \} ,
        \bnt{columns_number} ,
        \kw{initial time} , \{ \kw{from file} | \bnt{initial_time} \} ,
        \kw{time step} , \{ \kw{from file} | \bnt{time_step} \} ,
        [ \kw{interpolation} , \{ \kw{linear} | \kw{const} \} , ]
        [ \{ \kw{pad zeroes} , \{ \kw{yes} | \kw{no} \}
            | \kw{bailout} , \{ \kw{none} | \kw{upper} | \kw{lower} | \kw{any} \} \} , ]
        " \bnt{file_name} "
\end{Verbatim}
%\end{verbatim}
If the keyword \kw{count} is used instead of the \nt{steps\_number},
the number of time records provided by the file is obtained
by counting the valid records.

If the keyword \kw{from file} is used instead of either 
\nt{initial\_time} or \nt{time\_step},
the value is read from the file.
In this case, comment lines of the form
\begin{verbatim}
# initial time: <initial_time>
# time step: <time_step>
\end{verbatim}
are expected; the corresponding values are parsed and checked.
A valid value provided in the input file always overrides
any corresponding value provided in the corresponding comment line.

The value at an arbitrary time is interpolated from the available data.
If the requested value is out of time bounds, zero is returned,
unless \kw{pad zeroes} is set to \kw{no}, which means that the first
or the last set of values is respectively used.
As an alternative, if \kw{bailout} is set to \kw{upper}, \kw{lower}
or \kw{any}, the simulation is stopped as soon as the time goes out
of the respective ''time`` bounds.

If \kw{interpolation} is \kw{linear} (the default), a linear interpolation
is used.
Otherwise, if \kw{interpolation} is \kw{const}, the value at the beginning
of the time step that includes the current time is used.

The file format is
\begin{verbatim}
# an arbitrary number of comment lines starting with `#'
#
# comment lines may provide special parameters
# like "initial time", "time step" as indicated above; for example
#
# initial time: 0
# time step: 0.01
#
# channel #1 channel #2 ... channel #n
  1.         2.         ... 100.
...
\end{verbatim}


\paragraph{Example.} \
\begin{verbatim}
    begin: drivers;
        file: 100,
            fixed step,
            100,       # n. steps
            1,         # n. channels
            0,         # initial_time
            0.01,      # time_step
            "input.dat";

        file: 200,
            fixed step,
            count,     # n. steps
            1,         # n. channels
            from file, # initial_time
            from file, # time_step
            "input.dat";
    end: drivers;
\end{verbatim}

\subsection{Variable Step}
%\begin{verbatim}
\begin{Verbatim}[commandchars=\\\{\}]
    \bnt{normal_arglist} ::= \kw{variable step} , 
        \bnt{channels_number} ,
        [ \kw{interpolation} , \{ \kw{linear} | \kw{const} \} , ]
        [ \{ \kw{pad zeroes} , \{ \kw{yes} | \kw{no} \} 
            | \kw{bailout} , \{ \kw{none} | \kw{upper} | \kw{lower} | \kw{any} \} , \} , ]
        " \bnt{file_name} "
\end{Verbatim}
%\end{verbatim}
The same considerations of the \kw{fixed step} type apply.

The file format is
\begin{verbatim}
# an arbitrary number of lines starting with `#'
#
# time channel#1 channel#2 ... channel#n
  0.   1.        2.        ... 100.
  0.01 1.2       1.8       ...  90.
...
\end{verbatim}
The number of available records is computed as the number
of non-comment lines (lines not starting with a hash mark `\texttt{\#}').
The first column contains the time, while the remaining
\nt{channels\_number} columns contain the values of the input channels
at that time.
Time values must grow monotonically.


\subsection{Socket}
%\begin{verbatim}
\begin{Verbatim}[commandchars=\\\{\}]
    \bnt{normal_arglist} ::= \kw{socket} , 
        \bnt{columns_number} ,
	[ \kw{initial values} , \bnt{value \#1} , ... , ]
        \{ \kw{local} , " \bnt{file_name} " 
            | [ \kw{port} , \bnt{port_number} , ] (\ty{AuthenticationMethod}) \bnt{authentication} \}
\end{Verbatim}
%\end{verbatim}
The driver binds to a socket listening on port \nt{port\_number} 
(defaults to 9011) or on the named pipe \nt{file\_name}; at the
beginning of each time step, in case of connection, the driver expects some
input data in text format, consisting in an authentication token 
(if required).
The authentication token is usually in the form
%\begin{verbatim}
\begin{Verbatim}[commandchars=\\\{\}]
    user: \bnt{user_name} NEWLINE
    password: \bnt{user_password} NEWLINE
\end{Verbatim}
%\end{verbatim}
where \texttt{NEWLINE} is a literal `newline' (i.e.\ `\verb;\n;').
White-spaces may be significant in \nt{user\_name}, 
and surely are in \nt{user\_password}.

The identification is followed by a label token, in the form
%\begin{verbatim}
\begin{Verbatim}[commandchars=\\\{\}]
    label: \bnt{label} NEWLINE
\end{Verbatim}
%\end{verbatim}
indicating the column being edited, followed by the desired changes; 
the connection is terminated by a single mark followed by a newline:
%\begin{verbatim}
\begin{Verbatim}[commandchars=\\\{\}]
    . NEWLINE
\end{Verbatim}
%\end{verbatim}
The permitted operations, at present, are:
%\begin{verbatim}
\begin{Verbatim}[commandchars=\\\{\}]
    value: \bnt{value} NEWLINE
\end{Verbatim}
%\end{verbatim}
sets the value the drive will assume from the current step on
%\begin{verbatim}
\begin{Verbatim}[commandchars=\\\{\}]
    inc: \{ \kw{yes} | \kw{no} \} NEWLINE
\end{Verbatim}
%\end{verbatim}
tells whether to switch \kw{on} or \kw{off} the increment mode, resulting in
subsequent value commands being ``set'' rather than ``add''
%\begin{verbatim}
\begin{Verbatim}[commandchars=\\\{\}]
    imp: \{ \kw{yes} | \kw{no} \} NEWLINE
\end{Verbatim}
%\end{verbatim}
tells whether to switch \kw{on} or \kw{off} the impulse mode;
when \kw{on}, subsequent value commands to be applied for one step only.
At present, impulse mode supersedes any incremental mode, namely
the value of the drive is reset to zero after one step.
This behavior may change in the future.



\subsection{RTAI Mailbox}\label{sec:RTAI_in}
This special drive is a variant of the Socket Stream
(Section~\ref{sec:Stream}; under development yet)
that reads the input data from a RTAI mailbox in non-blocking mode.
It is intended as a means of communication between different processes
running in real-time.
The syntax is:
%\begin{verbatim}
\begin{Verbatim}[commandchars=\\\{\}]
    \bnt{normal_arglist} ::= \kw{RTAI input} ,
        \kw{stream drive name} , " \bnt{stream_name} " ,
        [ \kw{create} , \{ \kw{yes} | \kw{no} \} , ]
        [ \kw{host} , " \bnt{host_name} " , ]
        [ \{ [ \kw{non} ] \kw{blocking} \} , [ ... ] ]
        \bnt{columns_number} ;
\end{Verbatim}
%\end{verbatim}
where
\begin{itemize}
\item \nt{stream\_name} is the name of the mailbox (a unique string 
no more than 6 characters long);
\item the \kw{create} keyword determines whether the mailbox 
must be created or looked-up as already existing on the system;
\item \nt{host\_name} is the name or the IP of the remote host where 
the mailbox resides; note that if this field is given, \kw{create} must
be set to \kw{no} and the mailbox must be already defined
on the remote host;
\item the number of channels \nt{columns\_number} determines how many
columns can be accessed via the \kw{file} drive caller mechanism 
of Section~\ref{sec:DriveCaller:FILE}, as for all the other \kw{file} drivers.
\end{itemize}
This part of the program is rapidly evolving, so do please not expect
too much documentation and backward compatibility.

\emph{Note: at present, these elements require that the simulation
be run in real-time mode (see Section~\ref{sec:REAL-TIME});
future development will allow to emulate the use of these elements
also when the simulation is not run in real-time, e.g.\ for modeling
or model debugging purposes.}



\subsection{Stream}\label{sec:Stream}
%\begin{verbatim}
\begin{Verbatim}[commandchars=\\\{\}]
    \bnt{normal_arglist} ::= \kw{stream} ,
        \kw{name} , " \bnt{stream_name} " ,
        \kw{create} , \{ \kw{yes} | \kw{no} \} ,
        [ \{ \kw{local} , " \bnt{path} " ,
            | [ \kw{port} , \bnt{port_number} , ] [ \kw{host} , " \bnt{host_name} " , ] \} ]
        [ \{ [ \kw{no} ] \kw{signal}
            | [ \kw{non} ] \kw{blocking} \} , [ ... ] , ]
        [ \kw{input every} , \bnt{steps} , ]
        [ \kw{receive first} , \{ \kw{yes} | \kw{no} \} , ]
        [ \kw{timeout} , \bnt{timeout} , ]
        [ \kw{echo} , " \bnt{echo_file_name} "
            [ , \kw{precision} , \bnt{precision} ]
            [ , \kw{shift} , \bnt{shift} ] , ]
        \bnt{columns_number}
            [ , \kw{initial values} , \bnt{value \#1} , ... ]
            [ , \kw{modifier} , \bnt{content_modifier} ]

    \bnt{content_modifier} ::=
        \kw{copy}
        | \kw{copy cast} ,
            \{ \kw{all} , \bnt{one_cast} | \bnt{one_cast} [...] \}
            [ , \kw{size} , (\ty{int})\bnt{buffer_size} ]

    \bnt{one_cast} ::=
        \kw{int8_t}
        | \kw{uint8_t}
        | \kw{int16_t}
        | \kw{uint16_t}
        | \kw{int32_t}
        | \kw{uint32_t}
        | \kw{float}     # assuming float is 4 bytes
        | \kw{double}    # assuming double is 8 bytes
        | \kw{skip} , (\ty{int})\bnt{num_bytes}
\end{Verbatim}
%\end{verbatim}
The \kw{stream} drive allows MBDyn to receive streamed inputs 
from remote processes both during regular simulations and during 
real-time simulations under RTAI.
If the simulation is run in real-time, it uses RTAI mailboxes, 
otherwise regular UNIX sockets are used, either in the \kw{local} or 
in the \kw{internet} namespace.

This drive type is intended to allow the development of real-time models
by running regular simulations for the purpose of debugging the model
and the control process without the overhead and the potential problems
of running in real-time.
Then the same model can be run in real-time without changing the details
of the communication with the controller process.

\paragraph{Non real-time simulation}
During non real-time simulations, streams operate in blocking mode.
The meaning of the parameters is:
\begin{itemize}
\item \nt{stream\_name} indicates the name the stream will be known as;
it is mostly useless, and must be no more than 6 characters long,
since it is only allowed for compatibility with RTAI's mailboxes;

\item the instruction \kw{create} determines whether the socket will be
created or looked for by MBDyn; if \kw{create} is set to \kw{no},
MBDyn will retry for 60 seconds and then give up;

\item the keyword \kw{local} indicates that a socket 
in the local namespace will be used; if \kw{create} is set to \kw{yes},
the socket is created, otherwise it must exist.

\item either of the keywords \kw{port} or \kw{host} indicate that a socket
in the internet namespace will be used;
if \kw{create} is set to \kw{yes}, \nt{host\_name} indicates 
the host that is allowed to connect to the socket; it defaults 
to any host (0.0.0.0); if \kw{create} is set to \kw{no},
\nt{host\_name} indicates what host to connect to; the default 
is localhost (127.0.0.1); the default port is 9011;

\item the keyword \kw{no signal} disables raising a \texttt{SIGPIPE}
in case the stream is read after it was closed by the peer;

\item the keyword \kw{input every} allows to read new driver values
every \nt{steps} time steps;

\item the keyword \kw{receive first} allows to enable/disable
receiving data at the initial time (defaults to \kw{yes},
thus data are expected at the initial time, during derivatives);

\item the keyword \kw{timeout} allows to set a timeout
for each read operation; the timeout is expressed in seconds
as a floating point number, with a theoretical micro-second resolution
(the actual resolution is left to the OS);

\item the keyword \kw{echo} allows to echo the streamed data
into file \nt{echo\_file\_name}, using the optional \kw{precision}
and a \kw{shift} for the time, to make sure step changes are captured
correctly (a small negative value is recommended);
the resulting file is suitable to re-execute the simulation
using a \kw{variable step} file driver, using a \kw{const} value
for the \kw{interpolation} option;

\item the keyword \kw{initial values},
followed by \nt{columns\_number} real values,
allows to set the initial values of each channel;
this is useful for example if the initial value of a channel
is needed during initial assembly, before the first set of values
is read from the peer.

\item the keyword \kw{modifier}... TODO
\end{itemize}
If no socket type is specified, i.e.\ none of the \kw{local}, \kw{port} 
and \kw{host} keywords are given, a socket is opened by default 
in the internet namespace with the default IP and port; the \kw{create}
keyword is mandatory.

\paragraph{Real-time simulation}
During real-time simulations, streams wrap non-blocking RTAI mailboxes.
The meaning of the parameters is:
\begin{itemize}
\item the parameter \nt{stream\_name} indicates the name the stream
will be known as in RTAI's resource namespace;
it must no more than 6 characters long, and actually represents the mailbox name;
\item the instruction \kw{create} determines whether the mailbox will be
created or looked for by MBDyn;
\item the keyword \kw{local} is ignored;
\item the keyword \kw{host} indicates that a mailbox on a remote host 
will be used; it is useless when \kw{create} is set to \kw{yes}, because
RTAI does not provide the possibility to create remote resources;
if none is given, a local mailbox is assumed;
\item the keyword \kw{port} is ignored.
\end{itemize}

The parameter \nt{columns\_number} determines how many
channels will be used.
A channel is a double typed number; a \kw{stream drive} can read
an arbitrary number of simultaneous channels.
