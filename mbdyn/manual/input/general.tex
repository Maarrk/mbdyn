\chapter{General}
The input is divided in blocks. 
This is a consequence of the fact that almost every module of the code 
needs data and each module is responsible for its data input. 
So it is natural to make each module read and interpret its data starting 
from the input phase.
Every command (or card) has the following ``grammar'':
\begin{verbatim}
    <card> ::= <description> [ : <arglist> ] ;
\end{verbatim}
\texttt{arglist} is a (optional) list of arguments, that is driven by the 
\texttt{description} that identifies the card. 
The keyword can be made of multiple words separated by spaces or tabs; 
the extra spaces are skipped\footnote{
    Anything that makes the ``C'' function \texttt{isspace()} return \texttt{true}
}, and the match is case insensitive. 
The arguments are usually separated by commas\footnote{
    A few exceptions require a colon to separate blocks of arguments;
    wherever it is feasible, those exceptions will be eliminated in future
    versions. Those cards will be marked as deprecated.
}.
A semicolon ends the card. 
Many cards allow for extra arguments that assume default values 
if not supplied by the user. 
Usually these arguments are at the end of the card and must follow some 
rules. 
A check for the existence of extra args is made, and they are read in a 
fixed sequence.
When structured arguments like matrices, vectors, or drive callers are
expected, they are parsed by dedicated functions.
Anyway, the structured data always follows the rules of the general data. 
Few exceptions are present, and will be eliminated soon.
Every data module is surrounded by the control statements \texttt{begin} and
\texttt{end}:
\begin{verbatim}
    begin: <module_name> ;
        ...
    end: <module_name> ;
\end{verbatim}
The general rules follow.



\section{Output}
The program outputs data to a set of files for each simulation.
The contents of the file is related to the extension of the input file.
If no input file is explicitly supplied, and input is obtained from 
\texttt{stdin}, the output file name defaults to ``MBDyn''; otherwise, unless
the output file name is explicitly set, the name of the input file is used.
The contents of the output files are described accordingly to the items
(nodes or elements) that generate them.
Only a general information file, with extension \texttt{.out}, is described
here. 
The file contains general information about the simulation; it is not
formatted. 
In detail, for each step the current time, time step, the number of
iterations and the error are supplied.



\section{Numeric Values}
Every time a numeric value is expected, a mathematical expression
can be used, including variable declaration and assignment (variable names
and values are kept in memory throughout the input phase and the
simulation) and simple single-value math functions.
Named variables and non-named constants are strongly typed; two types are
available, \texttt{integer} and \texttt{real}.
Operations account for the type and perform implicit cast when allowed.
For instance \texttt{1+2.5} returns a \texttt{real} \texttt{3.5}, since one of the 
two addenda is \texttt{real}, while \texttt{1/3} returns \texttt{0} because 
the integer division is used.
An empty field, delimited by a valid separator (a comma or a semicolon,
depending on whether other arguments are expected or not) returns the
(program supplied) default value for that field, if supplied by the caller, 
otherwise the parser automatically defaults to zero.
Multiple expressions can be used, provided they are enclosed in plain 
brackets and are separated by semicolons.
The result of the last expression will be used as the expected numeric value,
but all the expressions (typically variable declarations and assigments) 
will be evaluated.
Example:
\begin{verbatim}
    1.
    (real r = 2.*pi; integer i = 1.; sin(i*r*Time+.87))      
\end{verbatim}
Note that the constant \texttt{pi} is always defined
as the machine $ \pi $, as well as the constants \texttt{e} and 
\texttt{MAX\_RAND}.
The variable \texttt{Time} is declared, defined and initialized\footnote{
    A variable is \texttt{declared} when its name enters the namespace;
    it is \texttt{defined} when it can be referenced;
    it is \texttt{initialized} when it is first assigned a value
} from the beginning of the control data section, and during the solution 
phase it is assigned the value of the current time. 

\section{Higher Level Structures}
Every time a higher level structure is expected, it can be
substituted by a keyword that influences how the structure is read.
The main data structures are:
\subsection{3 x 1 vectors}
\begin{enumerate}
    \item general case: a sequence of 3 reals, comma-separated.
    \item null vector: keyword \texttt{null}; the vector is initialised
    with zeros.
\end{enumerate}
As an example, all the following lines define an empty 3 x 1 vector:
\begin{verbatim}
    null
    0.,0.,0.
    ,,
\end{verbatim} 
\subsection{6 x 1 vectors}
\begin{enumerate}
    \item general case: a sequence of 6 reals, comma-separated.
    \item null vector: keyword \texttt{null}; the vector is initialised 
    with zeros.	
\end{enumerate}
\subsection{3 x 3 matrices}
\begin{enumerate}
    \item general case: a sequence of 9 reals, comma-separated, which
    represent the row-oriented coefficients $ a_{11} $, $ a_{12}$ ,
    \ldots, $ a_{32} $, $ a_{33} $.
    \item symmetric matrix: keyword \texttt{sym}, followed by a sequence
    of 6 reals, comma-separated, that represents the upper triangle, 
    row-oriented coefficients of a symmetric matrix, 
    e.g. $ a_{11} $, \ldots , $ a_{13} $, $ a_{22} $, $ a_{23} $, $ a_{33} $.
    \item diagonal matrix: keyword \texttt{diag}, followed by a sequence
    of 3 reals, comma-separated, that represent the diagonal coefficients 
    of a diagonal matrix.
    \item identity matrix: keyword \texttt{eye}; the matrix is initialized
    as the identity matrix, that is a null matrix except for the diagonal 
    coefficients that are 1.
    \item null matrix: keyword \texttt{null}; the matrix is initialised 
    with zeros.
\end{enumerate}
\subsection{3 x 3 rotation matrix}
\begin{enumerate}
    \item general case: two vectors that define an orthonormal reference
    system, each of them preceded by its index in the final rotation 
    matrix. The first vector is normalized and assumed to represent the
    desired direction, while the second simply defines the plane the
    vector that is not given is normal to, e.a.:
    \begin{verbatim}
    1, 1.,0.,0., 2, 0.,1.,0.
    1, (real alpha=pi/6.; 
       cos(alpha)), sin(alpha), 0., 3, 0.,0.,1.
    \end{verbatim}
    the first example represents the identity matrix (no rotation),
    while the second one describes a rotation of $ \pi/6 $ rad.\ about
    axis 3.
    \item identity matrix: keyword \texttt{eye}; the identity matrix, which means
    no rotation.
    \item a complete rotation matrix: keyword \texttt{matr} followed by the
    nine, row-oriented, coefficients, namely $ r_{11} $, $ r_{12} $, \ldots,
    $ r_{33} $.
    \item Euler angles: keyword \texttt{euler}, followed by the three
    values, as output by structural nodes.
\end{enumerate}
\subsection{6 x 6 matrices}
\begin{enumerate}
    \item general case: a sequence of 36 reals, comma-separated, that
    represent the row-oriented coefficients $ a_{11} $, $ a_{12}$ ,
    \ldots, $ a_{65} $, $ a_{66} $.
    \item ANBA format: keyword \texttt{anba}, followed by 36 reals, 
    comma-separated, that represent the coefficients of the beam stiffness
    matrix as generated by the code ANBA, namely the following
    transformation is performed:
    \begin{itemize}
        \item axis $ x $, in the section plane in ANBA notation, 
	becomes axis 2 in MBDyn notation;    
	\item axis $ y $, in the section plane in ANBA notation, 
	becomes axis 3 in MBDyn notation;    
	\item axis $ z $, the beam axis in ANBA notation, 
	becomes axis 1 in MBDyn notation;    
    \end{itemize}
    \item symmetric matrix: keyword \texttt{sym}, followed by a sequence
    of 21 reals, comma-separated, that represents the upper triangle,
    row-oriented coefficients of a symmetric matrix, 
    e.g. $ a_{11} $, \ldots , $ a_{16} $, $ a_{22} $,
    \ldots , $ a_{26} $, \ldots, $ a_{66} $.
    \item diagonal matrix: keyword \texttt{diag}, followed by a sequence
    of 6 reals, comma-separated, that represent the diagonal coefficients 
    of a diagonal matrix.
    \item identity matrix: keyword \texttt{eye}; the matrix is initialized
    as the identity matrix, that is a null matrix except for the diagonal 
    coefficients that are 1.
    \item null matrix: keyword \texttt{null}; the matrix is initialised 
    with zeros.
\end{enumerate}
\subsection{6 x N matrices}
\begin{enumerate}
    \item general case: a sequence of 6 x N reals, comma-separated, that
    represent the row-oriented coefficients $ a_{11} $, $ a_{12}$ ,
    \ldots, $ a_{6\plbr{N-1}} $, $ a_{6N} $.
    \item ANBA format: keyword \texttt{anba}, followed by 6 x N reals,
    comma-separated, that represent the coefficients of the beam stiffness
    matrix as generated by the code ANBA, namely the following
    transformation is performed:
    \begin{itemize}
        \item axis $ x $, in the section plane in ANBA notation, 
	becomes axis 2 in MBDyn notation;    
	\item axis $ y $, in the section plane in ANBA notation, 
	becomes axis 3 in MBDyn notation;    
	\item axis $ z $, the beam axis in ANBA notation, 
	becomes axis 1 in MBDyn notation;    
    \end{itemize}
    \item null matrix: keyword \texttt{null}; the matrix is initialised 
    with zeros.
\end{enumerate}


\section{Input Related Cards} 
Everywhere in the input file the following statement cards can be used.
They don't affect the solution or the input of the simulation data in a
direct manner, because they are handled directly by the parsing object.
They are:

\subsection{Include}
The \texttt{include} directive:
\begin{verbatim}
    <card> ::= include : " <file_name> " ;
\end{verbatim}
where \texttt{file\_name} is a valid filename for the operative system in
use, that must be enclosed in double quotes (").
The full (absolute or relative) path must be given if the included file 
is not in the directory of the including one.
There is no check for recursive \texttt{include}s, so 
{\bf the user must take care of recursion}.
The \texttt{include} directive forces the parser to scan the included file
\texttt{file\_name} before continuing with the including one.
This is very useful if, for instance, a big model can be made of many
small models that are meaningful by themselves. It can be used to
replicate parts of the model, by simply using parametric labels for
nodes, elements, reference systems, and setting a bias value before 
multiple-including the same bulk data file.

\subsection{Set}
The \texttt{set} directive:
\begin{verbatim}
    <card> ::= set : <math_expression> ;
\end{verbatim}
This directive simply invokes the math parser to evaluate the expression
\texttt{math\_expression} and then discards the result. It can be useful
to declare new variables, or to set the values of existing ones.

\subsection{Remark}
The \texttt{remark} directive:
\begin{verbatim}
    <card> ::= remark : " <remark_string > [ , <math_expression> ] ;
\end{verbatim}
This directive simply prints to stdout the string \texttt{remark\_string} and
optionally evaluates the expression \texttt{math\_expression} in a way that is
analogous to the \texttt{set} directive.
It is used to allow rough input debugging, where the file name and line 
is logged, followed by a message, possibly followed by the evaluation 
of an expression. 
Example:
\begin{verbatim}
    remark: "square root of 2", sqrt(2);
\end{verbatim}

\subsection{Reference}
The \texttt{reference} directive:
\begin{verbatim}
    <card> ::= reference : <unique_label> , 
                           <absolute_position> ,
                           <absolute_rotation_matrix> ,
                           <absolute_velocity> ,
                           <absolute_angular_velocity> ;
\end{verbatim}
A \texttt{reference} system is declared and defined.
It must be given a unique identifier, scanned by the mathparser
(which means that any regular expression is allowed, and the result is
rounded up to the nearest unsigned integer).
The entries \texttt{absolute\_*} are parsed by routines that
compute absolute (i.e.\ referring to the global frame) entities
starting from a given entity in a given reference frame.
These routines are very general, and make intense use of the 
\texttt{reference} entries themselves, which means that a reference 
can be recursively defined by means of previously defined 
\texttt{reference} entries.

\subsubsection{Use of Reference Frames}
Every time an absolute or a relative geometric or physical entity is
required, it is processed by a set of routines that allow the entity to be
expressed in the desired reference frame.
The following cases are considered:
\begin{itemize}
    \item relative position (physical)
    \item absolute position (physical)
    \item relative rotation matrix (physical)
    \item absolute rotation matrix (physical)
    \item relative velocity (physical)
    \item absolute velocity (physical)
    \item reltive angular velocity (physical)
    \item absolute angular velocity (physical)
    \item relative arbitrary vector (geometric)
    \item absolute arbitrary vector (geometric)    
\end{itemize}
The caller is responsible for the final interpretation of the input. 
The caller always supplies the routines a default reference structure
the input must be referred to.
So, depending on the caller, the entry can be in the following forms:
\begin{enumerate}
    \item \texttt{<entity>}: \\ 
    the entry is in the default reference frame
    \item \texttt{reference , <reference\_type> , <entity>}: \\
    the entry is in \texttt{reference\_type} reference frame, 
    that is one of \texttt{global}, 
    \texttt{node} or \texttt{local}.
    \item \texttt{reference , <reference\_number> , <entity>}: \\
    the entry is in \texttt{reference reference\_number} reference frame. 
    This reference frame must be already defined. 
    % By default, the global reference frame has number 1, so it can't be redefined.    
\end{enumerate}
Examples:
\begin{itemize}
    \item absolute position:
    \begin{verbatim}
    null
    reference, global, null
    reference, 8, 1., sin(.3*pi), log(3.)
    \end{verbatim}
    \item relative rotation matrix (e.g.\ as required by many constraints and
    thus referred to a node):
    \begin{verbatim}
    eye
    reference, node, eye
    reference, 8, 3, 0., 1., 0., 
                  1, .5, sqrt(3)/2., 0.
    \end{verbatim}
\end{itemize}
Notes: 
\begin{itemize}
    \item the global reference frame has position $ \cubr{0, 0, 0} $,
    rotation matrix \texttt{eye}, velocity $ \cubr{0, 0, 0} $ and angular
    velocity $ \cubr{0, 0, 0} $.
    \item if the caller is not related to a node, the reference type
    \texttt{node} shouldn't be defined. 
    In this case it is considered equivalent to \texttt{local}.
    \item when processing a velocity or an angular velocity, the resulting
    value always accounts for the velocity and angular velocity of the frame
    the entry is referred to. 
    As an example, if a node is defined on a reference frame that has
    non-null angular velocity $ \Omega_R $, and its position 
    $ x_{input} $ is not in the origin $ X_R $ of the reference frame
    it is attached to, its global velocity and angular velocity result
    as the composition of the input values and of those of the reference 
    frame:
    \begin{eqnarray*}    
        w & = & R_R\omega_{input}+\Omega_R \\
	v & = & R_Rv_{input}+V_R+\Omega_R\times\plbr{R_Rx_{input}}
    \end{eqnarray*}
    This, for instance, eases the input of all the parts of a complex system
    that is moving as a rigid body, by defining a reference frame with the
    proper initial velocities, and then referring all the entities, e.g.\ the 
    nodes, to that frame, with null local velocity.
\end{itemize}  
{\em
    Recalling the declaration and the definition of reference frames,
    a simple reference frame definition, with all the entries referring 
    by default to the global system, would be:
    \begin{verbatim}
    reference: 1000, null,
                     eye,
                     null,
                     null;			 
    \end{verbatim}
    which represents a redefinition of the global system.
    A more verbose, and self-explanatory definition would be:
    \begin{verbatim}
    reference: 1000, reference, global, null,
                     reference, global, eye,
                     reference, global, null,
                     reference, global, null;			 
    \end{verbatim}
    the reference frame one is referring to must be repeated for all the entries
    since they must be allowed to refer to whatever frame is preferred 
    by the user.
    A fancier definition would be:
    \begin{verbatim}
    reference: Rotating_structure, 
               reference, Fixed_structure, null,
               reference, Spindle_1, 1, 0.,0.,1., 
                                     3, 0.,1.,0.,
               reference, Fixed_structure, null,
               reference, Spindle_1, 0.,0.,Omega_1;
    \end{verbatim}
}


\subsection{Hydraulic fluid}\label{sec:HYDRAULIC-FLUID}
The \texttt{hydraulic fluid} directive:
\begin{verbatim}
    <card> ::= hydraulic fluid : <unique_label> , 
                                 <fluid_type> , <fluid_properties> ;
\end{verbatim}
A \texttt{hydraulic fluid} is declared and defined.
It is stored for later retrieval to be used in hydraulic elements,
see Section~\ref{sec:HYDRAULIC-ELEMENT}.
The fluid is identified by a numerical label. 
The \texttt{fluid\_type}s, with the related \texttt{fluid\_properties}, are:

\subsubsection{Uncompressible}
\begin{verbatim}
    <fluid_type> ::= uncompressible
    <fluid_properties> ::= [ density , <density> ]
                           [ , viscosity , <viscosity> ]
                           [ , pressure , <pressure > ]
                           [ , temperature , <temperature> ]
\end{verbatim}

\subsubsection{Linearly compressible}
\begin{verbatim}
    <fluid_type> ::= linear compressible
    <fluid_properties> ::= [ density , <ref_density> , 
                                       <beta> , 
                                       <ref_pressure> ]
                           [ , viscosity , <viscosity> ]
                           [ , temperature , <temperature> ]
\end{verbatim}

\subsubsection{Linearly compressible, with thermal dependency}
\begin{verbatim}
    <fluid_type> ::= linear thermal compressible
    <fluid_properties> ::= [ density , <ref_density> , 
                                       <beta> , 
                                       <ref_pressure> ,
                                       <alpha> ,
                                       <ref_temperature> ]
                           [ , viscosity , <viscosity> ]
\end{verbatim}

\subsubsection{Super (linearly compressible, with thermal dependency)}
\begin{verbatim}
    <fluid_type> ::= x-super
    <fluid_properties> ::= [ density , <ref_density> , 
                                       <beta> , 
                                       <ref_pressure> ,
                                       <alpha> ,
                                       <ref_temperature> ]
                           [ , viscosity , <viscosity> ]
\end{verbatim}
according to equation
\begin{displaymath}
	\matr{ll}{
		\rho \ = \ \rho_0 + \rho_{ref}\cfrac{1}{2}\plbr{
			1 + \llk{tanh}\plbr{a\plbr{p-p_{ref}}}
		} & p < p_{ref} \\
		+= \ \cfrac{p-p_{ref}}{\beta} & p > p_{ref}
	}
\end{displaymath}
\emph{Note: highly experimental}

\subsubsection{AMESim's compressible fluid, with saturation}
\begin{verbatim}
    <fluid_type> ::= x-amesim
    <fluid_properties> ::= [ density , <ref_density> , 
                                       <beta> , 
                                       <ref_pressure> ,
                                       <alpha> ,
                                       <ref_temperature> , ]
                           [ viscosity , <viscosity> , ]
                           <psat>
\end{verbatim}
where \texttt{psat} is the saturation pressure, according to equation
\begin{displaymath}
	\matr{ll}{
		\rho \ = \ \rho_0 \e{\cfrac{p-p_0}{\beta}} &
		p > p_{sat} \\
		\rho \ = \ \rho_0 \e{1000\cfrac{p-p_0}{\beta}} &
		p < p_{sat}
	}
\end{displaymath}



\section{Node Degrees of Freedom}
A node in MBDyn is nothing but an entity that owns degrees of freedom and
can lend them to other entities. 
Usually elements access nodal degrees of freedom through well-defined
interfaces, at a high level. 
But in a few cases, nodal degrees of freedom must be accessed
at a very low level, with the bare knowledge of the node label,
the internal number of the degree of freedom, and the order 
(algebraic or differential, if any).
The data that allows an entity to track a nodal degree of freedom is read as
follows:
\begin{verbatim}
    <node_dof> :: = <node_label> , 
                    <node_type> 
                    [ , <dof_number> ] ,
                    [ { algebraic | differential } ]
\end{verbatim}
In case an algebraic node is addressed (e.g.\ a pressure node), 
the \texttt{dof\_order} field is not required.
The label and the type of the node are used to track the pointer to the
desired node. 
If the node is not scalar, the \texttt{dof\_number} field is required.
Finally, the order of the degree of freedom is checked, if required.
It must be one of \texttt{algebraic} or \texttt{differential}.
If the \texttt{dof\_number} degree of freedom is differential, both
of them can be addressed, while in case of an algebraic node there is no
choice, only the \texttt{algebraic} order can be addressed.
The \texttt{dof\_number} must range between 1 and the number of {\em dof}s that
belong to the node.
The second choice is used in those cases where the order is not meaningful,
as for instance when the node is used to address an equation (abstract
forces, discrete control elements). 
In those cases, the contribution to the equation is added regardless 
of the order of the degree of freedom.

\section{Drives and Drive Callers}
Everytime some entity can be ``driven'', that is a value can be
expressed as depedent on some ``external'' input, an object of the class 
\texttt{DriveCaller} is created. This requires the input of the properties of
such an object. 
The family of the \texttt{DriveCaller} object is very populated, 
and should require a chapter.
The type of the \texttt{DriveCaller} is declared as follows:
\begin{verbatim}
    <drive_caller> ::= <drive_caller_type> , <arglist>
\end{verbatim}    
\texttt{arglist} is a list of arguments separated by commas.
As an exception, a constant \texttt{DriveCaller} (that behaves exactly as a
numerical constant with little or no overhead depending on the optimizing
capability of the compiler) is assumed when a numeric value is used instead
of a keyword.
The \texttt{drive\_caller\_type}s are:

\subsection{Drives}

\subsubsection{Null drive}
\begin{verbatim}
    <drive_caller> ::= null
\end{verbatim}
Zero valued.

\subsubsection{One drive}
\begin{verbatim}
    <drive_caller> ::= one
\end{verbatim}
Always 1.

\subsubsection{Constant drive}
\begin{verbatim}
    <drive_caller> ::= [ const , ] <const_coef>                    
\end{verbatim}
The keyword \texttt{const} can be omitted thus highlighting the real nature
of this driver, that is completely equivalent to a constant, static real
value.

\subsubsection{Time drive}
\begin{verbatim}
    <drive_caller> ::= time
\end{verbatim}
Yields the current time.
  
\subsubsection{Linear drive}
\begin{verbatim}
    <drive_caller> ::= linear , <const_coef> , 
                                <slope_coef>
\end{verbatim}

\subsubsection{Parabolic drive}
\begin{verbatim}
    <drive_caller> ::= parabolic , <const_coef> , 
                                   <linear_coef> , 
                                   <parabolic_coef>
\end{verbatim}

\subsubsection{Cubic drive}
\begin{verbatim}
    <drive_caller> ::= cubic , <const_coef> , 
                               <linear_coef> ,
                               <parabolic_coef>, 
                               <cubic_coef>
\end{verbatim}

\subsubsection{Step drive}
\begin{verbatim}
    <drive_caller> ::= step , <initial_time> , 
                              <step_value> ,
                              <initial_value>
\end{verbatim}    

\subsubsection{Double step drive}
\begin{verbatim}
    <drive_caller> ::= double step , <initial_time> , 
                                     <final_time> ,
                                     <step_value> , 
                                     <initial_value>
\end{verbatim}

\subsubsection{Ramp drive}
\begin{verbatim}
    <drive_caller> ::= ramp , <slope> , 
                              <initial_time> ,
                              <final_time> , 
                              <initial_value>
\end{verbatim}
  
\subsubsection{Double ramp drive}
\begin{verbatim}
    <drive_caller> ::= double ramp , <asc_slope> , 
                                     <asc_initial_time> , 
                                     <asc_final_time> , 
                                     <desc_slope> , 
                                     <desc_initial_time> , 
                                     <desc_final_time> , 
                                     <initial_value>
\end{verbatim}

\subsubsection{Piecewise linear drive}
\begin{verbatim}
    <drive_caller> ::= piecewise linear , <num_points> ,
                                          <point> , <value> [ , ... ]
\end{verbatim}
Piecewise linear function; the first and the last point/value pairs are
extrapolated in case a value beyond the extremes is required.
Linear interpolation between pairs is used.

\subsubsection{Sine drive}
\begin{verbatim}
    <drive_caller> ::= sine , <initial_time> ,
                              <pulsation> ,
                              <amplitude> ,
                              <number_of_cycles> , 
                              <initial_value>
\end{verbatim}
the value of \texttt{number\_of\_cycles} determines the behavior of the
drive. If it is positive, \texttt{number\_of\_cycles}$-1/2$ oscillations are
performed. If it is equal to zero, the oscillations never end. If it is
negative, the oscillations end after 
\texttt{number\_of\_cycles}$-3/4$ cycles at the top of the sine, with null
tangent.

\subsubsection{Cosine drive}
\begin{verbatim}
    <drive_caller> ::= cosine , <initial_time> ,
                                <pulsation> ,
                                <amplitude> ,
                                <number_of_cycles> , 
                                <initial_value>
\end{verbatim}
this drive actually computes a function of the type $ 1-\llk{cos}\plbr{x} $.
The value of \texttt{number\_of\_cycles} determines the behavior of the
drive. If it is positive, \texttt{number\_of\_cycles} oscillations are
performed. If it is equal to zero, the oscillations never end. If it is
negative, the oscillations end after
\texttt{number\_of\_cycles}$-1/2$ cycles at the top of the cosine, with null
tangent.   

\subsubsection{Frequency sweep drive}
\begin{verbatim}
    <drive_caller> ::= frequency sweep , <initial_time> ,
                                         <pulsation_drive> ,
                                         <amplitude_drive> ,
                                         <initial_value> ,
                                         <final_time> ,
                                         <final_value>
\end{verbatim}
this drive recursively calls two other drives that supply the pulsation 
and the amplitude of the oscillation. Any drive can be used.

\subsubsection{Exponential drive}
\begin{verbatim}
    <drive_caller> ::= exponential , <amplitude_value> ,
                                     <time_constant_value> ,
                                     <initial_time> ,
                                     <initial_value>
\end{verbatim}

\subsubsection{Random drive}
\begin{verbatim}
    <drive_caller> ::= random , <amplitude_value> ,
                                <mean_value> ,
                                <initial_time> ,
                                <final_time> 
                       [ , steps , <steps_to_hold_value>]
                       [ , seed , { time | <seed_value>} ]
\end{verbatim}
the first optional entry, preceeded by the keyword \texttt{steps}, sets the
number of steps a random value must be held before generating a new
random number. The second optional entry, preceeded by the keyword
\texttt{seed}, sets the new seed for the random number generator. A numeric
value can be used, or the keyword \texttt{time} uses the current time from
the internal clock. A given seed can be used to ensure that two
simulations use exactly the same random sequency (concurrent settings 
are not managed, so it is not very reliable).

\subsubsection{File drive}
A family of file drivers is being planned.
At present only a multiple-valued, constant time-step file drive is
implemented.
The DriveCaller is attached to a file drive object that must be declared
and defined in the \texttt{drivers} section of the input file 
(see Section~\ref{sec:DRIVERS}).
\begin{verbatim}
    <drive_caller> ::= file , <drive_label> [ , <column_number> = 1 ]
\end{verbatim}
\texttt{drive\_label} is the label of the drive the caller is attached to, while
\texttt{column\_number} is the number of the column the caller refers to.

\subsubsection{String drive}
\begin{verbatim}
    <drive_caller> ::= string , " <expression_string> "
\end{verbatim}
\texttt{expression\_string} is a string, delimited by double quotes.
It is parsed by the math parser every time the driver is invoked.
The valiable \texttt{Time} is kept up to date and can be used in the 
string to compute the return value.
Another variable, \texttt{Var}, is set to the value provided by the caller
in case the drive is called with an explicit argument.

\subsubsection{Dof drive}
\begin{verbatim}
    <drive_caller> ::= dof , (node_dof) <driving_dof> ,
                             <func_drive>
\end{verbatim}
a \texttt{node\_dof}, namely the reference to a degree of freedom of a node,
is read. 
Then a recursive call to a drive data is read. 
The driver returns the value of the \texttt{func\_drive} using the value of the 
\texttt{node\_dof} as input instead of the time. 
This can be used as a sort of explicit feedback, to implement fancy
springs (where a force is driven through a function by the displacement
of the node it is applied to) or an active control system. 
Refer to the description of a \texttt{node\_dof} entry for further details.

\subsubsection{Array drive}
\begin{verbatim}
    <drive_caller> ::= array , <num_drives> ,
                               <drive_caller> 
                               [ , <drive_caller> [ , ... ] ]
\end{verbatim}
this is simply a front-end for the linear combination of \texttt{num\_drives} 
normal drives. \\
\texttt{num\_drives} must be at least 1, in which case a simple drive
caller is created. 
Otherwise an array of drive callers is created and at every call their value 
is added to give the final value of the array drive.

\subsection{Template drive}
A particular \texttt{DriveCaller} is the template drive caller. This is made
of a constant entity that multiplies a conventional drive to give a drive
entity of dimensionality higher than that of a simple scalar drive.
there are two types of template drive callers, the \texttt{single} template 
drive caller, described above, and the \texttt{array} template drive caller, 
that is nothing but the sum of an array of \texttt{single}
template drive callers. They are entered as follows:
\begin{verbatim}
    <tpl_drive_caller> :: = {
            single , <entity> , <drive_caller> 
        |
            array , <num_template_drive_callers> ,
                    <entity> , <drive_caller>
                    [ , <entity> , <drive_caller> [ , ... ] ]
        }
\end{verbatim}
where \texttt{<entity>} is a constant of the expected type (scalar, 3 x 1 vector,
6 x 1 vector are the types currently defined, but, since a C++ template has
been used, the implementation of other ones is straightforward). In case of
scalar values, the template reverts by default to a normal drive caller,
such that no overhead is added.
At least 1 drive caller is expected. 
If \texttt{num\_template\_drive\_callers} is exactly 1, only a single
template drive caller is actually costructed, thus avoiding the overhead 
related to the handling of the drive caller array.    



\section{Shapes}
The \texttt{shape} entities are objects that return a value depending on one
(or two, for 2D shapes) dimensionless abscissa, ranging $ \sqbr{-1,1} $.
At present, only 1D shapes are used, by aerodynamic elements.
A \texttt{shape} input format is:
\begin{verbatim}
    <shape_1D> ::= <shape_type> , <shape_arglist>
\end{verbatim}
The shapes currently available are:
\begin{enumerate}
    \item \texttt{const}
    \begin{verbatim}
    <shape_arglist> ::= <const_value>
    \end{verbatim}
    \item \texttt{linear}
    \begin{verbatim}
    <shape_arglist> ::= <value_at_-1> , 
                        <value_at_1>
    \end{verbatim}
    \item \texttt{parabolic}
    \begin{verbatim}
    <shape_arglist> ::= <value_at_-1> , 
                        <value_at_0> , 
                        <value_at_1>
    \end{verbatim}
\end{enumerate}
This form of input has been chosen since, being the shapes mainly used to
interpolate values, it looks more ``natural'' to insert the mapping values
at characteristic points.

\section{Constitutive Laws}
Everytime a ``deformable'' entity requires a constitutive law, a template
constitutive law is read. This has been implemented by means of the C++
templates in order to allow the definition of a general constitutive law
when possible. The ``deformable elements at present are \texttt{rod}s, 1D,
\texttt{deformable hinge}s and \texttt{deformable displacement hinge}s, 3D,  
and \texttt{beam}s, 6D.
The \texttt{beam} element that uses the 6D template constitutive law has not been
implemented yet.
Some constitutive laws are meaningful only when related to some precise
dimension. 
Table~\ref{tab:CONST-LAW-DIM} shows the availability of each constitutive law.

\begin{table}[h]
    \newlength{\constlawwidth}
    \setlength{\constlawwidth}{70mm}
    \centering
    \caption{\em Constitutive Laws Availability}\label{tab:CONST-LAW-DIM}
    \begin{tabular}{l|c|c|c} 
        \hline
        Constitutive Law & 1D & 3D & 6D \\ 
	\hline
	linear elastic, linear elastic isotropic               & x & x & x \\
	linear elastic generic                                 & x & x & x \\
	linear elastic generic axial torsion coupling          &   &   & x \\
	log elastic                                            & x &   &   \\
	linear elastic generic bi-stop                         & x & x & x \\
	double linear elastic                                  & x & x &   \\
	isotropic hardening elastic                            & x & x & x \\
	linear viscous, linear viscous isotropic               & x & x & x \\
	linear viscous generic                                 & x & x & x \\
	linear viscoelastic, linear viscoelastic isotropic     & x & x & x \\
	linear viscoelastic generic                            & x & x & x \\
	double linear viscoelastic                             & x & x &   \\
	turbulent viscoelastic                                 & x &   &   \\
	linear viscoelastic generic bi-stop                    & x & x & x \\
	GRAALL damper                                          & x &   &   \\
	\hline
    \end{tabular}
\end{table}

\noindent 
The contitutive laws are entered as follows:
\begin{verbatim}
    <const_law> ::= <specific_const_law>                        
            [ , prestress, (entity) <prestress>
            [ , prestrain, (entity_tpl_driver) <prestrain> ] ]
    <specific_const_law> ::= <const_law_name> , 
                             <const_law_data>
\end{verbatim}
where \texttt{const\_law\_name} is the name of the constitutive law and
\texttt{const\_law\_data} depend on the specific constitutive law. 
The following fields, the type of which depends on the dimension of the
constitutive law, usually are optional, under the assumption that the
constitutive law is the last field in a card or that any ambiguity can be
avoided.
The data specific to the constitutive laws currently available must be
entered as follows:


\subsubsection{Linear elastic, linear elastic isotropic}
\begin{verbatim}
    <specific_const_law> ::= linear elastic [ isotropic ] , 
                             (scalar) <stiffness>
\end{verbatim}
the isotropic stiffness coefficient
  
  
\subsubsection{Linear elastic generic}
\begin{verbatim}
    <specific_const_law> ::= linear elastic generic ,  
                             (derivative_of_entity) <stiffness>
\end{verbatim}
the stiffness matrix. In case of 1D, the type is scalar, while, in case of 
n x 1 vectors, the type is the corrisponding n x n matrix.

\subsubsection{Linear elastic generic axial torsion coupling}
\begin{verbatim}
    <specific_const_law> ::= 
        linear elastic generic axial torsion coupling ,  
            (derivative_of_entity) <stiffness> ,
            (scalar) <coupling_coefficient>
\end{verbatim}
this is defined only for 6 x 1 vectors, where the torsion stiffness,
coefficient $ a_{44} $ in the stiffness matrix, depends linearly on 
the axial strain, $ \varepsilon_1 $, by means of 
\texttt{coupling\_coefficient}.
  
\subsubsection{Log elastic}
\begin{verbatim}
    <specific_const_law> ::= log elastic ,
                             (derivative_of_entity) <stiffness>      
\end{verbatim}
this is defined only for scalars. The force is defined as:
\begin{displaymath}
    f \ = \ \texttt{stiffness} \ \llk{log}\plbr{1+\varepsilon}
\end{displaymath}
  
\subsubsection{Linear elastic bi-stop generic}
\begin{verbatim}
    <specific_const_law> ::= linear elastic bistop,
                             (derivative_of_entity) <stiffness> ,
                             [ initial state , { inactive | active } , ]
                             (DriveCaller)<activating_condition> ,
                             (DriveCaller)<deactivating_condition>
\end{verbatim}
  
\subsubsection{Double linear elastic}
\begin{verbatim}
    <specific_const_law> ::= double linear elastic ,
                             (scalar) <stiffness_1> ,
                             (scalar) <upper_strain> ,
                             (scalar) <lower_strain> ,
                             (scalar) <stiffness_2>
\end{verbatim}
this is defined for scalar and 3 x 1 vectors. In the scalar case the
meaning of the entries is straightforward, while in case of 3 x 1 vectors,
the constitutive law is isotropic but in the local direction 3, where, in
case of strain out of the upper or lower bound, the \texttt{stiffness\_2} is
used.

\subsubsection{Isotropic hardening elastic}
\begin{verbatim}
    <specific_const_law> ::= isotropic hardening elastic ,
                             (scalar) <stiffness> ,
                             (scalar) <reference_strain>
\end{verbatim}
this constitutive law is defined as follows:
\begin{displaymath}
    f \ = \ \texttt{stiffness} \ \frac{
        \alpha\shbr{\varepsilon}^2
    }{
        1+\alpha\shbr{\varepsilon}^2
    }\varepsilon
\end{displaymath}
where $ \alpha=3/\shbr{\texttt{reference\_strain}}^2 $. The resulting
constitutive law, in the scalar case, is somewhat soft when
$ \varepsilon $ is smaller than \texttt{reference\_strain}, while it grows to
quasi-linear for higher $ \varepsilon$s

\subsubsection{Linear viscous, linear viscous isotropic}
\begin{verbatim}
    <specific_const_law> ::= linear viscous [ isotropic ] , 
                             (scalar) <viscosity_coefficient>
\end{verbatim}
the linear viscous coefficient. \\
{\em 
    Note: this constitutive law doesn't require any prestrain template
    drive caller.
}
  
\subsubsection{Linear viscous generic}
\begin{verbatim}
    <specific_const_law> ::= linear viscous generic , 
                             (derivative_of_entity) <viscosity_matrix>
\end{verbatim}
the linear viscous matrix. \\
{\em 
    Note: this constitutive law doesn't require any prestrain template
    drive caller.
}
  
\subsubsection{Linear viscoelastic, linear viscoelastic isotropic}
\begin{verbatim}
    <specific_const_law> ::= linear viscoelastic [ isotropic ] ,
                             (scalar) <stiffness> ,
                             { (scalar) <viscosity_coefficient>
                             | proportional, (scalar) <factor> }
\end{verbatim}
the isotropic stiffness and viscosity coefficients
  
\subsubsection{Linear viscoelastic generic}
\begin{verbatim}
    <specific_const_law> ::= linear viscoelastic generic ,  
                             (derivative_of_entity) <stiffness> ,
                             { (derivative_of_entity) <viscosity_matrix> 
                             | proportional, (scalar) <factor> }
\end{verbatim}
the linear stiffness and viscosity matrices
  
\subsubsection{Double linear viscoelastic}
\begin{verbatim}
    <specific_const_law> ::= double linear elastic ,
                             (scalar) <stiffness_1> ,
                             (scalar) <upper_strain> ,
                             (scalar) <lower_strain> ,
                             (scalar) <stiffness_2> ,
                             (scalar) <viscosity_coefficient>
\end{verbatim}
this is analogous to the \texttt{double linear elastic} constitutive law,
except for the isotropic viscosity term.
  
\subsubsection{Turbulent viscoelastic}
\begin{verbatim}
    <specific_const_law> ::= turbulent viscoelastic ,
                             (scalar) <stiffness> ,
                             (scalar) <parabolic_viscous_coefficient>
                             [ , (scalar) <treshold> 
                                 [ , (scalar) <linear_viscous_coefficient> ] ]
\end{verbatim}
the constitutive law has the form:
\begin{displaymath}
    f \ = \ \texttt{stiffness} \ \varepsilon + k \ \dot{\varepsilon}
\end{displaymath}
where:
\begin{displaymath}
    k = \lcubr{\matr{lcr}{
        \texttt{linear\_viscous\_coefficient} & & 
            \shbr{\dot{\varepsilon}} \leq \texttt{treshold} \\
        \texttt{parabolic\_viscous\_coefficient} & &
            \shbr{\dot{\varepsilon}} > \texttt{treshold}
    }}
\end{displaymath}
if \texttt{treshold} is null, or not defined, the constitutive law is always
parabolic. If the \texttt{linear\_viscous\_coefficient} is not defined, it is
computed based on \texttt{parabolic\_viscous\_coefficient} and on 
\texttt{treshold} to give a continuous force curve (with discontinuous slope).
Otherwise, it can be set by the user to give a discontinuous force curve,
as observed in some fluids at intermediate Reynolds number.

\subsubsection{Linear viscoelastic bi-stop generic}
\begin{verbatim}
    <specific_const_law> ::= linear viscoelastic bistop,
                             (derivative_of_entity) <stiffness> ,
                             (derivative_of_entity) <viscosity_coefficient> ,
                             [ initial state , { inactive | active } , ]
                             (DriveCaller)<activating_condition> ,
                             (DriveCaller)<deactivating_condition>
\end{verbatim}
  
\subsubsection{GRAALL damper}
This is a very experimental constitutive law, based on a nonlinear model
for a hydraulic damper to be used in landing gear modelling.
Basically, it requires the user to supply the name of the GRAALL-style 
input file with damper data.
It will be documented as soon as it reaches an appreciable level of
stability.



\section{Authentication Methods}
Some authentication methods are defined and made available to specific
program modules; they are used to authenticate before accessing some
resources of the program while it is running.
The syntax is:
\begin{verbatim}
    <authentication_method> ::= <method> [ , <specific_data> ]
\end{verbatim}
Authentication methods in general expect some authentication tokens to be
input.
Usually a user name and a password are required. \\
{\em 
    Note: no encryption is used when communicating, so the authentication
    methods are very rough and should not be considered completely reliable.
    Secure Socket connetion or other SSL-like communication protocol may be
    considered in the future.
} \\
Available methods:
\subsubsection{No authentication}
\begin{verbatim}
    <authentication_method> ::= no auth
\end{verbatim}

\subsubsection{Password}
\begin{verbatim}
    <authentication_method> ::= password ,
        user , " <user_name> " ,
        credentials , { prompt | " <user_cred> " }
\end{verbatim}
In case the keyword \texttt{prompt} is given as credentials, the user is
prompted for a password.

\subsubsection{PAM (Pluggable Authentication Modules)}
\begin{verbatim}
    <authentication_method> ::= pam 
        [ , user , " <user_name> " ]
\end{verbatim}
The {\em Linux-PAM} Pluggable Authentication Modules can be used to
authenticate a user. 
If no user name is provided, the effective user id, as proovided by the 
\texttt{geteuid()} system function, is used to retrieve the username of the
owner of mbdyn process.
the \texttt{user} nust be valid. 
The authentication is performed through a system-dependent \texttt{pam}
configuration file.
No checks on the validity of the account or on the permission of opening a
session are made; account, session and password changes should be explicitly
denied to \texttt{mbdyn} to avoid possible security breaks (see the following
example).
The interested reader should consult the documentation that comes with the
package, try for instance
\begin{verbatim}
    http://parc.power.net/morgan/Linux-PAM/index.html
\end{verbatim}
An example is provided with the pachage, in  \texttt{/etc/pam.d/mbdyn}, reading:
\begin{verbatim}
    ### use either of the following:
    auth       required     /lib/security/pam_unix_auth.so
    # auth       required     /lib/security/pam_pwdb.so
    #
    ### no account, session or password allowed
    account    required     /lib/security/pam_deny.so
    session    required     /lib/security/pam_deny.so
    password   required     /lib/security/pam_deny.so
\end{verbatim}
which allows authentication by using standard Un*x or \texttt{libpwdb} based
authentication.




\section{Miscellaneous}
Finally there are some miscellaneous points:
\begin{itemize}
    \item (UN*X systems) Enviroment variables whose name starts with MBDYN may
    be defined and passed to an execution of the mbdyn command.
    The following are recognised at present:
  
    \begin{enumerate}
  
        \item \texttt{MBDYNVARS=<expr\_list>}
	where \texttt{expr\_list} is a series of mathematical expressions
	separated by semicolons. 
	They are parsed and evaluated; if variables are declared, they are
	added to the symbol table to be used durig the whole execution of the
	program.
    
        \item \texttt{MBDYN\_<type>\_<name>=<value>},
	where \texttt{type} is a legal mbdyn type (\texttt{integer} or \texttt{real}),
	\texttt{name} is a legal symbol name and \texttt{value} is a legal
	mathematical expression.
    
    \end{enumerate}
    
    \item Newlines and indentations are not meaningful. But good indentation
    habits can lead to better and more readable input files.
    
    \item Everything that follows the character \texttt{`\#'} is considered a
    remark, and is discarded until the end of the line. 
    This can occur everywhere in the file, even inside a math expression 
    (if any problems occur, please let me know, because chances are 
    it is a bug!)
    
    \item A new style for comments has been introduced, resembling the 
    C programming language style: everything comprised between the marks 
    \begin{verbatim}
    /*
        useful comments make input files readable!
    */
    \end{verbatim}
    is regarded as a remark. 
    This can happen everywhere in the text {\bf except} in the middle 
    of a keyword.
    
    \item (UN*X systems) Whenever a file name is required, the shell-like
    syntax for home directories (i.e.\ \verb1~/filename1
    or \verb1~user/filename1 is automatically resolved if legal [user and]
    filename values are inserted.

    \item The \texttt{license} and the \texttt{warranty} commands
    respectively show the license and the warranty statement under 
    which the code is released on the standard output.
    They do not affect the simulation.
    
\end{itemize}  
 
