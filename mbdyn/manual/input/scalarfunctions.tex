% MBDyn (C) is a multibody analysis code.
% http://www.mbdyn.org
%
% Copyright (C) 1996-2009
%
% Pierangelo Masarati  <masarati@aero.polimi.it>
%
% Dipartimento di Ingegneria Aerospaziale - Politecnico di Milano
% via La Masa, 34 - 20156 Milano, Italy
% http://www.aero.polimi.it
%
% Changing this copyright notice is forbidden.
%
% This program is free software; you can redistribute it and/or modify
% it under the terms of the GNU General Public License as published by
% the Free Software Foundation (version 2 of the License).
% 
%
% This program is distributed in the hope that it will be useful,
% but WITHOUT ANY WARRANTY; without even the implied warranty of
% MERCHANTABILITY or FITNESS FOR A PARTICULAR PURPOSE.  See the
% GNU General Public License for more details.
%
% You should have received a copy of the GNU General Public License
% along with this program; if not, write to the Free Software
% Foundation, Inc., 59 Temple Place, Suite 330, Boston, MA  02111-1307  USA

\section{Scalar functions}\label{sec:SCALARFUNCS}
A \kw{ScalarFunction} object allows to compute the value of a function.
Almost every scalar function is of type \kw{DifferentiableScalarFunction},
derived from \kw{ScalarFunction}, and allows to compute the derivatives of
the function as well. Currently implemented scalar functions are
\begin{itemize}
\item \kw{const}: $f(x)=c$
\item \kw{exp}: $f(x)=m\cdot b^{c\cdot{x}}$
\item \kw{log}: $f(x)=m\cdot\textrm{log}_b(c\cdot{x})$
\item \kw{pow}: $f(x)=x^p$
\item \kw{linear}: linear interpolation between the two points $(x_1,y_1)$
and $(x_2,y_2)$
\item \kw{cubicspline}: cubic natural spline interpolation between the 
set of points $\{(x_i,y_i), i\in[1,k\geq3]\}$
\item \kw{multilinear}: multilinear interpolation between the 
set of points $\{(x_i,y_i), i\in[1,k\geq2]\}$
\item \kw{chebychev}: Chebychev interpolation between the 
set of points $\{a,b)$
\item \kw{sum}: $f(x)=f_1(x) + f_2(x)$
\item \kw{sub}: $f(x)=f_1(x) - f_2(x)$
\item \kw{mul}: $f(x)=f_1(x) \cdot f_2(x)$
\item \kw{div}: $f(x)=f_1(x) / f_2(x)$
\end{itemize}

\noindent
Every \kw{ScalarFunction} card follows the format
\begin{verbatim}
    <card> :: = scalar function : <unique_scalar_func_name>, 
                    <scalar_func_type>,
                    <scalar_func_args>
\end{verbatim}
\begin{verbatim}
    <scalar_func_name> :: = string 
\end{verbatim}

\noindent
The type of scalar function,
\kw{scalar\_func\_type}, together 
with relevant arguments, \kw{scalar\_func\_args},
are as follows:
\subsubsection{Const Scalar Function}
\begin{verbatim}
    <scalar_func_type> ::= const
    <scalar_func_args> ::= <const_coef>
\end{verbatim}
Note: if the \kw{scalar\_func\_type} is omitted,
a \kw{const} scalar function is assumed.

\subsubsection{Exp Scalar Function}
\begin{verbatim}
    <scalar_func_type> ::= exp
    <scalar_func_args> ::=
        [ base , <base> , ]
        [ coefficient , <coef> , ]
        <multiplier_coef>
\end{verbatim}
Note: the optional \kw{base} must be positive (defaults to $e$, Neper's number).

\subsubsection{Log Scalar Function}
\begin{verbatim}
    <scalar_func_type> ::= log
    <scalar_func_args> ::=
        [ base , <base> , ]
        [ coefficient , <coef> , ]
        <multiplier_coef>
\end{verbatim}
Note: the optional \kw{base} (defaults to $e$, Neper's number,
resulting in natural logarithms)
and the optional \kw{coefficient}
must be positive.
The argument must be positive as well, otherwise an exception is thrown.

\subsubsection{Pow Scalar Function}
\begin{verbatim}
    <scalar_func_type> ::= pow
    <scalar_func_args> ::= <exponent_coef>
\end{verbatim}

\subsubsection{Linear Scalar Function}
\begin{verbatim}
    <scalar_func_type> ::= linear
    <scalar_func_args> ::= <point>, <point>
    <point>            ::= <x>, <y>
\end{verbatim}

\subsubsection{Cubic Natural Spline Scalar Function}
\begin{verbatim}
    <scalar_func_type> ::= cubicspline
    <scalar_func_args> ::= [ do not extrapolate , ]
        <point>, 
        <point>, 
        <point>
        [, ...]
        [end]
    <point>            ::= <x>, <y>
\end{verbatim}
The \kw{end} delimiter is required if the card needs to continue;
it can be omitted if the card ends with a semicolon.

\subsubsection{Multilinear Scalar Function}
\begin{verbatim}
    <scalar_func_type> ::= multilinear
    <scalar_func_args> ::= [ do not extrapolate , ]
        <point>, 
        <point>, 
        [, ...]
        [end]
    <point>            ::= <x>, <y>
\end{verbatim}
Unless \kw{do not extrapolate} is set, when the input is outside
the provided values of \kw{x} the value is extrapolated using the slope
of the nearest point pair.

\subsubsection{Chebychev Scalar Function}
\begin{verbatim}
    <scalar_func_type> ::= chebychev
    <scalar_func_args> ::= 
        <lower_bound> , <upper_bound> ,
        [ do not extrapolate , ]
        <coef_0> [ , <coef_1> [ ... ] ],
        [end]
\end{verbatim}
Chebychev polynomials of the first kind are defined as
\begin{equation}
	T_n\plbr{\xi} = \cos\plbr{n \cos^{-1}\plbr{\xi}} ,
\end{equation}
with
\begin{equation}
	\xi = 2 \frac{x}{b - a} + \frac{b + a}{b - a} ,
\end{equation}
which corresponds to the series
\begin{align}
	T_0\plbr{\xi} &= 1 \\
	T_1\plbr{\xi} &= x \\
	\ldots \nonumber \\
	T_n\plbr{\xi} &= 2 \xi T_{n-1}\plbr{\xi} - T_{n-2}\plbr{\xi} .
\end{align}
This scalar function implements the truncated series in the form
\begin{equation}
	f\plbr{x} = \sum_{k=0,n} a_k T_k\plbr{\xi} .
\end{equation}
The first derivative of the series is obtained by considering
\begin{align}
	\frac{\mathrm{d}}{\mathrm{d}\xi}T_0\plbr{\xi} &= 0 \\
	\frac{\mathrm{d}}{\mathrm{d}\xi}T_1\plbr{\xi} &= 1 \\
	\ldots \nonumber \\
	\frac{\mathrm{d}}{\mathrm{d}\xi}T_n\plbr{\xi} &= 
		2 T_{n-1}\plbr{\xi}
		+ 2 \xi \frac{\mathrm{d}}{\mathrm{d}\xi}T_{n-1}\plbr{\xi}
		- \frac{\mathrm{d}}{\mathrm{d}\xi}T_{n-2}\plbr{\xi} ,
\end{align}
so the first derivative of the scalar function is
\begin{equation}
	\frac{\mathrm{d}}{\mathrm{d}x}f\plbr{x} = \frac{\mathrm{d}\xi}{\mathrm{d}x} \sum_{k=1,n} a_k \frac{\mathrm{d}}{\mathrm{d}\xi} T_k\plbr{\xi} .
\end{equation}
Subsequent derivatives follow the rule
\begin{equation}
	\frac{\mathrm{d}^i}{\mathrm{d}\xi^i}T_n\plbr{\xi} = 
		2 i \frac{\mathrm{d}^{i-1}}{\mathrm{d}\xi^{i-1}}T_{n-1}\plbr{\xi}
		+ 2 \xi \frac{\mathrm{d}^i}{\mathrm{d}\xi^i}T_{n-1}\plbr{\xi}
		- \frac{\mathrm{d}^i}{\mathrm{d}\xi^i}T_{n-2}\plbr{\xi} .
\end{equation}
Differentiation of order higher than 1 is not currently implemented.




\subsubsection{Sum Scalar Function}
\begin{verbatim}
    <scalar_func_type> ::= sum
    <scalar_func_args> ::= (ScalarFunction)<f1>, (ScalarFunction)<f2>
\end{verbatim}

\subsubsection{Sub Scalar Function}
\begin{verbatim}
    <scalar_func_type> ::= sub
    <scalar_func_args> ::= (ScalarFunction)<f1>, (ScalarFunction)<f2>
\end{verbatim}

\subsubsection{Mul Scalar Function}
\begin{verbatim}
    <scalar_func_type> ::= mul
    <scalar_func_args> ::= (ScalarFunction)<f1>, (ScalarFunction)<f2>
\end{verbatim}

\subsubsection{Div Scalar Function}
\begin{verbatim}
    <scalar_func_type> ::= div
    <scalar_func_args> ::= (ScalarFunction)<f1>, (ScalarFunction)<f2>
\end{verbatim}
Note: division by zero is checked, and an exception is thrown.

