% MBDyn (C) is a multibody analysis code.
% http://www.mbdyn.org
%
% Copyright (C) 1996-2006
%
% Pierangelo Masarati  <masarati@aero.polimi.it>
%
% Dipartimento di Ingegneria Aerospaziale - Politecnico di Milano
% via La Masa, 34 - 20156 Milano, Italy
% http://www.aero.polimi.it
%
% Changing this copyright notice is forbidden.
%
% This program is free software; you can redistribute it and/or modify
% it under the terms of the GNU General Public License as published by
% the Free Software Foundation (version 2 of the License).
% 
%
% This program is distributed in the hope that it will be useful,
% but WITHOUT ANY WARRANTY; without even the implied warranty of
% MERCHANTABILITY or FITNESS FOR A PARTICULAR PURPOSE.  See the
% GNU General Public License for more details.
%
% You should have received a copy of the GNU General Public License
% along with this program; if not, write to the Free Software
% Foundation, Inc., 59 Temple Place, Suite 330, Boston, MA  02111-1307  USA

\section{Scalar functions}\label{sec:SCALARFUNCS}
A \kw{ScalarFunction} object allows to compute the value of a function.
Almost every scalar function is of type \kw{DifferentiableScalarFunction},
derived from \kw{ScalarFunction}, and allows to compute the derivatives of
the function as well. Currently implemented scalar functions are
\begin{itemize}
\item \kw{const}: $f(x)=c$
\item \kw{log}: $f(x)=m\textrm{log}(x)$
\item \kw{pow}: $f(x)=x^p$
\item \kw{linear}: linear interpolation between the two points $(x_1,y_1)$
and $(x_1,y_1)$
\item \kw{cubicspline}: cubic natural spline interpolation between the 
set of points $\{(x_i,y_i), i\in[1,k\geq3]\}$
\item \kw{multilinear}: multilinear interpolation between the 
set of points $\{(x_i,y_i), i\in[1,k\geq2]\}$
\item \kw{sum}: $f(x)=f_1(x) + f_2(x)$
\item \kw{sub}: $f(x)=f_1(x) - f_2(x)$
\item \kw{mul}: $f(x)=f_1(x) \cdot f_2(x)$
\item \kw{div}: $f(x)=f_1(x) / f_2(x)$
\end{itemize}

\noindent
Every \kw{ScalarFunction} card follows the format
\begin{verbatim}
    <card> :: = scalar function : <unique_scalar_func_name>, 
                    <scalar_func_type>,
                    <scalar_func_args>
\end{verbatim}
\begin{verbatim}
    <scalar_func_name> :: = string 
\end{verbatim}

\noindent
The type of scalar function,
\kw{scalar\_func\_type}, together 
with relevant arguments, \kw{scalar\_func\_args},
are as follows:
\subsubsection{Const Scalar Function}
\begin{verbatim}
    <scalar_func_type> ::= const
    <scalar_func_args> ::= <const_coef>
\end{verbatim}
Note: if the \kw{scalar\_func\_type} is omitted,
a \kw{const} scalar function is assumed.

\subsubsection{Log Scalar Function}
\begin{verbatim}
    <scalar_func_type> ::= log
    <scalar_func_args> ::= [ { base , <base> } , ]
        [ { coefficient , <coef> } , ]
        <multiplier_coef>
\end{verbatim}
Note: the optional base and the optional coefficient
must be positive.
The argument must be positive as well, otherwise an exception is thrown.

\subsubsection{Pow Scalar Function}
\begin{verbatim}
    <scalar_func_type> ::= pow
    <scalar_func_args> ::= <exponent_coef>
\end{verbatim}

\subsubsection{Linear Scalar Function}
\begin{verbatim}
    <scalar_func_type> ::= linear
    <scalar_func_args> ::= <point>, <point>
    <point>            ::= <x>, <y>
\end{verbatim}

\subsubsection{Cubic Natural Spline Scalar Function}
\begin{verbatim}
    <scalar_func_type> ::= cubicspline
    <scalar_func_args> ::= [ do not extrapolate , ]
        <point>, 
        <point>, 
        <point>
        [, ...]
    <point>            ::= <x>, <y>
\end{verbatim}

\subsubsection{Multilinear Scalar Function}
\begin{verbatim}
    <scalar_func_type> ::= multilinear
    <scalar_func_args> ::= [ do not extrapolate , ]
        <point>, 
        <point>, 
        [, ...]
    <point>            ::= <x>, <y>
\end{verbatim}

\subsubsection{Sum Scalar Function}
\begin{verbatim}
    <scalar_func_type> ::= sum
    <scalar_func_args> ::= (ScalarFunction)<f1>, (ScalarFunction)<f2>
\end{verbatim}

\subsubsection{Sub Scalar Function}
\begin{verbatim}
    <scalar_func_type> ::= sub
    <scalar_func_args> ::= (ScalarFunction)<f1>, (ScalarFunction)<f2>
\end{verbatim}

\subsubsection{Mul Scalar Function}
\begin{verbatim}
    <scalar_func_type> ::= mul
    <scalar_func_args> ::= (ScalarFunction)<f1>, (ScalarFunction)<f2>
\end{verbatim}

\subsubsection{Div Scalar Function}
\begin{verbatim}
    <scalar_func_type> ::= div
    <scalar_func_args> ::= (ScalarFunction)<f1>, (ScalarFunction)<f2>
\end{verbatim}
Note: division by zero is checked, and an exception is thrown.

