% MBDyn (C) is a multibody analysis code.
% http://www.mbdyn.org
%
% Copyright (C) 1996-2007
%
% Pierangelo Masarati  <masarati@aero.polimi.it>
%
% Dipartimento di Ingegneria Aerospaziale - Politecnico di Milano
% via La Masa, 34 - 20156 Milano, Italy
% http://www.aero.polimi.it
%
% Changing this copyright notice is forbidden.
%
% This program is free software; you can redistribute it and/or modify
% it under the terms of the GNU General Public License as published by
% the Free Software Foundation (version 2 of the License).
% 
%
% This program is distributed in the hope that it will be useful,
% but WITHOUT ANY WARRANTY; without even the implied warranty of
% MERCHANTABILITY or FITNESS FOR A PARTICULAR PURPOSE.  See the
% GNU General Public License for more details.
%
% You should have received a copy of the GNU General Public License
% along with this program; if not, write to the Free Software
% Foundation, Inc., 59 Temple Place, Suite 330, Boston, MA  02111-1307  USA

\documentclass[12pt,fleqn]{article}
% $Header$
% Copyright (C) 1996-2013 Pierangelo Masarati <masarati@aero.polimi.it>
% Dipartimento di Ingegneria Aerospaziale, Politecnico di Milano
%
% Parentesi: tonde, quadre, curly, dritte, doppie e angolari.
\newcommand{\plbr}[1]{ \left( #1 \right) }
\newcommand{\sqbr}[1]{ \left[ #1 \right] }
\newcommand{\cubr}[1]{ \left\{ #1 \right\} }
\newcommand{\shbr}[1]{ \left| #1 \right| }
\newcommand{\nrbr}[1]{ \left\| #1 \right\| }
\newcommand{\anbr}[1]{ \langle #1 \rangle }

% Parentesi solo a sinistra: tonde, quadre, curly, dritte, doppie e angolari.
\newcommand{\lplbr}[1]{ \left( #1 \right. }
\newcommand{\lsqbr}[1]{ \left[ #1 \right. }
\newcommand{\lcubr}[1]{ \left\{ #1 \right. }
\newcommand{\lshbr}[1]{ \left| #1 \right. }
\newcommand{\lnrbr}[1]{ \left\| #1 \right. }
\newcommand{\lanbr}[1]{ \langle #1 \right. }

% Parentesi solo a destra: tonde, quadre, curly, dritte,doppie e angolari.
\newcommand{\rplbr}[1]{ \left. #1 \right) }
\newcommand{\rsqbr}[1]{ \left. #1 \right] }
\newcommand{\rcubr}[1]{ \left. #1 \right\} }
\newcommand{\rshbr}[1]{ \left. #1 \right| }
\newcommand{\rnrbr}[1]{ \left. #1 \right\| }
\newcommand{\ranbr}[1]{ \left. #1 \rangle }

% Vettori verticali:
\newcommand{\vvect}[2]{ \begin{array}{ #1 } #2 \end{array} }
\newcommand{\cvvect}[1]{ \begin{array}{c} #1 \end{array} }
\newcommand{\lvvect}[1]{ \begin{array}{l} #1 \end{array} }
\newcommand{\rvvect}[1]{ \begin{array}{r} #1 \end{array} }

% Vettori orizzontali:
\newcommand{\hvect}[2]{ \begin{array}{ #1 } #2 \end{array} }

% Matrici:
\newcommand{\matr}[2]{ \begin{array}{ #1 } #2 \end{array} }

% Integrali: uso \intg{inf}{sup}{arg}{dvar}
\newcommand{\intg}[4]{ \int_{#1}^{#2} {#3} \ {#4} }

% Limite: uso \limt{var}{lim}{arg}
\newcommand{\limt}[3]{ \lim_{{#1} \rightarrow {#2}} {#3}}

% LogLike functions
\newcommand{\llk}[1]{\ensuremath{\mathrm{#1}}}

\newcommand{\diag}[0]{\llk{diag}}
\newcommand{\tr}[0]{\llk{tr}}
\newcommand{\sym}[0]{\llk{sym}}
\newcommand{\skw}[0]{\llk{skw}}

\newcommand{\step}[0]{\llk{step}}
\newcommand{\imp}[0]{\llk{imp}}

\newcommand{\grad}[0]{\llk{grad}}
\newcommand{\divr}[0]{\llk{div}}
\newcommand{\rot}[0]{\llk{rot}}

% In italiano ...
\newcommand{\sca}[0]{\llk{sca}}

% first, second, etc
\newcommand{\first}[0]{1\ensuremath{^{\mathrm{st}}}}    % 1^st
\newcommand{\second}[0]{2\ensuremath{^{\mathrm{nd}}}}   % 2^nd
\newcommand{\third}[0]{3\ensuremath{^{\mathrm{rd}}}}    % 3^rd
\newcommand{\rth}[0]{\ensuremath{^{\mathrm{th}}}}       %  ^th

\newcommand{\degr}[0]{\ensuremath{^{\mathrm{o}}}}

% esponenziale
\providecommand{\e}[1]{\llk{e}^{#1}}


\begin{document}

\title{Scalar Filter}
\author{Pierangelo Masarati}
\date{}
\maketitle

Consider a filter in the rational form
\begin{displaymath}
    y\plbr{s} \ = \ \frac{
        b_0 + b_1 s + {\ldots} + b_{nb} s^{nb}
    }{
        a_0 + a_1 s + {\ldots} + a_{na} s^{na}
    } u\plbr{s}
\end{displaymath}
with $na\geq{nb}$.
It can be realized in the state space form
\begin{displaymath}
    \cubr{\cvvect{
        \dot{x}_0 \\
	\dot{x}_1 \\
	\vdots \\
	\dot{x}_{na-1}
    }} \ = \ \sqbr{\matr{cccc}{
        0 & {\ldots} & 0 & \alpha_0 \\
	1 & {\ldots} & 0 & \alpha_1 \\
	\vdots & \ddots & \vdots & \vdots \\
	0 & {\ldots} & 1 & \alpha_{na-1}
    }}\cubr{\cvvect{
        x_0 \\
	x_1 \\
	\vdots \\
	x_{na-1}
    }} + \cubr{\cvvect{
        \beta_0 \\
	\beta_1 \\
	\vdots \\
	\beta_{na-1}
    }}
\end{displaymath}
\begin{displaymath}
    y \ = \ \sqbr{\matr{cccc}{
        0 & {\ldots} & 0 & 1
    }}\cubr{\cvvect{
        x_0 \\
	x_1 \\
	\vdots \\
	x_{na-1}
    }} + \beta_{na} u
\end{displaymath}
By differentiating each row of the state equation as many times as the row
index, the following relation results
\begin{eqnarray*}
    \dot{x}_0 & = & \alpha_0 x_{na-1} + \beta_0 u \\
    \ddot{x}_1 & = & \dot{x}_0 + \alpha_1 \dot{x}_{na-1} + \beta_1 \dot{u} \\
    & & {\ldots} \\
    x_{na-2}^{\plbr{na-1}} & = & 
        x_{na-3}^{\plbr{na-2}} + \alpha_{na-2} x_{na-1}^{\plbr{na-2}}
	+ \beta_{na-2} u^{\plbr{na-2}} \\
    x_{na-1}^{\plbr{na}} & = & 
        x_{na-2}^{\plbr{na-1}} + \alpha_{na-1} x_{na-1}^{\plbr{na-1}}
	+ \beta_{na-1} u^{\plbr{na-1}}
\end{eqnarray*}
by considering the last equation, and by recursively substituting each of
the preceding equations into it, the following relation results
\begin{eqnarray*}
    x_{na-1}^{\plbr{na}} & = & 
        \alpha_0 x_{na-1}^{\plbr{0}} + \beta_0 u^{\plbr{0}}
        + \alpha_1 x_{na-1}^{\plbr{1}} + \beta_1 u^{\plbr{1}} \\
	& & \mbox{} + {\ldots} 
	+ \alpha_{na-2} x_{na-1}^{\plbr{na-2}} + \beta_{na-2} u^{\plbr{na-2}} \\
	& & \mbox{}
	+ \alpha_{na-1} x_{na-1}^{\plbr{na-1}} + \beta_{na-1} u^{\plbr{na-1}}
\end{eqnarray*}
and, by considering that 
\begin{displaymath}
    x_{na-1} \ = \ y - \beta_{na} u
\end{displaymath}
and that
\begin{displaymath}
    x_{na-1}^{\plbr{j}} \ = \ y^{\plbr{j}} - \beta_{na} u^{\plbr{j}}
\end{displaymath}
the relation 
\begin{eqnarray*}
    \lefteqn{
        y^{\plbr{na}} - \alpha_{na-1} y^{\plbr{na-1}}
	- {\ldots} - \alpha_1 y^{\plbr{1}} - \alpha_0 y^{\plbr{0}}
    } \\
    & = & \beta_{na} u^{\plbr{na}} 
    + \plbr{\beta_{na-1}-\alpha_{na-1}\beta_{na}} u^{\plbr{na-1}} 
    + {\ldots} \\
    & & \mbox{}
    + \plbr{\beta_1-\alpha_1\beta_{na}} u^{\plbr{1}}
    + \plbr{\beta_0-\alpha_0\beta_{na}} u^{\plbr{0}}
\end{eqnarray*}
which corresponds to the initial equation by setting
\begin{eqnarray*}
    \alpha_i & = & \frac{a_i}{a_{na}} \\
    \beta_{na} & = & \frac{b_{na}}{a_{na}} \\
    \beta_i & = & \frac{b_i}{a_{na}} + \alpha_i \beta_{na}
\end{eqnarray*}

\end{document}
